\documentclass[11pt]{book}

\usepackage[a4paper,left=20mm,right=20mm,top=40mm,bottom=40mm]{geometry}
\usepackage[utf8]{inputenc}
\usepackage{amsmath, amsthm, amsfonts, kotex, enumitem, setspace, hyperref, lipsum, amssymb, tabularray, mathtools}
\usepackage{chngcntr}
\usepackage{fancyhdr}
\usepackage{tikz}
\usepackage{graphicx}
\usetikzlibrary{matrix,calc,arrows}

\setstretch{1.4}

\numberwithin{equation}{chapter}

\pagestyle{fancy}
\fancyhf{}
\fancyhead[RE,LO]{\rightmark}
\fancyfoot[RO]{\leftmark}
\fancyhead[LE,RO]{\thepage}

% 공통

\def\NN{\mathbb{N}}
\def\ZZ{\mathbb{Z}}
\def\QQ{\mathbb{Q}}
\def\RR{\mathbb{R}}
\def\CC{\mathbb{C}}

\def\SS{\mathbb{S}}
\def\PP{\mathbb{P}}
\def\BB{\mathbb{B}}
\def\TT{\mathbb{T}}
\def\DD{\mathbb{D}}
\def\HH{\mathbb{H}}

\def\RP{\mathbb{RP}}

\def\eps{\epsilon}


\def\calT{\mathcal{T}}
\def\calB{\mathcal{B}}
\def\calS{\mathcal{S}}
\def\calA{\mathcal{A}}
\def\calC{\mathcal{C}}
\def\calU{\mathcal{U}}
\def\calF{\mathcal{F}}
\def\calG{\mathcal{G}}
\def\calP{\mathcal{P}}
\def\calR{\mathcal{R}}

\def\GL{\mathbf{GL}}
\def\SL{\mathbf{SL}}
\def\OO{\mathbf{O}}
\def\SO{\mathbf{SO}}
\def\UU{\mathbf{U}}
\def\SU{\mathbf{SU}}


\def\g{\mathfrak{g}}
\def\gl{\mathfrak{gl}}
\def\sl{\mathfrak{sl}}
\def\o{\mathfrak{o}}
\def\so{\mathfrak{so}}
\def\u{\mathfrak{u}}
\def\su{\mathfrak{su}}

\def\XX{\mathfrak{X}}
\def\M{\mathfrak{M}}

\def\id{\text{id}}
\def\tr{\text{tr}}


\def\rarrow{\rightarrow}
\def\onto{\twoheadrightarrow}
\def\inclusion{\xhookrightarrow{}}

\newcommand{\inner}[2]{\langle#1, #2\rangle}
\newcommand{\abs}[1]{\left\vert#1\right\vert}
\newcommand{\norm}[1]{\left\Vert#1\right\Vert}

\newcommand{\paren}[1]{\left(#1\right)}
\newcommand{\sqbracket}[1]{\left[#1\right]}

\newcommand{\gen}[1]{\left\langle#1\right\rangle}

\def\rk{\text{rk}}
\newcommand{\pd}[2]{\frac{\partial #1}{\partial #2}}
\newcommand{\doublepd}[3]{\frac{\partial^2 #1}{\partial #2 \partial #3}}

\newcommand{\lowint}[2]{\underline{\int_{#1}^{#2}}}
\newcommand{\upint}[2]{\overline{\int_{#1}^{#2}}}


\def\normal{\trianglelefteq}
\def\hasnormal{\rhd}
\def\im{\text{im}}

\def\res{\text{res}}

\def\textif{\text{if}}
\def\otw{\text{otherwise}}

\def\hom{\text{Hom}}

\def\Int{\text{int}}

\def\diam{\text{diam}}



\theoremstyle{definition}
\newtheorem{thm}{Theorem}[section]
\newtheorem{cor}[thm]{Corollary}
\newtheorem{lem}[thm]{Lemma}
\newtheorem{prop}[thm]{Proposition}
\newtheorem{defn}[thm]{Definition}
\newtheorem{notn}[thm]{Notation}
\newtheorem*{rem}{Remark}
\newtheorem*{ex}{Example}

\makeatletter
\newtheorem*{rep@theorem}{\rep@title}
\newcommand{\newreptheorem}[2]{%
\newenvironment{rep#1}[1]{%
 \def\rep@title{#2 \ref*{##1}}%
 \begin{rep@theorem}}%
 {\end{rep@theorem}}}
\makeatother

\newreptheorem{thm}{Theorem}

\allowdisplaybreaks[1]


\newenvironment{enum}
	{\begin{enumerate}[label=(\alph*), leftmargin=2\parindent]}
	{\end{enumerate}}


\title{{\huge\bfseries 해석개론의 정수\\
Essence of Undergraduate Real Analysis}\\
{\LARGE\bfseries 1\textsuperscript{st} edition}}
\author{}
\date{{\LARGE \today}}
\begin{document}
\maketitle

\chapter*{Preface}
\addcontentsline{toc}{chapter}{Preface}

{\large\bfseries 필자 소개}

본 {\LaTeX} 문서를 조판한 사람은 초판이 완성된 시점에서 서울대학교 의예과 및 수리과학부(복수전공)에 재학 중이다.

\noindent{\large\bfseries 이 문서는 어떤 문서인가?}

이 문서의 성격은 `복습 노트'이다. 학부 2학년 해석개론을 공부하고 나서, 필자가 이해한 바에 따라 해당 과목의 내용을 한 파일 안에 정리한 것이다. ``정수''라는 제목이 오만하게 느껴질 수도 있겠으나 입에 착 달라붙어서 그냥 썼다. \ref{ref1}과 \ref{ref2}\를 따라 필자가 공부하였으므로, 이 노트는 두 책의 내용과 구성을 상당히 많이 따르고 있음을 미리 밝힌다.

\noindent{\large\bfseries 왜 만들었는가?}

2021년 말-2022년 초 겨울방학에 {\LaTeX}을 처음 배운 필자는 뭔가 의미 있는 문서를 한번 만들어 보고 싶어 해석개론 노트를 쓰기 시작했는데, 여러 사정으로 {\textendash} 주로 게으름 때문에 {\textendash} 끝까지 완성하지 못하였다. 미완성인 채로 남아있는 것이 마음에 계속 걸려서 2022년 여름에 노트 작성을 처음부터 다시 시작하였고, 방학 동안 한 달이 조금 넘는 작업 끝에 본 문서를 세상에 내놓게 되었다.

\noindent{\large\bfseries 노트의 구성}

앞에서 밝혔듯이 기본적으로 \ref{ref1}과 \ref{ref2}의 내용과 구성을 따라가고 있으며, 해당 참고문헌의 서술이 불충분하다고 판단한 경우 해석개론 수준에서 이해할 수 있는 새로운 내용(주로 학부 위상수학(\ref{ref4})의 내용)을 추가하였다. 다변수해석학과 측도론은 다루지 않았다. 1장부터 9장까지가 해석개론 1의 범위, 10장부터 15장까지가 해석개론 2의 범위이다.

\noindent{\large\bfseries 노트의 특징}

가급적 일반적인 범위에서 논의를 진행함으로써 더 넓은 시각에서 해당 내용을 조망할 수 있게 하였다. 예를 들어, 일반적인 거리공간에 대해 성립하는 성질은 거리공간의 개념으로 증명하고 유클리드공간에서 특히 성립하는 성질에 대해서만 논의의 범위를 유클리드공간으로 한정하였다. 그리고 대부분의 해석개론 교과서에서는 ``거리공간에서 옹골집합, 극한점 옹골집합, 수열 옹골집합이 동치이다''라는 사실을 증명하지 않는데, 해당 내용을 증명하는 것이 그렇게 어렵지 않고 이를 가정함으로써 이후 아르젤라-아스콜리 정리에서 더 풍부한 내용을 다룰 수 있다고 판단하여 해당 사실의 증명을 수록하였다.

\noindent{\large\bfseries 해석개론을 어떻게 공부하면 좋은가?}

고등학교 및 학부 1학년 수준의 미적분학을 대충 알고 있다면 해석개론의 내용을 따라가는 데는 충분하다. 하지만 해석개론은 대부분의 학생들이 처음 만나는 ``진지한 수학''이기 때문에, 증명이나 문제 풀이의 방법에 익숙해지려면 많은 노력이 필요하다. 증명과 문제 풀이에서 자주 쓰이는 테크닉을 익히고 직접 써 보는 것이 중요하다. 그리고 증명만큼이나 중요한 것이 수많은 예시와 반례를 알고 있는 것이다. 필자의 능력 부족 및 귀찮음으로 이 노트에는 많은 예시가 들어있지 않으며 이것이 이 노트의 한계 중 하나라 생각한다. [\ref{ref1}]\과 [\ref{ref2}]의 연습문제를 많이 풀어보면 좋겠다. 대부분의 문제를 못 푸는 것이 정상이니 못 푼다고 좌절하지 말고, 풀이집을 참고하여 증명과 문제풀이의 아이디어들을 계속 얻어 가면 실력이 크게 향상될 것이다.

\noindent{\large\bfseries 문의 및 정오표}

오타 및 오류 제보, 조판이 이상한 부분, 질문, 의견 등은 \href{mailto:bih011122@gmail.com}{\texttt{bih011122@gmail.com}}으로 메일을 보내주거나 \href{https://susiljob.tistory.com/53}{\texttt{https://susiljob.tistory.com/53}}에 댓글을 달아주면 된다. 정오표는 해당 티스토리 글에 수시로 업데이트할 예정이다.

\tableofcontents

\chapter{실수와 복소수}

중학교에서 무리수를 배운 이래로 실수를 계속 사용해 왔지만, 한 번도 실수가 무엇인지 정의한 적이 없었다. 해석개론의 첫 번째 목표는 실수 집합 $\RR$를 제대로 정의하는 것이다.

\section{체}

    \begin{defn}
        집합 \(X\)에 대하여 함수 \(*: X \times X \to X\)를 \(X\) 위의 \textbf{이항 연산(binary operation) \(*\)}이라고 한다.
    \end{defn}

    \begin{defn}
        집합 \(F\)에 이항연산 \(+, \cdot: F \times F \to F\)가 정의되어 다음 (F1)-(F9)를 만족할 때, \((F, +, \cdot)\) 또는 간단히 \(F\)를 \textbf{체(field)}라고 부르고 이때 \(+\)와 \(\cdot\)을 각각 덧셈과 곱셈이라고 부른다.
        \begin{enumerate} [label=(F\arabic*), leftmargin=2\parindent]
			\item
			모든 \(x, y, z \in F\)에 대해 \(x + (y+z) = (x+y)+z\).
			\item
			\(0 \in F\)가 존재하여 [모든 \(x \in F\)에 대해 \(x+0=0+x=x\)].
			\item
			각 \(x \in F\)에 대하여 \(-x \in F\)가 존재하여 \(x+(-x)=(-x)+x=0\).
			\item
			모든 \(x, y \in F\)에 대해 \(x+y=y+x\).
			\item
			모든 \(x, y, z \in F\)에 대해 \(x(yz)=(xy)z\).
			\item
			\(1 \in F \setminus \{0\}\)이 존재하여 [모든 \(x \in F\)에 대해 \(x\cdot 1=1 \cdot x=x\)].
			\item
			각 \(x \in F \setminus \{0\}\)에 대하여 \(x^{-1} \in F\)가 존재하여 \(xx^{-1}=x^{-1}x=1\).
			\item
			모든 \(x, y \in F\)에 대해 \(xy=yx\).
			\item
			모든 \(x, y, z \in F\)에 대해 \(x(y+z)=xy+xz, (x+y)z = xz+yz\).
		\end{enumerate}
    \end{defn}

다음 성질들은 당연히 만족해야 할 성질들이다.

\begin{prop} \label{prop 1.1.3}
    체 \(F\)와 \(x, y, z \in F\)에 대하여 다음이 성립한다.
    \begin{enumerate} [label=(\alph*), leftmargin=2\parindent]
        \item
        \(x + y = x + z\)이면 \(y = z\).
        \item
        \(y \in F\)가 [모든 \(x \in F\)에 대해 \(x + y = y + x = x\)]를 만족하면 \(y = 0\).
        \item
        \(x + y = y + x = 0\)이면 \(y = -x\).
        \item
        \(-(-x) = x\).
    \end{enumerate}
\end{prop}
\begin{proof} \quad


    \begin{enumerate} [label=(\alph*), leftmargin=2\parindent]
        \item
        \(y = 0 + y = (-x + x) + y = -x + (x + y) = -x + (x + z) = (-x + x) + z = 0 + z = z\).
        \item
        \(x + y = x = x + 0\)이므로, (a)에 의해 \(y=0\).
        \item
        \(x + y = 0 = x + (-x)\)이므로, (a)에 의해 \(y = -x\).
        \item
        \(-x + x = x + (-x) = 0\)이므로, (c)에 의해 \(x = -(-x)\).
    \end{enumerate}
\end{proof}

\begin{prop} \label{prop 1.1.4}
    체 \(F\)와 \(x, y, z \in F\)에 대하여 다음이 성립한다.
    \begin{enumerate} [label=(\alph*), leftmargin=2\parindent]
        \item
        \(x \ne 0\)이고 \(xy = xz\)이면 \(y = z\).
        \item
        \(y \in F\)가 [모든 \(x \in F\)에 대해 \(xy = yx = x\)]를 만족하면 \(y = 1\).
        \item
        \(x \ne 0\)이고 \(xy = yx = 1\)이면 \(y = x^{-1}\).
        \item
        \(x \ne 0\)에 대하여 \((x^{-1})^{-1} = x\).
    \end{enumerate}
\end{prop}
\begin{proof}
    명제 \ref{prop 1.1.3}의 증명과 거의 같다.
\end{proof}

명제 \ref{prop 1.1.3}\과 \ref{prop 1.1.4}에 의해, 0과 1을 각각 \textbf{덧셈의 항등원(additive identity), 곱셈의 항등원(multiplicative identity)}이라고 부를 수 있다. 또 \(x \in F\)에 대해 \(-x\)를 \textbf{덧셈의 역원(additive inverse)}, \(x \ne 0\)일 때 \(x^{-1}\)을 \textbf{곱셈의 역원(multiplicative inverse)}라고 부를 수 있다.

\begin{ex}\quad


    \begin{enumerate} [label=(\alph*)]
        \item \(\QQ\)에 일상적인 의미의 덧셈과 곱셈을 정의하면 체가 된다. 따라서 \(\QQ\)를 유리수체라고 부른다. 
        \item 집합 \(\{0, 1\}\)에 다음과 같이 덧셈과 곱셈을 정의하자.
            \begin{alignat*}{4}
                &0 + 0 = 0, \quad &&0 + 1 = 1, \quad &&1 + 0 = 1, \quad&&1 + 1 = 0\\
                &0 \cdot 0 = 0, \quad &&0 \cdot 1 = 0, \quad &&1 \cdot 0 = 0, \quad&&1 \cdot 1 = 1
            \end{alignat*}
            그러면 집합 \(\{0, 1\}\)이 체의 구조를 가지는 것은 (F1)-(F9)를 확인함으로써 알 수 있다.
        \item 체 \(F\)에 대하여 \(F\)를 계수로 하는 유리식들의 집합 \(F(x)\) 역시 일상적인 의미의 덧셈과 곱셈을 정의하면 체가 된다.
    \end{enumerate}
\end{ex}

\begin{prop}
    체 \(F\)와 \(x, y \in F\)에 대하여 다음이 성립한다.
    \begin{enumerate} [label=(\alph*), leftmargin=2\parindent]
        \item
        \(0x = 0\).
        \item
        \(xy = 0\)이면 \(x = 0\) or \(y = 0\).
        \item
        \((-1)x = -x\).
        \item
        \((-x)y = x(-y) = -(xy)\)이고 \((-x)(-y)=xy\).
    \end{enumerate}
\end{prop}
\begin{proof}
    \quad

    \begin{enumerate} [label=(\alph*), leftmargin=2\parindent]
        \item
        \(0x + 0 = 0x = (0+0)x = 0x + 0x\)이므로 명제 \ref{prop 1.1.3} (a)에 의해.
        \item
        \(xy = 0\)이면서 \(x \neq 0, y \neq 0\)이면 \(x^{-1}, y^{-1}\)이 존재하므로
    \begin{gather*}
        0 = 0(y^{-1}x^{-1}) = (xy)(y^{-1}x^{-1}) = xx^{-1} = 1
    \end{gather*}
    이므로 모순.
        \item
        \(x + (-x) = 0 = 0x = (1 + (-1))x = 1x + (-1)x = x + (-1)x\)이므로 명제 \ref{prop 1.1.3} (a)에 의해.
        \item
        (c)에 의해 당연.
    \end{enumerate}
\end{proof}

\section{순서}

실수 집합에는 다들 알고 있듯이 덧셈과 곱셈 이외에 `순서'라는 구조가 주어져 있다.

    \begin{defn}
        집합 \(X\)에 주어진 관계(relation) \(\ge\)가 다음 조건을 만족하면 \(\ge\)를 \textbf{부분순서(partial order)}라고 한다.
        \begin{enumerate} [label=(O\arabic*), leftmargin=2\parindent]
            \item
            \(x \in X\)에 대하여 \(x \ge x\).
            \item
            \(x, y \in X\)에 대하여 \(x \ge y\)이고 \(y \ge x\)이면 \(x = y\).
            \item
            \(x, y, z \in X\)에 대하여 \(x \ge y\)이고 \(y \ge z\)이면 \(x \ge y \ge z\). 
        \end{enumerate}
        부분순서 \(\ge\)가 다음 조건까지 만족하면 \(\ge\)를 \textbf{전순서(total order)}라고 한다.
        \begin{enumerate}[label=(O\arabic*), leftmargin=2\parindent]
            \setcounter{enumi}{3}
            \item
            \(x, y \in X\)에 대하여 \(x \ge y\) or \(y \ge x\).
        \end{enumerate}
        부분순서와 전순서가 주어진 집합을 각각 \textbf{부분순서집합(paritally ordered set), 전순서집합(totally ordered set)}이라고 한다.
    \end{defn}

    \begin{ex}
        \(\NN, \ZZ, \QQ\)에 정의된 일상적 의미의 \(\ge\)가 전순서임은 쉽게 확인할 수 있다.
    \end{ex}
    \begin{defn}
        집합 \(X\)에 부분순서 \(\ge\)가 주어졌을 때, \(\le, >, <\)을 다음과 같이 정의한다.
        \[
        x \le y \iff y \ge x, \quad x > y \iff x \ge y \land x \ne y, \quad x < y \iff y > x
        \]
    \end{defn}

    \begin{defn}
        전순서 \(\ge\)가 주어진 체 \(F\)가 다음 조건을 만족할 때 \((F, \ge)\) 또는 간단히 \(F\)를 \textbf{순서체(ordered field)}라고 한다.
        \begin{enumerate}[label=(\alph*), leftmargin=2\parindent]
            \item \(x, y, z \in F\)에 대하여 \(x > y\)이면 \(x + z > y + z\).
            \item \(x, y \in F\)에 대하여 \(x, y > 0\)이면 \(xy > 0\).
        \end{enumerate}
        이때 \(x > 0\)이면 \(x\)를 \textbf{양수(positive number)}라고 부르고, \(x < 0\)이면 \(x\)를 \textbf{음수(negative number)}라고 부른다.
    \end{defn}
    \begin{ex}
        \(\QQ\)는 순서체이다.
    \end{ex}

    \begin{prop} \label{prop 1.2.4}
        순서체 \(F\)와 \(x, y, z \in F\)에 대하여 다음이 성립한다.
        \begin{enumerate}[label=(\alph*), leftmargin=2\parindent]
            \item \(x > 0\)이면 \(-x < 0\)이다.
            \item \(x > y \iff x - y > 0\).
            \item \(x > 0\)이고 \(y > z\)이면 \(xy > xz\)이다.
            \item \(x < 0\)이고 \(y < z\)이면 \(xy < xz\)이다.
            \item \(x \ne 0\)이면 \(x^2 > 0\)이다. 특히 \(1 > 0\)이다.
            \item \(0 < x < y\)이면 \(0 < y^{-1} < x^{-1}\)이다.
            \item \(x, y > 0\)이면 \(x > y \iff x^2 > y^2\).
        \end{enumerate}
    \end{prop}
    \begin{proof}
        \quad

        \begin{enumerate}[label=(\alph*), leftmargin=2\parindent]
            \item \(-x = -x + 0 < -x + x = 0\).
            \item 양변에 \(y\) 또는 \(-y\)를 더해서 얻을 수 있다.
            \item \(xy - xz = x(y-z) > 0\).
            \item \(xz - xy = (-x)(z - y) > 0\).
            \item \(x > 0\)이면 순서체의 정의에 의해 \(x^2 > 0\)이다. \(x < 0\)이면 \(-x > 0\)이므로 \(x^2 = (-x)^2 > 0\). 이때 \(1 = 1^2\)이므로 \(1 > 0\)이다.
            \item \(x^{-1} \le 0\)이면 \(1 = xx^{-1} \le 0\)에서 모순이므로 \(x^{-1} > 0\)이고, 마찬가지로 \(y^{-1} > 0\)이다. 따라서 \(x^{-1}y^{-1} > 0\)이므로 \(x < y\)의 양변에 \(x^{-1}y^{-1}\)을 곱하면 \(y^{-1} < x^{-1}\)을 얻는다.
        \end{enumerate}
    \end{proof}

    \begin{prop}
        순서체 \(F\)는 자연수 집합과 동형인 집합을 포함한다. 따라서 \(F\)는 무한집합이다.
    \end{prop}
    \begin{proof}
        자연수 \(n \in \NN\)에 대하여
        \[
        n \cdot 1 := \underbrace{1 + \ldots + 1}_{n\text{-번}}
        \]
        로 정의하자. 이때 \(n \mapsto n \cdot 1\)이 덧셈과 곱셈을 보존하는 것은 쉽게 확인할 수 있다. 이제 \(n \mapsto n \cdot 1\)이 단사인 것만 보이면 된다. 자연수 \(n, m\)에 대하여 \(n \cdot 1 = m \cdot 1\)이라고 가정하자. 일반성을 잃지 않고 \(n \ge m\)이라고 하면, \((n - m) \cdot 1 = 0\)이다. 그런데 1이 양수이므로, 순서체의 정의에 의해
        \[
        (n - m) \cdot 1 = \underbrace{1 + \ldots + 1}_{(n-m)\text{-번}} = 0
        \]
        이려면 \(n - m = 0\)이어야 한다. 따라서 순서체 \(F\)는 자연수 집합과 동형인 집합 \(\{n \cdot 1 \in F: n \in \NN\}\)을 포함한다.
    \end{proof}

    \begin{cor} \label{cor 1.2.7}
        순서체 \(F\)는 유리수체와 동형인 부분체를 포함한다.
    \end{cor}
    \begin{proof}
        음의 정수 \(n \in \ZZ \)에 대하여 \(n \cdot 1 := -((-n) \cdot 1)\)로 정의하고, 유리수 \(r = m/n \in \QQ \ (m, n \in \ZZ, n \ne 0)\)에 대하여
        \[
        r \cdot 1 := (m \cdot 1)(n \cdot 1)^{-1}    
        \]
        로 정의하자(\(r \cdot 1\)의 정의가 \(m, n\)의 선택에 무관함은 쉽게 확인할 수 있다). 이때 \(r \mapsto r \cdot 1\)이 덧셈과 곱셈을 보존하는 단사함수이므로, \(F\)는 \(\QQ\)와 동형인 부분체 \(\{r \cdot 1: r \in \QQ\}\)를 포함한다.
    \end{proof}

    \begin{defn}
        전순서집합 \(X\)의 부분집합 \(E\)에 대하여 [모든 \(x \in E\)에 대해 \(x \le \alpha\)]인 \(\alpha \in X\)가 존재하면 \(E\)가 \textbf{위로 유계(bounded above)}라고 하고 \(\alpha\)를 \(E\)의 \textbf{상계(upper bound)}라고 한다. 마찬가지 방법으로 \textbf{아래로 유계(bounded below), 하계(lower bound)}도 정의한다. \(E \subseteq X\)가 위로 유계이면서 아래로 유계이면 \textbf{유계(bounded)}라고 한다.
    \end{defn}
    \begin{ex}
        \quad

        \begin{enumerate}[label=(\alph*), leftmargin=2\parindent]
            \item 전순서집합 \(X\)의 유한 부분집합은 항상 유계이다.
            \item 자연수 집합 \(\NN\)은 자기 자신 안에서 위로 유계가 아니다. \(n \in \NN\)이 \(\NN\)의 상계라고 하면, \(n + 1 \in \NN\)인데 \(n < n+1\)이기 때문이다.
        \end{enumerate}
    \end{ex}

    \begin{defn}
        전순서집합 \(X\)의 위로 유계인 부분집합 \(E\)에 대하여, 다음을 만족하는 \(\alpha \in X\)가 존재한다고 하자.
        \begin{enumerate}[label=(\alph*), leftmargin=2\parindent]
            \item \(\alpha\)는 \(E\)의 상계이다.
            \item \(\beta \in X\)가 \(E\)의 상계이면 \(\alpha \le \beta\)이다.
        \end{enumerate}
        이때 \(\alpha\)를 \(E\)의 \textbf{최소상계=상한(least upper bound; supremum)}이라고 하고 \(\alpha = \sup E\)로 쓴다. 마찬가지로 아래로 유계인 부분집합 \(E\)에 대해 \(E\)의 \textbf{최대하계=하한(greatest lower bound; infimum)}을 정의할 수 있고(만약 존재한다면), 이를 \(\inf E\)로 쓴다.
    \end{defn}

    \begin{defn} \label{def 1.2.7}
        전순서집합 \(X\)가 \textbf{완비성(completeness)} 또는 \textbf{최소상계 성질(least upper bound property)}를 가진다는 것은 다음의 두 동치 조건을 만족한다는 것이다.
        \begin{enumerate} [label=(\alph*), leftmargin=2\parindent]
            \item 공집합이 아니고 위로 유계인 부분집합 \(E \subseteq X\)는 \(X\)에서 상한을 가진다.
            \item 공집합이 아니고 아래로 유계인 부분집합 \(E \subseteq X\)는 \(X\)에서 하한을 가진다.
        \end{enumerate}
        특히 순서체 \(F\)가 완비성을 가지면 \(F\)를 \textbf{완비순서체(complete ordered field)}라고 한다.
    \end{defn}

    \begin{prop}
        정의 \ref{def 1.2.7}의 (a)와 (b)는 동치이다.
    \end{prop}
    \begin{proof}
        두 방향이 대칭적이므로 (a)\(\Leftarrow\)(b)만 보이면 충분하다. \(E \subseteq X\)가 공집합이 아니고 아래로 유계인 부분집합이라고 하고, \(B = \{x \in X: x \ \text{is a lower bound of} \ E.\}\)로 정의하자. \(B\)가 아래로 유계이므로 \(B\)는 공집합이 아니다. 한편 \(E\)가 공집합이 아니므로 \(E\)의 원소를 하나 골랐을 때 이는 \(B\)의 상계가 된다. 따라서 \(B\)는 공집합이 아니고 위로 유계인, \(X\)의 부분집합이다. 가정에 의해 \(\alpha = \sup B\)가 존재한다. 우리의 주장은 \(\alpha = \inf B\)라는 것이다. 먼저 임의의 \(x \in E, y \in B\)에 대해 \(y \le x\)이므로 \(x\)는 \(B\)의 상계이다. 상한의 정의에 의해 \(\alpha \le x\)이고, \(x\)가 \(E\)의 임의의 원소였으므로 \(\alpha\)는 \(E\)의 하계이다. 한편 \(\beta \in X\)가 \(E\)의 하계라고 하면 \(B\)의 정의에 의해 \(\beta \in B\)이고, 상한의 정의에 의해 \(\beta \le \alpha\)이다. 따라서 \(\alpha = \inf E \in X\)가 증명되었다.
    \end{proof}

\section{완비순서체}

이제 우리의 질문은 완비순서체가 정말로 존재하는지이다. 먼저 우리가 잘 아는 \(\QQ\)는 완비순서체가 아니다. \(\QQ\)가 완비순서체라고 가정하고 집합 \(E = \{x \in \QQ : x \ge 0, x^2 < 2\}\)를 생각하자. \(0 \in E\)이므로 \(E\)는 공집합이 아니고, \(2\)가 \(E\)의 상계이므로 \(E\)는 위로 유계이다. 따라서 가정에 의해 \(\alpha = \sup E \in \QQ\)가 존재한다. 그런데 \(\alpha^2 = 2\)인 유리수 \(\alpha\)가 존재하지 않는 것은 중학교 때 증명했으므로, \(\alpha^2 = 2\)인 것만 보이면 증명이 끝난다.\\
먼저 다음과 같이 정의된 유리수 \(\beta\)를 생각하자.
\[
\beta = \alpha - \frac{\alpha^2 - 2}{\alpha + 2} = \frac{2\alpha+2}{\alpha+2}    
\]
이때 다음을 관찰할 수 있다.
\[
\beta^2 - 2 = \frac{2(\alpha^2 - 2)}{(\alpha + 2)^2}
\]
이제 \(\alpha^2 < 2\)라고 가정하자. 그러면 \(\beta\)의 정의에 의해 \(\beta > \alpha\)인데 \(\beta^2 < 2\)이므로 \(\beta \in E\)이다. 이는 \(\alpha\)가 \(E\)의 상계라는 것에 모순이다. 또 \(\alpha^2 > 2\)라고 가정하면, \(\beta < \alpha\)인데 \(\beta^2 > 2\)이므로 임의의 \(x \in E\)에 대하여 \(\beta \ge x\)이다. 즉 \(\beta\)는 \(E\)의 상계이며, 이는 \(\alpha = \sup E\)인 것에 모순이다. 따라서 \(\alpha^2 = 2\)일 수밖에 없으며, 이러한 유리수 \(\alpha\)는 존재하지 않음을 알고 있다. 따라서 \(\QQ\)는 완비순서체가 아니다.

다음은 이 노트에서 \textbf{처음 등장하는 정리}이다.

    \begin{thm}
        \(\QQ\)를 포함하는 완비순서체가 존재한다.
    \end{thm}
    \begin{proof}
        사실 자체에 비해 증명은 이후 story에서 크게 중요하지 않은 것 같다. 관심 있는 독자는 이 장의 Appendix를 참고하라.
    \end{proof}

완비성을 이용하여 다음 당연해 보이는 사실을 증명할 수 있다.

    \begin{thm}
        완비순서체 \(F\)에서 \(\NN\)은 위로 유계가 아니다.
    \end{thm}
    \begin{proof}
        \(\NN\)이 \(F\)에서 위로 유계라고 가정하면 완비성에 의해 \(\alpha = \sup \NN\)이 존재한다. \(\alpha - 1 < \alpha\)이므로 \(\alpha - 1\)은 \(\NN\)의 상계가 아니고, 따라서 \(\alpha - 1 < n \le \alpha\)를 만족하는 \(n \in \NN\)이 존재한다. 그런데 \(n + 1 \in \NN\)이고 \(\alpha < n + 1\)이므로 \(\alpha = \sup \NN\)에 모순이다. 따라서 \(\NN\)은 \(F\)에서 위로 유계가 아니다.
    \end{proof}
    
    \begin{cor}
        완비순서체 \(F\)의 임의의 양수 \(a > 0\)에 대하여 \(0 < n^{-1} < a\)인 \(n \in \NN\)이 존재한다.
    \end{cor}
    \begin{proof}
        \(a^{-1}\)이 \(\NN\)이 상계가 아니므로, \(a^{-1} < n\)인 \(n \in \NN\)이 존재한다. 양변에 역수를 취하면 원하는 결론을 얻는다.
    \end{proof}
    \begin{cor} \label{cor 1.3.5}
        완비순서체 \(F\)의 임의의 양수 \(a > 0\)와 \(b \in F\)에 대하여 \(na > b\)인 \(n \in \NN\)이 존재한다.
    \end{cor}
    \begin{proof}
        \(a^{-1} b\)가 \(\NN\)의 상계가 아니므로, \(a^{-1}b < n\)인 \(n \in \NN\)이 존재한다. 양변에 \(a\)를 곱하면 원하는 결론을 얻는다.
    \end{proof}

따름정리 \ref{cor 1.3.5}\를 \textbf{아르키메데스 성질(Archimedean property)}이라고 하고, 이러한 성질을 가지는 순서체를 \textbf{아르키메데스체(Archimedean field)}라고 한다. 도대체 이런 당연한 성질에 이름이 왜 붙었는지 궁금하겠지만\(\ldots\)

    \begin{ex}
        실수계수 다항식들의 비로 표시되는 유리식들의 집합 \(\RR(x)\)에 자연스러운 덧셈과 곱셈을 주면 \(\RR(x)\)는 체의 구조를 가진다. \(f \in \RR(x)\)에 대하여, \(f = p / q\) (\(p, q\)는 다항식)로 썼을 때 \(p, q\)의 최고차항의 비가 양수이면 \(f > 0\) 으로 정의하자. 이때 \(f \ge g \iff f - g \ge 0\)으로 정의하면 이는 \(\RR(x)\)에 전순서 구조를 주고, 이에 의해 \(\RR(x)\)가 순서체가 됨은 쉽게 확인할 수 있다. 그런데 \(f(x) = x \in \RR(x)\)와 임의의 자연수 \(n \in \RR(x)\)에 대하여 \(f > n\)이므로, \(\NN\)은 \(\RR(x)\)에서 위로 유계이다. 즉 \(\RR(x)\)는 비-아르키메데스체이다. 따라서 \(\RR(x)\)는 최소상계 성질을 가지지 않는다.
    \end{ex}

다음 정리는 완비순서체의 유일성을 말해 준다.

    \begin{thm}
        완비순서체는 유일하다. 즉, 두 완비순서체 \(F, G\)가 있으면 순서체 동형사상 \(f : F \to G\)가 존재하여 다음을 만족한다.
        \begin{enumerate} [label=(\alph*), leftmargin=2\parindent]
            \item 모든 \(x, y \in F\)에 대하여 \(f(x+y) = f(x) + f(y), f(xy) = f(x)f(y)\).
            \item 모든 \(x, y \in F\)에 대하여 \(x > y\)이면 \(f(x) > f(y)\).
        \end{enumerate}
    \end{thm}
    \begin{proof}[Sketch of Proof]
        \(F, G\)의 곱셈에 대한 항등원을 각각 \(1_F, 1_G\)로 쓰자. 따름정리 \ref{cor 1.2.7}의 표기법을 사용한다. 각 \(x \in F\)에 대해, \(G\)의 부분집합
        \[
            G_x = \{r \cdot 1_G \in G: r \in \QQ, \ r \cdot 1_F < x\}
        \]
        를 생각하자. 아르키메데스 성질에 의해 \(n \cdot 1_F > -x\)인 \(n \in \NN\)이 존재하고, 이때 \((-n) \cdot 1_G \in G_x\)이므로 \(G_x\)는 공집합이 아니다. 한편 아르키메데스 성질에 의해 \(m \cdot 1_F > x\)인 \(m \in \NN\)이 존재하여 \(m \cdot 1_G\)가 \(G_x\)의 상계이므로 \(G_x\)는 공집합이 아니고 위로 유계인 부분집합이다. 따라서 \(\sup G_x \in F\)가 잘 정의된다. 이제 함수 \(f : F \to G : x \mapsto \sup G_x\)를 정의하자. 우리의 주장은 \(f\)가 ordered field isomorphism이라는 것이다. 먼저 \(f\)가 bijection임을 보이기 위해 \(f\)의 역함수를 찾는다. 각 \(y \in G\)에 대해 \(F\)의 부분집합
        \[
        F_y = \{r \cdot 1_F \in F: r \in \QQ, \ r \cdot 1_G < y\}
        \]
        를 생각하면 \(F_y\)는 공집합이 아니고 위로 유계인 부분집합이므로 \(\sup G_y\)가 잘 정의된다. 이때 함수 \(g : G \to F : y \mapsto \sup G_y\)를 정의하자. 이때 \(f\)와 \(g\)가 역함수 관계임을 증명하고, \(f\)가 덧셈과 곱셈을 보존하는 것을 증명하면 된다.
    \end{proof}

완비순서체의 존재성과 유일성에 의해 다음과 같이 실수체를 정의할 수 있다.

    \begin{defn}
        유일하게 존재하는 완비순서체를 \textbf{실수체(real field)}라고 부른다.
    \end{defn}

이제 드디어 제곱근과 지수를 제대로 정의할 수 있게 되었다. 중학교에서 \(\sqrt{2}\)가 유리수가 아님은 증명하였지만, \(\sqrt{2}\)라는 것이 진짜 있는지, 즉 제곱해서 2가 되는 양수가 진짜 있는지는 증명하지 않았다. 마찬가지로 실수 지수에 대해서도 제대로 정의하지 않고 넘어갔다. 이제 우리는 실수체를 완벽히 알고 있으므로 다음 사실들을 막힘 없이 증명할 수 있다.

    \begin{lem}
        \(a, b \in \RR\)에 대하여 \(0 < a < b\)이면 \(b^n - a^n < (b - a)nb^{n-1}\)이다.
    \end{lem}
    \begin{proof}
        \(b^n - a^n = (b - a)(b^{n-1} + ab^{n-2} + \ldots + a^{n-1}) < (b-a)nb^{n-1}\).
    \end{proof}

    \begin{thm}
        양수\footnote{앞으로 양수는 양의 실수를 의미하는 것으로 약속한다.} \(x\)와 양의 정수 \(n \in \ZZ_+\)에 대하여, \(y^n = x\)인 양수 \(y\)가 유일하게 존재한다. 이러한 \(y\)를 \(\sqrt[n]{x}\) 또는 \(x^{1/n}\)으로 쓴다.
    \end{thm}
    \begin{proof}
        \(n \in \ZZ_+\)에 대하여 \(0 < y_1 < y_2\)이면 \(0 < y_1^n < y_2^n\)이므로, 원하는 조건을 만족하는 양수 \(y\)가 최대 1개 존재하는 것은 명백하다.\\
        이제 집합 \(E := \{t \in \RR_+: t^n \le x\} \)를 생각하자. 이때
        \[
        \paren{\frac{x}{1+x} }^n \le \frac{x}{1+x} \le x
        \]
        이므로 \(E\)는 공집합이 아니고,
        \[
        (1+x)^n \ge 1+x \ge x
        \]
        이므로 \(E\)는 \(1+x\)를 상계로 가진다. 따라서 \(E\)는 실수 안에서 상한을 가지며, 이 값을 \(y\)라 하자. \(y\)가 양수인 것은 당연하다. 이때 우리의 주장은 \(y^n = x\)라는 것이다.\\
        먼저 \(y^n < x\)라 가정하자. 이때 양수 \(h\)를 다음과 같이 정의하자.
        \[
            h = \min\left\{\frac{x - y^n}{n(y+1)^{n-1}}, 1\right\}
        \]
        그러면 바로 위 보조정리에 의해
        \[
        (y+h)^n - y^n < hn(y+h)^{n-1} \le hn(y+1)^{n-1} \le x - y^n
        \]
        이므로 \((y+h)^n < x\), 즉 \(y + h \in E\)이다. 이는 \(y\)가 \(E\)의 상계임에 모순이다.\\
        다음으로 \(y^n > x\)라 가정하자. 이때 양수 \(k\)를 다음과 같이 정의하자.
        \[
        k = \min\left\{\frac{y^n - x}{ny^{n-1}}, \frac{y}{2}\right\} 
        \]
        그러면 \(0 < k < y\)이므로 \(y - k > 0\)이다. 따라서
        \[
        y^n - (y - k)^n < kny^{n-1} \le y^n - x    
        \]
        이므로 \((y - k)^n > x\)이다. 이는 \(y - k\)가 \(E\)의 상계임을 의미하므로, \(y = \sup E\)에 모순이다. 따라서 \(y^n = x\)가 성립한다.
    \end{proof}

다음은 지수의 정의이다. 정수 지수에 관해서는 (사실 앞에서 계속 사용하였는데) 정의 및 지수법칙을 이미 알고 있다고 가정하겠다. 고등학교에서 배운 증명과 다른 점이 없기 때문이다. 

    \begin{defn} \label{def 1.3.10}
        양수 \(a > 0\)과 유리수 \(r = m/n \in \QQ \ (m, n \in \ZZ, n \ne 0)\)에 대하여 \(a^r := (a^m)^{1/n}\)으로 정의한다.
    \end{defn}

이런 종류의 정의를 보았을 때는 항상 잘 정의됨(well-definiteness)을 고려하여야 한다.

    \begin{prop}
        양수 \(a > 0\)과 정수 \(m, n, p, q \in \ZZ, \ n, q \ne 0\)에 대하여 \(m/n = p/q\)이면 \((a^m)^{1/n} = (a^p)^{1/q}\)이다. 따라서 정의 \ref{def 1.3.10}에 의해 유리수 지수는 잘 정의된다.
    \end{prop}
    \begin{proof}
        먼저 \(x = (a^m)^{1/n}\)이라 두면,
        \[
        x^{nq} = (x^n)^q = (a^m)^q = a^{mq}   
        \]
        이므로 \((a^m)^{1/n} = (a^{mq})^{1/nq}\)이다. 마찬가지 방법으로, \((a^p)^{1/q} = (a^{np})^{1/nq}\)이다. 그런데 가정에 의해 \(mq = np\)이므로
        \[
            (a^m)^{1/n} = (a^{mq})^{1/nq} = (a^{np})^{1/nq} = (a^p)^{1/q}
        \]
        이다.
    \end{proof}

다음이 성립하지 않으면 정의 \ref{def 1.3.10}\이 제대로 된 정의라고 할 수 없을 것이다.

    \begin{prop}
        유리수 지수에 대해 지수법칙이 성립한다. 즉, 양수 \(a, b > 0\)과 유리수 \(r, s \in \QQ\)에 대하여 다음이 성립한다.
        \begin{enumerate}[label=(\alph*), leftmargin=2\parindent]
            \item \(a^r a^s = a^{r+s}\).
            \item \((a^r)^s = a^{rs}\).
            \item \(a^r b^r = (ab)^r\).
            \item \(r > 0\)이면 \(a < b \iff a^r < b^r\).
            \item \(a > 1\)이면 \(r < s \iff a^r < b^r\).
        \end{enumerate}
    \end{prop}
    \begin{proof}
        연습문제로 남긴다.
    \end{proof}

드디어 실수 지수를 정의한다.

    \begin{defn} \label{def 1.3.13}
        양수 \(a > 1\)과 실수 \(x \in \RR\)에 대하여
        \[
            a^x := \sup \{a^r \in \RR : r \in \QQ, r \le x\}
        \]
        로 정의한다. \(a = 1\)일 때는 \(a^x := 1\), \(a < 1\)일 때는 \(a^x := ((a^{-1})^x)^{-1}\)로 정의한다.
    \end{defn}

    \begin{prop}
        양수 \(a > 1\)과 실수 \(x \in \RR\)에 대하여 집합 \(E := \{a^r \in \RR : r \in \QQ, r \le x\}\)는 상한을 가진다. 따라서 정의 \ref{def 1.3.13}에 의해 실수 지수는 잘 정의된다.
    \end{prop}
    \begin{proof}
        \(h = a - 1 > 0\)와 양의 정수 \(n \in \ZZ_+\)에 대하여,
        \[
        a^n = (1+h)^n \ge 1 + nh    
        \]
        인 것을 이용한다. 아르키메데스 원리에 의해
        \[
        1 + n_1 h > 1/x    
        \]
        인 \(n_1 \in \ZZ_+\)이 존재하는데, 이때 \(a^{-n_1} \le (1 + n_1 h)^{-1} < x\)이므로 \(E\)는 공집합이 아니다. 한편 아르키메데스 원리에 의해
        \[
        1 + n_2 h > x    
        \]
        인 \(n_2 \in \ZZ_+\)이 존재하는데, 이때 \(a^{n_2}\)는 \(E\)의 상계가 된다. 따라서 \(E\)는 상한을 가진다.
    \end{proof}

마찬가지로 다음이 성립하지 않으면 정의 \ref{def 1.3.13}\이 제대로 된 정의라고 할 수 없을 것이다.

    \begin{prop}
        실수 지수에 대해 지수법칙이 성립한다. 즉, 양수 \(a, b > 0\)과 유리수 \(x, y \in \RR\)에 대하여 다음이 성립한다.
        \begin{enumerate}[label=(\alph*), leftmargin=2\parindent]
            \item \(a^x a^y = a^{x+y}\).
            \item \((a^x)^y = a^{xy}\).
            \item \(a^x b^x = (ab)^x\).
            \item \(x > 0\)이면 \(a < b \iff a^x < b^x\).
            \item \(a > 1\)이면 \(x < y \iff a^x < b^x\).
        \end{enumerate}
    \end{prop}
    \begin{proof}
        연습문제로 남긴다.
    \end{proof}


\section{복소수체}

    \begin{prop} \label{prop 1.4.1}
        \(\RR^2\)에 다음과 같이 이항연산 \(+, \cdot\)을 정의하면
        \[
        (a, b) + (c, d) = (a+c, b+d), \quad (a, b)\cdot(c, d) = (ac - bd, ad+bc)    
        \]
        \(\RR^2\)는 체의 구조를 가진다.
    \end{prop}
    \begin{proof}
        고등학교 때에 증명하였다.
    \end{proof}

    \begin{defn}
        명제 \ref{prop 1.4.1}\과 같이 이항연산이 주어진 \(\RR^2\)를 \textbf{복소수체(complex field)}라고 부르며, \((a, b)\)를 \(a + ib\)로 쓴다.
    \end{defn}

복소수의 연산과 관련된 성질은 고등학교 때 배운 것과 완전히 같으므로 생략한다.

    \begin{prop}
        복소수체는 순서체가 아니다.
    \end{prop}
    \begin{proof}
        복소수체를 순서체가 되게 하는 전순서 \(\ge\)가 존재한다고 하자. 이때 \(i^2 = -1 < 0\)이므로 모순이다.
    \end{proof}

\section{Appendix: 완비순서체의 존재성}

완비순서체를 건설하는 방법에는 대표적으로 Cantor의 방법과 Dedekind의 방법이 있는데, 여기서는 후자를 사용하였다. 본 Appendix에서는 완비성을 가지는 어떤 전순서집합을 건설하고, 여기에 덧셈과 곱셈을 정의할 것이다. 이 이항연산이 체의 공리 (F1)-(F9)를 만족하는 것에 대한 증명은 필자의 귀찮음으로 인하여 생략할 것이다. 자세한 증명은[\ref{ref2}]를 참고하라.\\
\textbf{Step 1.} 집합 \(F\)를 다음 조건을 만족하는 모든 \(\alpha \subseteq \QQ\)들의 집합으로 정의한다. 이러한 집합 \(\alpha\)들을 절단(cut)이라고 한다.
    \begin{enumerate} [label=(\alph*), leftmargin=2\parindent]
        \item \(\alpha \ne \varnothing, \alpha \ne \QQ\)
        \item \(p, q \in \QQ\)에 대하여 \(p \in \alpha\)이고 \(q < p\)이면 \(q \in \alpha\).
        \item \(p\in \QQ\)에 대하여 \(p \in \alpha\)이면 \(p < r\)인 \(r \in \alpha\)가 존재.
    \end{enumerate}
즉 \(\{p \in \QQ: p < 0\}\)이나 \(\{p \in \QQ : p < 0 \lor p^2 < 2\}\)와 같은 집합들이 \(F\)의 원소이다.\\
\textbf{Step 2.} \(\alpha, \beta \in F\)에 대하여 \(\alpha \ge \beta \iff \alpha \supseteq \beta\)로 정의하면, 이는 \(F\)에 전순서를 준다. 부분순서인 것은 자명하고, \(\alpha, \beta \in F\)에 대하여 \(\alpha \le \beta\) or \(\beta \le \alpha\)인 것만 증명하면 된다. \(\alpha \le \beta\)가 아니라고 가정하자. 그러면 \(p \in \alpha \setminus \beta\)가 존재한다. 임의의 \(q \in \beta\)가 \(\alpha\)의 원소임을 보이면 되는데, 이를 위해서 \(q < p\)인 것만 보이면 된다. 가정에 의해 \(q = p\)일 수는 없다. 만약 \(q > p\)이면 \(p \in \beta\)이므로 모순이다. 따라서 \(q < p\)이므로 \(q \in \alpha\)이다.\\
\textbf{Step 3.} \(F\)가 완비 전순서집합임을 보인다. 먼저 \(E \subseteq F\)가 공집합이 아니고 위로 유계인 부분집합이라고 하자. 이때 집합 \(\alpha\)를
\[
\alpha = \bigcup_{\beta \in E} \beta \subseteq \QQ
\]
로 정의하자. \(\alpha \in F\)인 것은 절단의 정의로부터 쉽게 확인할 수 있다. 우리의 주장은 \(\alpha = \sup E\)라는 것이다. 먼저 \(\alpha\)가 \(E\)의 상계인 것은 당연하다. 어떤 \(\alpha' \in F\)가 \(E\)의 상계라고 하면, 임의의 \(\beta \in E\)에 대해 \(\beta \le \alpha'\), 즉 \(\beta \subseteq \alpha'\)이므로 \(\alpha = \bigcup_{\beta \in E}\beta \subseteq \alpha'\)이다. 따라서 \(\alpha = \sup E\)이고, \(F\)는 완비성을 가진다.\\
\textbf{Step 4.} \(F\)에서 덧셈을 다음과 같이 정의한다.
\[
\alpha + \beta := \{p + q \in \QQ : p \in \alpha, q \in \beta\}    
\]
\(\alpha + \beta \in F\)인 것은 절단의 정의로부터 쉽게 확인할 수 있다. 이때 덧셈의 항등원은
\[
0 = \{p \in \QQ : p < 0\}
\]
이고, \(\alpha \in F\)의 덧셈에 대한 역원은
\[
-\alpha = \{p \in \QQ : \exists r > 0 \ \text{s.t.} \ -p -r \notin \alpha\}    
\]
이다. 다음으로 \(F\)에서의 곱셈은 먼저 0보다 큰 원소들에 대해서만 정의한다. \(\alpha, \beta > 0\)에 대하여
\[
\alpha\beta := \{p \in \QQ : \exists r \in \alpha, s \in \beta \ \text{s.t.} \ r, s > 0, p \le rs\}
\]
로 정의한다. \(\alpha\beta \in F\)인 것은 절단의 정의로부터 쉽게 확인할 수 있다. 이제 임의의 \(\alpha \in F\)에 대하여
\[
\alpha0 := 0, \quad 0\alpha := 0
\]
와
\[
\alpha\beta := 
    \begin{cases}
        (-\alpha)(-\beta) &\textif \alpha < 0, \beta < 0\\
        -((-\alpha)\beta) &\textif \alpha < 0, \beta > 0\\
        -(\alpha(-\beta)) &\textif \alpha > 0, \beta < 0
    \end{cases}
\]
로 정의하면, \(F\)에서의 곱셈이 잘 정의된다. 이때 곱셈의 항등원은
\[
1 = \{p \in \QQ : p < 1\}    
\]
이고, \(\alpha \in F, \alpha > 0\)의 곱셈에 대한 역원은
\[
\alpha^{-1} = \QQ_{\le 0} \cup \{p \in \QQ_+ : \exists r > 1 \ \text{s.t.} \ p^{-1}r^{-1} \notin \alpha\},
\]
\(\alpha \in F, \alpha < 0\)의 곱셈에 대한 역원은
\[
\alpha^{-1} = (-\alpha)^{-1}
\]
이다. 이제 위에서 정의한 덧셈과 곱셈이 (F1)-(F9)를 만족하는지 확인하면 된다.\\
\textbf{Step 5.} 이제 \(F\)가 순서체인 것만 확인하면 되는데, 이는 덧셈과 곱셈의 정의에 의해 거의 당연하다. 따라서 \(F\)는 완비순서체이며, 유리수 \(r\)에 대해
\[
r \cdot 1 = \{p \in \QQ : p < r\}
\]
이 성립한다.




\chapter{수열}

수열은 정의역이 \(\NN\)인 함수인데, 해석개론에서 매우 중요한 위치에 있다. 해석개론 수준에서 가장 일반적인 setting은 거리공간의 수열이다. 일반적인 거리공간에서 수열의 수렴을 논하고, 실수열의 몇 가지 성질을 알아보겠다.

\section{거리공간}

    \begin{defn}
        집합 \(X\)에 대하여 다음 조건 (M1)-(M3)을 만족하는 함수 \(d : X \times X \to \RR\)가 주어졌을 때 \((X, d)\) 또는 간단히 \(X\)를 \textbf{거리공간(metric space)}이라고 하고, \(d\)를 \textbf{\(X\)의 거리(metric on \(X\))}라고 한다.
        \begin{enumerate} [label=(M\arabic*), leftmargin=2\parindent]
            \item \(x, y \in X\)에 대하여 \(d(x, y) = d(y, x)\)
            \item \(x, y, z \in X\)에 대하여 \(d(x, z) \le d(x, y) + d(y, z)\).
            \item \(x, y \in X\)에 대하여 \(d(x, y) = 0 \iff x = y\).
        \end{enumerate}
    \end{defn}
    \begin{ex}
        \quad

        \begin{enumerate} [label=(\alph*), leftmargin=2\parindent]
			\item
			\(\RR\) 또는 \(\CC\)에서 \(d(x, y) = \abs{x - y}\)로 정의하면 \((\RR, d), (\CC, d)\)는 거리공간이다.
            \item \(\RR^d\)에서
            \[
            d((x_1, \ldots, x_d), (y_1, \ldots, y_d)) = \sqbracket{\sum_{i=1}^d (x_i - y_i)^2}^{1/2}    
            \]
            로 정의하면 \((\RR^d, d)\)는 거리공간이다. 증명은 \ref{sec 3}장으로 미룬다.
			\item
			아무 집합 \(X\)에서 \(x, y \in X\)에 대해 \(d(x, y)\)를 \(x = y\)일 때 0, \(x \neq y\)일 때 1로 정의하면 \((X, d)\)는 거리공간이다. 이러한 거리공간을 \textbf{이산 거리공간(discrete metric space)}이라고 한다.
		\end{enumerate}
    \end{ex}

    \begin{prop}
        거리공간 \(X\)와 \(x, y, z \in X\)에 대해 다음이 성립한다.
        \begin{enumerate} [label=(\alph*), leftmargin=2\parindent]
            \item \(d(x, y) \ge 0\).
            \item \(\abs{d(x, y) - d(y, z)} \le d(x, z) \le d(x, y) + d(y, z)\). 
        \end{enumerate}
    \end{prop}
    \begin{proof}
        \quad

        \begin{enumerate} [label=(\alph*), leftmargin=2\parindent]
            \item \(0 = d(x, x) \le d(x, y) + d(y, x) = 2d(x, y)\).
            \item \(d(x, y) - d(y, z) \le d(x, z)\)이고 \(d(y, z) - d(x, y) \le d(x, z)\)이므로.
        \end{enumerate}
    \end{proof}



\section{수열의 수렴}

    \begin{defn}
        집합 \(X\)에 대하여, 함수 \(x : \NN \to X\)(또는 \(x : \ZZ \to X\)나 \(x : \ZZ_+ \to X\))를 \(X\) 안의 \textbf{수열(sequence)}이라고 하고 \((x_n)_{n \in \NN} = x, \ x_n = x(n)\)으로 표기한다.
    \end{defn}
    \begin{defn}
        거리공간 \(X\)의 수열 \((x_n)_{n \in \NN}\)이 \(x \in X\)로 \textbf{수렴(converge)}한다는 것은 임의의 양수 \(\eps > 0\)에 대하여 다음을 만족시키는 자연수 \(N\)이 존재한다는 것이다.
        \[
        n \ge N \implies d(x_n, x) < \eps    
        \]
        이때
        \[
        \lim_{n \to \infty} x_n = x, \quad x_n \to x    
        \]
        와 같이 쓰고, \(x\)를 \((x_n)_{n \in \NN}\)의 \textbf{극한(limit)}이라고 한다. 어떤 \(x\)로도 수렴하지 않는 수열을 \textbf{발산(diverge)}한다고 한다.
    \end{defn}


    \begin{thm}
        거리공간 \(X\)의 수열 \((x_n)_{n \in \NN}\)이 수렴하면 그 극한은 유일하다.
    \end{thm}
    \begin{proof}
        서로 다른 \(x, y \in X\)에 대하여 수열 \((x_n)_{n \in \NN}\)이 \(x, y\)로 각각 수렴한다고 가정하자. \(r = d(x, y) > 0\)으로 두면, 정의에 의해 다음을 만족하는 자연수 \(N_1, N_2\)가 존재한다.
        \[
        n \ge N_1 \implies d(x_n, x) < \frac{r}{2}, \quad n \ge N_2 \implies d(x_n, y) < \frac{r}{2}
        \]
        \(N = \max \{N_1, N_2\}\)로 두면,
        \[
        d(x, y) \le d(x, x_N) + d(x_N, y) < r    
        \]
        이므로 모순이다. 따라서 거리공간에서 수렴하는 수열 \((x_n)_{n \in \NN}\)의 극한은 유일하다.
    \end{proof}

    \begin{defn}
        거리공간 \(X\)의 부분집합 \(E \subseteq X\)가 \textbf{유계(bounded)}라는 것은, 어떤 양수 \(M > 0\)이 존재하여 [모든 \(x, y \in E\)에 대해 \(d(x, y) < M\)]라는 것이다.\\
        \(X\)의 수열 \((x_n)_{n \in \NN}\)이 \textbf{유계(bounded)}라는 것은 \(\{x_n \in X: n \in \NN\}\)가 유계라는 것이다.
    \end{defn}

    \begin{rem}
        앞에서 \(E \subseteq \RR\)가 유계인 것을 상계와 하계가 존재하는 것으로 정의하였는데, 이는 \(\RR\)에 표준적인 거리가 주어졌을 때 위의 정의와 동치이다.
    \end{rem}

    \begin{prop}
        거리공간 \(X\)에서 수렴하는 수열은 유계이다.
    \end{prop}
    \begin{proof}
        \(X\) 안의 수열 \((x_n)_{n \in \NN}\)이 \(x \in X\)로 수렴한다고 하자. 적당한 자연수 \(N\)이 존재하여 \(n \ge N\)이면 \(d(x_n, x) < 1\)이므로, 집합 \(\{x_n \in X : n \ge N\}\)은 유계이다. 한편 \(\{x_n \in X : n < N\}\)은 유한집합이므로 유계이다. 따라서 \(\{x_n \in X : n \in \NN\}\)은 유계이다.
    \end{proof}

지금부터 이 장이 끝날 때까지, 표준적인 거리 \(d(x, y) = \abs{x - y}\)가 주어진 \(\RR\) 안의 실수열을 생각한다.

다음은 고등학교에서 증명하지 않고 넘긴 내용이다.

    \begin{prop}
        실수열 \((x_n)_{n \in \NN}\)과 \((y_n)_{n \in \NN}\)이 각각 \(x, y \in \RR\)로 수렴하고 모든 자연수 \(n\)에 대해 \(x_n \le y_n\)이면 \(x \le y\)이다.
    \end{prop}
    \begin{proof}
        \(x > y\)라고 가정하고 \(\eps = (x - y) / 2\)로 두자. 가정에 의해, 자연수 \(N\)이 존재하여 \(n \ge N\)이면 \(\abs{x_n - x} < \eps, \abs{y_n - y} < \eps\)가 성립한다. 그런데
        \[
        x_N - y_N = (x_N - x) + (x- y) + (y - y_N) > -\eps + 2\eps -\eps = 0
        \]
        이므로 가정에 모순이다. 따라서 \(x \le y\).
    \end{proof}

    \begin{prop}
        실수열 \((x_n)_{n \in \NN}, (y_n)_{n \in \NN}, (z_n)_{n \in \NN}\)에 대하여 \(x_n \le y_n \le z_n\)이고 \(x_n \to \alpha, z_n \to \alpha\)이면 \(y_n \to \alpha\)이다.
    \end{prop}
    \begin{proof}
        극한의 정의로부터 거의 당연하다.
    \end{proof}

    \begin{prop}
        복소수열 \((x_n)_{n \in \NN}\)과 \((y_n)_{n \in \NN}\)이 각각 \(x, y \in \RR\)로 수렴할 때 다음이 성립한다.
        \begin{enumerate} [label=(\alph*), leftmargin=2\parindent]
            \item
            \(x_n + y_n \to x + y\).
            \item
            \(c \in \RR\)에 대해 \(cx_n \to cx\).
            \item
            \(x_ny_n \to xy\).
            \item
            \(y_n \ne 0, y \ne 0\)이면 \(x_n/y_n \to x/y\).
        \end{enumerate}
    \end{prop}
    \begin{proof}
        \quad

        \begin{enumerate} [label=(\alph*), leftmargin=2\parindent]
            \item
            양수 \(\eps > 0\)에 대하여, 다음을 만족하는 자연수 \(N\)이 존재한다.
            \[
            n \ge N \implies \abs{x_n - x} < \frac{\eps}{2}, \abs{y_n - y} < \frac{\eps}{2}    
            \]
            따라서 \(n \ge N\)이면
            \[
            \abs{(x_n + y_n) - (x + y)} \le \abs{x_n - x} + \abs{y_n - y} < \eps    
            \]
            이다.
            \item
            \(c = 0\)이면 \(cx_n\)이 항상 0이므로 자명하다. \(c \ne 0\)이면, 양수 \(\eps > 0\)에 대하여 [\(n \ge N \implies \abs{x_n - x} < \eps / \abs{c}\)]를 만족하는 자연수 \(N\)이 존재한다. 따라서 \(n \ge N\)이면
            \[
            \abs{cx_n - cx} = \abs{c}\abs{x_n - x} < \eps    
            \]
            이다.
            \item
            수열 \((x_n)_{n \in \NN}\)이 수렴하므로 유계이고, 모든 자연수 \(n\)에 대해 \(\abs{x_n} < M\)인 양수 \(M\)을 찾을 수 있다. 이제
            \[
            \abs{x_n y_n - xy} = \abs{x_n y_n - x_n y + x_n y - xy} \le \abs{x_n} \abs{y_n - y} + \abs{y} \abs{x_n - x} \le M \abs{y_n - y} + \abs{y} \abs{x_n - x}   
            \]
            이다. 임의의 양수 \(\eps > 0\)에 대해 다음을 만족하는 자연수 \(N\)이 존재한다.
            \[
            n \ge N \implies \abs{x_n - x} < \frac{\eps}{2(\abs{y}+1)}, \abs{y_n - y} < \frac{\eps}{2M}    
            \]
            따라서 \(n \ge N\)이면
            \[
                \abs{x_n y_n - xy} \le M \abs{y_n - y} + \abs{y} \abs{x_n - x}  < \frac{\eps}{2} + \frac{\eps}{2} = \eps
            \]
            이다.
            \item
            (c)를 이용하면, \(1/y_n \to 1/y\)인 것만 보이면 된다. 먼저 다음을 관찰하자.
            \[
            \abs{\frac{1}{y_n} - \frac{1}{y}} = \frac{\abs{y_n - y}}{\abs{y_n y}}    
            \]
            이때 임의의 양수 \(\eps > 0\)에 대해 다음을 만족하는 자연수 \(N\)이 존재한다.
            \[
            n \ge N \implies \abs{y_n - y} < \min \left\{\frac{\abs{y}}{2}, \frac{\eps \abs{y}^2}{2}\right\}  
            \]
            따라서 \(n \ge N\)이면 \(\abs{y_n} > \abs{y} / 2\)이고, 
            \[
                \abs{\frac{1}{y_n} - \frac{1}{y}} = \frac{\abs{y_n - y}}{\abs{y_n y}} < \frac{\eps \abs{y}^2/ 2}{\abs{y}^2 / 2} = \eps 
            \]
            이다.
        \end{enumerate}
    \end{proof}

    \begin{prop}
        \(p > 0, \abs{r} < 1, x \in \RR\)에 대하여 다음이 성립한다.
        \begin{enumerate} [label=(\alph*), leftmargin=2\parindent]
            \item \(n^{-p} \to 0\).
            \item \(p^{1/n} \to 1\).
            \item \(n^{1/n} \to 1\).
            \item \(n^x / (1 + p)^n \to 0\).
            \item \(x^n \to 0\).
        \end{enumerate}
    \end{prop}
    \begin{proof}
        \quad

        \begin{enumerate} [label=(\alph*), leftmargin=2\parindent]
            \item 양수 \(\eps > 0\)에 대하여 \(N > (1/\eps)^{1/p}\)인 자연수 \(N\)이 존재하고, \(n \ge N\)이면 \(n^{-p} \le N^{-p} < \eps\)이다.
            \item \(p = 1\)일 때는 자명하고, \(0 < p < 1\)에 대하여 \(p^{1/n} = 1/(1/p)^{1/n}\)이므로 \(p > 1\)일 때만 보이면 된다. 이때
            \[
            p = (1 + (p^{1/n} - 1))^n > n(p^{1/n} - 1)
            \]
            이므로
            \[
            0 < p^{1/n} - 1 < \frac{p}{n}.
            \]
            \item
            \[
            \sqrt{n} = (1 + (\sqrt{n}^{1/n} - 1))^n > n(\sqrt{n}^{1/n} - 1)    
            \]
            이므로
            \[
            0 < \sqrt{n}^{1/n} - 1 < \frac{1}{\sqrt{n}}
            \]
            이다. 즉 \(\sqrt{n}^{1/n} \to 1\)이고, \(n^{1/n} = (\sqrt{n}^{1/n})^2 \to 1\)이다.
            \item
            \(N > x\)인 양의 정수 \(N\)을 고정하자. \(n > 2N\)에 대하여
            \[
            (1 + p)^n > \frac{n}{N} p^N = \frac{n!}{(n-N)! N!} p^N > \frac{(n-N+1)^N}{N!} p^N > \frac{n^N}{2^N \cdot N!} p^N
            \]
            이므로
            \[
            0 < \frac{n^x}{(1+p)^x} < \frac{2^N \cdot N!}{p^N} n^{x - N}.
            \]
            \item (d)에서 \(x = 0\)으로 두면 된다.
        \end{enumerate}
    \end{proof}

\section{단조수열}

어떤 수열이 수렴함을 정의대로 보이려면 수렴할 것으로 예상되는 점을 먼저 찾아야 한다. 그런데 많은 경우에 수렴하는 것으로 예상되는 점을 찾는 것이 어려운 경우가 많다. 이번 절에서 배울 단조수렴정리는 수열이 어디로 수렴하는지 모르는 상태에서 수렴을 판정하는 방법 중 하나이다.

\begin{defn}
    실수열 \((x_n)_{n \in \NN}\)이 \textbf{단조증가수열(monotonically increasing sequence; nondecreasing sequence)}이라는 것은 모든 \(n\)에 대해 \(x_n \le x_{n+1}\)이라는 것이다. 마찬가지 방법으로 \textbf{단조감소수열(monotonically decreasing sequence; nonincreasing sequence)}도 정의한다. 단조증가수열 또는 단조감소수열을 \textbf{단조수열(monotonic sequence)}이라고 한다.
\end{defn}

\begin{thm} [단조수렴정리]
    실수열 \((x_n)_{n \in \NN}\)이 단조수열일 때, \((x_n)_{n \in \NN}\)이 수렴할 필요충분조건은 \((x_n)_{n \in \NN}\)이 유계인 것이다.
\end{thm}
\begin{proof}
    \((\Rightarrow)\) 방향은 이미 증명하였다. 역으로 단조수열 \((x_n)_{n \in \NN}\)이 유계라고 가정하자. 일반성을 잃지 않고 \((x_n)_{n \in \NN}\)이 단조증가수열이라고 가정할 수 있다. 그러면 \(E := \{x_n \in \RR : n \in \NN\}\)은 \(\RR\)의 공집합이 아니고 위로 유계인 부분집합이므로 \textbf{\(\RR\)의 완비성에 의해} \(x := \sup E\)가 존재한다. 우리의 주장은 \(x_n \to x\)라는 것이다. 양수 \(\eps > 0\)이 주어졌을 때, 상한의 성질에 의해 \(x_N > x - \eps\)인 자연수 \(N\)이 존재한다. 따라서 \(n \ge N\)이면
    \[
    \abs{x_n - x} = x - x_n \le x- x_N < \eps    
    \]
    이다.
\end{proof}

\section{상극한과 하극한}

먼저 무한대 기호 \(\infty\)를 정식으로 등장시키자.

\begin{defn}
    실수 집합 \(\RR\)에 두 원소 \(+\infty\)와 \(-\infty\)를 추가한 집합 \(\RR \cup \{+\infty, -\infty\}\)를 생각하자. \(\RR\)의 원소들끼리의 순서는 그대로 보존하고, 임의의 실수 \(x \in \RR\)에 대해 \(-\infty < x < \infty\)로 정의하자. 이때 \(\RR \cup \{+\infty, -\infty\}\)가 전순서 집합의 구조를 가짐은 분명하며, 이 집합을 \textbf{확장된 실수(extended real number system)}라고 한다.
\end{defn}

\(\infty\)는 ``무한히 커지고 있는 상태'' 따위가 아니다. 그냥 0, 1, \(e\)와 같은 기호일 뿐이며, 이를 도입함으로써 표현이 좀 더 간결해지는 측면이 있기 때문에 \(\infty\)라는 기호를 사용하는 것이다.

\begin{notn}
    다음과 같은 표기를 사용한다. 직관적이므로(오히려 이렇게 하지 않는 것이 부자연스럽다) 이해에 어려움은 없을 것이다.
    \begin{enumerate} [label=(\alph*), leftmargin=2\parindent]
        \item \(+\infty\)에서 + 부호는 생략하기도 한다.
        \item 확장된 실수를 \(\RR \cup \{\pm \infty\}, [-\infty, \infty]\) 등으로 쓰기도 한다.
        \item 확장된 실수에서 실수들끼리의 덧셈과 곱셈은 그대로 보존된다. 그리고 실수 \(x \in \RR\)에 대하여,
        \[
            x \pm \infty = \pm \infty, \quad \frac{x}{\pm \infty} = 0, \quad x \cdot (\pm \infty) = \pm \infty \ (x > 0), \quad  x \cdot (\pm \infty) = \mp \infty \ (x < 0)
        \]
        로 정의한다. \(\infty - \infty, \ \infty + (-\infty), 0 \cdot (\pm \infty)\) 등은 정의하지 않는다.
        \item 집합 \(E \subseteq \RR\)이 위로 유계가 아니면 \(\sup E = \infty\)로 쓴다. 마찬가지로 \(E \subseteq \RR\)이 아래로 유계가 아니면 \(\inf E = -\infty\)로 쓴다.
        \item 실수열 \((x_n)_{n \in \NN}\)이 있다고 하자. 임의의 실수 \(M\)에 대하여
        \[
        n \ge N \implies x_n > M
        \]
        을 만족시키는 자연수 \(N\)이 존재하면 \(\lim_{n \to \infty } x_n = \infty\) 또는 \(x_n \to \infty\)로 쓴다. \(\lim_{n \to \infty} x_n = -\infty, \ x_n \to -\infty\)도 비슷한 방법으로 정의한다.
    \end{enumerate}
\end{notn}

\(\infty\) 표기를 이용하면 다음과 같은 깔끔한 표현이 가능하다.

    \begin{prop} \label{prop 2.4.2}
        임의의 단조 실수열 \((x_n)_{n \in \NN}\)에 대하여 \(\lim_{n} x_n \in \RR \cup \{\pm \infty\}\)이다.
    \end{prop}
    \begin{proof}
        \((x_n)_{n \in \NN}\)이 유계이면 단조수렴정리에 의해 \(\lim_{n} x_n\)의 값은 실수로 존재한다. 유계가 아니고 \((x_n)_{n \in \NN}\)이 단조증가하면 \(\lim_{n} x_n = \infty\)이다. 유계가 아니고 \((x_n)_{n \in \NN}\)이 단조감소하면 \(\lim_{n} x_n = -\infty\)이다. 
    \end{proof}

이제 상극한과 하극한을 정의하자.

    \begin{defn} \label{def 2.4.3}
        실수열 \((x_n)_{n \in \NN}\)에 대하여 확장된 실수 안의 수열 \((y_n)_{n \in \NN}\)을
        \[
        y_n = \sup \{x_k \in \RR : k \ge n\}    
        \]
        으로 정의하자. \((y_n)_{n \in \NN}\)은 단조감소수열이므로(왜 그런가?), 명제 \ref{prop 2.4.2}에 의해 \(\lim_{n} y_n\)의 값이 \(\RR \cup \{\pm \infty\}\)안에서 존재한다. 이때 \(\alpha := \lim_{n} y_n \in \RR \cup \{\pm \infty\}\)의 값을 수열 \((x_n)_{n \in \NN}\)의 \textbf{상극한(upper limit)}이라고 하고 \(\limsup_{n \to \infty} x_n = \alpha\)로 쓴다. 마찬가지 방법으로 \textbf{하극한(lower limit)}을 정의하며, \(\liminf_{n \to \infty} x_n\)으로 쓴다. 다시 말해,
        \[
        \limsup_{n} x_n = \lim_n \sup_{k \ge n} x_k, \quad \liminf_{n} x_n = \lim_n \inf_{k \ge n} x_k.
        \]
    \end{defn}

\begin{ex}
    \quad

    \begin{enumerate} [label=(\alph*), leftmargin=2\parindent]
        \item 수열
        \[
        x_n = \begin{cases}
            1 + 2^{-n} &\textif \ n\text{은 홀수}\\
            -2^{-n}  &\textif \ n\text{은 짝수}
        \end{cases}
        \]
        에 대하여,
        \[
        \sup_{k \ge n} x_k =
            \begin{cases}
                1 + 2^{-n} &\textif \ n\text{은 홀수}\\
                1 + 2^{-n-1} &\textif \ n\text{은 짝수}
            \end{cases}
        , \quad \inf_{k \ge n} =
            \begin{cases}
                -2^{-n-1} &\textif \ n\text{은 홀수}\\
                -2^{-n} &\textif \ n\text{은 짝수}
            \end{cases}
        \]
        이므로
        \[
            \limsup_{n} x_n = \lim_n \sup_{k \ge n} x_k = 1, \quad \limsup_{n} x_n = \lim_n \inf_{k \ge n} x_k = 0.
        \]
        \item 수열
        \[
            x_n = \begin{cases}
                1 - 2^{-n} &\textif \ n\text{은 홀수}\\
                -n  &\textif \ n\text{은 짝수}
            \end{cases}
        \]
        에 대하여,
        \[
        \sup_{k \ge n} x_k = 1
        , \quad \inf_{k \ge n} = -\infty
        \]
        이므로
        \[
            \limsup_{n} x_n = \lim_n \sup_{k \ge n} x_k = 1, \quad \liminf_{n} x_n = \lim_n \inf_{k \ge n} x_k = -\infty.
        \]
        \item 수열 \(x_n = 1/(x+1)\)에 대하여,
        \[
            \sup_{k \ge n} x_k = \frac{1}{n+1}, \quad \inf_{k \ge n} k = 0
        \]
        이므로
        \[
            \limsup_{n} x_n = \lim_n \sup_{k \ge n} x_k = 0, \quad \liminf_{n} x_n = \lim_n \inf_{k \ge n} x_k = 0.
        \]
    \end{enumerate}
\end{ex}

    \begin{prop}
        실수열 \((x_n)_{n \in \NN},  (y_n)_{n \in \NN}\)에 대하여 \(\liminf_n x_n \le \limsup_n x_n\)이다. 한편 모든 자연수 \(n\)에 대해 \(x_n \ge y_n\)이면 \(\limsup_n x_n \ge \limsup_n y_n, \ \liminf_n x_n \ge \liminf_n y_n\)이다.
    \end{prop}
    \begin{proof}
        연습문제로 남긴다.
    \end{proof}

다음은 증명에서 자주 사용하는 동치 조건이다.
    \begin{prop} \label{prop 2.4.6}
        실수열 \((x_n)_{n \in \NN}\)과 \(\alpha \in \RR\)에 대하여 \(\limsup_{n} x_n = \alpha\)일 필요충분조건은 다음 (a), (b)가 성립하는 것이다.
            \begin{enumerate} [label=(\alph*), leftmargin=2\parindent]
                \item 임의의 양수 \(\eps > 0\)에 대하여
                \[
                n \ge N \implies x_n < \alpha + \eps    
                \]
                를 만족하는 자연수 \(N\)이 존재한다.
                \item 임의의 양수 \(\eps > 0\)에 대하여 \(x_n > \alpha - \eps\)인 자연수 \(n\)이 무수히 많이 존재한다.
            \end{enumerate}
    \end{prop}
    \begin{proof}
        정의 \ref{def 2.4.3}에서와 마찬가지로 수열 \((y_n)_{n \in \NN}\)을 정의한다. 동치임을 보이기 위하여 다음 두 사실을 보이면 충분하다.
        \[
        \text{(a)} \iff \limsup_{n} x_n \le \alpha, \quad \text{(b)} \iff \limsup_{n} x_n \ge \alpha 
        \]
        먼저 (a)를 가정하고 양수 \(\eps > 0\)이 주어졌다고 하자. 그러면 적당한 자연수 \(N\)에 대해 \(y_N \ge \alpha + \eps\)이고, \((y_n)_{n \in \NN}\)이 단조감소수열이므로 \(\lim_{n} y_n \le \alpha + \eps\)이다. 그런데 \(\eps > 0\)은 임의의 양수였으므로 \(\limsup_n x_n = \lim_{n} y_n \le \alpha\)이다. 역으로 \(\limsup_{n} x_n = \lim_n y_n \le \alpha\)을 가정하자. 양수 \(\eps > 0\)에 대해
        \[
        n \ge N \implies y_n < \alpha + \eps    
        \]
        이도록 하는 자연수 \(N\)이 존재하며, 이때 \(n \ge N\)이면 \(x_n \le y_n < \alpha + \eps\)이다.\\
        다음으로 (b)를 가정하고 양수 \(\eps > 0\)이 주어졌다고 하자. 임의의 \(n\)에 대해 \(k \ge n\)이면서 \(x_k > \alpha - \eps\)인 \(k\)가 존재하므로 \(y_n > \alpha - \eps\), 따라서 \(\lim_{n} y_n \ge \alpha - \eps\)이다. 그런데 \(\eps > 0\)은 임의의 양수였으므로 \(\limsup_n x_n = \lim_{n} y_n \ge \alpha\)이다. 역으로 어떤 \(\eps > 0\)에 대하여 \(x_n > \alpha - \eps\)인 자연수 \(n\)이 유한 개만 존재한다고 가정하자. 그러면 충분히 큰 자연수 \(N\)에 대하여 \(n \ge N\)이면 \(x_n \le \alpha - \eps\)이고, 따라서 \(y_n \le \alpha - \eps\)이다. 그러므로 \(\limsup_{n} x_n = \lim_n y_n \le \alpha - \eps < \alpha\)이다.
    \end{proof}

    \begin{prop}
        실수열 \((x_n)_{n \in \NN}\)과 \(\alpha \in \RR\)에 대하여 \(\liminf_{n} x_n = \alpha\)일 필요충분조건은 다음 (a), (b)가 성립하는 것이다.
        \begin{enumerate} [label=(\alph*), leftmargin=2\parindent]
            \item 임의의 양수 \(\eps > 0\)에 대하여
            \[
            n \ge N \implies x_n > \alpha - \eps    
            \]
            를 만족하는 자연수 \(N\)이 존재한다.
            \item 임의의 양수 \(\eps > 0\)에 대하여 \(x_n < \alpha + \eps\)인 자연수 \(n\)이 무수히 많이 존재한다.
        \end{enumerate}
    \end{prop}
    \begin{proof}
        연습문제로 남긴다.
    \end{proof}

상극한과 하극한이 유용한 이유 중 하나는 다음 따름정리 때문이다.

    \begin{cor} \label{2.4.8}
        실수열 \((x_n)_{n \in \NN}\)과 \(\alpha \in \RR\)에 대하여
        \[
        \lim_n x_n = \alpha \iff \limsup_n x_n = \liminf_n x_n = \alpha.    
        \]
    \end{cor}
    \begin{proof}
        극한의 정의와 위의 동치조건들로부터 당연하다.
    \end{proof}

    \begin{ex}
        실수열 \((s_n)_{n \in \NN}\)에 대하여 실수열 \((\sigma_n)_{n \in \ZZ_+}\)을
        \[
        \sigma_n = \frac{1}{n} \sum_{k=0}^{n-1} s_k    
        \]
        로 정의하자. 실수 \(s \in \RR\)에 대하여 \(s_n \to s\)이면 \(\sigma_n \to s\)임을 보이려 한다. 먼저 \(s = 0\)일 때 성립한다고 가정하자. 그러면 임의의 실수 \(s\)에 대해 \((s_n - s) \to 0\)이므로,
        \[
        \sigma_n = s + \frac{1}{n} \sum_{k=0}^{n-1} (s_k - s) \to s
        \]
        이다. 따라서 \(s = 0\)일 때만 생각하면 된다. 양수 \(\eps > 0\)이 주어졌다고 하자. 그러면 가정에 의해, \(n \ge N\)이면 \(\abs{s_n} < \eps\)이도록 하는 자연수 \(N\)이 존재한다. \(n > N\)에 대해
        \begin{align*}
			\abs{\sigma_n} = \frac{1}{n}\abs{\sum_{k=0}^{n-1}s_k} &\le \frac{1}{n}\abs{\sum_{k=0}^{N}s_k} + \frac{1}{n}\sum_{k=N+1}^{n}\abs{s_k}\\
			&< \frac{1}{n}\abs{\sum_{k=0}^{N}s_k} + \frac{(n-N)\eps}{n}\\
			&<  \frac{1}{n}\abs{\sum_{k=0}^{N}s_k} + \eps
		\end{align*}
        이므로, 양변에 (\(\lim\)이 아니라) \(\limsup\)을 취하면
        \[
        \limsup_n \abs{\sigma_n} \le \limsup_n \paren{\frac{1}{n}\abs{\sum_{k=0}^{N}s_k} + \eps} = \lim_n \paren{\frac{1}{n}\abs{\sum_{k=0}^{N}s_k} + \eps} = \eps
        \]
        이다. 그런데 \(\eps > 0\)은 임의의 양수였으므로 \(\limsup_n \abs{\sigma_n} \le 0\)이고, 항상 \(\abs{\sigma_n} \ge 0\)이므로 
        \[
        0 \le \liminf_n \abs{\sigma_n} \le \limsup_n \abs{\sigma_n} \le 0,
        \]
        즉
        \[
            \liminf_n \abs{\sigma_n} = \limsup_n \abs{\sigma_n} = 0
        \]
        이다. 따라서 \(\lim_n \abs{\sigma_n} = 0\)이므로 \(\lim_n {\sigma_n} = 0\)이다.
    \end{ex}



\chapter{유클리드공간} \label{sec 3}
지난 두 장에서 실수의 성질과 수열, 그 중에서도 실수열에 대해 꽤 많은 분량을 할애하여 살펴보았다. 잠시 숨을 돌려 우리의 마음의 고향인 벡터공간의 성질을 해석개론에 필요한 만큼만 알아보도록 하자. 선형대수학을 이전에 공부했다면 거의 아는 내용일 것이다. 이 장의 목표는 \(\RR^d\) 또는 \(\CC^d\)에 표준적인 거리 함수를 주는 것이다.

\section{벡터공간}
\begin{defn}
    집합 \(V\)와 체 \(F\)에 대하여, \(V\) 위의 이항 연산 \(+ : V \times V \to V\)와, 함수 \(\cdot : F \times V \to V\)\footnote{\(\cdot\)은 표기에서 주로 생략한다.}가 정의되어 다음 (V1)-(V8)을 만족할 때 \((V, +, \cdot)\) 또는 간단히 \(V\)를 \textbf{\(F\)-벡터공간(vector space over \(F\))}라고 부르고 \(V\)의 원소를 \textbf{벡터(vector)}라고 부른다. 그리고 \(+\)와 \(\cdot\)을 각각 덧셈(addition)과 스칼라곱(scalar multiplication)이라 부른다.
    \begin{enumerate}[label=(VS\arabic*), leftmargin=2\parindent]
        \item
        모든 \(u, v, w \in V\)에 대해 \((u+v)+w=u+(v+w)\).
        \item
        \(0 \in V\)가 존재하여 [모든 \(v \in V\)에 대해 \(v+0=0+v=v\)].
        \item
        각 \(v \in V\)에 대하여 \(-v \in V\)가 존재하여 \(v+(-v)=(-v)+v=0\).
        \item
        모든 \(v, w \in V\)에 대해 \(v+w=w+v\).
        \item
        모든 \(v \in V\)에 대해 \(1v = v\).
        \item
        모든 \(a, b \in F, v \in V\)에 대해 \((ab)v = a(bv)\).
        \item
        모든 \(a, b \in F, v \in V\)에 대해 \((a+b)v=av+bv\).
        \item
        모든 \(a \in F, v, w \in V\)에 대해 \(a(v+w) = av + aw\).
    \end{enumerate}
\end{defn}

\begin{prop} \label{prop 3.1.2}
    \(F\)-벡터공간 \(V\)와 \(v, w, u \in F\)에 대하여 다음이 성립한다.
    \begin{enumerate} [label=(\alph*), leftmargin=2\parindent]
        \item \(v+w = v+u\)이면 \(w = u\).
        \item \(w \in V\)가 [모든 \( v \in V\)에 대해 \(v+w = w+v = v\)]를 만족하면 \(v = 0\).
        \item \(v+w = w+v = 0\)이면 \(w = -v\).
        \item \(-(-v) = v\).
    \end{enumerate}
\end{prop}
\begin{proof}
    명제 \ref{prop 1.1.3}의 증명과 완전히 같다.
\end{proof}

위 명제로부터 \(0 \in V\)를 \textbf{덧셈의 항등원(additive identity)} 또는 \textbf{영벡터(zero vector)}라고 부를 수 있고, \(v \in V\)에 대해 \(-v\)를 \textbf{덧셈의 역원(additive inverse)}이라고 부를 수 있다.

\begin{rem}
    항상 0이 어디에 사는 0인지 구분해야 한다. 체 \(F\)에는 덧셈이 정의되어 있고 이 덧셈의 항등원을 0이라고 한다. 한편 \(F\)-벡터공간에 정의된 덧셈의 항등원도 0이라고 한다. \textbf{항상} 0을 보면 어디 사는 0인지 생각하는 연습을 하라. \(F\)의 0과 \(V\)의 \textit{0}(또는 \(\mathbf{0}\))을 글씨체로 구분하는 책들도 있는데 필자의 귀찮음으로 이 노트에서는 구분하지 않을 것이니 넓은 양해 바란다.
\end{rem}

\begin{prop}
    \(F\)-벡터공간 \(V\)와 \(c \in F, v \in V\)에 대하여 다음이 성립한다.
    \begin{enumerate} [label=(\alph*), leftmargin=2\parindent]
        \item \(0v = 0\).
        \item \(c0 = 0\).
        \item \(cv = 0\)이면 \(c = 0\) 또는 \(v = 0\).
        \item \((-1)v = -v\).
        \item \((-c)v = c(-v) = -(cv)\)이고 \((-c)(-v) = cv\).
    \end{enumerate}
\end{prop}
\begin{rem}
    (a)-(c)에는 총 7개의 0이 등장한다. 각 0은 어디 사는 0인가?\footnote{순서대로 \(F, V, V, V, V, F, V\).}
\end{rem}
    \begin{proof}
        \quad

        \begin{enumerate} [label=(\alph*), leftmargin=2\parindent]
            \item \(0v + 0 = 0v = (0 + 0)v = 0v + 0v\)이므로 명제 \ref{prop 3.1.2} (a)에 의해.
            \item 임의의 \(v \in V\)에 대해 \(c0 + v = c0 + cc^{-1}v = c(0 + c^{-1}v) = cc^{-1}v = v\).
            \item \(v + (-v) = 0 = (1+(-1))v = 1v + (-1)v = v + (-1)v\)이므로 명제 \ref{prop 3.1.2} (a)에 의해.
            \item \(cv = 0\)이고 \(c \ne 0\)이면 \(v = 1v = c^{-1}cv = c^{-1}0 = 0\).
            \item (d)에 의해 당연.
        \end{enumerate}
    \end{proof}

시간이 남는다면 아래 예시들이 \(F\)-벡터공간이 되는 것을 직접 증명해 보라.

\begin{ex}
    \quad

    \begin{enumerate} [label=(\alph*), leftmargin=2\parindent]
        \item 체 \(F\)에 대하여 \(F\)의 \(d\)-tuple\footnote{\(d\)-tuple이란 무엇인가? 즉, 순서쌍이란 무엇인가? ㅋㅋ. 수학적 기초론에 빠지는 것은 위험하다.}들의 집합을
        \[
        F^d := \{(x_1, \ldots, x_d) : x_i \in F\}    
        \]
        로 쓰자. 이때 \(x_i, y_i, c \in F\)에 대하여
        \[
        (x_1, \ldots, x_d) + (y_1, \ldots, y_d) := (x_1 + y_1, \ldots, x_d + y_d), \quad c(x_1, \ldots, x_d) = (cx_1, \ldots cx_d)
        \]
        로 정의하면 \(F^d\)은 자연스러운 \(F\)-벡터공간의 구조를 가진다.
        \item 집합 \(X\)에서 \(F\)로 가는 모든 함수들의 집합을 \(F^X\)로 쓰자. 이때 \(f, g \in F^X, c \in F\)에 대하여
        \[
        (f+g) : x \mapsto f(x) + g(x), \quad cf : x \mapsto c(f(x))   
        \]
        로 정의하면 \(F^X\)는 자연스러운 \(F\)-벡터공간의 구조를 가진다.
        \item 체 \(F\) 안의 수열은 \(\NN\)에서 \(F\)로 가는 함수이다. 따라서 (b)의 특수한 경우로, \(F\) 안의 수열들의 공간은 자연스러운 \(F\)-벡터공간의 구조를 가진다.
    \end{enumerate}
\end{ex}

\begin{rem}
    모든 \(\CC\)-벡터공간은 자연스러운 \(\RR\)-벡터공간의 구조를 가진다.
\end{rem}


\section{내적공간}
이 장이 끝날 때까지 \(F\)는 \(\RR\) 또는 \(\CC\)이다.

\begin{defn}
    \(F\)-벡터공간 \(V\)와 함수 \(\inner{\cdot}{\cdot}: V \times V \rightarrow F\)가 모든 \(u, v, w \in V, c \in F\)에 대해 (IP1)-(IP3)을 만족할 때, \(\inner{\cdot}{\cdot}\)를 \textbf{\(V\)의 내적(inner product on \(V\))}이라고 하고, \((V, \inner{\cdot}{\cdot})\) 또는 간단히 \(V\)를 \textbf{내적공간(inner product space)}이라고 한다.
    \begin{enumerate}[label=(IP\arabic*), leftmargin=2\parindent]
        \item
        (첫 번째 자리에서의 선형성) \(\inner{u+v}{w} = \inner{u}{w}+ \inner{v}{w}\)이고 \(\inner{cv}{w} = c\inner{v}{w}\).
        \item
        (켤레 대칭성) \(\inner{v}{w} = \overline{\inner{w}{v}}\).\footnote{물론 \(F = \RR\)인 경우에 (IP2)는 \(\inner{v}{w}\ = \inner{w}{v}\)를 의미한다.}
        \item
        (양의 정부호성) \(\inner{v}{v} \ge 0\)이고, \(\inner{v}{v} = 0\)이면 \(v = 0\)이다.
    \end{enumerate}
\end{defn}


고등학교나 미적분학 시간에 내적이라는 이름으로 배운 연산은 사실 진짜 내적의 특수한 경우였을 뿐이다.

\begin{prop} 다음 두 함수는 각각 \(\RR^d\)와 \(\CC^d\)의 내적이 된다. 이들을 \(\RR^d\)와 \(\CC^d\)의 \textbf{표준적인 내적(standard inner product)} 또는 \textbf{점곱(dot product)}라고 한다.
    \begin{align*}
        &\inner{\cdot}{\cdot} : \RR^d \times \RR^d \to \RR : ((x_1, \ldots, x_d), (y_1, \ldots, y_d)) \mapsto \sum_{i=1}^d x_i y_i\\
        &\inner{\cdot}{\cdot} : \CC^d \times \CC^d \to \CC : ((x_1, \ldots, x_d), (y_1, \ldots, y_d)) \mapsto \sum_{i=1}^d x_i \overline{y_i}
    \end{align*}
\end{prop}
\begin{proof}
    연습문제로 남긴다.
\end{proof}

하나의 벡터공간에 줄 수 있는 내적은 유일하지 않다. 가령
\[
\inner{\cdot}{\cdot} : \RR^2 \times \RR^2 \to \RR : ((x_1, x_2), (y_1, y_2)) \mapsto 2 x_1 y_1 + 3x_2 y_2    
\]
도 (표준적인 내적과 다른) \(\RR^2\)의 내적이 됨을 쉽게 확인할 수 있다.

내적의 정의 중 (IP3)에 의해, 다음이 잘 정의된다.

\begin{notn} \label{not 3.2.3}
    내적공간 \(V\)에서 함수
    \[
    \norm{\cdot} : V \to \RR : v \mapsto \inner{v}{v}^{1/2} 
    \]
    를 정의하자.
\end{notn}

그리고 다음 표기를 도입하는 것이 편리하다.

\begin{defn}
    내적공간 \(V\)와 \(v, w \in V\)에 대하여 \(\inner{v}{w} = 0\)이면 \(v\)와 \(w\)는 서로 \textbf{수직(orthogonal)}이라고 하고, \(v \perp w\)라고 쓴다.
\end{defn}

\begin{prop} \label{prop 3.2.5}
    내적공간 \(V\)와 \(v,w \in V, c \in F\)에 대하여 다음이 성립한다.
    \begin{enumerate}[label=(\alph*), leftmargin=2\parindent]
        \item \(\norm{cv} = \abs{c} \norm{v}\).
        \item \(\norm{v} = 0\) 이면 \(v = 0\).
        \item (평행사변형 법칙\footnote{Parallelogram law.}) \(\norm{v+w}^2 + \norm{v-w}^2 = 2(\norm{v}^2 + \norm{w}^2)\).
        \item (피타고라스의 법칙\footnote{Pythagorean law.}) \(v \perp w\)이면 \(\norm{v + w}^2 = \norm{v}^2 +\norm{w}^2\).
    \end{enumerate}
\end{prop}
\begin{proof}
    정의로부터 거의 당연하다.
\end{proof}

고등학교 때 \(\RR^3\)의 점곱에 대해 배운 다음 부등식은 사실 \textbf{모든 종류의 내적}에서 성립한다.

\begin{thm} [코시-슈바르츠 부등식\footnote{Cauchy-Schwarz Inequality.}]
    내적공간 \(V\)와 \(v, w \in V\)에 대하여 \(\abs{\inner{v}{w}} \le \norm{v}\norm{w}\)이고, 등호가 성립할 필요충분조건은 \(v, w\) 중 하나가 다른 하나의 스칼라배인 것이다.
\end{thm}
\begin{proof}
    \(w = 0\)이면 자명하므로 \(w \ne 0\)이라고 가정하자. 이때
    \[
    c = \frac{\inner{v}{w}}{\inner{w}{w}}    
    \]
    라고 두면, \(\inner{v - cw}{cw} = 0\)이다. 따라서
    \[
    \norm{v}^2 = \norm{v - cw}^2 + \norm{cw}^2 = \norm{v - cw}^2 + \abs{c}^2 \norm{w}^2
    \]
    이고 \(\norm{v - cw}^2 \ge 0\)이므로
    \[
    \norm{v}^2 \ge \abs{c}^2 \norm{w}^2  = \frac{\abs{\inner{v}{w}}^2}{\norm{w}^2}
    \]
    을 얻는다. 이를 정리하면 원하는 부등식을 얻는다. 이때 등호가 성립할 필요충분조건은 \(v = cw\)인 것이다.
\end{proof}

\section{노름공간}

\begin{defn} \label{def 3.3.1}
    \(F\)-벡터공간 \(V\)와 함수 \(\norm{\cdot} : V \to \RR\)가 모든 \(v, w \in V, c \in F\)에 대해 (N1)-(N3)을 만족할 때, \(\norm{\cdot}\)를 \(V\)의 \textbf{노름(norm)}이라고 하고, \((V, \norm{\cdot})\) 또는 간단히 \(V\)를 \textbf{노름공간(normed vector space)}이라고 한다.
    \begin{enumerate} [label=(N\arabic*), leftmargin=2\parindent]
        \item \(\norm{cv} = \abs{c}\norm{v}\).
        \item (삼각부등식\footnote{Triangular inequality.}) \(\norm{v + w} \le \norm{v} + \norm{w}\).
        \item \(\norm{v} = 0\)이면 \(v = 0\)이다.
    \end{enumerate}
\end{defn}

\begin{prop}
    노름공간 \(V\)의 모든 원소 \(v \in V\)에 대해 \(\norm{v} \ge 0\)이다.
\end{prop}
\begin{proof}
    \(0 = \norm{v + (-v)} \le \norm{v} + \norm{-v} = 2\norm{v}\).
\end{proof}

그런데 내적공간에서 \(\norm{\cdot}\)이라는 표기는 \(\inner{v}{v}^{1/2}\)를 나타내기로 표기법 \ref{not 3.2.3}에서 약속하였다. 기대하는 바와 같이, 내적공간에서 \(\norm{\cdot}\)은 정말로 정의 \ref{def 3.3.1}의 노름이 된다.

\begin{thm}
    내적공간 \(V\)에서 \(\norm{\cdot} : V \to \RR : v \mapsto \inner{v}{v}^{1/2}\)로 정의하면 \(\norm{\cdot}\)은 \(V\)의 노름이 된다. 즉 내적에 의해 자연스럽게 유도되는 노름이 존재한다.
\end{thm}
\begin{proof}
    (N1)과 (N3)은 각각 명제 \ref{prop 3.2.5}의 (a), (b)에서 확인하였으므로 (N2)만 확인하면 된다. 노름은 0 이상의 실수이므로, \(v, w \in V\)에 대하여 \(\norm{v+w}^2 \le (\norm{v} + \norm{w})^2\)인 것을 보이는 것과 같다. 이때
        \begin{align*}
            \norm{v+w}^2 &= \inner{v}{v} + \inner{v}{w} + \inner{w}{v} + \inner{w}{w}\\
            &= \norm{v}^2 + \inner{v}{w} + \overline{\inner{v}{w}} + \norm{w}^2\\
            &= \norm{v}^2 + 2 \mathrm{Re}\inner{v}{w} + \norm{w}^2\\
            &\le \norm{v}^2 + 2\abs{\inner{v}{w}} + \norm{w}^2\\
            &\le \norm{v}^2 + 2\norm{v}\norm{w} + \norm{w}^2 \quad (\because \ \text{Cauchy-Schwarz})\\
            &= (\norm{v} + \norm{w})^2
        \end{align*}
    이므로 삼각부등식이 증명되었다.
\end{proof}

따라서 \(F^d\)의 표준적인 내적으로부터 유도되는 \textbf{표준적인 노름(standard norm)} 또는 \textbf{유클리드 노름(Euclidean norm)}은 다음과 같이 정의된다.
\[
\norm{\cdot} : (x_1, \ldots, x_d) \mapsto \sqbracket{\sum_{i=1}^d \abs{x_i}^2}^{1/2}   
\]

그러나 모든 노름이 내적으로부터 유도된 노름일 필요는 없다. 노름의 정의에는 내적이 포함되지 않았기 때문이다.

\begin{ex}
     \(p = 1, 2, \infty\)에 대해 함수 \(\norm{\cdot}_p : F^d \to \RR\)를 다음과 같이 정의하자.
    \[
        \norm{(x_1, \ldots, x_d)}_p :=
        \begin{cases}
            \displaystyle
            \sqbracket{\sum_{i=1}^d \abs{x_i}^p}^{1/p} &\textif p < \infty\\
            \displaystyle
            \max_{1 \le i \le d} \abs{x_i} &\textif \ p = \infty
        \end{cases}
    \]
    \(p = 2\)일 때가 앞에서 확인한 유클리드 노름의 경우이고, \(p = 1\) 또는 \(p = \infty\)일 때 \(\norm{\cdot}_p\)가 실제로 노름이 되는 것을 보이는 것은 연습문제로 남긴다.\\
    그런데 \(d \ge 2\)일 때\footnote{\(d = 1\)이면 모든 \(p\)에 대해 \(\norm{\cdot}_p = \norm{\cdot}\)일 뿐이므로 별 재미가 없다.} \(\norm{\cdot}_p\)가 내적으로부터 유도된 노름일 필요충분조건은 \(p = 2\)인 것이다. \((\Leftarrow)\) 방향은 증명하였고, 역으로 \(\norm{\cdot}_p\)가 내적으로부터 유도된 노름이라고 가정하자. 그러면 명제 \ref{prop 3.2.5} (c)의 평행사변형 법칙이 성립한다. 이제
    \[
        v = (1, 0, 0, \ldots, 0) \in F^d, \quad w = (0, 1, 0, \ldots, 0) \in F^d
    \]
    라고 두면
    \[
    \norm{v+w}^2 + \norm{v-w}^2 = 2 \cdot 2^{2/p}
    \]
    이고
    \[
    2(\norm{v}^2 + \norm{w}^2) = 4
    \]
    이므로, 평행사변형 법칙에 의해 \(p = 2\)이어야 한다.
\end{ex}

사실, 평행사변형 법칙은 어떤 노름이 내적으로부터 유도된 노름일 필요충분조건이다.

\begin{thm} \label{thm 3.3.4}
    노름공간 \(V\)에 대하여 다음이 동치이다.
    \begin{enumerate} [label=(\alph*), leftmargin=2\parindent]
        \item {[모든 \(v \in V\)에 대해 \(\inner{v}{v} = \norm{v}^2\)]}인 \(V\)의 내적 \(\inner{\cdot}{\cdot}\)이 존재한다.
        \item 평행사변형 법칙이 성립한다. 즉, 모든 \(v, w \in V\)에 대해서
        \[
            \norm{v+w}^2 + \norm{v-w}^2 = 2(\norm{v}^2 + \norm{w}^2)    
        \]
        이다.
    \end{enumerate}
\end{thm}
\begin{proof} [Sketch of Proof]
    \((\Rightarrow)\) 방향은 이미 증명하였다. 역으로 (b)를 가정하고, \(\inner{\cdot}{\cdot} : V \times V \to F\)를 다음과 같이 정의한다.
    \[
    \inner{v}{w} =
        \begin{cases}
            \frac{1}{4}(\norm{v+w}^2 - \norm{v-w}^2) &\textif \ F = \RR\\
            \frac{1}{4}(\norm{v+w}^2 - \norm{v-w}^2 + i\norm{v+iw}^2 -i\norm{v-iw}^2) &\textif \ F = \CC
        \end{cases}    
    \]
    이때 모든 \(v \in V\)에 대해 \(\inner{v}{v} = \norm{v}^2\)인 것은 쉽게 확인할 수 있다. \(F = \RR\)이면
    \[
    \inner{v}{v} = \frac{1}{4}(\norm{2v}^2 + \norm{0}^2) = \norm{v}^2  
    \]
    이고, \(F =\CC\)이면
    \[
    \inner{v}{v} = \frac{1}{4}(\norm{2v}^2 - \norm{0}^2 + i\norm{(1+i)v}^2 -i\norm{(1-i)v}^2) = \norm{v}^2    
    \]
    이기 때문이다. 이제 \(\inner{\cdot}{\cdot}\)이 실제로 내적이 되는 것만 확인하면 되는데, 이 증명에 (아직 배우지 않은) 연속함수의 개념이 사용된다. 따라서 이후 증명은 이 장의 Appendix로 넘기며, \ref{sec conti}장을 공부하고 돌아와서 다시 읽어보라(다행히 \ref{sec conti}장까지 이 정리를 사용할 일은 없기 때문에, 순환논증은 아니다\(\ldots\)).
\end{proof}

잠시 다른 이야기를 좀 했는데, 이제 우리 목표에 거의 도달했다. 이 장의 서문에서 밝혔듯이, 우리의 목표는 \(F^d\)에 표준적인 거리 함수를 주는 것이다.

\begin{thm}
    노름공간 \(V\)에서 \(d : V \times V \to \RR : (v, w) \mapsto \norm{v - w}\)로 정의하면 \(d\)는 \(V\)의 거리가 된다. 즉 노름에 의해 자연스럽게 유도되는 거리가 존재한다.
\end{thm}
\begin{proof}
    노름의 정의로부터 거의 당연하다.
\end{proof}

따라서, 내적공간은 자연스러운 노름공간과 거리공간의 구조를 가진다. 특히 \(F^d\)의 표준적인 내적이 존재하며 그로부터 노름과 거리 함수가 존재한다. 앞으로 아무 말 없이 \(\RR^d\)와 \(\CC^d\)라고 하면 이 표준적인 내적공간 / 노름공간 / 거리공간의 구조를 가지고 있는 것으로 생각할 것이며, 특히 위의 구조가 주어진 \(\RR^d\)를 \(d\)-차원 \textbf{유클리드공간(Euclidean space)}이라고 한다.

\(\RR^d\)와 \(\CC^d\)에 거리공간의 구조가 주어졌으므로 이 안의 수열을 생각할 수 있다.

\begin{defn}
    \(F\)-벡터공간 \(V\)에 주어진 두 노름 \(\norm{\cdot}_a\)와 \(\norm{\cdot}_b\)가 \textbf{립시츠 동등(Lipschitz equivalent)}하다는 것은 양수 \(m, M > 0\)이 존재하여
    \[
    m \norm{\cdot}_a \le \norm{\cdot}_b \le M \norm{\cdot}_a
    \]
    라는 것이다. \(\norm{\cdot}_a\)와 \(\norm{\cdot}_b\)가 립시츠 동등하면 \(\norm{\cdot}_a \sim \norm{\cdot}_b\)로 쓴다.
\end{defn}

\begin{prop}
    \(F\)-벡터공간 \(V\)의 노름들 사이의 립시츠 동등 관계는 동치 관계이다. 즉, \(V\)의 임의의 노름 \(\norm{\cdot}_a, \norm{\cdot}_b, \norm{\cdot}_c\)에 대하여 다음이 성립한다.
    \begin{enum}
        \item \(\norm{\cdot}_a \sim \norm{\cdot}_a\).
        \item \(\norm{\cdot}_a \sim \norm{\cdot}_b\)이면 \(\norm{\cdot}_b \sim \norm{\cdot}_a\).
        \item \(\norm{\cdot}_a \sim \norm{\cdot}_b\)이고 \(\norm{\cdot}_b \sim \norm{\cdot}_c\)이면 \(\norm{\cdot}_a \sim \norm{\cdot}_c\).
    \end{enum}
\end{prop}
\begin{proof}
    연습문제로 남긴다.
\end{proof}

\begin{thm} \label{3.3.8}
    \(F\)-벡터공간 \(V\)에 주어진 두 노름 \(\norm{\cdot}_a\)와 \(\norm{\cdot}_b\)가 립시츠 동등하다고 하자. 이때 \(V\) 안의 수열 \((x_n)_{n \in \NN}\)이 \(x \in V\)로 \(\norm{\cdot}_a\)에 대하여 수렴하는 것과 \(\norm{\cdot}_b\)에 대하여 수렴하는 것이 동치이다.
\end{thm}
\begin{proof}
    가정에 의해 \(\norm{\cdot}_b \le M \norm{\cdot}_a\)인 양수 \(M > 0\)이 존재한다. \(\norm{\cdot}_a\)에 대해 \(x_n \to x\)라 하면, 임의의 양수 \(\eps > 0\)에 대해
    \[
        n \ge N \implies \norm{x_n - x}_{a} < \frac{\eps}{M}
    \]
    인 자연수 \(N\)이 존재한다. 이제 \(n \ge N\)이면
    \[
        \norm{x_n - x}_b \le M \norm{x_n - x}_a < \eps
    \]
    이므로, \(\norm{\cdot}_b\)에 대해서도 \(x_n \to x\)이다. 반대 방향도 마찬가지로 증명된다.
\end{proof}

\begin{prop} \label{3.3.9}
    \(F^d\)에 주어진 노름 \(\norm{\cdot}_1, \norm{\cdot}_2, \norm{\cdot}_\infty\)는 모두 립시츠 동등하다.
\end{prop}
\begin{proof}
    먼저 \(\norm{\cdot}_1\)과 \(\norm{\cdot}_2\)가 립시츠 동등함을 보이자.
    \[
        v := (v_1, \ldots, v_d), \quad \tilde{v} := (\abs{v_1}, \ldots, \abs{v_d}), \quad w := (1, \ldots, 1) \in F^d
    \]
    로 두고 코시-슈바르츠 부등식을 적용하면
    \[
    \norm{v}_1 = \sum_{i=1}^d \abs{v_i} = \tilde{v} \cdot w \le \norm{\tilde{v}}_2 \norm{w}_2 = \sqrt{d} \norm{v}_2
    \]
    이고
    \[
        \norm{v}_2^2 = \sum_{i=1}^d \abs{v_i}^2 \le \sum_{1 \le i < j \le d} 2\abs{v_i}\abs{v_j} = \paren{\sum_{i=1}^d \abs{v_i}}^2 = \norm{v}_1^2
    \]
    이므로
    \[
        \frac{\norm{\cdot}_1}{\sqrt{d}} \le \norm{\cdot}_2 \cdot \norm{\cdot}_1
    \]
    이다. 다음으로 \(\norm{\cdot}_1\)과 \(\norm{\cdot}_\infty\)가 립시츠 동등한 것은 다음 부등식으로부터 나온다.
    \[
        \norm{\cdot}_\infty \le \norm{\cdot}_1 \le d \norm{\cdot}_\infty
    \]
\end{proof}

\begin{cor}
    표준적인 거리가 주어진 \(F^d\) 안의 수열 \((x_n)_{n \in \NN}\)을 생각하고 \(x_n = (x_n^1, \ldots, x_n^d)\)로 쓰자. 이때 다음이 성립한다. 
    \begin{enumerate}[label=(\alph*), leftmargin=2\parindent]
        \item \(p \in \{1, 2, \infty\}\)에 대하여, 수열 \((x_n)_{n \in \NN}\)이 \(x \in F^d\)로 표준적인 거리에 대하여 수렴할 필요충분조건은 수열 \((x_n)_{n \in \NN}\)이 \(x \in F^d\)로 \(\norm{\cdot}_p\)에 의해 주어진 거리에 대하여 수렴하는 것이다.
        \item 수열 \((x_n)_{n \in \NN}\)이 \(x = (x^1, \ldots, x^d) \in F^d\)로 수렴할 필요충분조건은 각 \(i = 1, \ldots, d\)에 대하여 수열 \((x_n^i)_{n \in \NN}\)이 \(F\)에서 \(x^i\)로 수렴하는 것이다.
    \end{enumerate}
\end{cor}
\begin{proof}
    (a)는 명제 \ref{3.3.8}\과 정리 \ref{3.3.9}의 직접적인 결과이다. 그리고 (b)에서 [각 \(i = 1, \ldots, d\)에 대하여 수열 \((x_n^i)_{n \in \NN}\)이 \(F\)에서 \(x^i\)로 수렴하는 것]은 \((x_n)_{n \in \NN}\)이 \(\norm{\cdot}_\infty\)에 대하여 \(x\)로 수렴하는 것과 동치이므로 (a)에 의해 증명된다.
\end{proof}

\section{Appendix: 노름으로부터의 내적 건설}

정리 \ref{thm 3.3.4}의 \((\Leftarrow)\) 방향을 증명한다. 즉, 모든 \(v, w \in V\)에 대해서
\[
    \norm{v+w}^2 + \norm{v-w}^2 = 2(\norm{v}^2 + \norm{w}^2)    
\]
이면
\[
    \inner{v}{w} =
        \begin{cases}
            \dfrac{1}{4}(\norm{v+w}^2 - \norm{v-w}^2) &\textif \ F = \RR\\
            \dfrac{1}{4}(\norm{v+w}^2 - \norm{v-w}^2 + i\norm{v+iw}^2 -i\norm{v-iw}^2) &\textif \ F = \CC
        \end{cases}  
\]
로 정의된 함수 \(\inner{\cdot}{\cdot} : V \times V \to \RR\)가 내적이 됨을 보일 것이다. 양의 정부호성, \(\inner{v}{v} = \norm{v}^2\)인 것을 증명하였기 때문에 성립한다. 켤레 대칭성 역시 성립한다. \(F = \RR\)이면 대칭성은 당연하고, \(F = \CC\)이면
\begin{align*}
    \inner{w}{v} &= \frac{1}{4}(\norm{w+v}^2 - \norm{w-v}^2 + i\norm{w+iv}^2 - i\norm{w-iv}^2)\\
    &= \frac{1}{4}(\norm{v+w}^2 - \norm{v-w}^2 + i\norm{-i(w+iv)}^2 - i\norm{i(w-iv)}^2)\\
    &= \frac{1}{4}(\norm{v+w}^2 - \norm{v-w}^2 + i\norm{v-iw}^2 - i\norm{v+iw}^2)\\
    &= \overline{\inner{v}{w}}
\end{align*}
이기 때문이다. 다음으로 \(u, v, w \in V\)에 대해
\begin{equation} \label{eq1}
    \inner{u + v}{w} = \inner{u}{w} + \inner{v}{w}
\end{equation}
인 것을 보이는데, 그 전에 다음을 관찰하자.
\begin{lem} \label{lem 3.4.1}
    노름공간 \(V\)의 노름 \(\norm{\cdot}\)이 평행사변형 법칙을 만족할 때, \(v, w \in V\)에 대하여 \(\norm{v+2w}^2 - \norm{v-2w}^2 = 2(\norm{v+w}^2 - \norm{v-w}^2)\).
\end{lem}
\begin{proof}
    \begin{align*}
        \norm{v+2w}^2 - \norm{v-2w}^2 &= (\norm{v+2w}^2 + \norm{v}^2) - (\norm{v-2w}^2 + \norm{v}^2)\\
        &= 2(\norm{v+w}^2 + \norm{w}^2) - 2(\norm{v-w}^2 + \norm{w}^2)\\
        &= 2(\norm{v+w}^2 - \norm{v-w}^2).
    \end{align*}
\end{proof}
\noindent\textbf{Case 1.} \(F = \RR\)\\
이 경우에 (\ref{eq1})\은
\[
\norm{u + v + w}^2 - \norm{u + v - w}^2 \stackrel{?}{=} \norm{u+w}^2 - \norm{u-w}^2 + \norm{v+w}^2 - \norm{v-w}^2
\]
과 동치이고, 이는 다시
\begin{equation} \label{eq2}
    2(\norm{u + v + w}^2 + \norm{u-w}^2) - 2(\norm{u + v - w}^2 + \norm{u+w}^2) \stackrel{?}{=} 2(\norm{v+w}^2 - \norm{v-w}^2)
\end{equation}
과 동치이다. (\ref{eq2})의 좌변에 평행사변형 법칙을 적용하면
\begin{align*}
    &2(\norm{u + v + w}^2 + \norm{u-w}^2) - 2(\norm{u + v - w}^2 + \norm{u+w}^2)\\
    =\,&  (\norm{2u + v}^2 + \norm{v + 2w}^2) - (\norm{2u + v}^2 + \norm{v - 2w}^2)\\
    =\,&  \norm{v + 2w}^2 - \norm{v - 2w}^2
\end{align*}
이므로, 보조정리 \ref{lem 3.4.1}에 의해 (\ref{eq2})\가 증명되어 (\ref{eq1})\이 증명된다.\\
\textbf{Case 2.} \(F = \CC\)\\
이 경우에 (\ref{eq1})\은
    \begin{align*}
        &\norm{u+v+w}^2 - \norm{u+v-w}^2 +i\norm{u+v+iw}^2 - i\norm{u+v-iw}^2\\
        \stackrel{?}{=}\,&  \norm{u+w}^2 - \norm{u-w}^2 + i\norm{u+iw}^2 -i\norm{u-iw}^2\\
        +\,&  \norm{v+w}^2 - \norm{v-w}^2 + i\norm{v+iw}^2 -i\norm{v-iw}^2
    \end{align*}
과 동치이고, 이는 다시
    \begin{align} \label{eq3}
        \begin{split}
            &2(\norm{u+v+w}^2 + \norm{u-w}^2) - 2(\norm{u+v-w}^2 + \norm{u+w}^2)\\
            +\,&  2i(\norm{u+v+iw}^2 + \norm{u-iw}^2) - 2i(\norm{u+v-iw}^2 + \norm{u+iw}^2)\\
            \stackrel{?}{=}  \,& 2(\norm{v+w}^2 - \norm{v-w}^2) + 2i(\norm{v+iw}^2 - \norm{v-iw}^2)
        \end{split}
    \end{align}
과 동치이다. (\ref{eq3})의 좌변에 평행사변형 법칙을 적용하면
    \begin{align*}
        &2(\norm{u+v+w}^2 + \norm{u-w}^2) - 2(\norm{u+v-w}^2 + \norm{u+w}^2)\\
        +\,& 2i(\norm{u+v+iw}^2 + \norm{u-iw}^2) - 2i(\norm{u+v-iw}^2 + \norm{u+iw}^2)\\
        =\,& (\norm{2u+v}^2 + \norm{v+2w}^2) - (\norm{2u+v}^2 + \norm{v-2w}^2)\\
        +\,& i(\norm{2u+v}^2 + \norm{v+2iw}^2) - i(\norm{2u+v}^2 + \norm{v-2iw}^2)\\
        =\,& (\norm{v+2w}^2 - \norm{v-2w}^2) + i(\norm{v+2iw}^2 - \norm{v-2iw}^2)
    \end{align*}
이므로, 보조정리 \ref{lem 3.4.1}에 의해 (\ref{eq2})\가 증명되어 (\ref{eq1})\이 증명된다.\\
이제 마지막으로, \(v, w \in V, c \in F\)에 대해
\begin{equation} \label{eq4}
    \inner{cv}{w} = c\inner{v}{w}
\end{equation}
임을 보이는 것만 남았다. 먼저 \(c = 0\)일 때 (\ref{eq4})\가 성립하는 것은 자명하다. 다음으로 어떤 \(n \in \NN\)에 대해 (\ref{eq4})\가 \(c = n\)에서 성립한다고 가정하면, (\ref{eq1})에 의해
\[
\inner{(n+1)v}{w} = \inner{nv}{w} + \inner{v}{w} = n\inner{v}{w} + \inner{v}{w} = (n+1)\inner{v}{w}    
\]
이므로 \(c = n+1\)일 때도 (\ref{eq4})\가 성립한다. 따라서 수학적 귀납법에 의해 (\ref{eq4})\는 모든 \(c \in \NN\)에 대해 성립한다. 그리고 \(n \in \ZZ_+\)에 대해
\[
    \inner{-nv}{w} + \inner{nv}{w} = \inner{-nv + nv}{w} = \inner{0}{w} = 0
\]
에서 \(\inner{(-n)v}{w} = -n\inner{v}{w}\)이므로, 모든 \(c \in \ZZ\)에 대해 (\ref{eq4})\가 성립한다. 유리수 \(r = m/n \in \QQ \ (m, n \in \ZZ, n \ne 0)\)에 대해
\[
    n\inner{rv}{w} = \inner{nrv}{w} = \inner{mv}{w} = m\inner{v}{w}
\]
에서 \(\inner{rv}{w} = r\inner{v}{w}\)이므로, 모든 \(c \in \QQ\)에 대해 (\ref{eq4})\가 성립한다. 그런데 \(v, w \in V\)를 고정했을 때 (\ref{eq4})의 양변은 \(c\)에 관한 연속함수이고, \(\QQ\)가 \(\RR\)에서 조밀하므로 모든 \(c \in \RR\)에 대해 (\ref{eq4})\가 성립한다(이 부분이 유일하게 아직 배우지 않은 부분이다). 이렇게 \(F = \RR\)일 때의 증명이 완료되었다. \(F = \CC\)일 때의 증명을 마무리하기 위해, \(c = i\)일 때만을 보이면 충분하다. 이는
    \begin{align*}
        \inner{iv}{w} &= \frac{1}{4}(\norm{iv+w}^2 - \norm{iv-w}^2 + i\norm{iv+iw}^2 -i\norm{iv-iw}^2)\\
        &= \frac{1}{4}(\norm{v-iw}^2 - \norm{v+iw}^2 + i\norm{v+w}^2 -i\norm{v-w}^2)\\
        &= \frac{i}{4}(\norm{v+w}^2 - \norm{v-w}^2 +i\norm{v+iw}^2 -i\norm{v-iw}^2)\\
        &= i\inner{v}{w}
    \end{align*}
에서 성립한다. 따라서 \(F = \CC\)일 때의 증명도 완료되었다.



\chapter{거리공간의 위상 (1)}

이 장에서는 해석개론에 필요한 정도의 위상수학을 공부한다. 실수에서 열린구간이나 닫힌구간의 개념을 일반화하여, 일반적인 거리공간에서의 열린집합과 닫힌집합을 정의하고 관련된 수열과 관련된 몇 가지 개념들을 살펴볼 것이다. 그리고 거리공간에서 아주 중요한 개념인 코시수열을 정의한다. 코시수열이 수렴하는 거리공간을 완비거리공간이라고 하는데, 유클리드공간이 이에 해당함을 증명할 것이다. 지금으로서는 이 모든 것들이 지적 유희 같겠지만\(\ldots\) 이후 이를 바탕으로 엄청난 일들을 하게 되니 잘 따라오기 바란다.

\section{열린집합과 닫힌집합}

\begin{defn}
    거리공간 \(X\)에 대하여 다음을 정의한다.
    \begin{enumerate} [label=(\alph*), leftmargin=2\parindent]
        \item \(x \in X, r > 0\)에 대하여 집합 \(B(x, r) := \{y \in X : d(x, y) < r\}\)를 \(x\)의 \textbf{열린 공(open ball)}이라고 한다.
        \item \(U \subseteq X\)가 \textbf{\(X\)의 열린집합(open set in \(X\))}이라는 것은, 임의의 \(x \in U\)에 대해 적당한 \(r > 0\)가 존재하여 \(B(x, r) \subseteq U\)인 것이다. \(x\)를 포함하는 열린집합을 \(x\)의 \textbf{근방(neighborhood)}이라고 한다.
        \item \(F \subseteq X\)가 \textbf{\(X\)의 닫힌집합(closed set in \(X\))}이라는 것은 \(X \setminus F\)가 \(X\)의 열린집합이라는 것이다.
    \end{enumerate}
\end{defn}
\begin{rem}
    \quad

    \begin{enumerate}[label=(\alph*), leftmargin=2\parindent]
        \item ``근방''이라는 단어를 열린 공의 의미로 쓰는 책들도 있는데, 이 노트에서는 일반적인 열린집합의 의미로 사용한다.
        \item 이후에 배우게 되겠지만 열린집합과 닫힌집합은 그 집합이 속한 공간에 따라 달라지는 개념이다. 따라서 \(X\)의 열린(닫힌)집합이라는 표현을 쓴다. 단, 혼동의 여지가 없을 때는 그냥 열린(닫힌)집합이라고도 한다.
        \item 닫힌집합의 정의는 ``열린집합이 아닌 집합''이 \textbf{아니다}. 열린집합이지만 닫힌집합이 아닌 집합, 닫힌집합이지만 열린집합이 아닌 집합, 열린집합이면서 닫힌집합인 집합, 열린집합도 닫힌집합도 아닌 집합 모두 가능하다.
    \end{enumerate}
\end{rem}

이 장이 끝날 때까지 \(X\)는 아무 말이 없으면 거리공간을 의미한다.

\begin{prop}
    \(x \in X, r >0\)에 대하여 \(B(x, r)\)은 \(X\)의 열린집합이다.
\end{prop}
\begin{proof}
    \(y \in B(x, r)\)에 대하여, \(r' = r - d(x, y) > 0\)으로 두면 임의의 \(z \in B(y, r')\)에 대해 \(d(x, z) \le d(x, y) + d(y, z) < d(x, y) + r' = r\)이므로 \(z \in B(x, r)\)이다. 따라서 \(B(y, r') \subseteq B(x, r)\)이므로 \(B(x, r)\)은 열린집합이다.
\end{proof}

\begin{thm}
    거리공간 \(X\)에서 다음이 성립한다.
    \begin{enumerate} [label=(\alph*), leftmargin=2\parindent]
        \item 공집합과 \(X\)는 열린집합이다.
        \item 열린집합들의 합집합은 열린집합이다.
        \item 열린집합들의 유한 교집합은 열린집합이다.
    \end{enumerate}
\end{thm}
\begin{proof}
    \quad

    \begin{enumerate} [label=(\alph*), leftmargin=2\parindent]
        \item 아무리 생각해도 쓸 말이 없다.
        \item 열린집합들의 집합족(collection) \(\{U_i\}_{i \in I}\)에 대하여 \(U := \bigcup_{i \in I} U_i\)라고 하자. 임의의 \(x \in U\)에 대해 \(x \in U_i\)인 \(i \in I\)가 존재하므로, \(r > 0\)이 존재하여 \(B(x, r) \subseteq U_i \subseteq U\)이다. 따라서 \(U\)는 열린집합이다.
        \item 유한 개의 열린집합 \(U_1, \ldots, U_n\)에 대하여 \(U := \bigcap_{i =1}^n U_i\)라고 하자. 임의의 \(x \in U\)에 대해, 각 \(r_i > 0\)이 존재하여 \(B(x, r_i) \subseteq U_i\)이다. 따라서 \(r := \min\{r_1, \ldots, r_n\}\)이라고 하면 각 \(i\)에 대해 \(B(x, r) \subseteq B(x, r_i) \subseteq U_i\)이므로 \(B(x, r) \subseteq U\)이다. 따라서 \(U\)는 열린집합이다.
    \end{enumerate}
\end{proof}

\begin{cor}
    거리공간 \(X\)에서 다음이 성립한다.
    \begin{enumerate} [label=(\alph*), leftmargin=2\parindent]
        \item 공집합과 \(X\)는 닫힌집합이다.
        \item 닫힌집합들의 교집합은 닫힌집합이다.
        \item 닫힌집합들의 유한 합집합은 닫힌집합이다.
    \end{enumerate}
\end{cor}

\begin{ex}
    \quad

    \begin{enumerate} [label=(\alph*), leftmargin=2\parindent]
        \item \(a, b \in \RR, a < b\)에 대하여
        \[
            (a, b), (a, \infty), (-\infty, b)    
        \]
        는 \(\RR\)의 열린집합이다. 한편 \([a, b], [a, \infty), (\infty, b]\)는 \(\RR\)의 닫힌집합이다.
        \item \(X\)의 한점 부분집합들은 모두 닫힌집합이다. \(x \in X\)에 대하여 \(X \setminus \{x\}\)가 열린집합임을 보이면 된다. 임의의 \(y \in X \setminus \{x\}\)에 대하여 \(d(x, y) > 0\)이고, \(B(x, d(x,y)) \subseteq X \setminus \{x\}\)이기 때문에 성립한다. 따라서 \(X\)의 유한 부분집합은 한점 부분집합들의 유한 합집합이므로 항상 닫힌집합이다.
        \item \(X\)가 이산 거리공간이면 임의의 \(x \in X\)에 대해 \(\{x\} = B(x, 1)\)이므로 한점 부분집합들은 모두 열린집합이고, \(X\)의 모든 부분집합은 한점 부분집합들의 합집합이므로 결국 \(X\)의 모든 부분집합이 열린집합이다. 이는 \(X\)의 모든 부분집합이 닫힌집합임을 의미한다.
    \end{enumerate}
\end{ex}

\begin{rem}
    열린집합들의 무한 교집합은 열린집합이 아닐 수 있고, 닫힌집합들의 무한 합집합은 닫힌집합이 아닐 수 있다. 예를 들어 \((-1/n, 1/n)\)은 \(\RR\)에서 열린집합이지만 \(\{0\} = \bigcap_{n=1}^\infty (-1/n, 1/n)\)은 열린집합이 아니다. 한편 \([0, 1-1/n]\)은 \(\RR\)에서 닫힌집합이지만 \([0, 1) = \bigcup_{n=1}^\infty [0, 1-1/n]\)은 닫힌집합이 아니다.
\end{rem}

\begin{thm}
    \(x, y \in X, x \ne y\)에 대하여, \(x\)와 \(y\)의 적당한 근방 \(U, V \subseteq X\)가 존재하여 \(U \cap V = \varnothing\)이다.
\end{thm}
\begin{proof}
    \(r := d(x, y) > 0\)에 대하여 \(U := B(x, 2/r), V := B(y, r/2)\)로 정의하면 \(U \cap V = \varnothing\)이다.
\end{proof}

\begin{thm}
    \(U \subseteq X\)가 열린집합일 필요충분조건은 \(U\)가 열린 공들의 합집합인 것이다.
\end{thm}
\begin{proof}
    열린집합들의 합집합은 열린집합이므로 \((\Leftarrow)\) 방향은 당연하다. 역으로 \(U\)가 열린집합이면 각 \(x \in U\)에 대해 \(B(x, r_x) \subseteq U\)인 \(r_x > 0\)이 존재한다. 우리의 주장은
    \[
    U = \bigcup_{x \in U} B(x, r_x)    
    \]
    라는 것인데, 이는 쉽게 확인할 수 있다.
\end{proof}

수열의 수렴을 열린집합을 이용하여 다시 정의할 수도 있다.

\begin{prop}
    \(X\) 안의 수열 \((x_n)_{n \in \NN}\)이 \(x \in X\)로 수렴할 필요충분조건은, \(x\)의 임의의 근방 \(U\)에 대하여 다음을 만족하는 자연수 \(N\)이 존재하는 것이다.
    \[
    n \ge N \implies x_n \in U    
    \]
\end{prop}
\begin{proof}
    연습문제로 남긴다.
\end{proof}

열린집합, 닫힌집합과 관련된 몇 가지 단어를 소개한다.

\begin{defn}
    \(E \subseteq X\)에 대하여, \(E\)에 포함되는 가장 큰 \(X\)의 열린집합을 \textbf{\(E\)의 내부(interior of \(E\))}라고 하고 \(\Int E\)로 쓴다. 한편 \(E\)를 포함하는 가장 작은 \(X\)의 닫힌집합을 \textbf{\(E\)의 닫힘(closure of \(E\))}라고 하고 \(\overline{E}\) 또는 \(\text{cl}_X E\)라고 쓴다.
\end{defn}

물론 \(A\)가 어떤 성질을 만족하는 가장 큰(작은) 집합이라는 것은, (a) \(A\)가 그 성질을 만족하면서 (b) 그 성질을 만족하는 집합이 항상 \(A\)에 포함된다(\(A\)를 포함한다)는 뜻이다. 경우에 따라 이런 \(A\)는 존재하지 않을 수도 있으므로\footnote{``\(\RR\)의 유한 부분집합들 중 가장 큰 집합'' 따위는 존재하지 않는다.}, 항상 존재성을 확인해야 한다.

\begin{prop}
    \(E \subseteq X\)에 대하여 \(\Int E\)와 \(\overline{E}\)는 항상 존재한다. 구체적으로, \(E\)에 포함되는 모든 \(X\)의 열린집합들의 집합을 \(\calU\)라고 하고 \(E\)를 포함하는 모든 \(X\)의 닫힌집합들의 집합을 \(\calF\)라고 하면
    \[
    \Int E = \bigcup_{U \in \calU} U, \quad \overline{E} = \bigcap_{F \in \calF} F    
    \]
    이다.
\end{prop}
\begin{proof}
    연습문제로 남긴다.
\end{proof}

\begin{prop}
    \(E \subseteq X\)에 대하여 다음이 성립한다.
    \begin{enumerate}[label=(\alph*), leftmargin=2\parindent]
        \item \(E\)가 열린집합일 필요충분조건은 \(E = \Int E\)인 것이다.
        \item \(\Int E = \{x \in X : \exists \ r > 0 \ \text{s.t.} \ B(x, r) \subseteq E\}\).
    \end{enumerate}
\end{prop}
\begin{proof}
    (a)는 내부의 정의에 의해 당연하다. (b)에서 \((\supseteq)\) 방향을 보이기 위해, 우변(RHS)이 \(E\)에 포함되는 열린집합임을 보이면 된다. 우변의 임의의 원소 \(x\)에 대하여 \(B(x, r) \subseteq E\)이도록 하는 \(r > 0\)이 존재한다. 따라서 \(\text{RHS} \subseteq E\)이다. 그런데 임의의 \(y \in B(x, r)\)에 대하여 \(B(y, r - d(x, y)) \subseteq B(x, r) \subseteq E\)이므로 \(y \in \text{RHS}\)이다. 따라서 \(B(x, r) \subseteq \text{RHS}\)이므로, 우변도 열린집합이다. 이제 등호를 보이기 위해 \(E\)에 포함되는 모든 열린집합이 우변에 포함됨을 보이면 된다. \(U \subseteq E\)가 \(X\)의 열린집합이라고 하자. 임의의 \(x \in U\)에 대하여 \(r > 0\)이 존재하여 \(B(x, r) \subseteq U \subseteq E\)이므로 \(x \in \text{RHS}\)이고, 따라서 \(U \subseteq \text{RHS}\)이다. 따라서 증명이 완료되었다.
\end{proof}

\begin{prop} \label{prop 4.1.10}
    \(E \subseteq X\)에 대하여 다음이 성립한다.
    \begin{enumerate}[label=(\alph*), leftmargin=2\parindent]
        \item \(E\)가 닫힌집합일 필요충분조건은 \(E = \overline{E}\)인 것이다.
        \item \(\overline{E} = \{x \in X : \forall \ r > 0, \ B(x, r) \cap E \ne \varnothing\}\).
    \end{enumerate}
\end{prop}
\begin{proof}
    (a)는 닫힘의 정의에 의해 당연하다. (b)에서 \((\subseteq)\) 방향을 보이기 위해, 우변(RHS)이 \(E\)를 포함하는 닫힌집합임을 보이면 된다. 임의의 \(x \in X \setminus \text{RHS}\)에 대하여, 우변의 정의에 의해 \(B(x, r) \cap E = \varnothing\)인 \(r > 0\)이 존재한다. 특히 \(x \notin E\)이므로, \(E \subseteq \text{RHS}\)가 성립한다. 한편 임의의 \(y \in B(x, r) \)에 대하여 \(B(y, r - d(x, y)) \subseteq B(x, r) \subseteq X \setminus E\)이므로 \(y \in X \setminus \text{RHS}\)이다. 따라서 \(B(x, r) \subseteq X \setminus \text{RHS}\)이므로, \(\text{RHS}\)는 닫힌집합이다. 이제 등호를 보이기 위해 \(E\)를 포함하는 모든 닫힌집합이 우변을 포함함을 보이면 된다. \(F \supseteq E\)가 \(X\)의 닫힌집합이라고 하자. 임의의 \(x \in X \setminus F\)에 대하여 \(r > 0\)이 존재하여 \(B(x, r) \subseteq X \setminus F \subseteq X \setminus E\)이므로 \(x \in X \setminus \text{RHS}\)이고, 따라서 \(\text{RHS} \subseteq F\)이다. 따라서 증명이 완료되었다.
\end{proof}

\begin{cor}
    \(E \subseteq X\)에 대하여 \(x \in \overline{E}\)일 필요충분조건은 \(x\)의 모든 근방이 \(E\)와 만나는\footnote{두 집합이 ``만난다(meet, intersect)''는 것은 교집합이 공집합이 아니라는 뜻이다.} 것이다.
\end{cor}
\begin{proof}
    연습문제로 남긴다.
\end{proof}

\begin{ex}
    \quad

    \begin{enumerate} [label=(\alph*), leftmargin=2\parindent]
        \item \(X\) 전체의 내부와 닫힘은 모두 \(X\) 자기 자신이다.
        \item \(X = \RR\)에서 구간 \((a, b)\)의 닫힘은 \([a, b]\)이고, 구간 \([a, b]\)의 내부는 \((a, b)\)이다.
        \item \(X = \RR\)에서 집합 \(\{1/n \in \RR : n \in \ZZ_+\}\)의 닫힘은 \(\{1/n \in \RR : n \in \ZZ_+\} \cup \{0\}\)이다.
        \item \(X\)가 노름공간일 때, \(B(x, r)\)의 닫힘은 \(\{y \in X : \norm{x - y} \le r\}\)이고 \(\{y \in X : \norm{x - y} \le r\}\)의 내부는 \(B(x, r)\)이다. 증명은 연습문제로 남긴다.\footnote{일반적인 거리공간에서는 성립하지 않을 수 있다. }
    \end{enumerate}
\end{ex}

열린집합과 닫힌집합의 개념은 전체 공간에 의존한다고 앞에서 언급했었다.

\begin{defn} \label{def 4.1.11}
    \(E \subseteq Y \subseteq X\)일 때 \(E\)가 \textbf{\(Y\)의 열린집합(open set in \(Y\))}라는 것은, \(X\)의 어떤 열린집합 \(U\)가 존재하여 \(E = Y \cap U\)라는 것이다. \(E\)가 \textbf{\(Y\)의 닫힌집합(closed set in \(Y\))}라는 것은, \(X\)의 어떤 열린집합 \(F\)가 존재하여 \(E = Y \cap F\)라는 것이다.
\end{defn}

\begin{prop}
    \(E \subseteq Y \subseteq X\)에 대하여 다음이 성립한다. 
    \begin{enumerate} [label=(\alph*), leftmargin=2\parindent]
        \item \(E\)가 \(Y\)의 열린집합일 필요충분조건은, 임의의 \(x \in E\)에 대하여 어떤 \(r > 0\)이 존재하여 \(B_Y(x, r) := B_X(x, r) \cap Y = \{y \in Y : d(x, y) < r\} \subseteq E\)인 것이다.
        \item \(E\)가 \(Y\)의 닫힌집합일 필요충분조건은 \(Y \setminus E\)가 \(Y\)의 열린집합인 것이다.
        \item \(\text{cl}_Y E = Y \cap \text{cl}_X E \).
        \item \(Y\)가 \(X\)의 열린집합이면, \(E\)가 \(Y\)의 열린집합일 필요충분조건은 \(E\)가 \(X\)의 열린집합인 것이다.
        \item \(Y\)가 \(X\)의 닫힌집합이면, \(E\)가 \(Y\)의 닫힌집합일 필요충분조건은 \(E\)가 \(X\)의 닫힌집합인 것이다.
    \end{enumerate}
\end{prop}
\begin{proof}
    연습문제로 남긴다.
\end{proof}
\begin{rem}
    \(Y \subseteq X\)일 때, \(Y\)도 \(X\)에서 물려받은 거리공간의 구조를 가진다. 따라서 \(Y\)를 하나의 거리공간으로 보았을 때 그 안에서 열린집합과 닫힌집합의 개념을 정의할 수 있으며, (a)와 (b)가 말해주는 것은 \(Y\)를 거리공간으로 보았을 때의 정의와 정의 \ref{def 4.1.11}의 정의가 일치한다는 것이다.
\end{rem}

\begin{ex}
    \quad

    \begin{enumerate} [label=(\alph*), leftmargin=2\parindent]
        \item \(X\)의 모든 부분집합은 자기 자신의 열린집합이자 닫힌집합이다.
        \item \(X = \RR\)에서 \((0, 1]\)은 열린집합이 아니지만 \([0, 1]\)에서는 열린집합이다.
        \item \(X = \RR\)에서 
    \end{enumerate}
\end{ex}

어떤 공간 \(X\)가 너무 클 때, \(X\)의 모든 원소를 ``근사''할 수 있는 적당한 부분집합을 찾는 것이 유용할 때가 많다.

\begin{defn}
    \(D \subseteq X\)가 \textbf{\(X\)에서 조밀(dense in \(X\))}하다는 것은 \(\overline{D} = X\)라는 것이다.
\end{defn}

\begin{ex}
    \quad

    \begin{enumerate} [label=(\alph*), leftmargin=2\parindent]
        \item \(\QQ\)는 \(\RR\)에서 조밀하다(유리수의 조밀성). 이를 보이기 위해, 임의의 \(x \in X, r > 0\)에 대하여 \(q \in B(x, r) = (x -r, x+r)\)인 \(q \in \QQ\)가 존재함을 보이면 된다. 이제 \(1/n < 2r\)인 양의 정수 \(n\)을 고르면, \(m/n \in B(x, r)\)이도록 하는 \(m \in \ZZ\)가 적어도 하나 존재한다. 따라서 \(\QQ\)는 \(\RR\)에서 조밀하다.
        \item \(\RR \setminus \QQ\)도 \(\RR\)에서 조밀하다(무리수의 조밀성). 증명은 연습문제로 남긴다.
        \item 같은 방법으로 \(\{(x_1, \ldots, x_d) \in \RR^d : x_i \in \QQ\}\)가 \(\RR^d\)에서 조밀한 집합임을 확인할 수 있다. 따라서 \(\RR^d\)는 \textbf{가산 조밀 부분집합(countable dense subset)}을 가진다고 할 수 있다.
    \end{enumerate}
\end{ex}



\section{극한점과 수열}

\begin{defn}
    \(E \subseteq X\)에 대하여 \(x \in X\)가 \textbf{\(E\)의 극한점(limit point of \(E\))}라는 것은 임의의 \(r > 0\)에 대해 \((B(x, r)\setminus \{x\}) \cap E \ne \varnothing\)이라는 것이다. \(E\)의 모든 극한점들의 집합을 \(E'\)이라고 쓰고 \textbf{유도집합(derived set)}이라고 한다. \(E \setminus E'\)의 원소를 \textbf{\(E\)의 고립점(isolated point of \(E\))}라고 한다.
\end{defn}

\begin{prop}
    \(E \subseteq X\)에 대하여 \(x \in E'\)일 필요충분조건은 \(x\)의 모든 근방이 \(E \setminus \{x\}\)와 만나는 것이다.
\end{prop}
\begin{proof}
    연습문제로 남긴다.
\end{proof}

\begin{ex}
    \quad

    \begin{enumerate} [label=(\alph*), leftmargin=2\parindent]
        \item \(X = \RR\)에서 집합 \(\{0\} \cup [1, 2]\)의 유도집합은 \([1, 2]\)이다.
        \item \(X = \RR\)에서 집합 \(\{1/n \in \RR : n \in \ZZ_+\}\)의 유도집합은 \(\{0\}\)이다.
        \item 집합 \(E \subseteq \RR\)에 대하여 \(E\)가 유계일 때, \(\sup E\)와 \(\inf E\)는 \(E\)의 극한점이다.
    \end{enumerate}
\end{ex}

\begin{prop}
    \(E \subseteq X\)에 대하여 \(x \in E'\)이면, \(x\)의 모든 근방은 \(E\)의 원소를 무한히 포함한다.
\end{prop}
\begin{proof}
    \(x\)의 어떤 근방 \(U\)가 \(E \setminus \{x\}\)의 원소를 유한 개만 포함한다고 가정하자. \(U \cap (E \setminus \{x\}) = \{x_1, \ldots, x_n\}\)으로 쓰면, \(U \cap (E \setminus \{x\})\)는 닫힌집합이므로 \(V := U \setminus (U \cap (E \setminus \{x\}))\)는 \(x\)를 포함하는 열린집합이다. 그런데 \(V \cap (E \setminus \{x\}) = \varnothing\)이므로 \(x \in E'\)에 모순이다.
\end{proof}

다음은 닫힘의 새로운 정의이다. 원래 정의와 다음 정의 둘 다 중요하다.

\begin{thm}
    \(E \subseteq X\)에 대하여 \(\overline{E} = E \cup E'\)이다.
\end{thm}
\begin{proof}
    \quad\\
    \((\subseteq)\) \(x \in \overline{E}\)를 택하고 \(x \notin E\)라고 가정하자. 명제 \ref{prop 4.1.10} (b)에 의해, 임의의 \(r > 0\)에 대해 \(B(x, r) \cap E \ne \varnothing\)인데, \(x \notin E\)이므로 \((B(x, r) \setminus \{x\}) \cap E = B(x, r) \cap E \ne \varnothing\)이다. 따라서 \(x \in E'\).\\
    \((\supseteq)\) \(E \subseteq \overline{E}\)인 것은 닫힘의 정의이고, 명제 \ref{prop 4.1.10} (b)에 의해 \(E' \subseteq \overline{E}\)이다.
\end{proof}

이로부터 닫힌집합의 필요충분조건 하나를 찾을 수 있다.

\begin{cor}
    \(F \subseteq X\)에 대하여 다음이 동치이다.
    \begin{enumerate}[label=(\alph*), leftmargin=2\parindent]
        \item \(F\)는 닫힌집합이다.
        \item \(F' \subseteq F\).
    \end{enumerate}
\end{cor}
\begin{proof}
    \(F\)가 닫힌집합인 것과 \(\overline{F} = F \cup F' = F\)가 동치이고, 이는 \(F' \subseteq F\)와 동치이다.
\end{proof}

\begin{cor} \label{cor 4.2.6}
    \(F \subseteq \RR\)가 닫힌집합이라고 하자. \(F\)가 위로 유계이면 \(\sup F \in F\)이고 \(F\)가 아래로 유계이면 \(\inf F \in F\)이다.
\end{cor}
\begin{proof}
    거의 당연하다.
\end{proof}

\begin{cor}
    \(E \subseteq X\)에 대하여 \(E'\)는 닫힌집합이다.
\end{cor}
\begin{proof}
    \((E')' \subseteq E'\)인 것만 보이면 된다. 즉 임의의 \(x \in (E')'\)와 \(r > 0\)에 대하여 \((B(x, r) \setminus \{x\}) \cap E \ne \varnothing\)임을 보이면 된다. \(x \in (E')'\)이므로, \(y \in (B(x, r) \setminus \{x\}) \cap E'\)가 존재한다. 이제 \(r' := \min\{d(x,y), r - d(x, y)\} > 0\)으로 두면, \(y \in E'\)이므로 \(z \in (B(y, r') \setminus \{y\}) \cap E\)가 존재한다. 우리의 주장은 \(z \in (B(x, r) \setminus \{x\}) \cap E\)라는 것이다. 즉 \(0 < d(x, z) < r\)임을 보이면 된다. 이는
    \[
    0 = d(x, y) - d(x, y) < d(x,y) - d(y,z) \le d(x, z) \le d(x,y) + d(y, z) < r
    \]
    에서 성립한다.
\end{proof}

거리공간에서 극한점과 닫힘은 수열과 깊은 관련이 있다.

\begin{thm}
    \(E \subseteq X\)에 대하여 다음이 성립한다.
    \begin{enumerate}[label=(\alph*), leftmargin=2\parindent]
        \item \(x \in \overline{E}\)일 필요충분조건은 \(x\)로 수렴하는 \(E\) 안의 수열이 존재하는 것이다.
        \item \(x \in E'\)일 필요충분조건은 \(x\)로 수렴하지만 절대 \(x\)가 되지 않는 \(E\) 안의 수열이 존재하는 것이다.
    \end{enumerate}
\end{thm}
\begin{proof}
    (b)의 증명은 (a)와 유사하므로 연습문제로 남긴다. (a)에서 \(x \in \overline{E}\)이면, 임의의 \(n \in \ZZ_+\)에 대해 \(x_n \in B(x, 1/n) \cap E\)가 존재한다. 이때 수열 \((x_n)_{n \in \NN}\)이 \(x\)로 수렴하는 것은 거의 당연하다. 역으로 \(x\)로 수렴하는 \(E\) 안의 수열 \((x_n)_{n \in \NN}\)가 주어지면, 임의의 양수 \(r > 0\)에 대해 자연수 \(N\)이 존재하여 \(x_N \in B(x, r) \cap E\)이므로 \(x \in \overline{E}\)이다.
\end{proof}

마지막으로 닫힌집합과 수열의 관계이다.

\begin{prop}
    \(F \subseteq X\)에 대하여 다음이 동치이다.
    \begin{enumerate}[label=(\alph*), leftmargin=2\parindent]
        \item \(F\)는 닫힌집합이다.
        \item \(F\) 안의 수열 \((x_n)_{n \in \NN}\)이 \(x \in X\)로 수렴하면 \(x \in F\)이다.
    \end{enumerate}
\end{prop}
\begin{proof}
    \quad\\
    \((\Rightarrow)\) \(F\) 안의 수열 \((x_n)_{n \in \NN}\)이 \(x \in X\)로 수렴하면 \(x \in \overline{F}\)인데, \(F\)가 닫힌집합이면 \(F = \overline{F}\)이므로 \(x \in F\)이다.\\
    \((\Leftarrow)\) \(x \in \overline{F}\)에 대하여 \(F\) 안의 수열 \((x_n)_{n \in \NN}\)이 존재하는데, 가정에 의해 \(x \in F\)이므로 \(\overline{F} \subseteq F\), 즉 \(F\)는 닫힌집합이다.
\end{proof}

\section{부분수열}

\begin{defn}
    함수 \(n : \NN \to \NN\)이 모든 \(k \in \NN\)에 대하여 \(n(k) < n(k+1)\)을 만족한다고 하자. 이때 수열 \((x_n)_{n \in \NN}\)에 대하여 수열 \((x_{n(k)})_{k \in \NN}\)를 \textbf{\((x_n)_{n \in \NN}\)의 부분수열(subsequence of \((x_n)_{n \in \NN}\))}이라고 한다.
\end{defn}

\begin{ex}
    \(X = \RR\)에서 \(x_n = (-1)^{n}\)일 때, \(n : k \mapsto 2k\)로 정의하면 부분수열 \((x_{n(k)})_{k \in \NN}\)은 항상 1을 값으로 가지는 상수수열이다.
\end{ex}

\begin{prop}
    \(X\) 안의 수열 \((x_n)_{n \in \NN}\)과 \(x \in X\)에 대하여 다음이 동치이다.
    \begin{enumerate} [label=(\alph*), leftmargin=2\parindent]
        \item \((x_n)_{n \in \NN}\)이 \(x\)로 수렴한다.
        \item \((x_n)_{n \in \NN}\)의 모든 부분수열이 \(x\)로 수렴한다.
    \end{enumerate}
\end{prop}
\begin{proof}
    모든 수열은 자기 자신의 부분수열이므로 \((\Leftarrow)\) 방향은 자명하다. 역으로 (a)를 가정하고, \((x_{n(k)})_{k \in \NN}\)가 \((x_n)_{n \in \NN}\)의 부분수열이라고 하자. 임의의 양수 \(\eps > 0\)이 주어졌을 때 \(n \ge N\)이면 \(d(x_n, x) < \eps\)이도록 하는 자연수 \(N\)이 존재한다. 그런데 \(n : k \mapsto n(k)\)가 자연수에서 증가하는 함수이므로 \(n(K) \ge N\)이 되도록 하는 자연수 \(K\)가 존재한다. 따라서 \(k \ge K\)이면 \(d(x_{n(k)}, x) < \eps\)이므로, \((x_{n(k)})_{k \in \NN}\)도 \(x\)로 수렴한다.
\end{proof}

\begin{prop} \label{prop 4.3.3}
    \(X\) 안의 수열 \((x_n)_{n \in \NN}\)에 대하여 \(x \in X\)가 집합 \(\{x_n \in X : n \in \NN\}\)의 극한점이면 \(x\)로 수렴하는 \((x_n)_{n \in \NN}\)의 부분수열이 존재한다.
\end{prop}
\begin{proof}
    \(S := \{x_n \in X : n \in \NN\}\)이라고 하자. 먼저 \(x_{n(0)} \ne x\)인 자연수 \(n(0)\)을 고른다(만약 이러한 \(n(0)\)이 존재하지 않으면 \((x_n)_{n \in \NN}\)은 항상 값이 \(x\)인 상수수열이므로 더 증명할 것이 없다). \(\delta := d(x_{n(0)}, x) > 0\)이라고 하자. 다음을 만족하는 증가함수 \(n : \NN \to \NN\)을 찾으려고 한다.
    \begin{equation} \label{eq5}
        d(x_{n(k)}, k) < 2^{1-k}\delta
    \end{equation}
    \(k = 0\)일 때 성립하는 것은 자명하다. 이제 (\ref{eq5})\를 만족하는 \(n(0) < \ldots < n(l-1)\)을 찾았다고 가정하자. \(x\)가 \(S\)의 극한점이므로 \(B(x, 2^{1-l} \delta) \cap S\)는 무한집합이다. 즉 \(x_{n(l)} \in B(x, 2^{1-l} \delta)\)인 자연수 \(n(l) > n(l-1)\)이 존재한다. 따라서 귀납적으로 (\ref{eq5})\를 만족하는 증가함수 \(n : \NN \to \NN\)을 잡을 수 있으며, 이때 부분수열 \((x_{n(k)})_{k \in \NN}\)가 \(x\)로 수렴하는 것은 (\ref{eq5})에 의해 당연하다. 
\end{proof}

\begin{thm}
    \(X\) 안의 수열 \((x_n)_{n \in \NN}\)에 대하여, \((x_n)_{n \in \NN}\)의 모든 수렴하는 부분수열들의 극한(subsequential limit)들의 집합은 닫힌집합이다.
\end{thm}
\begin{proof}
    \((x_n)_{n \in \NN}\)의 부분수열들의 극한들의 집합을 \(E\)라 할 때 \(E' \subseteq E\)인 것을 보이면 된다. \(x \in E'\)에 대하여, \(x_{n(0)} \ne x\)인 자연수 \(n(0)\)을 고른다(만약 이러한 \(n(0)\)이 존재하지 않으면 \((x_n)_{n \in \NN}\)은 항상 값이 \(x\)인 상수수열이고 \(E = \{x\}\)이므로 증명할 것이 없다). \(\delta := d(x_{n(0)}, x) > 0\)이라고 하자. 다음을 만족하는 증가함수 \(n : \NN \to \NN\)을 찾으려고 한다.
    \begin{equation} \label{eq6}
        d(x_{n(k)}, x) < 2^{1-k}\delta 
    \end{equation}
    \(k = 0\)일 때 성립하는 것은 자명하다. 이제 (\ref{eq6})\를 만족하는 \(n(0) < \ldots < n(l-1)\)을 찾았다고 가정하자. \(x \in E'\)이므로, \(d(x, y) < 2^{-l} \delta\)를 만족하는 \(y \in E\)가 존재한다. 또 \(E\)의 정의에 의해, \(d(x_{n(l)}, y) < 2^{-l} \delta\)를 만족하는 자연수 \(n(l) > n(l-1)\)을 잡을 수 있다. 그러면 \(d(x_{n(l)}, x) < 2^{1-l} \delta\)이다. 따라서 귀납적으로 (\ref{eq6})\를 만족하는 증가함수 \(n : \NN \to \NN\)을 잡을 수 있으며, 이때 부분수열 \((x_{n(k)})_{k \in \NN}\)가 \(x\)로 수렴하는 것은 (\ref{eq6})에 의해 당연하다. 따라서 \(x \in E\)이므로 \(E' \subseteq E\)이다.
\end{proof}

\section{코시수열}

코시수열은 해석학에서 정말 중요한 개념이다. 아직 이런 걸 왜 하는지 도저히 모르겠지만, 다음 장에서 코시수열의 첫 번째 응용인 급수를 만나게 될 것이다.

\begin{defn}
    거리공간 \(X\) 안의 수열 \((x_n)_{n \in \NN}\)이 \textbf{코시수열(Cauchy sequence)}이라는 것은, 임의의 \(\eps > 0\)에 대해 다음을 만족하는 자연수 \(N\)이 존재하는 것이다.
    \[
    m, n \ge N \implies d(x_m, x_n) < \eps    
    \]
\end{defn}

\begin{defn}
    \(E \subseteq X\)에 대하여 \textbf{\(E\)의 지름(diameter of \(E\))}을 \(\sup \{d(x, y) \in \RR : x, y \in E\}\)로 정의하고 \(\diam E\)로 쓴다.
\end{defn}

\begin{prop}
    \(X\) 안의 수열 \((x_n)_{n \in \NN}\)에 대하여 \(E_n := \{x_k \in X : k \ge n\}\)으로 정의하자. 그러면 \((x_n)_{n \in \NN}\)이 코시수열일 필요충분조건은 \(\lim_{n \to \infty} \diam E_n = 0\)인 것이다.
\end{prop}
\begin{proof}
    연습문제로 남긴다.
\end{proof}

\begin{prop}
    거리공간의 코시수열은 유계이다.
\end{prop}
\begin{proof}
    \((x_n)_{n \in \NN}\)이 \(X\)의 코시수열이라고 하자. \(m, n \ge N\)이면 \(d(x_m, x_n) < 1\)이도록 하는 자연수 \(N\)이 존재한다. 이때 \(\{x_n \in X : n < N\}\)은 유한집합이므로 유계이고, \(\{x_n \in X : n \ge N\}\)은 \(B(x_N, 1)\)의 부분집합이므로 유계이다. 따라서 \((x_n)_{n \in \NN}\)은 유계수열이다.
\end{proof}

\begin{prop} \label{prop 4.4.5}
    거리공간의 수렴하는 수열은 코시수열이다.
\end{prop}
\begin{proof}
    \((x_n)_{n \in \NN}\)이 \(x \in X\)로 수렴한다고 하자. 양수 \(\eps > 0\)에 대하여 \(n \ge N\)이면 \(d(x_n, x) < \eps/2\)이도록 하는 자연수 \(N\)이 존재한다. 그런데 \(m, n \ge N\)이면 \(d(x_m, x_n) \le d(x_m, x) + d(x_n, x) < \eps\)이므로 정의에 의해 \((x_n)_{n \in \NN}\)이 코시수열이다.
\end{proof}

\begin{ex}
    명제 \ref{prop 4.4.5}의 역은 성립하지 않는다. \(X = \RR\)에서 \(\sqrt{2}\)로 수렴하는 유리수열 \((x_n)_{n \in \NN}\)이 존재한다. 이때 \((x_n)_{n \in \NN}\)이 표준적인 거리가 주어진 거리공간 \(\QQ\)의 수열이라고 생각해도 된다. 이때 \((x_n)_{n \in \NN}\)은 코시수열이지만, \(\QQ\)에서 수렴하지 않는다.
\end{ex}

\begin{defn}
    거리공간 \(X\)가 \textbf{완비 거리공간(complete metric space)}이라는 것은 \(X\)의 임의의 코시수열이 \(X\) 안에서 수렴한다는 것이다.
\end{defn}

\begin{rem}
    두 가지 종류의 완비성이 있다. 첫 번째 완비성은 전순서집합에서 최소상계 성질을 의미하고, 두 번째 완비성은 거리공간에서 코시수열이 수렴함을 의미한다. 해석개론에서 전순서집합이면서 거리공간인 집합은 \(\RR\)(또는 그 부분집합들)뿐이며, 실제로 \(\RR\)는 두 가지 의미의 완비성을 모두 가진다. 다루는 대상이 \(\RR\)가 아닐 때는 이것이 전순서집합인지 거리공간인지가 문맥에 따라 명백할 것이기 때문에 혼동의 여지가 없다.
\end{rem}

\begin{defn}
    내적공간 \(V\)에 대하여 \(V\)의 내적에 의해 유도되는 거리공간의 구조가 완비 거리공간일 때 \(V\)를 \textbf{힐베르트공간(Hilbert space)}이라고 한다. 노름공간 \(V\)에 대하여 \(V\)의 노름에 의해 유도되는 거리공간의 구조가 완비 거리공간일 때 \(V\)를 \textbf{바나흐공간(Banach space)}이라고 한다.
\end{defn}

모든 힐베르트공간이 바나흐공간임은 자명하다. 일단 지금 우리의 목표는 유클리드공간이 힐베르트공간임을 증명하는 것이다. 다음 명제가 우리의 갈 길을 알려준다.

\begin{prop} \label{prop 4.4.7}
    거리공간에서 수렴하는 부분수열을 가지는 코시수열은 수렴한다.
\end{prop}
\begin{proof}
    수열 \((x_n)_{n \in \NN}\)이 \(X\)의 코시수열이라고 하고, 부분수열 \((x_{n(k)})_{k \in \NN}\)가 어떤 \(x \in X\)로 수렴한다고 하자. 우리의 주장은 \((x_n)_{n \in \NN}\)도 \(x\)로 수렴한다는 것이다. 임의의 양수 \(\eps > 0\)에 대해 다음을 만족하는 자연수 \(N, K\)가 존재한다.
    \[
    m, n \ge N \implies d(x_m, x_n) < \frac{\eps}{2}, \quad k \ge K \implies d(x_{n(k)}, x) < \frac{\eps}{2}, \quad n(K) \ge N
    \]
    이때 \(n \ge n(K)\)에 대해
    \[
    d(x_n, x) \le d(x_n, x_{n(K)}) + d(x_{n(K)}, x) < \frac{\eps}{2} + \frac{\eps}{2} = \eps
    \]
    이므로 \(x_n \to x\)이다.
\end{proof}

\section{유클리드공간의 완비성}

\begin{defn}
    집합 \(X\)의 부분집합들의 집합열 \((E_n)_{n \in \NN}\)이 각 \(n \in \NN\)에 대하여 \(E_{n} \supseteq E_{n+1}\)을 만족할 때, \((E_n)_{n \in \NN}\)을 \textbf{축소열(nested sequence)}이라고 한다.
\end{defn}

\begin{thm} [축소구간정리\footnote{Nested intervals theorem.}]
    \(\RR\)의 유계닫힌구간들의 축소열 \((I_n)_{n \in \NN}\)이 축소열일 때, \(\bigcap_n I_n \ne \varnothing\)이다.
\end{thm}

\begin{proof}
    각 \(n \in \NN\)에 대하여 \(I_n = [a_n, b_n]\)이라고 쓰면, \((a_n)_{n \in \NN}\)은 위로 유계인 단조증가수열이므로 \(x := \sup_{n \in \NN} a_n \in \RR\)로 수렴한다. 우리의 주장은 \(x \in \bigcap_n I_n\)이라는 것이다. 이는 모든 \(n \in \NN\)에 대해 \(a_n \le x \le b_n\)임을 보이는 것과 같은데, \(a_n \le x\)임은 명백하므로 \(x \le b_n\)인 것만 보이면 된다. 그런데 각 \(b_n\)이 \(\{a_n \in \RR : n \in \NN\}\)의 상계이므로 \(x \le b_n\)도 성립한다. 
\end{proof}

\begin{rem}
    \quad

    \begin{enumerate} [label=(\alph*), leftmargin=2\parindent]
        \item 축소구간정리의 증명에 단조수렴정리가 사용되었고, 단조수렴정리는 실수체의 완비성에 의존한다. 한편 축소구간정리를 가정하면 실수체의 완비성을 증명할 수 있다. 따라서 순서체에서 완비성, 단조수렴정리, 축소구간정리는 모두 동치이다.
        \item \(I_n\)들이 유계닫힌구간이라는 조건을 빼면 \(\bigcap_n I_n = \varnothing\)일 수 있다. \(I_n = (0, 1/{n+1})\) 또는 \(I_n = [n \infty)\)가 반례가 된다.
    \end{enumerate}
\end{rem}

\begin{cor} \label{cor 4.5.2}
    각 \(1 \le i \le d, n \in \NN\)에 대하여 \(I_n^i \subseteq \RR\)가 유계닫힌구간이고 \(I_n^i \supseteq I_{n+1}^i\)일 때, \(B_n := I_n^1 \times \ldots \times I_n^d\)라고 하자. 이때 \(\bigcap_n B_n \ne \varnothing\)이다.
\end{cor}
\begin{proof}
    각 좌표에 대하여 축소구간정리를 적용하면 된다.
\end{proof}

\begin{defn}
    \(I_1, \ldots, I_d \subseteq \RR\)가 유계닫힌구간들일 때, 집합 \(B = I_1 \times \ldots I_d \subseteq \RR^d\)를 \textbf{상자(\(d\)-cell)}라고 한다.
\end{defn}

\begin{thm} [볼차노-바이어슈트라스 정리\footnote{Bolzano-Weierstrass theorem.}]
    \(E \subseteq \RR^d\)가 유계집합이면 극한점을 가진다.
\end{thm}
\begin{proof}
    \(E\)가 유계집합이므로 어떤 상자 \(B_0 = I_0^1 \times \ldots \times I_0^d\) 안에 포함된다. 각 \(n\)에 대하여 다음을 만족하는 상자 \(B_n = I_n^1 \times \ldots \times I_n^d\)을 찾으려고 한다.
    \begin{equation} \label{eq7}
        B_n \supseteq B_{n+1}, \quad B_n \cap E \ \text{is infinite.}
    \end{equation}
    (\ref{eq7})\을 만족하도록 상자 \(B_0 \supseteq \ldots \supseteq B_{m-1}\)이 주어졌다고 하자. 이때 각 구간 \(I_{m-1}^i\)를 이등분하여 \(B_{m-1}\)을 \(2^d\)개의 sub-상자들의 합집합으로 나타낼 수 있다. \(B_{m-1} \cap E\)가 무한집합이었으므로, \(2^d\)개의 sub-상자들 중 적어도 하나는 \(E\)와의 교집합이 무한집합이며, 이러한 sub-상자를 \(B_m\)이라고 하자. 따라서 귀납적으로 (\ref{eq7})\을 만족하는 상자들의 열 \((B_n)_{n \in \NN}\)을 얻을 수 있다. 이제 따름정리 \ref{cor 4.5.2}에 의해 \(x \in \bigcap_n B_n\)이 존재한다. 우리의 주장은 \(x\)가 \(E\)의 극한점이라는 것이다.\\
    양수 \(r > 0\)이 주어졌다고 하자. \(M := \diam B_0 < \infty\)이고, \(\diam B_n = 2^{-n} M\)이므로 \(\diam B_n \to 0\)임은 쉽게 알 수 있다. 따라서 충분히 큰 \(N\)에 대하여 \(B_N \subseteq B(x, r)\)이다. 그런데 \(B_N \cap E\)가 무한집합이었으므로 \(B(x, r) \cap E\)도 무한집합이고, 따라서 \(B(x, r) \cap (E \setminus \{x\})\)는 공집합이 아니다. \(r > 0\)이 임의의 양수였으므로 \(x\)는 \(E\)의 극한점이다. 
\end{proof}

\begin{cor} \label{cor 4.5.5}
    \(\RR^d\)의 유계수열은 수렴하는 부분수열을 가진다.
\end{cor}
\begin{proof}
    볼차노-바이어슈트라스 정리와 명제 \ref{prop 4.3.3}의 직접적인 결과이다.
\end{proof}

다음 정리는 이 장의 최종 목표이다. 따름정리이지만 그 중요성을 고려하여 정리로 적는다.

\begin{thm} [유클리드공간의 완비성]
    \(\RR^d\)는 힐베르트공간이다.
\end{thm}
\begin{proof}
    따름정리 \ref{cor 4.5.5}와 명제 \ref{prop 4.4.7}의 직접적인 결과이다.
\end{proof}

\begin{rem}
    표준적인 거리가 주어진 \(\CC^{d}\)는 \(\RR^{2d}\)와 거리공간으로서 완전히 동일하다. 따라서 \(\CC^d\)도 힐베르트공간이다.
\end{rem}

유클리드공간의 완비성을 증명하는 데 가장 결정적으로 사용된 것은 볼차노-바이어슈트라스 정리였으며, 이는 실수체의 (전순서집합으로서의) 완비성에 의존했다는 점을 기억하기 바란다.

볼차노-바이어슈트라스 정리를 이용하면 한 가지 부수적인 수확을 얻을 수 있는데, 이는 상극한과 하극한의 또 다른 정의이다. 원래 정의만큼이나 중요하다.

\begin{thm}
    실수열 \((x_n)_{n \in \NN}\)의 부분수열들의 극한들(\(\pm \infty\)도 허용해서)의 집합을 \(E\)라고 할 때, \(E\)는 \(\RR \cup \{\pm \infty\}\)에서 최댓값과 최솟값을 가지며 각각 \((x_n)_{n \in \NN}\)의 상극한, 하극한과 같다.
\end{thm}
\begin{proof}
    최댓값에 대해서만 보이면 최솟값에 대한 증명은 거의 같으므로 생략한다.\\
    \textbf{Case 1.} \((x_n)_{n \in \NN}\)이 위로 유계가 아닌 경우\\
    이 경우에는 각 \(k \in \NN\)에 대해 \(x_{n(k)} > k\)이도록 하는 증가함수 \(n : \NN \to \NN\)을 잡을 수 있고 이때 \(x_{n(k)} \to \infty\)이다. 즉 \(\infty \in E\)이고, \(\limsup_n x_n = \infty = \max E\)임은 자명하다.\\
    \textbf{Case 2.} \((x_n)_{n \in \NN}\)이 위로 유계이고 \(\limsup_n x_n = - \infty\)인 경우.\\
    이때 \((x_n)_{n \in \NN}\)은 아래로 유계가 아니고, \textbf{Case 1}에서와 같은 방법으로 \(-\infty \in E\)임을 알 수 있다. 만약 \(E\)가 어떤 \(x \in \RR\)를 포함한다고 하자. 그러면 \(x\)로 수렴하는 부분수열 \((x_{n(k)})_{k \in \NN}\)가 존재하고, 어떤 \(K\)에 대해 \(k \ge K\)이면 \(x_{n(k)} > x -1 \)이게 할 수 있다. 그러므로 모든 \(n\)에 대하여 \(\sup_{l \ge n} x_l \ge x -1 \)이고 상극한의 정의에 의해 \(\limsup_n x_n \ge x-1\)이어야 하므로 모순이다. 즉 \(E = -\infty\)이므로 \(\limsup_n x_n = -\infty = \max E\)이다.\\
    \textbf{Case 3.} \((x_n)_{n \in \NN}\)이 위로 유계이고 \(\limsup_n x_n \in \RR\)인 경우.\\
    따름정리 \ref{cor 4.2.6}에 의해 \(E\)는 \(\RR\)에서 유계인 닫힌집합이므로 최댓값 \(\alpha \in \RR\)를 가진다. 우리의 주장은 \(\limsup_n x_n = \alpha\)라는 것이다. 먼저 \(\alpha \in E\)이므로 \(\alpha\)로 수렴하는 부분수열 \((x_{n(k)})_{k \in \NN}\)가 존재한다. 임의의 양수 \(\eps > 0\)이 주어졌다고 할 때, \(k \ge K\)이면 \(x_{n(k)} > \alpha - \eps\)이도록 하는 자연수 \(K\)가 존재하므로 \(x_n > \alpha - \eps\)인 자연수 \(n\)은 무한히 많이 존재한다. 한편 \(x_n \ge \alpha + \eps\)인 자연수 \(n\)이 무한히 많이 존재한다고 가정하면, 항상 \(\alpha + \eps\)보다 큰 값을 가지는 부분수열 \((x_{n'(k)})_{k \in \NN}\)을 잡을 수 있다. 그런데 \((x_n)_{n \in \NN}\)이 유계수열이므로 부분수열도 유계이고, 따라서 \((x_{n'(k)})_{k \in \NN}\)의 부분수열 중 어떤 \(x' \ge \alpha + \eps\)로 수렴하는 \((x_{n''(k)})_{k \in \NN}\)를 잡을 수 있다. 이는 원래 수열 \((x_n)_{n \in \NN}\)의 부분수열이고, \(\alpha + \eps\)로 수렴하므로 \(\alpha + \eps \in E\)이다. 이는 \(\alpha = \max E\)에 모순이므로, \(n \ge N\)이면 \(x_n < \alpha + \eps\)인 자연수 \(N\)이 존재한다. 따라서 명제 \ref{prop 2.4.6}에 의해 \(\alpha = \limsup_n x_n\)이다.
\end{proof}

아직 해결하지 않은 의문이 하나 있다. 장의 제목이 ``거리공간의 위상''인데, 도대체 위상이 무엇인가? 집합 \(X\)에 대하여 \(X\)의 \textbf{위상(topology)}은 \(X\)의 모든 열린집합들의 집합을 의미한다. 그리고 가장 일반적인 의미에서 열린집합은 거리의 개념도 요구하지 않는다. 다행히 해석개론에서는 거리가 없는 공간은 다루지 않을 것이다.



\chapter{급수}

미적분학에서 급수의 수렴 판정법을 배웠는데, 이 장에서는 예전에 배운 급수의 수렴 판정법을 제대로 증명하고 새로운 판정법들도 배울 것이다. 그리고 급수를 이용하여 실수 \(e\)를 정의할 것이다. 


\section{코시 판정법}

\begin{defn}
    노름공간 \(V\) 안의 수열 \((x_n)_{n \in \NN}\)에 대하여 무한합
    \[
    x_0 + x_1 + x_2 + \ldots    
    \]
    를 \textbf{급수(series)}라고 하고, \(\sum_{n=0}^\infty x_n\)으로 쓴다. 이때
    \[
    s_n = \sum_{k=0}^\infty x_k    
    \]
    로 정의된 수열 \((s_n)_{n \in \NN}\)을 \textbf{부분합(parital sum)}이라고 한다. 수열 \((s_n)_{n \in \NN}\)이 \(s \in V\)로 수렴하면 급수 \(\sum_{n=0}^\infty x_n\)이 \textbf{수렴(converge)}한다고 하고 \(\sum_n x_n = s\)로 쓴다. 만약 그렇지 않으면 급수 \(\sum_{n=0}^\infty x_n\)이 \textbf{발산(diverge)}한다고 한다.
\end{defn}

이 절이 끝날 때까지 아무 말이 없으면 수열 또는 급수는 \(\CC\) 안의 수열 또는 급수이다.

\begin{prop}
    수열 \((x_n)_{n \in \NN}\)과 \((y_n)_{n \in \NN}\)이 각각 \(x, y \in \CC\)로 수렴할 때 다음이 성립한다.
    \begin{enumerate} [label=(\alph*), leftmargin=2\parindent]
        \item
        \(x_n + y_n \to x + y\).
        \item
        \(c \in \CC\)에 대하여 \(cx_n \to cx\).
    \end{enumerate}
\end{prop}
\begin{proof}
    수열의 극한에 관한 성질로부터 나온다.
\end{proof}

\(\CC\)는 힐베르트공간이므로, 다음을 얻는다. 첫 번째로 등장하는 급수의 수렴 판정법이며, 코시 판정법(Cauchy criterion)이라고 한다.

\begin{thm}
    수열 \((x_n)_{n \in \NN}\)에 대하여 다음이 동치이다.
    \begin{enumerate} [label=(\alph*), leftmargin=2\parindent]
        \item
        급수 \(\sum_n x_n\)이 수렴한다.
        \item
        임의의 양수 \(\eps > 0\)에 대하여 \(m \ge n \ge N\)이면 다음을 만족하는 자연수 \(N\)이 존재한다.
        \[
        \abs{\sum_{k=n}^m x_k} < \eps    
        \]
    \end{enumerate}
\end{thm}
\begin{proof}
    (b)는 급수 \(\sum_n x_n\)의 부분합수열이 코시수열인 것과 동치이고, 이는 부분합수열이 수렴하는 것과 동치이다.
\end{proof}

이로부터 다음 따름정리를 얻는다.

\begin{cor} \label{cor 5.1.4}
    급수 \(\sum_n x_n\)이 수렴하면 \(\lim_n x_n = 0\)이다.
\end{cor}
\begin{proof}
    양수 \(\eps > 0\)이 주어졌다고 하고 \(m \ge n \ge N\)이면 \(\abs{\sum_{k=n}^m x_k} < \eps\)인 자연수 \(N\)이 존재한다. \(m = n\)으로 두면, \(\abs{x_n} < \eps\)이고 따라서 \(x_n \to 0\)이다.
\end{proof}

물론 \(x_n = 1/n\)과 같은 반례가 있으므로 따름정리 \ref{cor 5.1.4}의 역은 성립하지 않는다.

\section{교대급수판정법}

\begin{defn}
    각 항이 음이 아닌 실수열 \((x_n)_{n \in \NN}\)에 대하여 \(\sum_n (-1)^n x_n\)\footnote{물론 \(\sum_n (-1)^{n-1} x_n\)이어도 상관없다.}을 \textbf{교대급수(alternating series)}라고 한다.
\end{defn}

\begin{lem} \label{lem 5.2.2}
    각 항이 음이 아닌 실수열 \((x_n)_{n \in \NN}\)이 단조감소할 때, \(m \ge n\)에 대하여
    \[
    \abs{\sum_{k=n}^m (-1)^k x_k } \le x_n
    \]
    이다.
\end{lem}
\begin{proof}
    \[
        \abs{\sum_{k=n}^m (-1)^k x_k } = \abs{x_n - x_{n+1} + \ldots + (-1)^{m-n} x_m}
    \]
    이다. 먼저 \(m - n\)이 홀수이면,
    \begin{align*}
        \abs{\sum_{k=n}^m (-1)^k x_k } &= \abs{(x_n - x_{n+1}) + \ldots + (x_{m-1} - x_m)}\\
        &= (x_n - x_{n+1}) + \ldots + (x_{m-1} - x_m)\\
        &\le x_n - (x_{n+1} - x_{n+2}) - \ldots - (x_{m-2} - x_{m-1})\\
        &\le x_n
    \end{align*}
    이다. 다음으로 \(m - n\)이 0보다 큰 짝수이면(0이면 증명할 것이 없다),
    \begin{align*}
        \abs{\sum_{k=n}^m (-1)^k x_k } &= \abs{(x_n - x_{n+1}) + \ldots + (x_{m-2} - x_{m-1}) + x_{m}}\\
        &= (x_n - x_{n+1}) + \ldots + (x_{m-2} - x_{m-1}) + x_{m}\\
        &= x_n - (x_{n+1} - x_{n+2}) - \ldots - (x_{m-1} - x_m)\\
        &\le x_n
    \end{align*}
    이다.
\end{proof}

\begin{thm}[교대급수판정법]
    각 항이 음이 아닌 실수열 \((x_n)_{n \in \NN}\)이 0으로 단조감소하면서 수렴할 때, 교대급수 \(\sum_n (-1)^n x_n\)은 수렴한다.
\end{thm}
\begin{proof}
    \(\sum_n (-1)^n x_n\)의 부분합 \((s_n)_{n \in \NN}\)이 코시수열임을 보이는 것과 같다. 임의의 양수 \(\eps > 0\)에 대하여 \(x_N < \eps\)인 자연수 \(N\)이 존재하며, 보조정리 \ref{lem 5.2.2}에 의해 \(m > n > N\)이면
    \[
    \abs{s_m - s_n} = \abs{\sum_{k=n+1}^m (-1)^k x_k} \le x_{n+1} \le x_N M \eps
    \]
    이다. 따라서 \((s_n)_{n \in \NN}\)이 코시수열이다.
\end{proof}

\section{비교판정법}

비교판정법(comparsion test)은 실수체의 완비성의 직접적인 결과로, 하나의 급수로부터 다른 급수를 판정하는 기준이 된다.

\begin{thm}
    각 항이 음이 아닌 실급수 \(\sum_n x_n\)이 수렴할 필요충분조건은 부분합수열이 유계인 것이다.
\end{thm}
\begin{proof}
    각 \(x_n\)이 0 이상이므로, 부분합수열은 단조증가수열이다. 따라서 부분합이 수렴할 필요충분조건은 부분합이 유계인 것이다.
\end{proof}

\begin{notn}
    각 항이 음이 아닌 실수열 \((x_n)_{n \in \NN}\)에 대하여 급수 \(\sum_n x_n\)은 어떤 0이 아닌 실수로 수렴하거나, 무한대로 발산한다. 전자의 경우 \(\sum_n x_n < \infty\), 후자의 경우 \(\sum_n x_n = \infty\)로 쓰겠다.
\end{notn}

\begin{thm}[바이어슈트라스 판정법]
    바나흐공간 \(V\) 안의 수열 \((x_n)_{n \in \NN}\)에 대하여, 각 \(n\)에 대해 \(\norm{x_n} \le M_n\)인 양수 \(M_n > 0\)이 존재한다고 하자. 이때 급수 \(\sum_n M_n\)이 수렴하면 급수 \(\sum_n x_n\)도 수렴하고, \(\abs{\sum_n x_n} \le \sum_n M_n\)이다.
\end{thm}
\begin{proof}
    코시 판정법에 의해, 임의의 양수 \(\eps > 0\)에 대해
    \[
    m > n \ge N \implies \sum_{k=n+1}^m M_k < \eps
    \]
    인 자연수 \(N\)이 존재한다. 이때 \(m > n \ge N\)이면
    \[
    \norm{\sum_{k=n+1}^n x_k} \le \sum_{k=n+1}^n \norm{x_k} \le \sum_{k+1}^n M_k < \eps
    \]
    이므로, \(\sum_n x_n\)의 부분합도 코시수열이다.
\end{proof}

\begin{cor}[비교판정법]
    복소수열 \((x_n)_{n \in \NN}\)에 대하여, 각 \(n\)에 대해 \(\abs{x_n} \le M_n\)인 양수 \(M_n > 0\)이 존재한다고 하자. 이때 급수 \(\sum_n M_n\)이 수렴하면 급수 \(\sum_n x_n\)도 수렴하고, \(\abs{\sum_n x_n} \le \sum_n M_n\)이다.
\end{cor}

\begin{cor} \label{cor 5.3.5}
    수열 \((x_n)_{n \in \NN}\)에 대하여 \(\sum_n \abs{x_n} < \infty\)이면 급수 \(\sum_n x_n\)도 수렴하고, \(\abs{\sum_n x_n} \le \sum_n \abs{x_n}\)이다.
\end{cor}
\begin{proof}
    비교판정법에서 \(M_n = \abs{x_n}\)으로 두면 된다.
\end{proof}

\begin{defn}
    수열 \((x_n)_{n \in \NN}\)에 대하여 \(\sum_n \abs{x_n} < \infty\)이면 급수 \(\sum_n x_n\)이 \textbf{절대수렴(absolutely converge)}한다고 한다. 급수 \(\sum_n x_n\)이 수렴하지만 \(\sum_n \abs{x_n} < \infty\)이면 급수 \(\sum_n x_n\)이 \textbf{조건수렴(conditionally converge)}한다고 한다.
\end{defn}

따름정리 \ref{cor 5.3.5}\는 다시 말해 절대수렴하는 급수는 수렴한다는 것이다.

\begin{ex}
    \quad

    \begin{enum}
        \item 각 항이 음이 아닌 실급수 \(\sum_n x_n\)은 수렴하는 것과 절대수렴하는 것이 동치이다.
        \item \(\sum_n (-1)^n / n^2\)은 절대수렴한다.
        \item \(\sum_n (-1)^n / n\)은 조건수렴한다.
        \item \(\sum_n (-1)^n / \log n\)은 조건수렴한다.
    \end{enum}
\end{ex}

\section{코시 응집판정법}

코시 응집판정법(Cauchy condensation test)은 다음과 같다.

\begin{thm} [코시 응집판정법]
    각 항이 음이 아닌 단조감소수열 \((x_n)_{n \in \ZZ_+}\)에 대하여 다음이 동치이다.
    \begin{enum}
        \item \(\sum_{n=1}^\infty x_n < \infty\).
        \item \(\sum_{k=0}^\infty 2^k x_{2^k} < \infty\).
    \end{enum}
\end{thm}
\begin{proof}
    (a)와 (b)의 급수의 부분합을 \((s_n)_{n \in \ZZ_+}, (t_k)_{k \in \NN}\)이라 쓰자.\\
    \((\Rightarrow)\) \(\sum_{n=1}^\infty x_n < \infty\)라고 가정하면, 각 \(k \ge 0\)에 대하여
    \begin{align*}
        \sum_{k=0}^\infty 2^k x_{2^k} &\le x_1 + \sum_{k=0}^\infty 2^k x_{2^k}\\
        &= 2 \paren{x_1 + \sum_{k=1}^\infty 2^{k-1} x_{2^k}}\\
        &\le 2 (x_1 + x_2 + (x_3 + x_4) + \ldots + (x_{2^{k-1}+1} + \ldots + x_{2^k}))\\
        &= 2 \sum_{n=1}^{2^k} x_n < \infty
    \end{align*}
    이다.\\
    \((\Leftarrow)\) \(\sum_{k=0}^\infty 2^k x_{2^k} < \infty\)라고 가정하자. 각 \(n \ge 1\)에 대해 \(n \le 2^K - 1\)인 자연수 \(K\)를 잡을 수 있다. 이때
    \begin{align*}
        s_n &\le s_{2^K - 1}\\
        &= x_1 + (x_2 + x_3) + \ldots + (x_{2^{K-1}} + \ldots + x_{2^K - 1})\\
        &\le x_1 + 2x_2 + \ldots 2^{K-1}x_{2^{K-1}}\\
        &= t_{K-1} < \infty
    \end{align*}
    이다.
\end{proof}

이로부터 다음 \(p\)-판정법(\(p\)-test)을 증명할 수 있다. 코시 응집판정법을 이용하는 것이 이를 증명하는 가장 깔끔한 방법이라 생각된다.

\begin{thm}[\(p\)-판정법]
    실수 \(p\)에 대하여 \(\sum_n 1/n^p < \infty\)일 필요충분조건은 \(p > 1\)인 것이다.
\end{thm}
\begin{proof}
    \(1/n^p\)는 각 항이 음이 아닌 단조감소수열이므로, 코시 응집판정법을 사용할 수 있다.
    \begin{align*}
        \sum_n \frac{1}{n^p} < \infty &\iff \sum_k \frac{2^k}{(2^k)^p}\\
        &\iff \sum_n \frac{1}{(2^{p-1})^k}\\
        &\iff 2^{p-1} < 1\\
        &\iff p > 1
    \end{align*}
\end{proof}

\begin{ex}
    급수 \(\sum_{n=2}^\infty 1/n(\log n)^p\)가 수렴할 필요충분조건은 \(p > 1\)인 것이다. 왜냐하면
    \begin{align*}
        \sum_{n=2}^\infty \frac{1}{n(\log n)^p} < \infty &\iff \sum_{k=1}^\infty \frac{2^k}{2^k (\log 2^k)^p} < \infty\\
        &\iff \sum_{k=1}^\infty \frac{1}{(\log 2)^p k^p} < \infty\\
        &\iff p > 1
    \end{align*}
    이기 때문이다.
\end{ex}

\section{근판정법과 비판정법}

근판정법(root test)과 비판정법(ratio test)은 비교판정법의 결과로서 실제로 급수의 수렴 판정에 직접적으로 가장 많이 사용되는 판정법들이다.

\begin{thm} [근판정법]
    수열 \((x_n)_{n \in \NN}\)에 대하여 \(\alpha := \limsup_n \sqrt[n]{\abs{x_n}} \in [0, \infty]\)이라 하자. 이때 \(\alpha < 1\)이면 급수 \(\sum_n x_n\)은 절대수렴하고 \(\alpha > 1\)이면 급수 \(\sum_n x_n\)은 발산한다.
\end{thm}
\begin{proof}
    \(\alpha < 1\)이면 \(\beta \in (\alpha, 1)\)을 택할 수 있다. 상극한의 성질에 의해, [\(n \ge N\)이면 \(\sqrt[n]{\abs{x_n}} < \beta\)]인 자연수 \(N\)이 존재한다. 즉 \(n \ge N\)이면 \(\abs{x_n} < \beta^n\)이므로
    \[
    \sum_{n=N}^\infty \abs{x_n} < \sum_{n = N}^{\infty} \beta^n < \infty    
    \]
    이다. \(\alpha > 1\)이면 \(\lim_{n \to \infty} \abs{x_n} = 0\)일 수 없다. 따라서 \(\sum_n x_n\)이 발산한다.
\end{proof}

\begin{ex}
    \quad

    \begin{enum}
        \item 급수 \(\sum_{n=2}^\infty \frac{1}{(\log n)^n}\)는 수렴한다. 왜냐하면
        \[
        \limsup_n \abs{\frac{1}{(\log n)^n}}^{1/n} = \lim_n \frac{1}{\log n} = 0
        \]
        이기 때문이다.
        \item \(x_n = 1/n, \quad y_n = 1/n^2\)에 대하여,
        \[
        \limsup_n \abs{x_n}^{1/n} = \limsup_n \abs{y_n}^{1/n} = 1
        \]
        이다. 그런데 \(\sum_n x_n = \infty, \sum_n y_n < \infty\)이므로 근판정법은 \(\alpha = 1\)인 경우에는 아무런 정보를 주지 못한다.
    \end{enum}
\end{ex}

\begin{thm}[비판정법]
    수열 \((x_n)_{n \in \NN}\)에 대하여 다음이 성립한다(단 \(a_n \ne 0\)).
    \begin{enum}
        \item \(\limsup_n \abs{x_{n+1} / x_n} < 1\)이면 급수 \(\sum_n x_n\)은 절대수렴한다.
        \item \(n \ge N \implies \abs{x_{n+1} / x_n} > 1\)인 \(N\)이 존재하면 급수 \(\sum_n x_n\)은 발산한다.
    \end{enum}
\end{thm}
\begin{proof}
    \(\alpha := \limsup_n \abs{x_{n+1} / x_n} < 1\)에 대하여 \(\beta \in (\alpha, 1)\)을 택할 수 있다. 이때 상극한의 성질에 의해, [\(n \ge M\)이면 \(\abs{x_{n+1} / x_n} < \beta\)]인 \(M\)이 존재한다. 따라서 \(n \ge N\)이면 \(\abs{x_n} < \beta^{n-N}\abs{x_N}\)이므로
    \[
    \sum_{n=N}^\infty \abs{x_n} \le \sum_{n=N}^\infty \beta^{n-N}\abs{x_N} < \infty
    \]
    이다. 한편 어떤 \(N\)에 대해 \(n \ge N \implies \abs{x_{n+1} / x_n} > 1\)이면 \(\lim_{n \to \infty} \abs{x_n} = 0\)일 수 없다. 따라서 \(\sum_n x_n\)이 발산한다.
\end{proof}

\begin{ex}
    \quad

    \begin{enum}
        \item 급수 \(\sum_{n=0}^\infty 1/n!\)은 수렴한다. 왜냐하면
        \[
        \limsup_n \frac{1/(n+1)!}{1/n!} = \lim_n \frac{1}{n+1} = 0    
        \]
        이기 때문이다.
        \item \(\liminf_n \abs{x_{n+1} / x_n} > 1\)이면 (b)의 경우에 해당한다.
    \end{enum}
\end{ex}

다음 명제는 근판정법과 비판정법의 판정 범위를 비교해 준다.

\begin{prop}
    양수열 \((x_n)_{n \in \NN}\)에 대하여 다음이 성립한다.
    \[
    \liminf_n \frac{x_{n+1}}{x_n} \le \liminf_n \sqrt[n]{x_n} \le \limsup_n \sqrt[n]{x_n} \le \limsup_n \frac{x_{n+1}}{x_n}    
    \]
\end{prop}
\begin{proof}
    두 번째 부등호는 당연하고, 첫 번째 부등호는 세 번째 부등호와 비슷한 방법으로 보일 수 있다.\\
    \(\alpha := \limsup_n x_{n+1} / x_n\)이라 하자. \(\alpha = \infty\)이면 더 증명할 것이 없으므로, \(\alpha < \infty\)라고 가정하자. 양수 \(\eps > 0\)에 대해, 상극한의 정의에 의해 [\(n \ge N\)이면 \(x_{n+1} / x_n < \beta\)]인 자연수 \(N\)이 존재한다. 이때 \(n \ge N\)에 대해
    \[
    x_n < (\alpha + \eps)^{n-N} x_N
    \]
    이므로
    \[
    \sqrt[n]{x_n} < (\alpha + \eps)^{1 - N/n} \sqrt[n]{x_N}    
    \]
    이 성립한다. 양변에 \(n \to \infty\)을 취하면
    \[
    \limsup_n \sqrt[n]{x_n} \le \alpha + \eps
    \]
    를 얻는다. \(\eps > 0\)은 임의의 양수였으므로,
    \[
        \limsup_n \sqrt[n]{x_n} \le \alpha
    \]
    가 성립한다.
\end{proof}

\begin{cor} \label{cor 5.5.4}
    수열 \((x_n)_{n \in \NN}\)에 대하여 \(\limsup_n \abs{x_{n+1} / x_n} < 1\)이면 \(\limsup_n \sqrt[n]{x_n} < 1\)이다. 즉, 비판정법으로 \(\sum_n x_n\)이 수렴함을 보일 수 있으면 근판정법으로도 보일 수 있다.
\end{cor}

\begin{ex}
    따름정리 \ref{cor 5.5.4}의 역은 성립하지 않는다. 다음과 같이 정의된 급수
    \[
    \sum_n x_n = \frac{1}{2^1} + \frac{1}{2^0} + \frac{1}{2^3} + \frac{1}{2^2} + \frac{1}{2^5} + \frac{1}{2^4} + \ldots    
    \]
    를 생각하자. 이때
    \[
    \liminf_n \sqrt[n]{x_n} = \limsup_n \sqrt[n]{x_n} = \frac{1}{2}    
    \]
    이므로 근판정법으로는 \(\sum_n x_n < \infty\)임을 보일 수 있다. 그러나
    \[
        \liminf_n \frac{x_{n+1}}{x_n} = \frac{1}{8}, \quad \limsup_n \frac{x_{n+1}}{x_n} = 2
    \]
    이므로 비판정법으로는 급수의 수렴을 판정할 수 없다. 
\end{ex}

\section{\(e\)}

비판정법에 의해 급수 \(\sum_n 1/n!\)은 수렴한다.

\begin{defn}
    \(e := \sum_{n=0}^\infty 1/n!\).
\end{defn}

\(n! \ge 2^{n-1}\)임을 이용하면
\[
2 = \frac{1}{0!} + \frac{1}{1!} < e \le 1 + \sum_{n=1}^\infty \frac{1}{2^{n-1}} = 3
\]
임을 알 수 있다.

고등학교에서는 다음 정리를 주어진 사실로 받아들이고 이를 \(e\)의 정의로 사용하였지만, 이제는 이를 증명할 수 있다.

\begin{thm}
    \(\lim_n (1 + 1/n)^n = e\).
\end{thm}
\begin{proof}
    \(n \ge 1\)에 대하여
    \[
    s_n := \sum_{k=0}^n \frac{1}{k!}, \quad t_n := \paren{1 + \frac{1}{n}}^n    
    \]
    으로 두자. 먼저 이항정리에 의해
    \begin{align*}
        t_n &= \sum_{k=0}^n \frac{n(n-1)\ldots(n-k+1)}{k!} \frac{1}{n^k}\\
        &= \sum_{k=0}^n \frac{1}{k!} \paren{1 - \frac{1}{n}} \ldots \paren{1 - \frac{k-1}{n}}\\
        &\le \sum_{k=0}^n \frac{1}{k!} = s_n
    \end{align*}
    이므로 양변에 \(n \to \infty\)를 취하면
    \[
    \limsup_n t_n \le e
    \]
    를 얻는다. 한편 임의의 두 자연수 \(n \ge m\)에 대하여
    \[
    t_n = \sum_{k=0}^n \frac{1}{k!} \paren{1 - \frac{1}{n}} \ldots \paren{1 - \frac{k-1}{n}} \ge \sum_{k=0}^m \frac{1}{k!} \paren{1 - \frac{1}{n}} \ldots \paren{1 - \frac{k-1}{n}}
    \]
    이다. \(m\)을 고정하고 양변에 \(n \to \infty\)를 취하면
    \[
    \liminf_n t_n \ge \sum_{k=0}^m \frac{1}{k!} = t_m
    \]
    이다. 그런데 위 부등식은 모든 자연수 \(m\)에 대해 성립하므로, \(m \to \infty\)를 취했을 때
    \[
    \liminf_n t_n \ge e    
    \]
    를 얻는다. 따라서 \(\limsup_n t_n \le e \le \liminf_n t_n\)이므로 \(t_n \to e\)이다.
\end{proof}

다음 사실도 증명할 수 있다.

\begin{thm}
    \(e\)는 무리수이다.
\end{thm}
\begin{proof}
    \(\sum_n 1/n!\)의 부분합
    \[
    s_n := \sum_{k=0}^n \frac{1}{k!} 
    \]
    을 정의하자. 이때 \((s_n)_{n \in \NN}\)은 \(e\)로 단조증가하면서 수렴한다. 만약 \(e\)가 유리수라고 가정하면, 어떤 양의 정수 \(p, q \in \ZZ_+\)에 대해 \(e = p/q\)로 쓸 수 있다. \(n > m\)일 때
    \[
        n! > m! \cdot m^{n-m}
    \]
    인 것을 이용하면,
    \begin{align*}
        0 < e - s_q &= \sum_{k=1}^\infty \frac{1}{(q+k)!}\\
        &< \frac{1}{(q+1)!} \sum_{k=1}^\infty \frac{1}{(q+1)^{k-1}}\\
        &= \frac{1}{(q+1)!} \cdot \frac{q+1}{q} = \frac{1}{q! \cdot q}
    \end{align*}
    이므로
    \[
    0 < q! \cdot e - q! \cdot s_q < \frac{1}{q}    
    \]
    이다. 그런데
    \[
        q! \cdot e - q! \cdot s_q = (q-1)! \cdot p - \sum_{k=0}^q \frac{q!}{k!} \in \ZZ   
    \]
    이므로 모순이다. 따라서 \(e\)는 무리수이다.
\end{proof}

\section{재배열급수}

급수와 관련하여 하나의 중요한 문제는 합의 순서를 바꾸는 것이다. 유한합에 대해서는 덧셈의 교환법칙에 의해 합의 순서를 바꾸어도 그 합이 보존되는 것을 알고 있는데, 무한합에서도 그렇게 될지 알아보자.

\begin{defn}
    급수 \(\sum_n x_n\)과 전단사함수 \(r : \NN \to \NN\)에 대하여 급수 \(\sum_n x_{r(n)}\)을 \(\sum_n x_n\)의 \textbf{재배열급수(rearrangement series)}라고 한다.
\end{defn}

급수 \(\sum_n x_n\)의 각 항이 음이 아닌 실수일 때는 재배열급수가 원래 급수와 같은 값\footnote{\(\infty\)도 허용한다.}으로 수렴함을 증명할 수 있다.

\begin{thm}
    각 항이 음이 아닌 실급수 \(\sum_n x_n\)과 그 재배열급수 \(\sum_n x_{r(n)}\)에 대하여 \(\sum_n x_n = \sum_n x_{r(n)}\)이다.
\end{thm}
\begin{proof}
    \(s := \sum_n x_n, t := \sum_n x_{r(n)} \in [0, \infty]\)로 두자. 각 \(n\)에 대하여 \(M(n) := \max_{k \le n} r(k)\)로 두면,
    \[
    \sum_{k=0}^n x_{r(k)} \le \sum_{k=0}^{M(n)} x_{k} \le s
    \]
    이므로 \(n \to \infty\)의 극한을 취하면 \(t = \sum_n x_{r(n)} \le s\)를 얻는다. 한편 \(r^{-1} : \NN \to \NN\)도 전단사함수이므로 \(\sum_n x_n\)은 \(\sum_n x_{r(n)}\)의 재배열급수이고, 위에서와 같은 논리로 \(s \le t\)이다. 따라서 \(s = t\)이다.
\end{proof}

다음은 절대수렴하는 급수의 재배열급수이다.

\begin{thm}
    절대수렴하는 급수 \(\sum_n x_n\)과 그 재배열급수 \(\sum_n x_{r(n)}\)에 대하여 \(\sum_n x_n = \sum_n x_{r(n)}\)이다.
\end{thm}
\begin{proof} [Proof 1]
    \(\sum_n x_n\)과 \(\sum_n x_{r(n)}\)의 부분합을 각각 \((s_n)_{n \in \NN}, (t_n)_{n \in \NN}\)이라고 하자. \(\sum_n \abs{x_n}\)이 수렴하므로, 양수 \(\eps > 0\)에 대해 다음을 만족하는 자연수 \(N\)이 존재한다.
    \[
    m \ge n \ge N \implies \sum_{k=n}^m \abs{x_k} < \frac{\eps}{2} 
    \]
    또 \(r\)이 전단사이므로, 자연수 \(K \ge N\)을
    \[
    \{0, \ldots, N\} \subseteq \{r(0), \ldots, r(K)\}    
    \]
    를 만족하도록 잡을 수 있다. 이제 \(M(n) := \max_{k \le n} r(k)\)로 두면 \(n > K\)일 때 \(M(n) \ge r(K) \ge N\)이고,
    \begin{align*}
        \abs{t_n - s_n} &= \abs{\sum_{k=0}^n x_{r(n)} - \sum_{k=0}^n x_n}\\
        &= \abs{\paren{\sum_{i=0}^N x_i + \sum_{\substack{0 \le k \le n \\ r(k) > N}} x_{r(k)}} - \paren{\sum_{i=0}^N x_i + \sum_{i=N+1}^n x_i}}\\
        &= \abs{\sum_{\substack{0 \le k \le n \\ r(k) > N}} x_{r(k)} -  \sum_{i=N+1}^n x_i}\\
        &\le \sum_{\substack{0 \le k \le n \\ r(k) > N}} \abs{x_{r(k)} } + \sum_{i=N+1}^n \abs{x_i}\\
        &\le \sum_{i=N+1}^{M(n)} \abs{x_i} + \sum_{i=N+1}^n \abs{x_i}\\
        &< \frac{\eps}{2} + \frac{\eps}{2 } = \eps
    \end{align*}
    이다. 양변에 \(n \to \infty\)를 취하면
    \[
    \limsup_n \abs{t_n - s_n} \le \eps    
    \]
    이고, \(\eps  > 0\)이 임의의 양수였으므로
    \[
    \lim_n t_n = \lim_n s_n    
    \]
    을 얻는다.
\end{proof}
\begin{proof} [Proof 2]
    먼저 \(\sum_n x_n\)이 절대수렴하는 실급수라고 하자. 그러면 \(\sum_n (\abs{x_n} - x_n)\)은 각 항이 음이 아닌 실급수이고, 유한한 값으로 수렴한다. 따라서
    \[
        \sum_n \abs{x_{r(n)}} = \sum_n x_{n} < \infty, \quad \sum_n (\abs{x_{r(n)}} - x_{r(n)}) = \sum_n (\abs{x_n} - x_n) < \infty
    \]
    이므로,
    \begin{align*}
        \sum_n x_{r(n)} &= \sum_n (\abs{x_{r(n)}} - (\abs{x_{r(n)}} - x_{r(n)}))\\
        &= \sum_n \abs{x_{r(n)}} - \sum_n (\abs{x_{r(n)}} - x_{r(n)})\\
        &= \sum_n \abs{x_n} - \sum_n (\abs{x_{r(n)}} - x_{r(n)})\\
        &= \sum_n (\abs{x_n} - (\abs{x_{n}} - x_{r(n)}))\\
        &= \sum_n x_n
    \end{align*}
    이다. 이제 \(\sum_n x_n\)이 절대수렴하는 복소급수라고 하고 각 \(x_n\)을 \(x_n = x_n^1 + ix_n^2 \ (x_n^1, x_n^2 \in \RR)\)로 쓰자\footnote{거듭제곱과 혼동될 여지는 없을 거라 생각한다.}. 그러면
    \[
    \sum_n \abs{x_n^1} \le \sum_n \abs{x_n} < \infty, \quad  \sum_n \abs{x_n^2} \le \sum_n \abs{x_n} < \infty   
    \]
    이므로 \(\sum_n x_n^1\)과 \(\sum_n x_n^2\)모두 절대수렴하는 실급수이다. 따라서 위의 논의에 의해
    \begin{align*}
        \sum_n x_{r(n)} &= \sum_n (x_{r(n)}^1 + ix_{r(n)}^2)\\
        &= \sum_n x_{r(n)}^1 + i \sum_n x_{r(n)}^2 \\
        &= \sum_n x_n^1 + i\sum_n x_n^2\\
        &= \sum_n (x_n^1 + ix_n^2)\\
        &= \sum_n x_n 
    \end{align*}
    이다.
\end{proof}

조건수렴하는 급수의 재배열급수는 일반적으로 같은 값으로 수렴하지 않는다. 오히려 실급수에 관하여 그 정반대의 사실이 성립한다. 즉, 조건수렴하는 실급수가 주어졌을 때 임의의 실수 값으로 수렴하는 재배열을 찾을 수 있다. 먼저 표기법 하나를 도입한다.

\begin{notn}
    집합 \(X\)에서 정의된 실함수 \(f : X \to \RR\)에 대하여 함수 \(f^+, f^- : X \to \RR\)를 다음과 같이 정의한다.
    \[
    f^+ : x \mapsto \frac{\abs{f(x)} + f(x)}{2}, \quad f^- : x \mapsto \frac{\abs{f(x)} - f(x)}{2}  
    \]
\end{notn}

이때 \(f^+, f^-\)는 항상 0 이상의 값만을 가지고,
\[
f = f^+ - f^-, \quad \abs{f} = f^+ + f^-    
\]
가 성립한다.

실수열도 자연수에서 정의된 실함수이므로, 수열 \((x_n^+)_{n \in \NN}, (x_n^-)_{n \in \NN}\)을 위의 표기법에 따라 정의할 수 있을 것이다.

\begin{thm}[리만 재배열 정리\footnote{Riemann rearrangement theorem.}]
    실급수 \(\sum_n x_n\)이 조건수렴하고, \(- \infty \le \alpha \le \beta \le \infty\)이라 하자. 그러면 적당한 전단사함수 \(r: \NN \to \NN\)이 존재하여, 재배열급수 \(\sum_n x_{r(n)}\)의 부분합 \((t_n)_{n \in \NN}\)이 다음을 만족한다.
    \[
        \liminf_n t_n = \alpha, \quad \limsup_n t_n = \beta
    \]
\end{thm}
\begin{proof}
    급수 \(\sum_n x_n\)이 조건수렴하므로,
    \[
    \sum_n x_n^+ = \infty, \quad \sum_n x_n^- = \infty    
    \]
    이다. 따라서 수열 \((x_n^+)_{n \in \NN}\)과 \((x_n^-)_{n \in \NN}\)은 0 이상의 항을 각각 무한히 많이 포함한다. 따라서 각 수열에서 0보다 큰 항을 순서대로 나열한 수열을 각각 \((P_n)_{n \in \NN}\), \((Q_n)_{n \in \NN}\)이라고 정의할 수 있다. 이때
    \[
    \sum_n P_n = \infty, \quad \sum_n Q_n = \infty, \quad \lim_n P_n = 0, \quad \lim_n Q_n = 0
    \]
    임을 쉽게 확인할 수 있다.\\
    한편 증가수열 \(r, s : \NN \to \NN\)에 대하여,
    \begin{equation} \label{eq8}
        P_0 + \ldots + P_{r(0)} - Q_0 - \ldots - Q_{s(0)} + P_{r(0)+1} + \ldots + P_{r(1)} - Q_{s(0)+1} - \ldots - Q_{s(1)} + \ldots
    \end{equation}
    은 \(\sum_n x_n\)의 재배열급수이다. 이제 이 재배열급수가 주어진 조건을 만족하도록 증가수열 \(r, s\)를 잡으려고 한다. 먼저 실수열 \((\alpha_n)_{n \in \NN}, (\beta_n)_{n \in \NN}\)을 다음 조건을 만족하도록 잡을 수 있다.
    \[
    \alpha_n \to 0, \quad \beta_n \to 0, \quad \alpha_n < \beta_n
    \]
    이제 \(r(0)\)을
    \[
        P_0 + \ldots + P_{r(0)} > \beta_0    
    \]
    을 만족하는 최소의 자연수로 정의한다. \(\sum_n P_n = \infty\)인 것에서 이러한 자연수 \(r(0)\)의 존재성이 보장된다. 그리고 \(s(0)\)을
    \[
        P_0 + \ldots + P_{r(0)} - Q_0 - \ldots - Q_{s(0)} < \alpha_0
    \]
    을 만족하는 최소의 자연수로 정의한다. \(\alpha_0 < \beta_0\)이고 \(\sum_n Q_n = \infty\)인 것에서 이러한 자연수 \(s(0)\)의 존재성이 보장된다. 다음으로 \(r(1)\)을
    \[
        P_0 + \ldots + P_{r(0)} - Q_0 - \ldots - Q_{s(0)} + P_{r(0)+1} + \ldots + P_{r(1)} > \alpha_1
    \]
    을 만족하는 최소의 자연수로 정의한다. 이렇게 귀납적으로 증가수열 \(r, s : \NN \to \NN\)을 정의할 수 있다. 이제 급수 (\ref{eq8})의 부분합을 \((t_n)_{n \in \NN}\)이라 하고, 다음과 같이 정의된 \((t_n)_{n \in \NN}\)의 두 부분수열 \((t_{n_1(k)})_{k \in \NN}\)과 \((t_{n_2(k)})_{k \in \NN}\)을 생각하자.
    \begin{align*}
        t_{n_1(k)} &= P_0 + \ldots + P_{r(0)} - \ldots + P_{r(k-1)+1} + \ldots + P_{r(k)}\\
        t_{n_2(k)} &= P_0 + \ldots + P_{r(0)} - \ldots - Q_{s(k-1)+1} + \ldots + P_{s(k)}
    \end{align*}
    만약 두 부분수열이 수렴한다면, 이들의 극한이 \((t_n)_{n \in \NN}\)의 부분수열들의 극한들 중 각각 최댓값과 최솟값임은 쉽게 확인할 수 있다. 따라서
    \[
    \lim_k t_{n_1(k)}  = \beta, \quad \lim_k t_{n_2(k)}  = \alpha
    \]
    인 것만 보이면 된다. 그런데 각 \(k\)에 대해 \(r(k)\)와 \(s(k)\)의 정의에 의해,
    \[
        t_{n_1(k)} - P_{r(k)} \le \beta_k < t_{n_1(k)}, \quad t_{n_2(k)} + Q_{s(k)} \ge \alpha_k > t_{n_2(k)}
    \]
    가 성립한다. 따라서
    \[
        0 <  t_{n_1(k)} - \beta_k \le P_{r(k)}, \quad 0 < \alpha_k - t_{n_2(k)} \le Q_{s(k)}
    \]
    이다. 각 식에서 \(k \to \infty\)를 취하면,
    \[
    \lim_k t_{n_1(k)} = \lim_k \beta_k = \beta, \quad \lim_k t_{n_2(k)} = \lim_k \alpha_k = \alpha    
    \]
    를 얻는다.
\end{proof}

\begin{cor}
    조건수렴하는 실급수 \(\sum_n x_n\)과 \(- \infty \le \alpha \le \infty\)에 대하여, 적당한 전단사함수 \(r : \NN \to \NN\)이 존재하여 \(\sum_n x_{r(n)} = \alpha\)이다.
\end{cor}
\begin{proof}
    리만 재배열 정리에서 \(\beta = \alpha\)인 경우이다.
\end{proof}

\begin{ex}
    교대급수판정법에 의해 \(\sum_{n=1}^\infty (-1)^{n-1}/n\)은 수렴하고, \(\sum_n 1/n = \infty\)이므로 \(\sum_n (-1)^{n-1}/n\)은 조건수렴하는 실급수이다. 이 급수의 부분합을 \((s_n)_{n \in \NN}\)이라 하고 그 극한을 \(s \in \RR\)라 하자.\footnote{나중에 \(s = \log 2\)임을 알게 된다. 정리 \ref{12.3.8} 아래의 예시 참고.} 그리고 다음 재배열급수의 부분합을 \((t_n)_{n \in \NN}\)이라 하자.
    \[
        1 + \frac{1}{3} - \frac{1}{2} + \frac{1}{5} + \frac{1}{7} - \frac{1}{4} + \frac{1}{9} +\frac{1}{11} - \frac{1}{6} + \ldots    
    \]
    우리의 주장은
    \[
    \lim_n t_n = \frac{3}{2}s    
    \]
    라는 것이다. 먼저 다음을 관찰하자.
    \begin{align*}
        t_{3n} &= \sum_{k=1}^n \paren{\frac{1}{4k-3} + \frac{1}{4k-1} - \frac{1}{2k}}\\
        &= \sum_{k=1}^n \paren{\frac{1}{4k-3} - \frac{1}{4k-2} + \frac{1}{4k-1} - \frac{1}{4k}} + \sum_{k=1}^n \paren{\frac{1}{4k-2} - \frac{1}{4k}}\\
        &= s_{4n} + \frac{1}{2}s_{2n}
    \end{align*}
    따라서
    \[
    \lim_n t_{3n} = \lim_n \paren{s_{4n} + \frac{1}{2}s_{2n}} = \frac{3}{2}s    
    \]
    이다. 한편
    \[
    \abs{t_{3n+1} - t_{3n}} < \abs{t_{3n+2} - t_{3n}} = \frac{1}{4n+1} + \frac{1}{4n+3}    
    \]
    이므로 \(\lim_n t_{3n+1} = \lim_n t_{3n+2} = \lim_n t_{3n} = 3s/2\)이다. 이제 \(\lim_n t_n = 3s/2\)인 것은 극한의 정의에 의해 쉽게 확인할 수 있다.
\end{ex}

\section{디리클레 판정법과 아벨 판정법} \label{sec5.8}

디리클레 판정법(Dirichlet's test)과 아벨 판정법(Abel's test)은 두 수열의 곱의 형태로 나타나는 급수의 수렴을 판정하기 위해 사용된다. 먼저 다음을 관찰하자.

\begin{lem}[아벨 보조정리]
    수열 \((a_n)_{n \in \NN}, (b_n)_{n \in \NN}\)에 대하여 \(\sum_n a_n\)의 부분합을 \((A_n)_{n \in \NN}\)이라 할 때, 임의의 두 자연수 \(m > n\)에 대해
    \[
        \sum_{k=n+1}^m a_k b_k = A_m b_{m+1} - A_n b_{n+1} - \sum_{k=n+1}^m A_k (b_{k+1} - b_k)
    \]
    이다.
\end{lem}
\begin{proof}
    표기의 편리함을 위해 \(A_{-1} := 0\)으로 정의하면 임의의 \(n\)에 대해 다음이 성립한다.
    \begin{align*}
        \sum_{k=0}^n a_k b_k &= \sum_{k=0}^n (A_{k} - A_{k-1}) b_k\\
        &= \sum_{k=0}^n A_k b_k - \sum_{k=0}^n A_{k-1} b_k\\
        &= \sum_{k=0}^n A_k b_k - \sum_{k=0}^n A_k b_{k+1} + A_n b_{n+1}\\
        &= A_n b_{n+1} - \sum_{k=0}^n A_k (b_{k+1} - b_k)
    \end{align*}
    따라서
    \[
    \sum_{k=0}^m a_k b_k - \sum_{k=0}^n a_k b_k    
    \]
    을 계산하면 원하는 등식을 얻는다.
\end{proof}

\begin{thm}[디리클레 판정법] \label{5.8.2}
    수열 \((a_n)_{n \in \NN}\)과 실수열 \((b_n)_{n \in \NN}\)이 다음을 만족한다고 하자.
    \begin{enum}
        \item \(\sum_n a_n\)의 부분합이 유계이다.
        \item \(b_n \to 0\).
        \item \(\sum_n \abs{b_{n+1} - b_n} < \infty\).
    \end{enum}
    이때 \(\sum_n a_n b_n\)은 수렴한다.
\end{thm}
\begin{proof}
    \(\sum_n a_n\)과 \(\sum_n a_n b_n\)의 부분합을 각각 \((A_n)_{n \in \NN}, (s_n)_{n \in \NN}\)이라 쓰자. (a)에 의해, [모든 \(n \in \NN\)에 대해 \(\abs{a_n} < M\)]인 양수 \(M\)이 존재한다. 임의의 양수 \(\eps > 0\)이 주어졌을 때, (b)와 (c)에 의해 다음을 만족하는 자연수 \(N\)이 존재한다.
    \[
    m > n \ge N \implies \abs{b_n} < \frac{\eps}{3M}, \sum_{k=n+1}^m \abs{b_{k+1} - b_k} < \frac{\eps}{3M}   
    \]
    이제 \(m > n \ge N\)에 대하여 아벨 보조정리를 쓰면
    \begin{align*}
        \abs{s_m - s_n} &\le \abs{A_m b_{m+1}} + \abs{A_n b_{n+1}} + \sum_{k=n+1}^m \abs{A_k (b_{k+1} - b_k)}\\
        &< M \cdot \frac{\eps}{3M} + M \cdot \frac{\eps}{3M} + M \cdot \sum_{k=n+1}^m \abs{b_{k+1} - b_k}\\
        &< \frac{\eps}{3} + \frac{\eps}{3} +\frac{\eps}{3} = \eps
    \end{align*}
    이므로, \((s_n)_{n \in \NN}\)이 코시수열이고 따라서 \(\sum_n a_n b_n\)이 수렴한다.
\end{proof}

\begin{cor} \label{5.8.3}
    수열 \((a_n)_{n \in \NN}\)과 실수열 \((b_n)_{n \in \NN}\)이 다음을 만족한다고 하자.
    \begin{enum}
        \item \(\sum_n a_n\)의 부분합이 유계이다.
        \item \((b_n)_{n \in \NN}\)은 0으로 수렴하는 단조감소수열이다.
    \end{enum}
    이때 \(\sum_n a_n b_n\)은 수렴한다.
\end{cor}

\begin{proof}
    \(\sum_n \abs{b_{n+1} - b_n} = \sum_n (b_n - b_{n+1}) = b_1 < \infty\).
\end{proof}

\begin{ex}
    \quad

    \begin{enum}
        \item 따름정리 \ref{5.8.3}에서 \(a_n = (-1)^n\)인 경우가 교대급수판정법이다.
        \item \(z \in \CC\)가 \(\abs{z} = 1, z \ne 1\)이고 실수열 \((b_n)_{n \in \NN}\)이 0으로 수렴하는 단조수열이라고 하자. 이때
        \[
            \abs{\sum_{k=0}^n z^k} = \abs{\frac{1 - z^{n+1}}{1 - z}} \le \frac{2}{\abs{1 - z}}
        \]
        이므로 \(\sum_n z^n\)의 부분합이 유계이다. 따라서 디리클레 판정법에 의해 급수 \(\sum_n b_n z^n\)이 수렴한다.
    \end{enum}
\end{ex}

\begin{thm}[아벨 판정법] 
    수열 \((a_n)_{n \in \NN}\)과 실수열 \((b_n)_{n \in \NN}\)이 다음을 만족한다고 하자.
    \begin{enum}
        \item \(\sum_n a_n\)이 수렴한다. 
        \item \((b_n)_{n \in \NN}\)은 유계인 단조감소수열이다.
    \end{enum}
    이때 \(\sum_n a_n b_n\)은 수렴한다.
\end{thm}
\begin{proof}
    \(\sum_n a_n\)과 \(\sum_n a_n b_n\)의 부분합을 각각 \((A_n)_{n \in \NN}, (s_n)_{n \in \NN}\)이라 쓰자. (b)에 의해 [모든 \(n \in \NN\)에 대해 \(\abs{b_n} < M\)]인 양수 \(M\)이 존재한다. 임의의 두 자연수 \(m > n\)에 대해
    \begin{align*}
        s_m - s_n &= A_m b_{m+1} - A_n b_{n+1} - \sum_{k=n+1}^m A_k (b_{k+1} - b_k)\\
        &= A_m b_{n+1} + \sum_{k=n+1}^m A_m (b_{k+1} - b_k) - A_n b_{n+1} - \sum_{k=n+1}^m A_k (b_{k+1} - b_k)\\
        &= (A_m - A_n) b_{n+1} + \sum_{k=n+1}^m (A_m - A_k)(b_{k+1}- b_k)
    \end{align*}
    이다. 이제 \((A_n)_{n \in \NN}\)이 수렴하고 \(\sum_n \abs{b_{n+1} - b_n} < \infty\)이므로 임의의 \(\eps > 0\)에 대해 다음을 만족하는 자연수 \(N\)이 존재한다.
    \[
    m > n \ge N \implies \abs{A_m - A_n} < \frac{\eps}{3M}
    \]
    따라서 \(m > n \ge N\)이면
    \begin{align*}
        \abs{s_m - s_n} &< \frac{\eps}{3M} \cdot M + \frac{\eps}{3M} \cdot \sum_{k=n+1}^m (b_k - b_{k+1})\\
        &< \frac{\eps}{3} + \frac{\eps}{3M} \cdot (b_{n+1} - b_{m+1}) \le \eps
    \end{align*}
    이므로 \((s_n)_{n \in \NN}\)은 코시수열이다. 따라서 \(\sum_n a_n b_n\)이 수렴한다.
\end{proof}

\section{코시곱} \label{sec5.9}

두 급수 \(\sum_n a_n, \sum_n b_n\)의 곱은 다음과 같이 정의하는 것이 자연스럽다.

\begin{defn}
    두 급수 \(\sum_n a_n, \sum_n b_n\)에 대하여 수열 \((c_n)_{n \in \NN}\)을
    \[
        c_n := \sum_{k=0}^n a_k b_{n-k}
    \]
    로 정의하자. 이때 급수 \(\sum_n c_n\)을 두 급수 \(\sum_n a_n, \sum_n b_n\)의 \textbf{코시곱(Cauchy product)}이라고 한다. 
\end{defn}

이 정의가 합리적인 정의이려면, 다음 두 질문에 대한 답이 긍정적이어야 한다: \(\sum_n a_n = A, \sum_n b_n = B\)로 수렴할 때

\begin{enum}
    \item \(\sum_n c_n\)이 수렴하는가?
    \item 수렴한다면 \(\sum_n c_n = AB\)인가?
\end{enum}

조건 하나만을 추가하면 두 질문에 대한 답이 긍정적이다. 그것은 \(\sum_n a_n, \sum_n b_n\) 중 적어도 하나가 절대수렴하는 것이다.

\begin{thm} \label{5.9.2}
    \(\sum_n a_n = A, \sum_n b_n = B\)이고 \(\sum_n a_n\)이 절대수렴한다고 하자. 이때 두 급수 \(\sum_n a_n, \sum_n b_n\)의 코시곱 \(\sum_n c_n\)은 수렴하고 \(\sum_n c_n = AB\)이다.
\end{thm}
\begin{proof}
    \(\sum_n a_n, \sum_n b_n, \sum_n c_n\)의 부분합을 각각 \((A_n)_{n \in \NN}, (B_n)_{n \in \NN}, (C_n)_{n \in \NN}\)이라 하자. \(A_n B \to AB\)임은 명백하므로, \(C_n - A_n B \to 0\)임을 보이면 충분하다.\\
    \(\beta_n := B_n - B\)로 두고 다음을 관찰하자.
    \begin{align*}
        C_n &= \sum_{k=0}^n \sum_{l=0}^k a_l b_{k-l}\\
        &= \sum_{k=0}^n a_k \sum_{l=0}^{n-k} b_l\\
        &= \sum_{k=0}^n a_k B_{n-k}\\
        &= \sum_{k=0}^n a_k (B + \beta_{n-k})\\
        &= A_n B + \sum_{k=0}^n a_k \beta_{n-k}
    \end{align*}
    따라서
    \[
    \gamma_n := \sum_{k=0}^n \beta_k a_{n-k} \to 0  
    \]
    인 것만 보이면 된다. 이제
    \[
    \sum_n \abs{a_n} =: M < \infty    
    \]
    로 두고, 양수 \(\eps > 0\)이 주어졌다고 하자. \(\beta_n = B_n - B \to 0\)이므로, [\(n \ge N\)이면 \(\abs{\beta_n} < \eps\)]인 자연수 \(N\)이 존재한다. \(n \ge N\)에 대하여
    \begin{align*}
        \abs{\gamma_n} &= \abs{\sum_{k=0}^n \beta_k a_{n-k}}\\
        &\le \abs{\sum_{k=0}^N \beta_k a_{n-k}} + \abs{\sum_{k=N+1}^n \beta_k a_{n-k}}\\
        &\le \abs{\sum_{k=0}^N \beta_k a_{n-k}} + \sum_{k=N+1}^n \abs{\beta_k a_{n-k}}\\
        &\le \abs{\sum_{k=0}^N \beta_k a_{n-k}} + \eps \sum_{k=N+1}^n \abs{a_{n-k}}\\
        &\le \abs{\sum_{k=0}^N \beta_k a_{n-k}} + M\eps
    \end{align*}
    이므로, \(N\)을 고정한 채로 \(n \to \infty\)를 취하면
    \[
    \limsup_n \abs{\gamma_n} \le M\eps    
    \]
    을 얻는다. \(\eps > 0\)이 임의의 양수였으므로, \(\abs{\gamma_n} \to 0\)을 얻는다.
\end{proof}

두 급수 \(\sum_n a_n, \sum_n b_n\)가 모두 조건수렴하면 코시곱이 수렴하지 않을 수도 있다.

\begin{ex}
    \[
    a_n = b_n = \frac{(-1)^n}{\sqrt{n+1}}    
    \]
    으로 정의했을 때, 교대급수판정법에 의해 급수 \(\sum_n a_n, \sum_n b_n\)은 수렴한다. 이때 두 급수의 코시곱을 \(\sum_n c_n\)으로 쓰면
    \[
    c_n =  \sum_{k=0}^n a_k b_{n-k} = (-1)^n \sum_{k=0}^n \frac{1}{\sqrt{(k+1)(n-k+1)}}   
    \]
    로 정의되는데, 산술-기하평균 부등식에 의해
    \begin{align*}
        \abs{c_n} &= \sum_{k=0}^n \frac{1}{\sqrt{(k+1)(n-k+1)}}\\
        &\ge \sum_{k=0}^n \frac{2}{n+2} = \frac{2(n+1)}{n+2}
    \end{align*}
    이므로 \(c_n \to 0\)이 아니다. 따라서 \(\sum_n c_n\)은 발산한다.
\end{ex}

여기서 다음과 같은 의문을 던질 수 있다: \(\sum_n c_n\)이 \(AB\)가 아닌 다른 값으로 수렴할 수 있는가? 이러한 경우는 불가능함이 알려져 있다. 증명은 정리 \ref{12.3.8}(아벨 정리)를 증명한 이후로 미룬다.

\begin{thm} \label{5.9.3}
    \(\sum_n a_n = A, \sum_n b_n = B\)이고 두 급수의 코시곱 \(\sum_n c_n\)이 \(C\)로 수렴하면 \(AB = C\)이다.
\end{thm}



\chapter{거리공간의 위상 (2)}

다시 거리공간의 위상으로 돌아왔다. 이 장에서는 옹골집합과 연결집합을 공부하는데, 이 둘은 각각 \(\RR\)의 유계닫힌집합과 구간의 거리공간으로의 일반화이다. 특히 \(\RR^d\)의 부분집합이 옹골집합일 필요충분조건을 찾는 데에 상당한 분량을 할애할 것이다. 이 장이 끝날 때까지 \(X\)는 거리공간이다.

\section{옹골집합}

\begin{defn}
    \quad

    \begin{enum}
        \item \(X\)의 부분집합들의 집합족 \(\{U_i\}_{i \in I}\)가 \(E \subseteq X\)의 \textbf{덮개(cover)}라는 것은 \(E \subseteq \bigcup_{i \in I} U_i\)라는 것이다. \(U_i\)들이 모두 \(X\)의 열린집합이면 \(\{U_i\}_{i \in I}\)를 \(E\)의 \textbf{열린덮개(open cover)}라고 한다.
        \item \(E\)의 덮개 \(\{U_i\}_{i \in I}\)와 \(J \subseteq I\)에 대하여 \(\{U_i\}_{i \in J}\)도 \(E\)의 덮개이면 \(\{U_i\}_{i \in J}\)는 \(\{U_i\}_{i \in I}\)의 \textbf{부분덮개(subcover)}라고 한다. \(J\)가 유한집합이면 \(\{U_i\}_{i \in J}\)를 \textbf{유한 부분덮개(finite subcover)}라고 한다.
        \item \(K \subseteq X\)가 \textbf{옹골집합} 또는 \textbf{컴팩트집합(compact set)}이라는 것은, \(K\)의 임의의 열린덮개가 유한 부분덮개를 가진다는 것이다. 즉, \(K\)의 임의의 열린덮개 \(\{U_i\}_{i \in I}\)에 대하여 \(I\)의 유한 부분집합 \(J\)가 존재하여 \(\{U_i\}_{i \in J}\)도 \(K\)의 덮개라는 것이다.
    \end{enum}
\end{defn}

\begin{ex}
    \quad

    \begin{enum}
        \item \(X\)의 유한집합은 항상 옹골집합이다. \(E = \{x_1, \ldots, x_n\} \subseteq X\)에 대하여 \(\{U_i\}_{i \in I}\)가 \(E\)의 열린덮개라고 하자. 덮개의 정의에 의해, 각 \(k = 1, \ldots, n\)에 대해 \(x_k \in U_{i_k}\)인 \(i_k \in I\)가 존재하므로 유한 부분덮개 \(\{U_{i_k}\}_{1 \le k \le n}\)가 존재한다.
        \item \(X = \RR\)에서 \(\RR\) 자신은 옹골집합이 아니다. \(\{(-n, n)\}_{n \in \ZZ_+}\)은 \(\RR\)의 열린덮개이다. 이 열린덮개가 유한 부분덮개 \(\{-n, n\}_{n \in J}\)를 가진다고 가정하자. 그러면 \(M := \max_{n \in J} n\)에 대하여 \(M+1 \notin \bigcup_{n \in J} (-n, n)\)이므로 모순이다. 따라서 \(\{(-n, n)\}_{n \in \ZZ_+}\)은 유한 부분덮개를 가지지 않는다.
        \item \(X = \RR\)에서 구간 \((0, 1)\)은 옹골집합이 아니다. \(\{(1/n, 1)\}_{n \in \ZZ_+}\)는 \((0, 1)\)의 열린덮개이다. 이 열린덮개가 유한 부분덮개 \(\{(1/n, 1)\}_{n \in J}\)를 가진다고 가정하자. 그러면 \(M := \max_{n \in J} n\)에 대하여 \(1/(M+1) \notin \bigcup_{n \in J} (1/n, n)\)이므로 모순이다. 따라서 \(\{(1/n, 1)\}_{n \in \ZZ_+}\)는 유한 부분덮개를 가지지 않는다.
        \item \(X = \RR\)에서 \(E := \{0\} \cup \{1/n \in \RR : n \in \ZZ_+\}\)는 옹골집합이다. \(\{U_i\}_{i \in I}\)가 \(E\)의 열린덮개라고 하자. 이때 \(0 \in U_j\)인 \(j \in I\)가 존재하고, \(U_j\)는 열린집합이므로 어떤 양수 \(r > 0\)에 대해 \(B(0, r) \subseteq U_j\)이다. 이제 \(1/N < r\)이도록 하는 자연수 \(N\)을 택하면, \(n > N\)에 대하여 \(1/n \in U_j\)이다. 한편 각 \(n = 1, \ldots, N\)에 대하여 \(1/n \in U_{i_n}\)이도록 하는 \(i_n \in I\)가 존재한다. 따라서 유한 부분덮개 \(\{U_j\} \cup \{U_{i_n}\}_{1 \le n \le N}\)을 찾을 수 있다.
    \end{enum}
\end{ex}

앞서 열린집합과 닫힌집합의 개념은 전체 공간에 따라 달라짐을 배웠는데, 옹골집합의 경우 그렇지 않다.

\begin{prop}
    \(K \subseteq Y \subseteq X\)에 대하여 다음이 동치이다.
    \begin{enum}
        \item \(K\)는 \(X\)의 부분집합으로서 옹골집합이다.
        \item \(K\)는 \(Y\)의 부분집합으로서 옹골집합이다.
    \end{enum}
\end{prop}
\begin{proof}
    \((\Rightarrow)\) 방향을 보이기 위해 (a)를 가정하고, \(\{V_i\}_{i \in I}\)가 \(Y\)에서 \(K\)의 열린덮개라고 하자. 이때 각 \(i \in I\)에 대해 \(V_i = U_i \cap Y\)인, \(X\)의 열린집합 \(U_i\)가 존재한다. 이제 가정에 의해 유한 부분덮개 \(\{U_i\}_{i \in J}\)가 존재하는데, \(K \subseteq Y\)이므로
    \[
    K = K \cap Y \subseteq \bigcup_{i \in J} U_i \cap Y = \bigcup_{i \in J} (U_i \cap Y) = \bigcup_{i \in J} V_i    
    \]
    이다. 따라서 \(\{V_i\}_{i \in I}\)의 유한 부분덮개 \(\{V_i\}_{i \in J}\)가 존재하므로 \(K\)는 \(Y\)의 부분집합으로서 옹골집합이다. 이 증명을 거꾸로 읽으면 반대 방향이 증명된다.
\end{proof}

따라서 ``\(K\)가 옹골집합이다''라는 표현이 전체 공간에 무관하게 잘 정의된다. 특히 옹골집합 \(K \subseteq X\)는 \(X\)에서 물려받은 거리공간의 구조를 가지고 있으므로 자기 자신의 부분집합으로서 옹골집합이며, 이때 \(K\)를 \textbf{옹골 거리공간(compact metric space)}이라고 할 수 있다.

다음은 \(K \subseteq X\)가 옹골집합일 필요조건이다.

\begin{thm} \label{thm 6.1.3}
    \(K \subseteq X\)가 옹골집합이면 \(K\)는 \(X\)에서 유계닫힌집합이다.
\end{thm}
\begin{proof}
    먼저 \(K\)가 유계임을 보이자. \(x_0 \in X\)를 고정했을 때, \(\{B(x_0, r)\}_{r \in \ZZ_+}\)는 \(K\)의 열린덮개이다. 가정에 의해 유한 부분덮개 \(\{B(x_0, r)\}_{r \in J}\)가 존재한다. 이때 \(R := \max_{r \in J} r \)로 두면 \(K \subseteq \bigcup_{r \in J} B(x_0, r) = B(x_0, R)\)이므로 \(K\)가 유계임이 증명된다.\\
    다음으로 \(K\)가 닫힌집합임을 보이기 위해, \(X \setminus K\)가 열린집합임을 보이면 된다. \(x \in X \setminus K\)를 고정했을 때, 임의의 \(y \in K\)에 대하여 \(U_y \cap V_y = \varnothing\)을 만족하는 \(x, y\) 각각의 근방 \(U_y, V_y \subseteq X\)가 존재한다. 이때 \(\{V_y\}_{y \in K}\)는 \(K\)의 열린덮개이고, 가정에 의해 유한 부분덮개 \(\{V_y\}_{y \in J'}\)가 존재한다. 이때 \(U := \bigcap_{y \in J'} U_y\)로 두면 \(U\)는 \(y\)를 포함하는 열린집합이고,
    \[
    U \cap K \subseteq \bigcap_{y \in J'} U_y \cap \bigcup_{y \in J'} V_y \subseteq \bigcup_{y \in J'} (U_y \cap V_y) = \varnothing     
    \]
    이므로 \(U \subseteq X \setminus K\)이다. \(y \in X \setminus K\)가 임의의 점이었으므로 \(X \setminus K\)는 열린집합이다.
\end{proof}

하지만 정리 \ref{thm 6.1.3}의 역은 일반적으로 성립하지 않는다. \ref{sec 6.3}절에서 유클리드공간에서 역이 성립함을 증명할 것이다.
\begin{ex}
    무한집합 \(X\)에 이산 거리공간의 구조가 주어졌다고 하자. 이때 어떤 점 \(x_0 \in X\)에 대하여 \(X \subseteq B(x_0, 2)\)이므로 \(X\)는 유계이고, \(X\)는 자기 자신의 닫힌집합이다. 하지만 \(X\)는 옹골집합이 아니다. 왜냐하면 \(\{B(x, 1)\}_{x \in X}\)는 \(X\)의 열린덮개이지만 유한 부분덮개를 가지지 않기 때문이다.
\end{ex}

\begin{thm}
    옹골 거리공간 \(K\)의 닫힌집합 \(F\)는 옹골집합이다.
\end{thm}
\begin{proof}
    \(\{U_i\}_{i \in I}\)가 \(K\)에서 \(F\)의 열린덮개라고 하자. 이때 \(K \setminus F\)는 \(K\)에서 열린집합이므로, \(\{U_i\}_{i \in I} \cup \{K \setminus F\}\)도 \(K\)의 열린덮개이다. 가정에 의해 \(\{U_i\}_{i \in I} \cup \{K \setminus F\}\)의 유한 부분덮개가 존재하는데, 일반성을 잃지 않고 이 부분덮개가 \(K \setminus F\)를 포함한다고 가정해도 된다(왜 그런가?). 따라서 \(I\)의 어떤 유한 부분집합 \(J\)에 대해
    \[
        F \subseteq \bigcup_{i \in J} U_i \cup (K \setminus F)
    \]
    인데, 이는
    \[
    F \subseteq   \bigcup_{i \in J} U_i  
    \]
    임을 의미한다. 따라서 \(\{U_i\}_{i \in I}\)의 유한 부분덮개 \(\{U_i\}_{i \in J}\)가 존재한다.
\end{proof}

옹골집합을 닫힌집합을 이용해서도 정의할 수 있다.

\begin{defn}
    집합 \(X\)의 부분집합들의 집합족 \(\{E_i\}_{i \in I}\)가 \textbf{유한 교집합 성질(finite intersection property; FIP)}을 가진다는 것은, 임의의 유한 sub-집합족 \(\{E_i\}_{i \in J}\)에 대하여 \(\bigcap_{i \in J} E_i \ne \varnothing\)이라는 것이다.
\end{defn}

\begin{ex}
    \quad

    \begin{enum}
        \item \(\{(0, 1/n)\}_{n \in \ZZ_+}\)는 유한 교집합 성질을 가진다.
        \item \(\{(n, n+1)\}_{n \in \ZZ_+}\)는 유한 교집합 성질을 가지지 않는다.
    \end{enum}
\end{ex}

\begin{thm} \label{thm 6.1.6}
    거리공간 \(K\)에 대하여 다음이 동치이다.
    \begin{enum}
        \item \(K\)는 옹골집합이다. 
        \item \(K\)의 닫힌집합들의 집합족 \(\{F_i\}_{i \in I}\)에 대하여, \(\{F_i\}_{i \in I}\)가 유한 교집합 성질을 가지면 \(\bigcap_{i \in I} F_i \ne \varnothing\)이다.
    \end{enum}
\end{thm}
\begin{proof}
    \((\Rightarrow)\)를 보이기 위해 (b)를 부정하자. 즉 어떤 닫힌집합들의 집합족 \(\{F_i\}_{i \in I}\)가 존재하여 \(\{F_i\}_{i \in I}\)가 유한 교집합 성질을 만족하지만 \(\bigcap_{i \in I} F_i = \varnothing\)이라고 하자. 이때 각 \(i \in I\)에 대해 \(U_i := K \setminus F\)로 두면
    \[
    \bigcup_{i \in I} U_i = \bigcup_{i \in I} (K \setminus F_i) = K \setminus \bigcap_{i \in I} K_i = X    
    \]
    이므로 \(\{U_i\}_{i \in I}\)는 \(K\)의 열린덮개이다. 그런데 \(I\)의 임의의 유한 부분집합 \(J\)에 대해
    \[
    \bigcup_{i \in J} U_i = \bigcup_{i \in J} (K \setminus F_i) = K \setminus \bigcap_{i \in J} K_i \subsetneq K     
    \]
    이므로 \(\{U_i\}_{i \in J}\)는 \(K\)의 덮개가 아니다. 즉 \(\{U_i\}_{i \in I}\)는 유한 부분덮개를 가지지 않으므로 \(K\)는 옹골집합이 아니다. 이 증명을 거꾸로 읽으면 반대 방향이 완성된다.
\end{proof}

\begin{cor} \label{cor 6.1.7}
    옹골 거리공간 \(K\)의 닫힌집합들의 축소열 \((K_n)_{n \in \NN}\)에 대하여, 각 \(K_n\)이 공집합이 아니라고 하자. 그러면 \(\bigcap_n K_n \ne \varnothing\)이다.
\end{cor}
\begin{proof}
    \((K_n)_{n \in \NN}\)이 축소열이고 각 \(K_n\)이 공집합이 아니므로, 집합족 \(\{K_n\}_{n \in \NN}\)은 유한 교집합 성질을 가진다. 따라서 정리 \ref{thm 6.1.6}의 \((\Rightarrow)\)에 의해, \(\bigcap_n K_n \ne \varnothing\)이다.
\end{proof}

\begin{rem}
    \ref{sec 6.3}절에서 \(\RR\)의 유계닫힌구간은 옹골집합임을 증명하게 된다. 즉, 축소구간정리는 따름정리 \ref{cor 6.1.7}의 특수한 경우이다.
\end{rem}

다음 정리는 고립점이 없는, 비어있지 않은 옹골집합이 가산집합이 될 수 없음을 말해 준다.

\begin{thm} \label{thm 6.1.8}
    공집합이 아닌 옹골 거리공간 \(K\)가 고립점을 가지지 않으면 \(K\)는 비가산집합이다.
\end{thm}
\begin{proof}
    \quad\\
    \textbf{Claim 1.} 임의의 공집합이 아닌 열린집합 \(U \subseteq K\)와 \(x \in K\)에 대하여, 공집합이 아닌 어떤 열린집합 \(V \subseteq U\)가 존재하여 \(x \notin \overline{V}\)이다.
    \begin{proof}[Proof of Claim 1.]
        \(U \setminus \{x\}\)가 공집합이 아니므로 \(y \in U \setminus \{x\}\)가 존재한다. 이때 \(W_1 \cap W_2 = \varnothing\)인 \(x, y\) 각각의 근방 \(W_1, W_2 \subseteq K\)가 존재한다. 이제 \(V := W_2 \cap U\)로 두면 \(V \subseteq U\)는 열린집합이고, \(y \in V\)이므로 \(V\)는 공집합이 아니다. 한편 \(V \subseteq K \setminus W_1\)에서 \(\overline{V} \subseteq K \setminus W_1\)이므로 \(x \in W_1\)이다.
    \end{proof}
    \noindent다음 Claim 2만 보이면 증명이 완료된다.\\
    \textbf{Claim 2.} 임의의 함수 \(f : \ZZ_+ \to K\)에 대하여 \(f\)는 전사함수가 아니다.
    \begin{proof}[Proof of Claim 2.]
        \(x_n := f(n)\)으로 두자. \(U = K\)에 대하여 Claim 1을 적용하면, 공집합이 아닌 어떤 열린집합 \(V_1 \subseteq K\)가 존재하여 \(x_1 \notin \overline{V_1}\)이다. 각 \(n\)에 대하여 공집합이 아닌 열린집합 \(V_n \subseteq K\)가 주어졌을 때, \(U = V_{n}\)에 대하여 Claim 1을 적용하여 공집합이 아닌 열린집합 \(V_{n+1} \subseteq V_n\)을 \(x_{n+1} \notin \overline{V_{n+1}}\)이도록 찾을 수 있다. 이때 각 \(\overline{V_n}\)은 옹골 거리공간의 닫힌집합이고, \((\overline{V_n})_{n \in \ZZ_+}\)가 축소열이므로 따름정리 \ref{cor 6.1.7}에 의해 \(x \in \bigcap_n \overline{V}\)가 존재한다. 이제 각 \(n\)에 대해 \(x_{n} \ne x\)이므로 \(f\)는 전사함수가 아니다.
    \end{proof}
\end{proof}

\begin{rem}
    \ref{sec 6.3}절에서 \(\RR\)의 유계닫힌구간은 옹골집합임을 증명하고 나면, 유계닫힌구간은 고립점을 가지지 않으므로 정리 \ref{thm 6.1.8}에 의해 비가산집합임을 알게 된다. 따라서 \(\RR\)도 비가산집합이다.
\end{rem}

\section{LPC와 SC}

\begin{defn}
    거리공간 \(K\)가 \textbf{극한점 옹골집합(limit point compact space; LPC)}이라는 것은 \(K\)의 임의의 무한 부분집합이 \(K\) 안에서 극한점을 가진다는 것이다.
\end{defn}

\begin{thm} \label{thm 6.2.2}
    옹골 거리공간 \(K\)는 LPC이다.
\end{thm}
\begin{proof}
    \(K\)가 LPC임을 보이는 것은 다음을 보이는 것과 같다:
    \begin{center}
        \(E \subseteq K\)가 극한점을 가지지 않으면 \(E\)는 유한집합이다.
    \end{center}
    \(E\)가 극한점을 가지지 않으므로 \(E\)는 \(K\)의 닫힌집합이고, 따라서 옹골집합이다. 각 \(x \in E\)가 \(E\)의 고립점이므로, 한점집합 \(U_x := \{x\}\)가 \(E\)의 열린집합이다. 따라서 \(\{U_x\}_{x \in E}\)는 \(E\) 자신 안에서 \(E\)의 열린덮개인데, \(E\)가 옹골집합이므로 \(E\)의 유한 부분집합 \(F\)가 존재하여
    \[
    E = \bigcup_{x \in F} \{x\} = F    
    \]
    이다. 따라서 \(E\)는 유한집합이다.
\end{proof}

\begin{defn}
    거리공간 \(K\)가 \textbf{수열 옹골집합(sequentially compact space; SC)}이라는 것은 \(K\) 안의 임의의 수열이 수렴하는 부분수열을 가진다는 것이다.
\end{defn}

\begin{thm} \label{thm 6.2.4}
    거리공간 \(K\)에 대하여 다음이 동치이다.
    \begin{enum}
        \item \(K\)는 옹골집합이다.
        \item \(K\)는 LPC이다.
        \item \(K\)는 SC이다.
    \end{enum}
\end{thm}
\begin{proof}
    \quad\\
    (a)\(\Rightarrow\)(b) 정리 \ref{thm 6.2.2}에서 증명하였다.\\
    (b)\(\Rightarrow\)(c) \(K\)가 LPC라고 하고 \((x_n)_{n \in \NN}\)이 \(K\) 안의 수열이라고 하자. \(E := \{x_n \in K : n \in \NN\}\)이 유한집합이면 어떤 \(x \in E\)에 대해 \(x_n = x\)인 자연수 \(n\)이 무한히 많이 존재하므로, \(x\)로 수렴하는 \((x_n)_{n \in \NN}\)의 부분수열을 잡을 수 있다. 이제 \(E\)가 무한집합이라고 가정하자. \(K\)가 LPC이므로 \(E\)의 극한점 \(x \in K\)가 존재한다. 명제 \ref{prop 4.3.3}에 의해 \(x\)로 수렴하는 \((x_n)_{n \in \NN}\)의 부분수열이 존재한다. 따라서 \(K\)가 SC임이 증명되었다.\\
    (c)\(\Rightarrow\)(a) 이 부분의 증명은 세 단계로 구성된다. \(K\)가 SC라고 하고 \(\{U_i\}_{i \in I}\)가 \(K\)의 열린덮개라고 하자.\\
    \textbf{Claim 1.} (\ref{eq9})\를 만족하는 양수 \(\delta > 0\)이 존재한다.
    \begin{equation} \label{eq9}
        E \subseteq K\text{에 대하여} \ \diam E < \delta \text{이면} \ E \subseteq U_i \text{인} \ i \in I \text{가 존재한다.}
    \end{equation}
    \begin{proof}[Proof of Claim 1.]
        먼저 다음 진술을 가정했을 때 모순이 발생함을 보이자.
    \begin{center}
        각 \(n \in \ZZ_+\)에 대해 \(E_n \subseteq K\)가 존재하여,\\
        \(\diam E_n < 1/n\)이지만 어떤 \(i \in I\)에 대해서도 \(E_n \not\subseteq U_i\)이다.
    \end{center}
    각 \(E_n\)은 공집합이 아니므로 어떤 \(x_n \in E_n\)을 고를 수 있다. 이제 \(K\)가 SC이므로 수열 \((x_n)_{n \in \ZZ_+}\)는 어떤 \(x \in K\)로 수렴하는 부분수열 \((x_{n(k)})_{k \in \NN}\)을 가진다. 이제 \(x \in U\)인 \(U \in \{U_i\}_{i \in I}\)가 존재하고, \(U\)가 열린집합이므로 적당한 양수 \(r > 0\)에 대하여 \(B(x, r) \subseteq U\)이다.\\
    이때 \(n(k_1) > 2/r\)인 자연수 \(k_1\)이 존재하고, [\(k \ge k_2\)이면 \(d(x_{n(k)}, x) < r/2\)]인 자연수 \(k_2\)가 존재한다. \(K := \max \{k_1, k_2\}\)에 대하여 \(\diam E_{n(K)} < 1/n(K) < r/2\)이므로, 각 \(y \in C_{n(K)}\)에 대해 \(d(x_{n(K)}, y) \le \diam E_{n(K)} <  r/2\)이다. 그런데 \(d(x, y) \le d(x, x_{n(K)}) + d(x_{n(K)}, y) < r/2 + r/2 = r\)에서 \(E_{n(K)} \subseteq B(x, r) \subseteq U\)이므로 모순이다.\\
    따라서 다음을 만족하는 \(n \in \ZZ_+\)이 적어도 하나 존재한다.
    \begin{center}
        \(E \subseteq K\)에 대하여 \(\diam E < 1/n\)이면 \(E \subseteq U_i\)인 \(i \in I\)가 존재한다.
    \end{center}
    이제 \(\delta := 1/n\)으로 두면 (\ref{eq9})\를 만족한다.
    \end{proof}
    \noindent\textbf{Claim 2.} 임의의 \(\eps > 0\)에 대하여, 유한 개의 점들 \(x_1, \ldots, x_n \in K\)가 존재하여 \(K = \bigcup_{k=1}^n B(x_k, \eps)\)이다.
    \begin{proof}[Proof of Claim 2.]
        이를 부정하여, 어떤 \(\eps > 0\)에 대해 그러한 유한 개의 점들이 존재하지 않는다고 하자. 그러면 다음과 같이 수열 \((x_n)_{n \in \ZZ_+}\)을 구성할 수 있다.\\
    \(x_1\)은 \(K\)의 아무 점으로 한다. \(n \ge 1\)에 대하여 \(x_1, \ldots, x_n\)이 주어지면, \(\bigcup_{k=1}^n B(x_k, \eps) \ne K\)이므로 \(x_{n+1} \in K \setminus \bigcup_{k=1}^n B(x_k, \eps)\)를 택할 수 있다. 이렇게 귀납적으로 \((x_n)_{n \in \ZZ_+}\)을 정의하면, 임의의 두 자연수 \(m > n\)에 대해 \(d(x_m, x_n) \ge \eps\)이므로 \((x_n)_{n \in \NN}\)의 어떤 부분수열도 코시수열이 아니다. 따라서 \(K\)가 SC임에 모순이다.
    \end{proof}
    이제 \(K\)가 옹골집합임을 보이자. Claim 1에서 구한 \(\delta > 0\)에 대하여 \(\eps := \delta / 3\)으로 두자. 그러면 Claim 2에 의하여, 유한 개의 점들 \(x_1, \ldots, x_n \in K\)가 존재하여 \(K = \bigcup_{k=1}^n B(x_k, \eps)\)이다. 그런데 각 \(k\)에 대해 \(\diam B(x_k, \eps) \le 2\eps < \delta\)이므로 \(B(x_k, \eps) \subseteq U_{i_k}\)인 \(i_k \in I\)가 존재한다. 따라서 유한 부분덮개 \(\{U_{i_k}\}_{1 \le k \le n}\)이 존재하므로 \(K\)는 옹골집합이다.
\end{proof}

\section{하이네-보렐 정리} \label{sec 6.3}

이제 유클리드공간에서 정리 \ref{thm 6.1.3}의 역을 증명할 준비가 되었다. The정리orem \ref{thm 6.2.4}\를 이용하면 거의 공짜나 마찬가지다.

\begin{thm}[하이네-보렐 정리\footnote{Heine-Borel theorem.}]
    \(K \subseteq \RR^d\)에 대하여 다음이 동치이다.
    \begin{enum}
        \item \(K\)는 옹골집합이다.
        \item \(K\)는 \(\RR^d\)에서 유계닫힌집합이다.
    \end{enum}
\end{thm}
\begin{proof}
    \((\Rightarrow)\) 방향은 일반적으로 성립하므로 \((\Leftarrow)\) 방향만 증명하면 되는데, 정리 \ref{thm 6.2.4}에 의해 이는 \(K\)가 LPC임을 보이는 것과 같다. \(E \subseteq K\)가 무한 부분집합이라고 하자. \(K\)가 유계이므로 \(E\)도 유계이고, 따라서 \(E\)는 \(\RR^d\)에서 극한점 \(x\)를 가진다. 그런데 \(K\)가 닫힌집합이므로 \(x \in E' \subseteq K' \subseteq K\)이다. 따라서 \(K\)는 LPC이다.
\end{proof}

\begin{cor}
    \(K \subseteq \RR^d\)에 대하여 다음이 동치이다.
    \begin{enum}
        \item \(K\)는 옹골집합이다.
        \item \(K\)는 \(\RR^d\)에서 유계닫힌집합이다.
        \item \(K\)의 임의의 무한 부분집합은 \(K\)에서 극한점을 가진다. 즉 \(K\)는 LPC이다.
        \item \(K\) 안의 임의의 수열은 \(K\) 안의 점으로 수렴하는 부분수열을 가진다. 즉 \(K\)는 SC이다.
    \end{enum}
\end{cor}

\begin{cor}
    \(\RR\)의 유계닫힌구간은 비가산집합이다.
\end{cor}

\begin{ex}
    \(I_0 := [0, 1]\)이라 하고, \(I_0\)을 삼등분했을 때 가운데의 열린구간 \((1/3, 2/3)\)을 제거한 집합을 \(I_1 := I_0 \setminus (1/3, 2/3)\)라 하자. \(I_1\)은 서로소인 두 유계닫힌구간의 합집합으로 나타나는데, 각각의 구간을 삼등분하여 가운데의 열린구간을 제거하여 얻어진 집합을 \(I_2\)라 하자. 이와 같이 귀납적으로
    \[
    I_n := I_{n-1} \setminus \bigcup_{k=0}^\infty \paren{\frac{1+3k}{3^n}, \frac{2+3k}{3^n}}    
    \]
    로 정의하자. 그리고 다음과 같이 정의된 집합
    \[
    C := \bigcap_{n \in \ZZ_+} I_n
    \]
    을 \textbf{칸토어 집합(Cantor set)}이라 한다. 이때 \(C \subseteq [0, 1]\)이므로 \(C\)는 유계이고, 각 \(I_n\)이 닫힌집합이므로 \(C\)도 닫힌집합이다. 즉 \(C\)는 \(\RR\)의 유계닫힌집합이므로 옹골집합이다. 이제 \(C\)가 비가산집합임을 보이기 위해, \(C\)의 모든 점이 극한점임을 보이면 된다(\(0 \in C\)이므로 \(C\)가 공집합이 아님은 자명하다).\\
    각 \(I_n\)은 \(2^n\)개의 서로소인 유계닫힌구간 \(I_n^1, \ldots, I_n^{2^n}\)들의 합집합으로 나타낼 수 있고, 각 구간읙 길이는 \(3^{-n}\)이다. 이때 임의의 \(n\)에 대해 구간 \(I_n^k\)들의 끝점은 \(C\)에 포함됨을 쉽게 확인할 수 있다. 이제 임의의 \(x \in C\)와 \(r > 0\)에 대하여, \(3^{-N} < r\)이도록 하는 자연수 \(N\)이 존재한다. 그리고 \(x \in I_N\)이므로 길이가 \(3^{-N}\)인 어떤 구간 \(I_N^k\)에 대하여 \(x \in I_N^k\)이다. \(I_N^k\)의 끝점 중 \(x\)가 아닌 것이 적어도 하나 있으므로 이 점을 \(y\)라 하면 \(\abs{y - x} \le 3^{-N} < r\)이다. 따라서 \(B(x, r) \cap (C \setminus \{x\}) \ne \varnothing\)이므로 \(x\)는 \(C\)의 극한점이다.
\end{ex}

\section{연결집합}

\begin{defn}
    \(C \subseteq X\)와 \(X\)의 열린집합 \(U, V\)에 대하여
    \[
    U \cap C \ne \varnothing, \quad V \cap C \ne \varnothing, \quad  U \cap V \cap C \ne \varnothing , \quad C \subseteq U \cup V  
    \]
    이면 \((U \cap C, V \cap C)\)를 \(C\)의 \textbf{분할(separation)}이라고 한다. \(C\)의 분할이 존재하지 않으면 \(C\)를 \textbf{연결집합(connected set)}이라고 한다.
\end{defn}

\begin{ex}
    \quad

    \begin{enum}
    \item \(\RR \setminus \{0\}\)은 \((-\infty, 0) \cup (0, \infty)\)의 분할을 가지므로 연결집합이 아니다.
    \item \(X\)의 원소가 2개 이상인 유한 부분집합은 연결집합이 아니다. \(E \subseteq X\)가 원소가 2개 이상인 유한집합이라고 하고 어떤 \(x \in E\)를 고정하자. 이때 \(x\)는 \(E\)의 고립점이므로 \(U \cap E = \{x\}\)인 \(X\)의 열린집합 \(U\)가 존재한다. 이제 \(V := X \setminus \{x\}\)로 두면, \(E = (U \cap E) \cup (V \cap E)\)가 \(E\)의 분할임은 쉽게 확인할 수 있다.
    \item 2개 이상의 원소를 가지는 집합 \(X\)가 이산 거리공간의 구조를 가지면 \(X\)는 연결집합이 아니다. 어떤 \(x \in X\)를 고정하면 \(U := \{x\}\)는 \(X\)의 열린집합이면서 닫힌집합이다. 따라서 \(X = U \cup (X \setminus U)\)는 \(X\)의 분할이다.
    \end{enum}
\end{ex}


옹골집합과 마찬가지로 연결집합은 전체 공간에 의존하지 않는다.

\begin{prop}
    \(C \subseteq Y \subseteq X\)에 대하여 다음이 동치이다.
    \begin{enum}
        \item \(C\)는 \(X\)의 부분집합으로서 연결집합이다.
        \item \(C\)는 \(Y\)의 부분집합으로서 연결집합이다.
    \end{enum}
\end{prop}
\begin{proof}
    연습문제로 남긴다.
\end{proof}

따라서 ``\(C\)가 연결집합이다''라는 표현이 전체 공간에 무관하게 잘 정의된다. 특히 연결집합 \(C \subseteq X\)는 \(X\)에서 물려받은 거리공간의 구조를 가지고 있으므로 자기 자신의 부분집합으로서 연결집합이며, 이때 \(C\)를 \textbf{연결 거리공간(connected metric space)}이라고 할 수 있다.

다음은 분할의 필요충분조건이다.
\begin{prop}
    \(C \subseteq X\)와 \(U, V \subseteq C\)에 대하여 다음이 동치이다.
    \begin{enum}
        \item \((U, V)\)가 \(C\)의 분할이다.
        \item \(U \ne \varnothing, V \ne \varnothing, U \cup V = C\)이고 \(U \cap \overline{V} = V \cap \overline{U} = \varnothing\)이다(여기서 \(\overline{U}, \overline{V}\)는 각각 \(X\)에서의 \(U, V\)의 닫힘이다).
    \end{enum}
\end{prop}
\begin{proof}
    \quad\\
    \((\Rightarrow)\) \((U, V)\)가 \(C\)의 분할이므로 \(U, V\)는 모두 \(C\)의 열린집합이다. 그런데 \(U \cap V = \varnothing, U \cup V = C\)이므로 \(U, V\)는 \(C\)의 닫힌집합이기도 하다. 따라서
    \[
    U \cap \overline{V} = U \cap (\overline{V} \cap C) = U \cap \text{cl}_C V = U \cap V = \varnothing 
    \]
    이고 마찬가지 방법으로 \(V \cap \overline{U} = \varnothing\)이다.\\
    \((\Leftarrow)\) 가정에 의해 \(U \ne \varnothing, V \ne \varnothing, U \cup V = C, U \cap V = \varnothing\)이 성립하므로 \(U, V\)가 각각 \(C\)의 열린집합인 것만 보이면 된다. 이는 \(U, V\)가 각각 \(C\)의 닫힌집합인 것을 보이는 것과 같다. 이때
    \[
    \text{cl}_C U = \overline{U} \cap C = U \cup (\overline{U} \cap V) = U    
    \]
    이므로 \(U\)는 \(C\)의 닫힌집합이고, 마찬가지로 \(V\)도 \(C\)의 닫힌집합이다.
\end{proof}

이제 주어진 연결집합으로부터 다른 연결집합을 만들어내는 방법을 볼 것이다.

\begin{lem} \label{lem 6.4.4}
    \((U, V)\)가 \(X\)의 분할이고 \(C \subseteq X\)가 연결집합이면 \(C \subseteq U\)이거나 \(C \subseteq V\)이다.
\end{lem}
\begin{proof}
    두 집합 \(U' := U \cap C, V' := V \cap C\)를 생각하자. 이때 \(U', V'\)는 \(C\)의 열린집합이고, \(U' \cup V' = C, U' \cap V' = \varnothing\)을 만족한다. 그런데 \(C\)는 연결집합이므로 \((U', V')\)는 \(C\)의 분할이 될 수 없고, 따라서 \(U'\)와 \(V'\) 중 적어도 하나는 공집합이다. 따라서 \(C \subseteq U\)이거나 \(C \subseteq V\)이다.
\end{proof}

\begin{thm} \label{thm 6.4.5}
    \(X\)의 연결집합들의 집합족 \(\{C_i\}_{i \in I}\)에 대하여, \(\bigcap_{i \in I} C_i \ne \varnothing\)이면 \(\bigcup_{i \in I} C_i\)는 연결집합이다.
\end{thm}
\begin{proof}
    \(C := \bigcup_{i \in I} C_i\)가 분할 \((U, V)\)를 가진다고 하자. 이때 가정에 의해 어떤 점 \(x \in \bigcap_{i \in I} C_i\)가 존재한다. 일반성을 잃지 않고 \(x \in U\)라고 가정할 수 있다. 보조정리 \ref{lem 6.4.4}에 의해 각 \(i \in I\)에 대하여 [\(C_i \subseteq U\) 또는 \(C_i \subseteq V\)]인데, \(x \in U\)이므로 모든 \(C_i\)가 \(U\)에 포함된다. 따라서 \(C \subseteq U\)이며, 이는 \(V = \varnothing\)임을 의미하므로 모순이다. 따라서 \(C\)는 연결집합이다.
\end{proof}

\begin{thm}
    \(C \subseteq X\)가 연결집합이고 \(C \subseteq D \subseteq \overline{C}\)이면 \(D\)도 연결집합이다.
\end{thm}
\begin{proof}
    \(D\)가 분할 \((U, V)\)를 가진다고 하자. 보조정리 \ref{lem 6.4.4}에 의해 \(C \subseteq U\)이거나 \(C \subseteq V\)이다. 일반성을 잃지 않고 \(C \subseteq U\)라고 가정할 수 있다. 그러면 \(D \subseteq \overline{C} \overline{U}\)인데 \(\overline{U} \cap V = \varnothing\)이므로 \(V = \varnothing\)이고, 이는 \((U, V)\)가 분할이라는 데 모순이다. 따라서 \(D\)는 연결집합이다.
\end{proof}

\section{연결집합의 예시}

\begin{thm} \label{thm 6.5.1}
    공집합이 아닌 \(C \subseteq \RR\)에 대하여 다음이 동치이다.
    \begin{enum}
        \item \(C\)는 연결집합이다.
        \item 임의의 \(x, y, z \in \RR\)에 대하여, \(x, y \in C\)이고 \(x < z < y\)이면 \(z \in C\)이다.
        \item \(C\)는 한점집합 또는 구간이다.
    \end{enum}
\end{thm}
\begin{proof}
    (b)\(\Leftrightarrow\)(c)는 거의 당연하다. (a)\(\Rightarrow\)(b)를 보이기 위해 (b)를 부정하면, \(x, z \in C\)이고 \(x < z < y\)이지만 \(z \notin C\)인 세 실수 \(x, y, z \in \RR\)가 존재한다. 이때
    \[
    C = ((-\infty, z) \cap C) \cup ((z, \infty) \cap C)    
    \]
    는 \(C\)의 분할이므로 \(C\)는 연결집합이 아니다.\\
    마지막으로 (b)\(\Rightarrow\)(a)를 보이기 위해 (b)를 가정하고 (a)를 부정하여, \(C\)가 분할 \((U, V)\)를 가진다고 하자. \(U\)와 \(V\)는 모두 공집합이 아니므로 \(x \in U\)와 \(y \in V\)를 택할 수 있고, 일반성을 잃지 않고 \(x < y\)라 할 수 있다. 이제 \(z := \sup (U \cap [x, y])\)라 정의하면 \(x \le z \le y\)이고, (b)에 의해 \(z \in C\)이다. 따라서 \(z \in U\)이거나 \(z \in V\)이다.\\
    \textbf{Case 1.} \(z \in U\)인 경우\\
    이때는 \(z \ne y\)이므로 \(x \le z < y\)이다. 한편 \(U \cap [x, y)\)는 \([x, y)\)의 열린집합이므로, 어떤 양수 \(r_1 > 0\)이 존재하여 \(B(z, r_1) \cap [x, y) \subseteq U \cap [x, y)\)이다. 따라서
    \[
    d_1 \in U \cap (z, \min\{z + r_1, y\}) \subseteq U \cap [x, y]  
    \]
    인 \(d_1\)이 존재하는데 이는 \(z\)가 \(U \cap [x, y]\)의 상계인 것에 모순이다.\\
    \textbf{Case 2.} \(z \in U\)인 경우\\
    이때는 \(z \ne x\)이므로 \(x < y \le z\)이다. 앞에서와 마찬가지로 양수 \(r_2 > 0\)이 존재하여 \(B(z, r_2) \cap (x, y] \subseteq V \cap (x, y] \)이다. 상한의 정의에 의해,
    \[
    d_2 \in (U \cap [x, y] ) \cap (\max\{z - r_2, x\}, z) \subseteq U
    \]
    인 \(d_2\)가 존재하는데, 이는 \(U \cap V = \varnothing\)인 것에 모순이다.\\
    따라서 (b)\(\Rightarrow\)(a)가 증명되었다.
\end{proof}

\begin{defn}
    \(\RR\)- 또는 \(\CC\)-벡터공간 \(V\)와 \(x, y \in V\)에 대하여 집합
    \[
    [x, y] := \{(1-t)x + ty \in V : t \in [0, 1]\}    
    \]
    를 \(x\)와 \(y\) 사이의 \textbf{선분(line segment)}이라고 한다.
\end{defn}

\begin{thm}
     노름공간 \(V\)와 \(x, y \in V\)에 대하여 \([x, y]\)는 연결집합이다.
\end{thm}
\begin{proof}
    \ref{thm 6.5.1}의 (b)\(\Rightarrow\)(a) 증명과 거의 같다.
\end{proof}

\begin{defn}
    \(\RR\)- 또는 \(\CC\)-벡터공간 \(V\)의 부분집합 \(C \subseteq V\)가 \textbf{볼록집합(convex set)}이라는 것은 임의의 \(x, y \in C\)에 대하여 \([x, y] \subseteq C\)라는 것이다.
\end{defn}
\begin{ex}
    \quad

    \begin{enum}
        \item \(\RR\)의 구간은 볼록집합이다.
        \item 노름공간 \(V\)와 \(x_0 \in V, r > 0\)에 대하여 \(B(x_0, r)\)은 볼록집합이다. 임의의 \(x, y \in B(x_0, r), t \in [0, 1]\)에 대하여
        \begin{align*}
            \norm{((1-t)x + ty) - x_0} &= \norm{(1-t)(x - x_0) + t(y - x_0)}\\
            &\le (1-t)\norm{x-x_0} + t\norm{y - x_0}\\
            &< (1-t)r+tr  = r 
        \end{align*}
        이므로 \([x, y] \subseteq B(x_0, r)\)이기 때문이다.
    \end{enum}
\end{ex}

\begin{cor}
    노름공간 \(V\)의 볼록집합은 연결집합이다.
\end{cor}
\begin{proof}
    \(C \subseteq V\)가 볼록집합이라고 하자. \(C\)가 공집합이면 자명하게 연결집합이므로, \(C\)가 공집합이 아니라고 가정하고 \(x \in C\)를 고정하자. 그러면 다음이 성립한다.
    \[
        C = \bigcup_{y \in C} [x, y], \quad \bigcap_{y \in C} [x, y] = \{x\}
    \]
    각 \([x, y]\)가 연결집합이므로 정리 \ref{thm 6.4.5}에 의해 \(C\)가 연결집합이다.
\end{proof}



\chapter{연속성} \label{sec conti}

이 장에서는 고등학교 때부터 배운 함수의 극한과 연속함수의 개념을 거리공간의 수준에서 드디어 엄밀하게 정의하고, 옹골집합과 연결집합에서 정의된 연속함수가 각각 어떤 성질을 가지는지 알아볼 것이다. 그리고 고등학교와 미적분학에서 배우지 않은 새로운 개념인 고른연속함수를 정의하고, 실수의 구간에서 정의된 실함수 중 단조함수의 성질을 공부할 것이다. 마지막으로 연결집합과 비슷하지만 다른 개념인 경로연결집합을 연속함수를 이용하여 정의하고 그 성질을 조금 알아볼 것이다. 이 장이 끝날 때까지 \(X\)와 \(Y\)는 각각 \(d_X\)와 \(d_Y\)라는 거리함수가 주어진 거리공간이다.

\section{함수의 극한과 연속함수}

\begin{defn}
    함수 \(f : E \subseteq X \to Y\)와 \(x_0 \in E', y_0 \in Y\)에 대하여,
    \[
    \lim_{x \to x_0} f(x) = y_0
    \]
    라는 것은 임의의 양수 \(\eps > 0\)에 대하여 양수 \(\delta > 0\)이 존재하여
    \[
    0 < d_X(x, x_0) < \delta \implies d_Y(f(x), y_0) < \eps \quad (x \in E)    
    \]
    라는 것이다. 이때 \(y_0\)를 \(x \to x_0\)일 때 \(f\)의 \textbf{극한}이라 한다.
\end{defn}

\begin{prop}
    함수 \(f : E \subseteq X \to Y\)와 \(x_0 \in E', y_0 \in Y\)에 대하여 다음이 동치이다.
    \begin{enum}
        \item \(\lim_{x \to x_0} f(x) = y_0\).
        \item \(x_0\)로 수렴하는 \(E \setminus \{x_0\}\) 안의 임의의 수열 \((x_n)_{n \in \ZZ_+}\)이 \(f(x_n) \to f(x_0)\)을 만족한다.
    \end{enum}
\end{prop}
\begin{proof}
    \quad\\
    \((\Rightarrow)\) \(\lim_{x \to x_0} f(x) = y_0\)라고 하고, \(x_0\)로 수렴하는 \(E \setminus \{x_0\}\) 안의 수열 \((x_n)_{n \in \ZZ_+}\)가 주어졌다고 하자. 이때 임의의 양수 \(\eps > 0\)에 대해 다음을 만족하는 양수 \(\delta > 0\)이 존재한다.
    \[
        0 < d_X(x, x_0) < \delta \implies d_Y(f(x), y_0) < \eps \quad (x \in E)
    \]
    그런데 \(x_n \to x_0\)이므로, 다음을 만족하는 자연수 \(N\)이 존재한다.
    \[
    n \ge N \implies d_X(x_n, x_0) < \delta    
    \]
    따라서 \(n \ge N\)이면 \(d_Y(f(x_n), y_0) < \eps\)이므로 \(f(x_n) \to y_0\)이다.\\
    \((\Leftarrow)\) (a)를 부정하여, 어떤 \(\eps > 0\)이 존재하여 다음을 만족한다고 하자. 
    \begin{equation} \label{eq10}
        \text{임의의} \ \delta > 0\text{에 대해} \ 0 < d_X(x, x_0) < \delta\text{이지만} \ d_Y(f(x), y_0) \ge \eps\text{인} \ x \in E\text{가 존재한다.}
    \end{equation}
    각 \(n \in \ZZ_+\)에 대하여 \(\delta = 1/n\)으로 둘 때 (\ref{eq10})\을 만족하는 \(x \in E \setminus \{x_0\}\)를 \(x_n\)이라 하면, \((x_n)_{n \in \ZZ_+}\)는 \(x_0\)로 수렴하지만 \(d_Y(f(x_n), y_0) \ge \eps\)이므로 \(f(x_n) \to y_0\)가 아니다.
\end{proof}

\begin{cor}
    함수 \(f : E \subseteq X \to Y\)와 \(x_0 \in E'\)에 대하여 \(x \to x_0\)일 때 \(f\)의 극한이 존재하면 유일하다.
\end{cor}
\begin{proof}
    거리공간의 수열의 극한의 유일성으로부터 나온다.
\end{proof}

함수의 극한과 수열의 극한의 관계로부터 다음 명제들은 거의 당연하므로 증명은 생략한다.

\begin{prop}
    함수 \(f, g : E \subseteq X \to \CC\)와 \(x_0 \in E'\)에 대하여 \(x \to x_0\)일 때 \(f(x) \to \alpha, g(x) \to \beta\)라고 하자. 이때 \(x \to x_0\)일 때 다음 함수들의 극한은 다음과 같다.
    \begin{enum}
        \item \(f(x)+g(x) \to \alpha + \beta\).
        \item \(c \in \CC\)에 대하여 \(cf(x) \to c\alpha\).
        \item \(f(x)g(x) \to \alpha\beta\).
        \item \(g \ne 0, \beta \ne 0\)일 때 \(f(x)/g(x) \to \alpha/\beta\).
    \end{enum}
\end{prop}

\begin{prop}
    각 \(i = 1, \ldots, d\)에 대하여 함수 \(f_i : E \subseteq X \to \RR\)가 주어졌다고 하고, \(f : E \subseteq X \to \RR^d\)를
    \[
    f : x \mapsto (f_1(x), \ldots, f_d(x))    
    \]
    로 정의하자. 이때 \(x_0 \in E'\)와 \(y_1, \ldots, y_d \in \RR\)에 대하여 \(x \to x_0\)일 때 \(f(x) \to (y_1, \ldots, y_d)\)일 필요충분조건은 각 \(i\)에 대하여 \(x \to x_0\)일 때 \(f_i(x) \to y_i\)인 것이다.
\end{prop}

\begin{defn}
    함수 \(f : X \to Y\)가 \(x_0 \in X\)에서 \textbf{연속(continuous)}이라는 것은, 임의의 양수 \(\eps > 0\)에 대해 다음을 만족하는 양수 \(\delta > 0\)이 존재하는 것이다.
    \[
    d_X(x, x_0) < \delta \implies d_Y(f(x), f(x_0)) < \eps \quad (x \in X)    
    \]
    \(f\)가 \(X\)의 모든 점에서 연속이면 \(f\)를 \textbf{연속함수(continuous function)}라고 한다.
\end{defn}

\begin{ex}
    \quad

    \begin{enum}
        \item 상수함수는 연속함수이다.
        \item 공집합이 아닌 거리공간 \(X\)의 한 점 \(x_0 \in X\)를 고정하자. 함수 \(f : X \to \RR\)를
        \[
        f : x \mapsto d(x, x_0)    
        \]
        로 정의하면 \(f\)는 연속함수이다. 임의의 \(x, y \in X\)에 대해
        \[
        \abs{f(x) - f(y)} = \abs{d(x, x_0) - d(y, x_0)} \le d(x, y)    
        \]
        가 성립한다. 따라서 임의의 \(x \in X, \eps > 0\)에 대하여 \(\delta := \eps\)로 두면
        \[
            d(x, y) < \delta \implies \abs{f(x) - f(y)} < \eps \quad (y \in X) 
        \]
        이므로 \(f\)는 \(x\)에서 연속이다.
        \item 노름공간 \(V\)에서 정의된 노름 \(\norm{\cdot} : V \to \RR\)는 연속함수이다. 증명은 (a)와 거의 같으므로 연습문제로 남긴다.
        \item 내적공간 \(V\)에서 한 점 \(v_0 \in V\)를 고정하자. 함수 \(f : X \to F\)를
        \[
        f : v \mapsto \inner{v}{v_0}    
        \]
        로 정의하면 \(f\)는 연속함수이다. \(v_0 = 0\)이면 \(f\)는 상수함수이므로 더 증명할 것이 없다. \(v_0 \ne 0\)이라고 하자. 이때 임의의 \(v, w \in V\)에 대해
        \[
        \abs{f(v) - f(w)} = \abs{\inner{v - w}{v_0}} \le \norm{v - w} \norm{v_0}    
        \]
        가 성립한다. 따라서 임의의 \(v \in V, \eps > 0\)에 대하여 \(\delta := \eps / \norm{v_0}\)로 두면
        \[
            \norm{v - w} < \delta \implies \abs{f(v) - f(v)} < \eps \quad (w \in X) 
        \]
        이므로 \(f\)는 \(v\)에서 연속이다.
        \item \(F = \RR\) 또는 \(\CC\)에서 \textbf{\(i\)번째 사영 함수(\(i\)-th projection map)} \(\pi_i : F^d \to F : (x_1, \ldots, x_d) \mapsto x_i\)는 연속함수이다.
    \end{enum}
\end{ex}

\begin{prop} \label{prop 7.1.7}
    함수 \(f : X \to Y\)와 \(x_0 \in X\)에 대하여 다음이 동치이다.
    \begin{enum}
        \item \(f\)는 \(x_0\)에서 연속이다.
        \item \(f(x_0)\)의 임의의 근방 \(V \subseteq Y\)에 대하여, \(x_0\)의 근방 \(U \subseteq X\)가 존재하여 \(f(U) \subseteq V\)이다.
        \item \(x_0\)로 수렴하는 \(X\) 안의 임의의 수열 \((x_n)_{n \in \ZZ_+}\)이 \(f(x_n) \to f(x_0)\)을 만족한다.
    \end{enum}
    특히 \(x_0 \in X'\)이면 위의 진술들과 \(\lim_{x \to x_0} f(x) = f(x_0)\)가 동치이다.
\end{prop}
\begin{proof}
    \quad\\
    (a)\(\Rightarrow\)(b) \(f(x_0)\)의 근방 \(V \subseteq Y\)가 주어졌다고 하자. 이때 어떤 양수 \(\eps > 0\)에 대하여 \(B_Y(f(x_0), \eps) \subseteq V\)이다. 한편 (a)에 의해
    \[
        d_X(x, x_0) < \delta \implies d_Y(f(x), f(x_0)) < \eps \quad (x \in X)
    \]
    이도록 하는 양수 \(\delta > 0\)이 존재한다. \(U := B_X(x_0, \eps)\)로 두면 (b)가 증명된다.\\
    (b)\(\Rightarrow\)(c) \(X\) 안의 수열 \((x_n)_{n \in \ZZ_+}\)이 \(x_0\)로 수렴한다고 하자. 양수 \(\eps > 0\)에 대하여 \(V := B_Y(f(x_0), \eps)\)로 두면 (b)에 의해 \(f(U) \subseteq V\)인 \(x_0\)의 근방 \(U \subseteq X\)가 존재한다. \(U\)는 \(X\)의 열린집합이므로 양수 \(\delta > 0\)이 존재하여 \(B_X(x_0, \delta) \subseteq U\)이다. \(x_n \to x_0\)이므로 [\(n \ge N\)이면 \(d_X(x_n, x_0) < \delta\)]인 자연수 \(N\)이 존재한다. 이제 \(n \ge N\)이면 \(x_n \to B_X(x_0, \delta) \subseteq U\)이므로 \(f(x_n) \in V = B_Y(f(x_0), \eps)\)이고, 따라서 \(f(x_n) \to f(x_0)\)이다.\\
    (c)\(\Rightarrow\)(a) (a)를 부정하여, 어떤 \(\eps > 0\)이 존재하여 다음을 만족한다고 하자.
    \begin{equation} \label{eq7.2}
        \text{임의의} \ \delta > 0\text{에 대해} \ d_X(x, x_0) < \delta\text{이지만} \ d_Y(f(x), f(x_0)) \ge \eps\text{인} \ x \in X\text{가 존재한다.}
    \end{equation}
    각 \(n \in \ZZ_+\)에 대하여 \(\delta = 1/n\)으로 둘 때 (\ref{eq7.2})\를 만족하는 \(x \in X\)를 \(x_n\)이라 하면, \((x_n)_{n \in \ZZ_+}\)은 \(x_0\)로 수렴하지만 \(d_Y(f(x_n), f(x_0)) \ge \eps\)이므로 \(f(x_n) \to f(x_0)\)가 아니다.\\
    이제 마지막으로 \(x_0 \in X'\)인 경우에, (a)의 정의와 \(\lim_{x \to x_0} f(x) = f(x_0)\)의 정의로부터 두 진술이 동치인 것은 당연하다.
\end{proof}
\begin{rem}
    \(x \in X \setminus X'\)인 경우에, 즉 \(x\)가 \(X\)의 고립점이면 \(f : X \to Y\)는 항상 \(x\)에서 연속이다. 왜냐하면 한점집합 \(\{x\}\)가 \(X\)의 열린집합이므로, \(U := \{x\}\)로 두면 (b)가 항상 참이기 때문이다.
\end{rem}

함수의 연속과 함수의 극한의 관계에 의해 다음 명제들은 거의 당연하므로 증명은 생략한다.

\begin{prop}
    함수 \(f, g : X \to \CC\)에 대하여 \(f, g\)가 \(x_0 \in X\)에서 연속이면 \(f+g, cf \ (c \in \CC), fg, f/g \ (g(x_0) \ne 0)\)도 \(x_0\)에서 연속이다.
\end{prop}

\begin{prop}
    각 \(i = 1, \ldots, d\)에 대하여 함수 \(f_i : X \to \RR\)가 주어졌다고 하고, \(f : X \to \RR^d\)를
    \[
    f : x \mapsto (f_1(x), \ldots, f_d(x))    
    \]
    로 정의하자. 이때 \(f\)가 \(x_0 \in X\)에서 연속일 필요충분조건은 각 \(i\)에 대하여 \(f_i\)가 \(x_0\)에서 연속인 것이다.
\end{prop}

\begin{thm}
    거리공간 \(X, Y, Z\)와 함수 \(f : X \to Y, g : Y \to Z\)에 대하여 \(f\)가 \(x_0 \in X\)에서 연속이고 \(g\)가 \(f(x_0)\)에서 연속이면 \(g \circ f : X \to Z\)도 \(x_0\)에서 연속이다.
\end{thm}
\begin{proof}
    \((g \circ f)(x_0)\)의 근방 \(W \subseteq Z\)에 대하여, \(g\)가 \(f(x_0)\)에서 연속이므로 명제 \ref{prop 7.1.7}에 의해 \(f(x_0)\)의 근방 \(V \subseteq Y\)가 존재하여 \(g(V) \subseteq W\)이다. 한편 \(f\)가 \(x_0\)에서 연속이므로 다시 명제 \ref{prop 7.1.7}에 의해 \(x_0\)의 근방 \(U \subseteq X\)가 존재하여 \(f(U) \subseteq V\)이다. 이때 \((g \circ f)(U) = g(f(U)) \subseteq g(V) \subseteq W\)이므로 증명이 완료되었다.
\end{proof}

다음은 어떤 함수가 연속함수일 필요충분조건들이다.

\begin{thm}
    함수 \(f : X \to Y\)에 대하여 다음이 동치이다.
    \begin{enum}
        \item \(f : X \to Y\)가 연속함수이다.
        \item \(Y\)의 임의의 열린집합 \(V \subseteq Y\)에 대하여 \(f^{-1}(V) \subseteq X\)가 \(X\)의 열린집합이다.
        \item \(Y\)의 임의의 닫힌집합 \(F \subseteq Y\)에 대하여 \(f^{-1}(F) \subseteq X\)가 \(X\)의 닫힌집합이다.
        \item 임의의 \(E \subseteq X\)에 대하여 \(f(\overline{E}) \subseteq \overline{f(E)}\)이다.
        \item 임의의 \(x_0 \in X\)와 \(f(x_0)\)의 근방 \(V \subseteq Y\)에 대하여, \(x_0\)의 근방 \(U \subseteq X\)가 존재하여 \(f(U) \subseteq V\)이다.
        \item 임의의 \(x_0 \in X\)에 대하여, \(x_0\)로 수렴하는 \(X\) 안의 임의의 수열 \((x_n)_{n \in \ZZ_+}\)이 \(f(x_n) \to f(x_0)\)을 만족한다.
        \item 임의의 \(x_0 \in X'\)에 대하여 \(x \to x_0\)일 때 \(f(x) \to f(x_0)\)이다.
    \end{enum}
\end{thm}
\begin{proof}
    (a)\(\Leftrightarrow\)(e)\(\Leftrightarrow\)(f)\(\Leftrightarrow\)(g)는 명제 \ref{prop 7.1.7}에서 증명하였다. 나머지 부분을 보이기 위해,
    \begin{center}
        (b)\(\Rightarrow\)(e)\(\Rightarrow\)(b), \quad (b)\(\Rightarrow\)(d)\(\Rightarrow\)(c)\(\Rightarrow\)(b)
    \end{center}
    를 보이겠다.\\
    (b)\(\Rightarrow\)(e) \(x_0 \in X\)와 \(f(x_0)\)의 근방 \(V \subseteq Y\)에 대하여, \(f^{-1}(V) \subseteq X\)가 \(X\)의 열린집합이다. 따라서 적당한 양수 \(\delta > 0\)에 대해 \(B_X(x_0, \delta) \subseteq f^{-1}(V)\)이며, \(U := B_X(x_0, \delta)\)로 두면 (e)가 증명된다.\\
    (e)\(\Rightarrow\)(b) \(V \subseteq Y\)가 열린집합이라 하자. 임의의 \(x \in f^{-1}(V)\)에 대해, \(V\)는 \(f(x)\)의 \(Y\)에서의 근방이므로 (e)에 의해 \(x\)의 근방 \(U_x \subseteq X\)가 존재하여 \(U_x \subseteq f^{-1}(V)\)이다. 이때 \(f^{-1}(V) = \bigcup_{x \in f^{-1}(V)} U_x\)이므로 \(f^{-1}(V)\)는 \(X\)의 열린집합이다.\\
    (b)\(\Rightarrow\)(d) \(E \subseteq X\)와 \(x \in \overline{E}\)에 대하여 \(f(x) \in \overline{f(E)}\)인 것을 보이면 된다. \(f(x)\)의 임의의 근방 \(V\)에 대하여, (b)에 의해 \(f^{-1}(V)\)가 \(x\)의 근방이므로 어떤 \(y \in f^{-1}(V) \cap E\)가 존재한다. 따라서 \(V\)는 \(f(E)\)의 원소 \(f(y)\)를 포함한다. \(V\)가 \(f(x)\)의 임의의 근방이었으므로 \(f(x) \in \overline{f(E)}\)이다.\\
    (d)\(\Rightarrow\)(c) \(F \subseteq Y\)가 닫힌집합이라 하고 \(G := f^{-1}(F)\)라 하자. 이때 \(f(G) \subseteq F\)이므로, 임의의 \(x \in \overline{G}\)에 대해
    \[
    f(x) \in f(\overline{G}) \subseteq \overline{f(G)} \subseteq \overline{F} \subseteq F    
    \]
    이다. 따라서 \(x \in G\)이므로 \(\overline{G} \subseteq G\)이고, 이는 \(G\)가 닫힌집합임을 의미한다.\\
    (c)\(\Rightarrow\)(b) \(V \subseteq Y\)가 열린집합이라 하면,
    \[
    f^{-1}(Y \setminus V) = f^{-1}(Y) \setminus f^{-1}(V) = X \setminus f^{-1}(V) 
    \]
    이다. 그런데 \(Y \setminus V\)는 \(Y\)의 닫힌집합이고, 가정에 의해 \(f^{-1}(Y \setminus V)\)도 \(X\)의 닫힌집합이다. 따라서 \(f^{-1}(V)\)는 \(X\)의 열린집합이다. 
\end{proof}


다음은 주어진 연속함수로부터 새로운 연속함수를 만드는 방법들이다. 

\begin{prop}
    거리공간 \(X, Y, Z\)에 대하여 다음이 성립한다.
    \begin{enum}
        \item \(A \subseteq X\)에 대하여 \textbf{포함 함수(inclusion function)} \(\iota: A \inclusion X\)는 연속함수이다.
        \item 함수 \(f : X \to Y\)와 \(g : Y \to Z\)가 연속함수이면 \(g \circ f : X \to Z\)도 연속함수이다.
        \item 연속함수 \(f : X \to Y\)와 \(A \subseteq X\)에 대해 \(f\)의 정의역을 \(A\)로 제한한 함수 \(f \vert_A : A \to Y\)도 연속함수이다.
        \item 함수 \(f : X \to Y\)가 연속함수이고, \(Y \subseteq Z\)이면서 \(d_Y = d_Z \vert_{Y \times Y}\)라고 하자. \(f(X) \subseteq B \subseteq Y\)에 대하여 \(f\)의 공역을 \(B\)로 제한한 함수 \(g : X \to B\)도 연속함수이고 \(f\)의 공역을 \(Z\)로 확장한 함수 \(h : X \to Z\)도 연속함수이다.
        \item \(\{U_i\}_{i \in I}\)가 \(X\)의 열린덮개일 때, 함수 \(f : X \to Y\)가 연속함수일 필요충분조건은 각 \(i \in I\)에 대해 \(f \vert_{U_i} : U_i \to Y\)가 연속함수인 것이다.
    \end{enum}
\end{prop}
\begin{proof}
    \quad

    \begin{enum}
        \item \(X\)의 열린집합 \(V \subseteq X\)에 대하여 \(\iota^{-1}(V) = A \cap V\)는 \(A\)의 열린집합이다.
        \item \(Z\)의 열린집합 \(V \subseteq Z\)에 대하여 \(g^{-1}(V) \subseteq Y\)는 \(Y\)의 열린집합이고, \((g \circ f)^{-1}(V) = f^{-1}(g^{-1}(V))\)는 \(X\)의 열린집합이다.
        \item 포함 함수 \(\iota : A \inclusion X\)에 대하여 \(f \vert_A = f \circ \iota\).
        \item \(V \subseteq B\)가 \(B\)의 열린집합이면 \(V = B \cap U\)인 \(Y\)의 열린집합 \(U \subseteq Y\)가 존재한다. 이때 \(g^{-1}(V) = f^{-1}(V) = f^{-1}(U)\)는 \(X\)의 열린집합이다. 또 \(V \subseteq Z\)가 \(Z\)의 열린집합이면 \(V \cap Y\)는 \(Y\)의 열린집합이므로 \(h^{-1}(V) = f^{-1}(V \cap Y)\)가 \(X\)의 열린집합이다.
        \item \((\Rightarrow)\) 방향은 (c)에 의해 자명하다. 이제 \((\Leftarrow)\) 방향을 보이기 위해, 각 \(i \in I\)에 대해 \(f \vert_{U_i} : U_i \to Y\)가 연속함수라고 가정하자. 이때 \(Y\)의 열린집합 \(V \subseteq Y\)에 대해
        \[
        f^{-1}(V) = \bigcup_{i \in I} (U_i \cap f^{-1}(V)) = \bigcup_{i \in I} (f \vert_{U_i})^{-1}(V)    
        \]
        이다. 그런데 각 \(f \vert_{U_i}\)가 연속이므로 \((f \vert_{U_i})^{-1}(V)\)는 \(U_i\)의 열린집합인데, \(U_i\)가 \(X\)의 열린집합이므로 \((f \vert_{U_i})^{-1}(V)\)는 \(X\)의 열린집합이다. 따라서 \(f^{-1}(V)\)가 \(X\)의 열린집합이다.
    \end{enum}
\end{proof}

\begin{prop} \label{prop 7.1.13}
    닫힌집합 \(F, G \subseteq X\)에 대하여 \(X = F \cup G\)라고 하고, \(f : F \to Y, g : G \to Y\)가 연속함수라고 하자. \(f \vert_{F \cap G} = g \vert_{F \cap G}\)이면 다음과 같이 정의된 함수 \(h : X \to Y\)도 연속함수이다.
    \[
      h : x \mapsto \begin{cases}
        f(x) &\textif \ x \in F\\
        g(x) &\textif \ x \in G
      \end{cases}  
    \]
\end{prop}
\begin{proof}
    \(f \vert_{F \cap G} = g \vert_{F \cap G}\)이므로 \(h\)는 잘 정의된다. 이제 임의의 닫힌집합 \(C \subseteq Y\)에 대하여
    \[
    h^{-1}(C) = f^{-1}(C) \cap g^{-1}(C)    
    \]
    이다. \(f, g\)가 연속이므로 \(f^{-1}(C)\)와 \(g^{-1}(C)\)는 각각 \(F, G\)의 닫힌집합이다. 그런데 \(F, G\)가 \(X\)의 닫힌집합이므로 \(f^{-1}(C), g^{-1}(C)\)도 \(X\)의 닫힌집합이고, \(h^{-1}(C)\)도 \(X\)의 닫힌집합이다. 따라서 \(h\)는 연속함수이다.
\end{proof}



\section{연속함수와 옹골집합}

고등학교 때부터 증명을 미룬 최대최소정리를 증명한다.

\begin{thm}
    옹골 거리공간 \(K\)에서 정의된 연속함수 \(f : K \to Y\)에 대하여 \(f\)의 상 \(f(K) \subseteq Y\)도 옹골집합이다.
\end{thm}
\begin{proof}
    \(f(K)\)의 열린집합들의 집합족 \(\{V_i\}_{i \in I}\)가 \(f(K)\)의 열린덮개라고 하자. 이때 각 \(f^{-1}(V_i)\)는 \(K\)의 열린집합이고, \(\bigcup_{i \in I} f^{-1}(V_i) = f^{-1} \paren{\bigcup_{i \in I} V_i} =f^{-1}(f(K)) = K\)이므로 \(\{f^{-1}(V_i)\}_{i \in I}\)는 \(K\)의 열린덮개이다. \(K\)가 옹골집합이므로 유한 부분덮개 \(\{f^{-1}(V_i)\}_{i \in J}\)가 존재한다. 따라서
    \[
    f(K) = f \paren{\bigcup_{i \in J} f^{-1}(V_i)} = \bigcup_{i \in J} f(f^{-1}(V_i)) = \bigcup_{i\in J} V_i
    \]
    이므로, \(\{V_i\}_{i \in I}\)의 유한 부분덮개 \(\{V_i\}_{i \in J}\)가 존재한다. 따라서 \(f(K)\)도 옹골집합이다.
\end{proof}

이제 최대최소정리는 당연한 따름정리이다.

\begin{cor}[최대최소정리\footnote{Extreme value theorem.}]
    옹골 거리공간 \(K\)에서 정의된 연속 실함수 \(f : K \to \RR\)에 대하여 \(f\)는 \(K\)에서 최댓값과 최솟값을 가진다.
\end{cor}
\begin{proof}
    \(f(K)\)가 \(\RR\)의 유계닫힌집합이므로 \(f(K)\)는 최댓값과 최솟값을 가진다.
\end{proof}
\begin{cor}
    \(\RR\)의 유계닫힌구간 \([a, b] \subseteq \RR\)에서 정의된 연속 실함수 \(f : [a, b] \to \RR\)에 대하여 \(f\)는 \([a, b]\)에서 최댓값과 최솟값을 가진다.
\end{cor}

다음으로 전단사 연속함수의 역함수도 연속함수일 충분조건을 제시한다.

\begin{thm}
    옹골 거리공간 \(K\)에서 정의된 전단사 연속함수 \(f : K \to Y\)에 대하여 \(f^{-1} : Y \to K\)도 연속함수이다.
\end{thm}
\begin{proof}
    \(K\)의 열린집합 \(U \subseteq K\)에 대하여 \(f(U) \subseteq Y\)가 \(Y\)의 열린집합인 것을 보이는 것과 같다. 이때 \(K \setminus U\)는 \(K\)의 닫힌집합이므로 옹골집합이다. 따라서 \(f(K \setminus U)\)는 거리공간의 옹골집합이므로 닫힌집합이다. 그런데 \(f\)가 전단사이므로 \(f(K \setminus U) = f(K) \setminus f(U) = Y \setminus f(U)\)이고, 따라서 \(f(U)\)는 \(Y\)의 열린집합이다.
\end{proof}

\begin{ex}
    정의역이 옹골집합이 아니면 전단사 연속함수의 역함수가 연속함수가 아닐 수 있다. 함수 \(X = \{0\} \cup (1, 2], Y = [0, 1]\)에 대하여 함수 \(f : X \to Y\)를
    \[
        f(x) =
        \begin{cases}
            0 &\textif \ x = 0\\
            x-1 &\textif \ 1 < x \le 2
        \end{cases}
    \]
    로 정의하면 \(f\)는 전단사 연속함수이지만,
    \[
    f^{-1}(y) =
    \begin{cases}
        0 &\textif \ y = 0\\
        y+1 &\textif \ 0 < y \le 1
    \end{cases}    
    \]
    는 연속함수가 아니다.
\end{ex}

옹골집합에서 정의된 연속함수를 이용하여 \(F^d\)의 노름에 관해 좀 더 알아보자. \(F = \RR\) 또는 \(\CC\).


\begin{lem} \label{lem 7.2.7}
    노름공간 \((V, \norm{\cdot}_V), (W, \norm{\cdot}_W)\)에 대하여, 선형사상 \(f : (V, \norm{\cdot}_V) \to (W, \norm{\cdot}_W)\)가 연속함수일 필요충분조건은 \(f\)가 \(0 \in V\)에서 연속인 것이다.
\end{lem}
\begin{proof}
    \((\Rightarrow)\) 방향은 정의에 의해 자명하다. \((\Leftarrow)\) 방향을 보이기 위해 \(f\)가 \(0 \in V\)에서 연속이라고 가정하자. 그러면 임의의 \(\eps > 0\)에 대해 다음을 만족하는 \(\delta > 0\)이 존재한다.
    \[
    \norm{v}_V < \delta \implies \norm{f(v)}_W < \eps \quad (v \in V)    
    \]
    이때 임의의 \(u, v \in V\)에 대해
    \[
        \norm{u - v}_V < \delta \implies \norm{f(u) - f(v)}_W = \norm{f(u - v)}< \eps
    \]
    가 성립한다. 따라서 \(f\)는 \(v\)에서 연속이다.
\end{proof}

\begin{lem} \label{lem 7.2.8}
    \(F\)-벡터공간 \(V\)에 두 노름 \(\norm{\cdot}_a\)와 \(\norm{\cdot}_b\)가 주어졌을 때, 양수 \(M\)이 존재하여
    \[
        \norm{\cdot}_b \le M \norm{\cdot}_a
    \]
    가 성립하면 \(\id : (V, \norm{\cdot}_a) \to (V, \norm{\cdot}_b)\)는 연속함수이다.
\end{lem}
\begin{proof}
    임의의 \(\eps > 0\)에 대하여 \(\delta := \eps / M\)으로 두면
    \[
    \norm{v}_a < \delta \implies \norm{v}_b < \eps \quad (v \in V)
    \]
    가 성립하므로 \(\id : (V, \norm{\cdot}_a) \to (V, \norm{\cdot}_b)\)는 0에서 연속이다. 따라서 보조정리 \ref{lem 7.2.7}에 의해 \(f\)는 연속함수이다.
\end{proof}

\begin{thm}
    \(F^d\)에 주어진 임의의 두 노름은 립시츠 동등하다.
\end{thm}
\begin{proof}
    \(\CC^d\)는 \(\RR^{2d}\)로 동일시할 수 있으므로 \(F = \RR\)인 경우만 증명해도 된다. 그리고 노름의 립시츠 동등 관계가 동치 관계이므로, \(\RR^d\)의 임의의 노름 \(\norm{\cdot}\)이 표준적인 노름 \(\norm{\cdot}_2\)와 립시츠 동등한 것을 보이면 충분하다.\\
    먼저 각 \(i = 1, \ldots, d\)에 대해 \(e_i\)를 \(i\)번째 성분만 1이고 나머지 성분은 0인 \(\RR^d\)의 원소로 정의하자. 그러면 임의의 \(v = (v_1, \ldots, v_d) \in \RR^d\)에 대하여 \(v = \sum_{i=1}^d v_i e_i\)로 쓸 수 있고, 보조정리 \ref{3.3.9}에 의해 어떤 양수 \(L > 0\)이 존재하여
    \[
        \norm{v} \le \sum_{i=1}^d \abs{v_i} \norm{e_i} \le \max_{1 \le i \le d} \norm{e_i} \cdot \norm{v}_1 \le L \cdot \max_{1 \le i \le d} \norm{e_i} \cdot \norm{v}_2
    \]
    가 성립한다. 따라서 \(M := L \cdot \max_{1 \le i \le d} \norm{e_i}\)로 두면 \(\norm{\cdot} \le M \norm{\cdot}_2\)를 얻는다.\\
    다음으로 \(S^{d-1} := \{v \in \RR^d : \norm{v}_2 = 1\}\)을 생각하자. \(S^{d-1}\)이 \(\norm{\cdot}_2\)에 관하여 유계인 것은 당연하고, 연속함수 \(\norm{\cdot}_2 : (\RR^d, \norm{\cdot}_2) \to \RR\)에 의한 닫힌집합 \(\{1\}\)의 역상이므로 \((\RR^d, \norm{\cdot}_2)\)에 관하여 닫힌집합이다. 따라서 하이네-보렐 정리에 의해 \(S^{d-1}\)은 옹골집합이다. 한편 보조정리 \ref{lem 7.2.8}에 의해 \(\id : (\RR^d, \norm{\cdot}_2) \to (\RR^d, \norm{\cdot})\)가 연속함수이고, \(\norm{\cdot} : (\RR^d, \norm{\cdot}) \to \RR\)도 연속함수이므로 \(\norm{\cdot} : (\RR^d, \norm{\cdot}_2) \to \RR\)가 연속함수이다. 따라서 최대최소정리에 의해 \(\norm{\cdot}\)은 \(S^{d-1}\)에서 최솟값 \(m\)을 가지는데, \(S^{d-1}\)에서 \(\norm{\cdot}\)의 값은 항상 양수이므로 \(m >  0\)이다. 이때 임의의 \(v \in \RR^d \setminus \{0\}\)에 대해,
    \[
        \norm{v} = \norm{v}_2 \norm{\frac{1}{\norm{v}_2}v} \ge m\norm{v}_2
    \]
    이므로 \(m\norm{\cdot}_2 \le \norm{\cdot}\)이 증명되었다. 따라서 \(\norm{\cdot}_2 \sim \norm{\cdot}\)이다.
\end{proof}

\begin{cor}
    \(F^d\)에 주어진 두 노름 \(\norm{\cdot}_a, \norm{\cdot}_b\)에 대하여 함수 \(\id : (F^d, \norm{\cdot}_a) \to (F^d, \norm{\cdot}_b)\)는 연속함수이고 그 역함수도 연속함수이다.
\end{cor}
\begin{proof}
    양수 \(m, M > 0\)이 존재하여
    \[
        m \norm{\cdot}_a \le \norm{\cdot}_b \le M \norm{\cdot}_a    
    \]
    가 성립한다. 따라서 보조정리 \ref{lem 7.2.8}에 의해 \(\id : (F^d, \norm{\cdot}_a) \to (F^d, \norm{\cdot}_b)\)와 \(\id : (F^d, \norm{\cdot}_b) \to (F^d, \norm{\cdot}_a)\) 모두 연속함수이다.
\end{proof}

\begin{cor}
    \(F^d\)에 주어진 임의의 노름 \(\norm{\cdot}\)에 대하여 다음이 성립한다.
    \begin{enum}
        \item \((F^d, \norm{\cdot})\)은 바나흐공간이다.
        \item \((F^d, \norm{\cdot})\)의 유계닫힌집합은 옹골집합이다.
    \end{enum}
\end{cor}
\begin{proof}
    \(\id : (F^d, \norm{\cdot}) \to (F^d, \norm{\cdot}_2)\)를 통해 임의의 노름이 주어진 \(F^d\)를 표준적인 노름이 주어진 \(F^d\)와 동일시할 수 있다.
\end{proof}

\section{연속함수와 연결집합}

고등학교 때부터 증명을 미룬 사잇값정리를 증명한다.

\begin{thm}
    연결 거리공간 \(C\)에서 정의된 연속함수 \(f : C \to Y\)에 대하여 \(f\)의 상 \(f(C) \subseteq Y\)도 연결집합이다.
\end{thm}
\begin{proof}
    \(f(C)\)가 연결집합이 아니라고 하자. 그러면 \(f(C)\)의 분할 \((U, V)\)가 존재한다. 이제 \(U' := f^{-1}(U)\)와 \(V' := f^{-1}(V)\)는 각각 \(X\)의 열린집합이고, \(U' \ne \varnothing, V' \ne \varnothing, U' \cup V' = C, U' \cap V' = \varnothing\)을 만족한다. 따라서 \((U', V')\)가 \(C\)의 분할이므로 \(C\)가 연결집합이라는 데에 모순이다.
\end{proof}

이제 사잇값정리는 당연한 따름정리이다.

\begin{cor}[사잇값정리\footnote{Intermediate value theorem.}]
    연결 거리공간 \(C \ne \varnothing\)에서 정의된 연속 실함수 \(f : C \to \RR\)에 대하여 \(f\)의 상 \(f(C) \subseteq \RR\)는 한점집합 또는 구간이다.
\end{cor}
\begin{cor}
    구간 \([a, b] \ne \RR\)에서 정의된 연속 실함수 \(f : [a, b] \to \RR\)에 대하여 \(k\)가 \(f(a)\)와 \(f(b)\) 사이의 실수이면 \(f(c) = k\)인 \(c \in [a, b]\)가 존재한다.
\end{cor}

연결 거리공간의 연속함수에 의한 상이 연속함수라는 것으로부터, 노름공간 \(V\)의 임의의 선분 \([x, y]\)가 연결집합이라는 것을 쉽게 증명할 수 있다. 구간 \([0, 1]\)이 연결집합인 것은 이미 알고 있으므로, 함수 \(f : [0, 1] \to V : t \mapsto (1-t)x + ty \)가 \([0, 1]\)에서 연속함수인 것만 보이면 되는데 이는
\[
    \norm{f(t) - f(s)} = \abs{(s-t)x + (t-s)y} = \abs{t-s}\norm{x-y} \quad (t, s \in [0, 1])
\]
인 것에서 성립한다.

\section{고른연속함수}

이 절에서는 해석개론에서 처음 정의하는 개념인 고른연속함수의 성질에 대해 공부한다.

\begin{defn}
    함수 \(f : X \to Y\)가 \textbf{고른연속함수(uniformly continuous function)}라는 것은, 임의의 양수 \(\eps > 0\)에 대하여 양수 \(\delta > 0\)이 존재하여 임의의 \(x, y \in X\)에 대해
    \[
    d_X(x, y) < \delta \implies d_Y(f(x), f(y)) < \eps    
    \]
    라는 것이다.
\end{defn}
\begin{rem}
    \quad

    \begin{enum}
        \item 모든 고른연속함수는 연속함수이다.
        \item 연속함수의 정의는 모든 점에서 연속인 것이므로, 연속성은 기본적으로 점별 성질이다. 그러나 고른연속성은 정의역 전체에 의존하는 성질이다. 즉 ``\(f : X \to Y\)가 \(x \in X\)에서 고른연속이다''라는 말 자체가 성립하지 않는다.
    \end{enum}
\end{rem}

\begin{ex}
    \quad 

    \begin{enum}
        \item 함수 \(f : X \to Y\)와 양수 \(M > 0\)에 대하여 \(f\)가 \textbf{\(M\)-립시츠 연속함수(\(M\)-Lipschitz continuous function)}라는 것은, 어떤 양수 \(M\)이 존재하여 임의의 \(x, y \in X\)에 대해
        \begin{equation} \label{eq7.3}
            d_Y(f(x), f(y)) \le M \cdot d_X(x, y)    
        \end{equation}
        라는 것이다. 어떤 양수 \(M\)에 대해 \(f\)가 \(M\)-립시츠 연속함수이면 \(f\)를 \textbf{립시츠 연속함수(Lipschitz continuous function)}라고 한다. 립시츠 연속함수는 고른연속인데, (\ref{eq7.3})\을 만족하는 \(M\)이 존재할 때 임의의 양수 \(\eps > 0\)에 대하여
        \[
        d_X(x, y) < \frac{\eps}{M} \implies d_Y(f(x), f(y)) \le M \cdot d_X(x, y) \eps    
        \]    
        가 성립하기 때문이다.
        \item 함수 \(f : [0, 1] \to \RR : x \mapsto x^{1/2}\)은 고른연속함수이다. 임의의 \(\eps > 0\)에 대해
        \[
        \delta := \min \left\{\eps^2, \frac{1}{\eps}\right\}  
        \]
        로 정의하자. 이때 \(0 \le x < y \le 1\)이 \(y - x < \delta\)를 만족하면, \(y < \eps^2\)일 때
        \[
            \abs{\sqrt{x} - \sqrt{y}} \le \sqrt{y} < \eps
        \]
        이고 \(y \ge \eps^2\)일 때
        \[
            \abs{\sqrt{x} - \sqrt{y}} = \frac{\abs{x-y}}{\sqrt{x} + \sqrt{y}} \le \frac{\abs{x-y}}{\sqrt{y}} < \frac{\delta}{\eps} \le \eps
        \]
        이다. 따라서 \(f\)는 고른연속함수이다. 그러나 \(f\)는 립시츠 연속은 아닌데, 이는 \(0 < x \le 1\)에 대해
        \[
        \frac{\abs{\sqrt{x} - \sqrt{0}}}{\abs{x-0}} = \frac{1}{\sqrt{x}}\to \infty \ \text{as} \ x \to 0    
        \]
        이기 때문이다. 따라서 (a)의 역은 성립하지 않는다.
        \item \(f : [a, b] \to \RR\)가 정의역에서 연속적으로 미분가능\footnote{물론 아직 미분가능성을 정의하지도 않았다.}하면 \(f\)는 고른연속이다. 최대최소정리에 의해 \(\abs{f'} < M\)인 양수 \(M > 0\)이 존재하는데, 평균값정리에 의해 \(f\)가 \(M\)-립시츠 연속이기 때문이다.
        \item 함수 \(f : [0, 1] \to \RR : x \to x^2\)은 (c)에 의해 고른연속이다. 그러나 \(g : \RR \to \RR : x \to x^2\)은 고른연속이 아니다. 만약 \(g\)가 고른연속이라고 가정하면, 다음을 만족하는 양수 \(\delta > 0\)이 존재한다.
        \[
        \abs{x - y} < \delta \implies \abs{x^2 - y^2} < 1 \quad (x, y \in \RR)    
        \]
        이때
        \[
        x = \frac{1}{\delta}, \quad y = \frac{1}{\delta} + \frac{\delta}{2}    
        \]
        로 두면 \(\abs{x-y} < \delta\)이지만
        \[
        \abs{x^2 - y^2} = \frac{\delta^2}{4} + 1 > 1    
        \]
        이므로 모순이다. 따라서 \(g\)는 고른연속이 아니다. 즉 고른연속성은 정의역에 의존하는 성질이다.
    \end{enum}
\end{ex}

이 절의 주된 결론은 옹골 거리공간에서 정의된 연속함수는 항상 고른연속함수라는 것이다. 이를 위해 조금의 준비가 필요하다.

\begin{lem} \label{lem 7.4.2}
    \(E \subseteq X\)가 공집합이 아니라고 하자. 이때 \(x \in X\)에 대하여
    \[
    d(x, E) := \inf \{d(x, y) \in \RR : y \in E\}
    \]
    로 정의하면 \(x \mapsto d(x, E)\)는 \(X\)에서 연속함수이다.
\end{lem}
\begin{proof}
    \(x, y \in X, z \in E\)에 대해
    \[
        d(x, E) \le d(x, z) \le d(x, y) + d(y, z)
    \]
    이므로
    \[
    d(x, E) - d(x, y) \le d(y, z)    
    \]
    이다. 이는 임의의 \(z \in E\)에 대해 성립하므로,
    \[
    d(x, E) - d(x, y) \le d(y, E)
    \]
    이다. 따라서 \(d(x, E) - d(y, E) \le d(x, y)\)인데, 마찬가지 방법으로 \(d(y, E) - d(x, E) \le d(y, x)\)이다. 즉
    \[
    \abs{d(x, E) - d(y, E)} \le d(x, y)    
    \]
    가 성립하며, 이는 \(x \mapsto d(x, E)\)가 립시츠 연속임을 의미한다.
\end{proof}

\begin{lem}[르베그 수 보조정리\footnote{Lebesgue number lemma}]
    \(\{U_i\}_{i \in I}\)가 옹골 거리공간 \(K\)의 열린덮개라고 하자. 이때 (\ref{eq7.4})\를 만족하는 양수 \(\delta > 0\)이 존재한다.
    \begin{equation} \label{eq7.4}
        E \subseteq K\text{에 대하여} \ \diam E < \delta \text{이면} \ E \subseteq U_i \text{인} \ i \in I \text{가 존재한다.}
    \end{equation}
    이러한 \(\delta > 0\)을 \(\{U_i\}_{i \in I}\)의 \textbf{르베그 수(Lebesgue number)}라고 한다.
\end{lem}
\begin{proof}
    \(U_i\)들 중 \(U_i = X\)인 것이 있으면 임의의 양수 \(\delta > 0\)이 (\ref{eq7.4})\를 만족하므로 증명이 끝난다. 이제 \(U_i\)들이 모두 \(X\)의 진부분집합이라 하고, \(\{U_i\}_{i \in I}\)의 유한 부분덮개 \(\{U_{i_k}\}_{1 \le k \le n}\)를 잡자. 각 \(i_k\)에 대해 \(F_k := K \setminus U_{i_k}\)로 두고, 함수 \(f : K \to \RR\)를
    \[
    f : x \mapsto \frac{1}{n}\sum_{k=1}^n d(x, F_{k})    
    \]
    로 정의하자. 임의의 \(x \in K\)에 대하여 \(x \in U_{i_k}\)인 어떤 \(k\)가 존재하는데, \(U_{i_k}\)가 열린집합이므로 적당한 \(r > 0\)에 대해 \(B(x, r) \subseteq U_{i_k}\)이어서 \(d(X, F_k) \ge r\)이다. 즉 임의의 \(x \in K\)에 대하여 \(f(x) > 0\)이다. Lemma \ref{lem 7.4.2}에 의해 \(f\)는 연속함수이므로 최대최소정리에 의해 \(f\)의 최솟값 \(\delta \ge 0\)이 존재한다. 우리의 주장은 이 \(\delta\)가 (\ref{eq7.4})\를 만족한다는 것이다. \(E \subseteq K\)가 \(\diam E < \delta\)를 만족한다고 하자. 그러면 어떤 \(x_0 \in E\)에 대해 \(E \subseteq B(x_0, \delta)\)이다. 이때 \(d(x_0, F_1), \ldots d(x_0, F_n)\) 중 가장 큰 것을 \(d(x_0, F_m)\)이라 하면,
    \[
    \delta \le f(x_0) \le d(x_0, F_m)
    \]
    이다. 즉 \(B(x_0, \delta) \subseteq K \setminus F_m = U_{i_m}\)이므로, \(E \subseteq U_{i_m}\)이 성립한다.
\end{proof}

\begin{thm}
    옹골 거리공간에서 정의된 연속함수는 고른연속함수이다.
\end{thm}
\begin{proof}
    함수 \(f : K \to Y\)가 옹골 거리공간 \(K\)에서 정의된 연속함수라고 하자. 주어진 양수 \(\eps > 0\)에 대하여 \(\{f^{-1}(B_Y(y, \eps/2))\}_{y \in Y}\)는 \(K\)의 열린덮개이다. 양수 \(\delta > 0\)이 이 열린덮개의 르베그 수라고 하자. \(x_1, x_2 \in K\)에 대하여 \(d_X(x_1, x_2) < \delta\)이면, \(\diam \{x_1, x_2\} < \delta\)이므로 어떤 \(y \in Y\)에 대하여 \(\{x_1, x_2\} \subseteq f^{-1}(B_Y(y, \eps/2))\)가 성립한다. 이는 \(f(x_1), f(x_2) \in B_Y(y, \eps/2)\)임을 의미하므로 \(d_Y(f(x_1), f(x_2)) < \eps\)이다. 따라서 \(f\)는 고른연속함수이다.
\end{proof}

고른연속의 좋은 점은 코시수열을 보존해준다는 것이다.

\begin{prop} \label{prop 7.4.5}
    함수 \(f : X \to Y\)가 고른연속이고 \((x_n)_{n \in \NN}\)이 \(X\) 안의 코시수열이면 \((f(x_n))_{n \in \NN}\)도 \(Y\) 안의 코시수열이다.
\end{prop}
\begin{proof}
    양수 \(\eps > 0\)이 주어졌을 때 다음을 만족하는 양수 \(\delta > 0\)이 존재한다.
    \[
    d_X(p, q) < \delta \implies d_Y(f(p), f(q)) < \eps \quad (p, q \in X)    
    \]
    그리고 다음을 만족하는 자연수 \(N\)도 존재한다.
    \[
    m > n \ge N \implies d_X(x_m, x_n) < \delta    
    \]
    따라서
    \[
        m > n \ge N \implies d_Y(f(x_m), f(x_n)) < \eps
    \]
    이므로, \((f(x_n))_{n \in \NN}\)은 \(Y\) 안의 코시수열이다.
\end{proof}

\begin{defn}
    연속함수 \(f : E \subseteq X \to Y\)에 대하여 \(\tilde{f} : X \to Y\)가 연속함수이고 \(tilde{f} \vert_E = f\)이면 \(tilde{f}\)를 \(f\)의 \textbf{연속적 확장(continuous extension)}이라고 한다.
\end{defn}

\begin{ex}
    \quad

    \begin{enum}
        \item \(f : [0, 1] \subseteq \to \RR : x \to x\)에 대하여, \(\tilde{f} : \RR \to \RR : x \to x\)는 \(f\)의 \(\RR\)로의 연속적 확장이다.
        \item \(f : (0, 1) \subseteq \to \RR : x \to 1/x\)는 \(\RR\)로의 연속적 확장을 가지지 않는다.
    \end{enum}
\end{ex}

공역이 완비 거리공간일 때 고른연속은 정의역의 닫힘으로의 연속적 확장이 존재할 충분조건이다.

\begin{thm} \label{7.4.7}
    \(D\)가 \(X\)에서 조밀하고 \(f : D \subseteq X \to Y\)가 연속함수일 때 다음이 성립한다.
    \begin{enum}
        \item \(f\)의 \(X\)로의 연속적 확장이 존재하면 그것은 유일하다. 
        \item \(f\)가 \(D\)에서 고른연속이고 \(Y\)가 완비 거리공간이면 \(f\)의 \(X\)로의 연속적 확장이 존재하고, 확장 역시 \(X\)에서 고른연속이다.
    \end{enum}
\end{thm}
\begin{proof}
    \quad

    \begin{enum}
        \item \(\tilde{f}_1, \tilde{f}_2 : X \to Y\)가 \(f\)의 \(X\)로의 연속적 확장이라고 하자. 임의의 \(x \in X\)에 대하여 \(x\)로 수렴하는 \(D\) 안의 수열 \((x_n)_{n \in \NN}\)이 존재하고, \(\tilde{f}_1, \tilde{f}_2\)가 \(x\)에서 연속이므로
        \[
        \tilde{f}_1(x) = \lim_{n } \tilde{f}_1(x_n) = \lim_{n } \tilde{f}_2(x_n) = \tilde{f}_2(x)
        \]
        이다. 따라서 \(\tilde{f}_1 = \tilde{f}_2\)이다.
        \item \textbf{Step 1.} 각 \(x \in X\)에 대하여, \(x\)로 수렴하는 \(D\) 안의 수열 \((x_n)_{n \in \NN}\)이 존재한다. 이때 \((x_n)_{n \in \NN}\)은 \(D\)의 코시수열이므로 명제 \ref{prop 7.4.5}에 의해 \((f(x_n))_{n \in \NN}\)도 \(Y\)의 코시수열이다. \(Y\)가 완비 거리공간이므로 \((f(x_n))_{n \in \NN}\)이 어떤 \(y \in Y\)로 수렴하며, 이때 \(\tilde{f}(x) := y\)로 정의하자. 이때 \(f(x)\)가 잘 정의됨을 보여야 한다. 즉, \(x\)로 수렴하는 \(D\) 안의 다른 수열 \((x'_n)_{n \in \NN}\)을 가져왔을 때
        \[
        \lim_{n } f(x_n) = \lim_{n } f(x'_n)   
        \]
        인지를 보여야 한다. \(f(x_n) \to \alpha, f(x'_n) \to \beta\)로 쓰자. 임의의 \(\eps > 0\)에 대하여, \(f\)의 고른연속성에 의해
        \begin{equation} \label{eq7.5}
            d_X(p, q) < \delta \implies d_Y(f(p), f(q)) < \eps \quad (p, q \in D)   
        \end{equation}
        를 만족하는 \(\delta > 0\)이 존재한다. 그리고 충분히 큰 \(N\)에 대하여
        \[
        n \ge N \implies d_X(x_n, x) < \frac{\delta}{2}, d_X(x'_n, x) < \frac{\delta}{2}    
        \]
        가 성립한다. 따라서 \(n, m \ge N\)이면 \(d_X(x_n, x'_m) < \delta\)이므로 \(d_Y(f(x_n), f(x'_m)) < \eps\)이다. 이제 양변에 \(m \to \infty\)를 취하면, 고정된 \(n\)에 대하여 \(d_Y(f(x_n), f(\cdot))\)가 연속함수이므로
        \[
        d_Y(f(x_n), \beta) \le \eps    
        \]
        를 얻는다. 또 \(n \to \infty\)를 취하면 \(d_Y(f(\cdot), \beta)\)가 연속함수이므로
        \[
            d_Y(\alpha, \beta) \le \eps
        \]
        를 얻는다. 이때 \(\eps > 0\)이 임의의 양수였으므로 \(\alpha = \beta\)이다. 따라서 \(\tilde{f} : X \to Y\)가 잘 정의된다.\\
        \textbf{Step 2.} \(\tilde{f} \vert_D = f\)임을 보이자. \(x \in D\)에 대해, \(x\)로 수렴하는 \(D\) 안의 수열 \((x_n)_{n \in \NN}\)이 존재한다. 이때 \(\tilde{f}(x)\)의 정의에 의해
        \[
        \tilde{f}(x) = \lim_{n } f(x_n) = f(x)    
        \]
        이다.\\
        \textbf{Step 3.} 마지막으로 \(\tilde{f}\)가 \(X\)에서 고른연속인 것만 보이면 된다. \(\eps > 0\)에 대하여 (\ref{eq7.5})를 만족하는 \(\delta > 0\)이 존재한다. 우리의 주장은
        \[
            d_X(p, q) < \frac{\delta}{4} \implies d_Y(\tilde{f}(p), \tilde{f}(q)) \le \eps \quad (p, q \in X) 
        \]
        라는 것이다. \(d_X(p, q) < \delta/4\)라고 하고, \(p, q\)로 각각 수렴하는 \(D\) 안의 수열 \((p_n)_{n \in \NN}, (q_n)_{n \in \NN}\)을 잡자. 이때 충분히 큰 자연수 \(N\)에 대하여
        \[
            n \ge N \implies d_X(p_n, p) < \frac{\delta}{4}, d_X(q_n, q) < \frac{\delta}{4}
        \]
        가 성립한다. 따라서 \(n, m \ge N\)이면
        \[
        d_X(p_n, q_m) \le d_X(p_n, p) + d_X(p, q) + d_X(q, q_m) < \delta    
        \]
        이므로
        \[
        d_Y(f(p_n), f(q_m)) < \eps    
        \]
        이다. 이제 양변에 \(m \to \infty\)를 취하면, 고정된 \(n\)에 대하여 \(d_Y(f(p_n), f(\cdot))\)가 연속함수이므로
        \[
        d_Y(f(p_n), f(q)) \le \eps    
        \]
        를 얻는다. 또 \(n \to \infty\)를 취하면 \(d_Y(f(\cdot), f(q))\)가 연속함수이므로
        \[
        d_Y(f(p), f(q)) \le \eps    
        \]
        를 얻는다. 따라서 증명이 완료되었다.
    \end{enum}
\end{proof}

\section{단조함수}

이 절에서는 실수의 구간에서 정의된 실함수만을 다룬다. 먼저 표기의 편리함을 위해 다음을 정의하자.

\begin{defn}
    임의의 실수 \(c \in \RR\)에 대하여 \((c, \infty]\)는 \(\RR \cup \{\pm \infty\}\)에서 \(\infty\)의 근방이고, \([-\infty, c)\)는 \(\RR \cup \{\pm \infty\}\)에서 \(-\infty\)의 근방이다.
\end{defn}

이 절이 끝날 때까지 \(-\infty \le a < b \le \infty\)이다.

\begin{defn}
    함수 \(f : (a, b) \to \RR\)와 \(\alpha \in \RR \cup \{\pm\infty\}\)에 대하여
    \[
    \lim_{x \to a+} f(x) = \alpha    
    \]
    라는 것은 \(\alpha\)의 \(\RR \cup \{\pm \infty\}\)에서 임의의 근방 \(V\)에 대하여, \(a\)의 \([a, b)\)에서의 근방 \(U\)가 존재하여 \(f(U \cap (a, b)) \subseteq V\)인 것이다. 이때 \(\alpha\)를 \(x \to a+\) 또는 \(x \searrow a\)일 때 \(f\)의 \textbf{우극한(right-handed limit)}이라고 하고, \(f(a+) := \alpha\)로 쓴다. 같은 방법으로 \(x \to b-\) 또는 \(x \nearrow b\)일 때 \(f\)의 \textbf{좌극한(left-handed limit)}을 정의한다.
\end{defn}

다음 명제들은 거의 당연하므로 증명은 생략한다.

\begin{prop}
    함수 \(f : (a, b) \to \RR\)와 \(\alpha \in \RR \cup \{\pm \infty\}\)에 대하여 다음이 동치이다.
    \begin{enum}
        \item \(f(a+) = \alpha\)일 필요충분조건은, \(x_n \to a\)인 \((a, b)\) 안의 임의의 수열 \((x_n)_{n \in \NN}\)이 \(f(x_n) \to \alpha\)를 만족하는 것이다.
        \item \(f(b-) = \alpha\)일 필요충분조건은, \(x_n \to b\)인 \((a, b)\) 안의 임의의 수열 \((x_n)_{n \in \NN}\)이 \(f(x_n) \to \alpha\)를 만족하는 것이다.
    \end{enum}
\end{prop}

\begin{prop}
    함수 \(f : (a, b) \to \RR\)와 \(c \in (a, b), \alpha \in \RR\)에 대하여 다음이 성립한다.
    \begin{enum}
        \item \(x \to c\)일 때 \(f(x) \to \alpha\)일 필요충분조건은 \(f(c+)\)와 \(f(c-)\)의 값이 모두 존재하고 \(f(c+) = f(c-) = \alpha\)인 것이다.
        \item \(f\)가 \(c\)에서 연속일 필요충분조건은 \(f(c+)\)와 \(f(c-)\)의 값이 모두 존재하고 \(f(c+) = f(c-) = f(c)\)인 것이다.
    \end{enum}
\end{prop}

\begin{defn}
    함수 \(f : (a, b) \to \RR\)가 \textbf{단조증가함수(monotonically increasing function; nondecreasing function)}이라는 것은 [\(a < x < y < b\)이면 \(f(x) \le f(y)\)]라는 것이다. 마찬가지 방법으로 \textbf{단조감소함수(monotonically decreasing function; nonincreasing function)}도 정의한다. 단조증가함수 또는 단조감소함수를 \textbf{단조함수(monotonic function)}라고 한다.
\end{defn}

다음 정리는 단조함수에서 좌극한과 우극한의 존재성을 보장해 준다. 

\begin{lem} \label{lem 7.5.6}
    단조증가함수 \(f : (a, b) \to \RR\)와 \(x \in (a, b]\)에 대해
    \[
        E_x := \{f(t) \in \RR : a < t < x\}, \quad \alpha := \sup E_x \in \RR \cup \{\infty\}
    \]
    라 하면, \(\alpha\)의 임의의 근방 \(V\)에 대하여 \(t \in (a, x)\)가 존재하여 \(f((t, x)) \subseteq V\)이다.
\end{lem}
\begin{proof}
    \(\alpha = \infty\)인 경우 \(V\)는 어떤 실수 \(c\)에 대해 \((c, \infty]\) 꼴이다. \(\alpha \in \RR\)인 경우, 적당한 양수 \(\eps > 0\)에 대해 \((\alpha - \eps, \alpha + \eps) \subseteq V\)이다. 따라서 처음부터 \(V = (\alpha - \eps, \alpha + \eps)\)였다고 가정해도 무방하다. 이때 상한의 정의에 의해 어떤 \(t \in (a, x)\)가 존재하여 \(f(t) \in V\)인데, \(f\)가 단조증가함수이므로 \(f((t, x)) \subseteq V\)이다.
\end{proof}

\begin{thm}
    단조증가함수 \(f : (a, b) \to \RR\)와 \(x \in (a, b)\)에 대해 \(f(x+), f(x-) \in \RR\)가 존재하고, \(f(a+), f(b-) \in \RR \cup \{\pm\infty\}\)가 존재한다. 구체적으로,
    \begin{equation}\label{eq7.6}
        f(x-) = \sup_{a<t<x}f(t) \ (x \in (a, b]), \quad f(x+) = \inf_{x<t<b}f(t) \ (x \in [a, b))
    \end{equation}
    이다. 따라서 \(x \in (a, b)\)에 대해 \(f(x-) \le f(x) \le f(x)\)이고, \(a \le x < y \le b\)에 대해 \(f(x+) \le f(y-)\)이다.
\end{thm}
\begin{proof}
    (\ref{eq7.6})만 증명하면 아래 내용은 당연하다. (\ref{eq7.6})의 두 식이 대칭적이므로 첫 번째 식만 보이겠다. \(x \in (a, b]\)에 대하여 \ref{lem 7.5.6}에서와 같이 \(E_x\)와 \(\alpha\)를 정의하면, \(\alpha\)의 임의의 근방 \(V\)에 대하여 \(f((t, x)) \subseteq V\)인 \(t \in (a, x)\)가 존재한다. \(U := (t, x]\)로 두면 \(U\)는 \(x\)의 \((a, x]\)에서의 근방이고, \(f(U \cap (a, x)) \subseteq V\)이다. 따라서 \(\alpha = f(x-)\)이다.
\end{proof}


\begin{cor}
    단조함수 \(f : (a, b) \to \RR\)의 불연속점은 가산집합이다.
\end{cor}
\begin{proof}
    일반성을 잃지 않고 \(f\)가 단조증가함수라고 하고, \(f\)의 모든 불연속점의 집합을 \(D\)라 하자. 그리고 유리수 집합 \(\QQ\)를 \(\{r_n\}_{n \in \NN}\)으로 쓰자. 이때 각 점 \(x \in D\)에 대하여 \(f(x-) < f(x+)\)이므로, \(\QQ \cap (f(x-), f(x+))\)는 공집합이 아니고, \(r_n \in (f(x-), f(x+))\)인 최소의 자연수 \(n\)이 존재한다. 이때 \(r(x) := r_n\)으로 쓰자. 그러면 \(x \mapsto r(x)\)는 단사함수이므로 \(\abs{D} \le \abs{\QQ}\)이다.
\end{proof}

\section{Appendix: 경로연결집합}

\begin{defn}
    \(X\)의 두 원소 \(x, y \in X\)에 대하여, \(x\)와 \(y\)를 연결하는 \textbf{경로(path)}는 \(f(0) = x, f(1) = y\)를 만족하는 연속함수 \(f : [0, 1] \to X\)이다. \(P \subseteq X\)가 \textbf{경로연결집합(path connected set)}이라는 것은 임의의 \(x, y \in P\)에 대하여 \(x\)와 \(y\)를 연결하고 그 상이 \(P\)에 포함되는 경로가 존재한다는 것이다.
\end{defn}

\begin{rem}
    \(f\)의 정의역은 실수의 유계닫힌구간이기만 하면 된다.
\end{rem}

\begin{ex}
    \quad

    \begin{enum}
        \item 노름공간 \(V\)의 볼록집합 \(C \subseteq V\)는 경로연결집합이다. 임의의 \(x, y \in C\)에 대하여 선분 \([x, y]\)가 \(x\)와 \(y\)를 연결하고 그 상이 \(C\)에 포함되는 경로이기 때문이다.
        \item \(X\)의 경로연결집합들의 집합족 \(\{P_i\}_{i \in I}\)에 대하여 \(\bigcap_{i \in I} P_i \ne \varnothing\)이면 \(P := \bigcup_{i \in I} P_i\)는 경로연결집합이다. 가정에 의해 어떤 \(x_0 \in \bigcap_{i \in I} P_i\)가 존재한다. 임의의 두 점 \(x, y \in P\)에 대해 \(x \in P_{i_1}, y \in P_{i_2}\)인 \(i_1, i_2 \in I\)가 존재한다. 이때 \(x\)와 \(x_0\)을 연결하는 \(P_{i_1}\) 안의 경로 \(f : [0, 1] \to X\)가 존재하고, \(x_0\)와 \(y\)를 연결하는 \(P_{i_2}\) 안의 경로 \(g : [1, 2] \to X\)가 존재한다. 이제 \(h : [0, 2] \to X\)를
        \[
        h : t \mapsto \begin{cases}
            f(t) &\textif \ t \in [0, 1]\\
            g(t) &\textif \ t \in [1, 2]
        \end{cases}    
        \]
        로 정의하면 \(h\)는 명제 \ref{prop 7.1.13}에 의해 그 상이 \(P\)에 포함되는 연속함수이고, 따라서 \(x\)와 \(y\)를 연결하는 \(P\) 안의 경로이다. 따라서 \(P\)는 경로연결집합이다.
    \end{enum}
    
\end{ex}

\begin{thm} \label{thm 7.6.2}
    경로연결집합은 연결집합이다.
\end{thm}
\begin{proof}
    \(P \subseteq X\)가 공집합이 아닌 경로연결집합이라고 하자. 한 점 \(x_0 \in X\)를 고정하면, 임의의 \(x \in P\)에 대해 \(x_0\)와 \(x\)를 연결하는 \(P\) 안의 경로가 존재한다. 이 경로의 상을 \(\gamma_x \subseteq P\)라 하면, \(\gamma_x\)는 연속함수에 의한 구간 \([0, 1]\)의 상이므로 연결집합이다. 한편
    \[
        \bigcup_{x \in P} \gamma_x = P, \quad x_0 \in \bigcap_{x \in P} \gamma_x
    \]
    이므로 \(P\)는 연결집합이다.
\end{proof}

\begin{ex}
    \textbf{위상수학자의 사인 곡선(topologist's sine curve)}은 다음과 같이 정의된다.
    \[
      S := \{\paren{x, \sin \frac{1}{x}} \in \RR^2 : x \in (0, 1]\}  
    \]
    이때 \(S\)는 연속함수에 의한 구간 \((0, 1]\)의 상이므로 \(\RR^2\)의 연결집합이고, 따라서 \(\overline{S}\)도 연결집합이다. 다음을 쉽게 확인할 수 있다.
    \[
      \overline{S} = S \cup \{(0, y) \in \RR^2 : y \in [-1, 1]\} 
    \]
    그런데 \(\overline{S}\)는 경로연결집합이 아니다. 이는 \ref{thm 7.6.2}의 역이 성립하지 않는 대표적인 반례이다.\\
    \(\overline{S}\)가 경로연결집합이라고 가정하자. 그러면 \((0, 0)\)과 \((1, \sin 1)\)을 연결하는 \(\overline{S}\) 안의 경로 \(f : [0, 1] \to \overline{S}\)가 존재한다. 이때
    \[
        E := \{t \in [0, 1] : f(t) = (0, 0)\}, \quad t_0 := \sup E  
    \]
    으로 정의하자. \(f\)가 연속함수이므로 \(E\)는 \([0, 1]\)의 닫힌집합이고, 따라서 \(t_0 = \max E\)이다. 한편 \(f\)가 \(t_0\)에서 연속이므로 양수 \(\delta > 0\)이 존재하여
    \[
        \abs{t - t_0} < \delta \implies \norm{f(t)} < 1 \quad (t \in [0, 1])
    \]
    이 성립한다. \(t_1 := t_0 + \delta / 2\)라고 하면, \(x_1 := (\pi_1 \circ f) (t_1) > 0\)이다. 이때 다음을 만족하는 자연수 \(N\)이 존재한다.
    \[
    x_2 := \frac{1}{2N\pi + \pi/2} < x_1   
    \]
    \(\pi_1 \circ f\)가 연속함수이므로, 사잇값정리에 의해 \((\pi_1 \circ f)(t_2) = x_2\)인 \(t_2 \in [t_0, t_1]\)이 존재한다. 이제 \(\abs{t_2 - t_0} < \delta\)이지만
    \[
    \norm{f(t_2)} = \norm{\paren{x_2, \sin \frac{1}{x_2}}} = \norm{(x_2, 1)} \ge 1
    \]
    이므로 모순이다. 따라서 \(\overline{S}\)는 경로연결집합이 아니다.
\end{ex}

\(F = \RR\) 또는 \(\CC\)일 때 \(F^d\)의 열린집합에 대해서는 \ref{thm 7.6.2}의 역이 성립한다. 먼저 보조정리 하나를 보이자.

\begin{lem} \label{lem 7.6.3}
    \(F^d\)의 열린집합 \(U \subseteq F^d\)와 \(x \in U\)에 대하여, 경로연결집합인 \(x\)의 적당한 근방 \(P\)가 존재하여 \(P \subseteq U\)이다.
\end{lem}
\begin{proof}
    \(F^d\)의 열린 공들은 볼록집합이므로 경로연결집합임을 알고 있다. \(U\)가 열린집합이므로 \(x\)를 중심으로 하며 \(U\)에 포함되는 열린 공이 존재하며, 이를 \(P\)라고 하면 된다.
\end{proof}


\begin{thm}
    \(F^d\)의 열린집합 \(U \subseteq F^d\)에 대하여 \(U\)가 연결집합인 것과 \(U\)가 경로연결집합인 것이 동치이다.
\end{thm}
\begin{proof}
    \((\Leftarrow)\) 방향만 보이면 된다. 임의의 \(x \in U\)에 대해 집합 \(U_x\)를 다음과 같이 정의하자.
    \[
      U_x := \{y \in U : x\text{와} \ y\text{를 연결하는} \ U \ \text{안의 경로가 존재한다.}\}  
    \]
    이때 보조정리 \ref{lem 7.6.3}에 의해 각 \(U_x\)는 \(U\)의 열린집합이다. 이제 \(U\)가 공집합이 아니라고 하고 한 점 \(x_0 \in U\)를 고정하자. 이때
    \[
    U \setminus U_{x_0} = \bigcup_{x \notin U_{x_0}} U_x    
    \]
    인 것을 쉽게 확인할 수 있다. 따라서 \(U_{x_0}\)는 \(U\)의 열린집합이면서 닫힌집합이다. 그런데 \(x_0 \in U_{x_0}\)이므로 \(U_{x_0}\)는 공집합이 아니고, \(U\)가 연결집합이므로 \(U = U_{x_0}\)이다. \(U_{x_0}\)가 경로연결집합이므로 \(U\)도 경로연결집합이다.
\end{proof}



\chapter{미분}

이 장에서는 새로운 내용은 딱히 없다. 미적분학에서 배운 일변수 실함수의 미분과 평균값 정리, 테일러 정리를 증명과 함께 복습할 것이다.

\section{미분의 정의}

\begin{defn}
    구간 \(I = (a, b)\)에서 정의된 함수 \(f : I \to \RR\)와 \(x_0 \in I\)에 대하여 극한
    \[
        \lim_{h \to 0} \frac{f(x_0+h) - f(x_0)}{h} 
    \]
    가 존재하면
    \begin{equation}\label{eq8.1}
        f'(x_0) := \lim_{h \to 0} \frac{f(x_0+h) - f(x_0)}{h} 
    \end{equation}
    로 정의하고 \(f\)가 \textbf{\(x_0\)에서 미분가능(differentiable at \(x_0\))}하다고 한다. 정의역이 \([a, b), (a, b], [a, b]\)의 꼴일 때 \(a\)나 \(b\)에서의 미분가능성은 (\ref{eq8.1})에서 우극한과 좌극한을 이용하여 정의한다. 구간 \(I\)에서 정의된 \(f : I \to \RR\)가 모든 \(x_0 \in I\)에서 미분가능하면 \(f\)는 \textbf{\(I\)에서 미분가능(differentiable on \(I\))}하다고 한다.
\end{defn}

이 장이 끝날 때까지 \(I\)는 아무 말이 없으면 실수의 구간이다.

다음 명제들의 증명은 고등학교 때와 완전히 같으므로 생략한다.

\begin{prop}
    함수 \(f : I \to \RR\)가 \(x_0 \in I\)에서 미분가능하면 \(f\)는 \(x_0\)에서 연속이다.
\end{prop}

\begin{prop}
    함수 \(f, g : I \to \RR\)가 \(x_0 \in I\)에서 미분가능할 때 다음이 성립한다.
    \begin{enum}
        \item \(f+g\)가 \(x_0\)에서 미분가능하고 \((f+g)'(x_0) = f'(x_0) + g'(x_0)\)이다.
        \item 실수 \(c \in \RR\)에 대해 \(cf\)가 \(x_0\)에서 미분가능하고 \((cf)'(x_0) = cf'(x_0)\)이다.
        \item \(fg\)가 \(x_0\)에서 미분가능하고 \((fg)'(x_0) = f'(x_0)g(x_0) + f(x_0)g'(x_0)\)이다.
        \item \(g(x_0) \ne 0\)이면 \(f/g\)도 \(x_0\)에서 미분가능하고 \((f/g)'(x_0) = [f'(x_0)g(x_0) - f(x_0)g'(x_0)] / [g(x_0)]^2\)이다.
    \end{enum}
\end{prop}

미분가능성의 동치 조건을 소개하기 전에, 먼저 표기법 하나를 소개한다.

\begin{notn}
    \(a \in \RR \cup \{\pm \infty\}\)를 제외한 근방에서 정의된 실함수 \(f, g\)에 대하여 \(g\)가 양의 값만을 가진다고 하자. 이때, \(x \to a\)일 때
    \[
    f(x) = o(g(x))
    \]
    라는 것은
    \[
    \lim_{x \to a} \frac{f(x)}{g(x)} = 0    
    \]
    이라는 것이다. 또 \(x \to a\)일 때
    \[
    f(x) = O(g(x))
    \]
    라는 것은
    \[
    \limsup_{x \to \infty} \frac{\abs{f(x)}}{g(x)} < \infty
    \]
    라는 것이다.
\end{notn}

\begin{prop} \label{prop 8.1.5}
    열린구간 \(I \subseteq \RR\)에서 정의된 함수 \(f : I \to \RR\)와 \(x_0 \in I, L \in \RR\)에 대하여 다음이 동치이다.
    \begin{enum}
        \item \(f\)가 \(x_0\)에서 미분가능하고 \(f'(x_0) = L\)이다.
        \item 원점 근방에서 정의된 함수 \(\eta\)가 존재하여 \(f(x_0 + h) - f(x_0) = L h + \eta(h)\)이고 \(\eta(h) = o(\abs{h})\).
    \end{enum}
\end{prop}

\begin{proof}
    \quad \((\Rightarrow)\) 방향은 거의 당연하다. \((\Leftarrow)\) 방향을 보이기 위해 (b)를 가정하고 미분가능성의 정의를 직접 쓰면
    \[
    \lim_{h \to 0} \frac{f(x_0 + h) - f(x_0)}{h} = \lim_{h \to 0} \frac{L h + \eta(h)}{h} = L 
    \]
    이다.
\end{proof}

\begin{prop}[연쇄법칙]
    열린구간 \(I, J\)에서 정의된 두 함수 \(f : I \to \RR, g : J \to \RR\)에 대하여 \(f(I) \subseteq J\)라 하자. \(f\)가 \(x_0 \in I\)에서 미분가능하고 \(g\)가 \(f(x_0) \in J\)에서 미분가능하면 함수 \(g \circ f : I \to \RR\)도 \(x_0\)에서 미분가능하고,
    \[
    (g \circ f)'(x_0) = g'(f(x_0))f'(x_0)    
    \]
    이다.
\end{prop}
\begin{proof}
    \(L := f'(x_0), M := g'(f(x_0))\)라 하자. 원점 근방에서 정의된 함수 \(\eta_1, \eta_2\)가 존재하여, \(h \to 0\)일 때 \(\eta_1(h), \eta_2(h) = o(\abs{h})\)이고
    \[
    f(x_0 + h) - f(x_0) = L h + \eta_1(h), \quad g(f(x_0) + h) - g(f(x_0)) = M h + \eta_2(h)
    \]
    이다. 이제
    \begin{align*}
        (g \circ f)(x_0 + h) - (g \circ f)(x_0) &= M[f(x_0 + h) - f(x_0)] + \eta_2(f(x_0 + h) - f(x_0))\\
        &= M[Lh + \eta_1(h)] + \eta_2(f(x_0 + h) - f(x_0))
    \end{align*}
    이고, 임의의 \(\eps > 0\)에 대하여 \(\abs{h}\)가 충분히 작으면
    \begin{align*}
        \abs{(g \circ f)(x_0 + h) - (g \circ f)(x_0) - MLh} &= \abs{M\eta_1(h) + \eta_2(f(x_0 + h) - f(x_0))}\\
        &\le M\eps\abs{h} + \eps\abs{f(x_0+h) - f(x_0)}\\
        &\le M\eps\abs{h} + \eps\abs{Lh + \eta_2(h)} = o(\abs{h})
    \end{align*}
    이다. 따라서
    \[
        (g \circ f)'(x_0) = ML = g'(f(x_0))f'(x_0)
    \]
    이 성립한다.
\end{proof}

\section{평균값 정리}

코시의 평균값 정리로부터 평균값 정리와 로피탈의 정리를 증명한다.

\begin{lem} \label{lem 8.2.1}
    \(f : (a, b) \to \RR\)가 \(c \in (a, b)\)에서 최댓값 또는 최솟값을 가지고 \(c\)에서 미분가능하면 \(f'(c) = 0\)이다.
\end{lem}
\begin{proof}
    일반성을 잃지 않고 \(f\)가 \(c\)에서 최댓값을 가진다고 하자. 그러면 \(h \ne 0\)에 대하여
    \[
    h > 0 \implies \frac{f(c + h) - f(c)}{h} \le 0, \quad h < 0 \implies \frac{f(c + h) - f(c)}{h} \ge 0
    \]
    이고, \(f\)가 \(c\)에서 미분가능하므로
    \[
    f'(c) = \lim_{h \searrow 0} \frac{f(c + h) - f(c)}{h} \le 0, \quad  f'(c) = \lim_{h \nearrow 0} \frac{f(c + h) - f(c)}{h} \ge 0  
    \]
    이다. 따라서 \(f'(c) = 0\)이다. 
\end{proof}

\begin{prop}[롤의 정리]
    \(f : [a, b] \to \RR\)가 \([a, b]\)에서 연속이고 \((a, b)\)에서 미분가능할 때, \(f(a) = f(b)\)이면 \(f'(c) = 0\)인 \(c \in (a, b)\)가 존재한다.
\end{prop}
\begin{proof}
    최대최소 정리에 의해 \(f\)는 \([a, b]\)에서 최댓값과 최솟값을 가진다. 만약 \(f(a) = f(b)\)가 \(f\)의 최댓값이자 최솟값이면 \(f\)는 상수함수이므로 자명하다. 만약 최댓값 또는 최솟값이 \(f(a) = f(b)\)의 값과 다르면, 어떤 \(c \in (a, b)\)에서 \(f\)가 최댓값 또는 최솟값을 가진다. 보조정리 \ref{lem 8.2.1}에 의해 \(f'(c) = 0\)이다.
\end{proof}

\begin{rem}
    \(f : [a, b] \to \RR\)가 \([a, b]\)에서 연속이고 \((a, b)\)에서 미분가능한 것이 \(f\)가 \([a, b]\)에서 미분가능함을 함축하지 않는다. 대표적인 반례로 다음과 같이 정의된
    \[
    f : x \mapsto \begin{cases}
        x \sin \frac{1}{x} &\textif \ x \in (0, 1]\\
        0 &\textif \ x = 0
    \end{cases}   
    \]
    가 있다 \(f\)는 \([0, 1]\)에서 연속이고 \((0, 1)\)에서 미분가능하지만 \(x = 0\)에서 미분가능하지 않다.
\end{rem}

\begin{thm}[코시의 평균값 정리]
    \(f, g : [a, b] \to \RR\)가 \([a, b]\)에서 연속이고 \((a, b)\)에서 미분가능하면,
    \[
    f'(c)[g(b) - g(a)] = g'(c)[f(b) - f(a)]    
    \]
    를 만족하는 \(c \in (a, b)\)가 존재한다.
\end{thm}
\begin{proof}
    \([a, b]\)에서 \(h(x) := [f(b) - f(a)]g(x) - [g(b) - g(a)]f(x)\)로 정의하면, \(h\)는 \([a, b]\)에서 연속이고 \((a, b)\)에서 미분가능하다. 또
    \[
    h(a) = f(b)g(a) - g(b)f(a) = h(b)    
    \]
    이므로, \(h\)에 롤에 정리를 적용하면 원하는 \(c \in (a, b)\)의 존재를 얻는다.
\end{proof}

\begin{cor} \label{cor 8.2.4}
    \(f, g : [a, b] \to \RR\)가 \([a, b]\)에서 연속이고 \((a, b)\)에서 미분가능하다고 하자. 이때 임의의 \(x \in (a, b)\)에 대해 \(g'(x) \ne 0\)이면
    \[
      \frac{f'(c)}{g'(c)} = \frac{f(b) - f(a)}{g(b) - g(a)}  
    \]
    를 만족하는 \(c \in (a, b)\)가 존재한다.
\end{cor}

\begin{cor}[평균값 정리]
    \(f : [a, b] \to \RR\)가 \([a, b]\)에서 연속이고 \((a, b)\)에서 미분가능하면 \(f(b) - f(a) = f'(c)(b - a)\)를 만족하는 \(c \in (a, b)\)가 존재한다.
\end{cor}
\begin{proof}
    코시의 평균값 정리에서 \(g(x) = x\)로 두면 된다.
\end{proof}

\begin{thm}[로피탈의 법칙] \label{thm 8.2.5}
    어떤 \(r > 0\)에 대하여 \((x_0 - r, x_0 + r) \setminus \{x_0\}\)에서 정의된 두 실함수 \(f, g\)가 다음을 만족한다고 하자.
    \begin{enum}
        \item \(x \to x_0\)일 때 \(f(x), g(x) \to 0\)이다.
        \item \(f, g\)는 \((x_0 - r, x_0 + r) \setminus \{x_0\}\)에서 미분가능하다.
        \item 임의의 \(x \in (x_0 - r, x_0 + r) \setminus \{x_0\}\)에 대하여 \(g'(x) \ne 0\)이다.\footnote{사실, (d)가 성립하는 것이 (c)를 함축한다.}
        \item \(x \to x_0\)일 때 \(f(x) / g(x) \to L \in \RR \cup \{\pm \infty\}\)이다.
    \end{enum}
    그러면 \(x \to x_0\)일 때 \(f(x)/g(x) \to L\)이다.
\end{thm}
\begin{proof}
    \(V \subseteq \RR \cup \{\infty\}\)가 \(L\)의 임의의 근방이라고 하자. 이때 \(L \in \RR\)이면 \(V = (L - \eps, L + \eps)\), \(L = \infty\)이면 \(V = (R, \infty]\), \(L = -\infty\)이면 \(V = [-\infty, M)\)의 형태라 생각해도 무방하다. (d)에 의해 다음을 만족하는 \(\delta > 0\)이 존재한다.
    \[
    0 < \abs{x - x_0} < \delta \implies \frac{f'(x)}{g'(x)} \in V
    \]
    \(x \in (x_0 - \delta, x_0 + \delta) \setminus \{x_0\}\)을 고정하고 구간 \([x_0, x]\) 또는 \([x, x_0]\)에서 따름정리 \ref{cor 8.2.4}를 적용하면, 다음을 만족하는 \(c_x \in (x_0, x)\) 또는 \(c_x \in (x, x_0)\)가 존재한다.
    \[
        \frac{f'(c_x)}{g'(c_x)} = \frac{f(x) - f(x_0)}{g(x) - g(x_0)} = \frac{f(x)}{g(x)}
    \]
    이때 \(0 < \abs{c_x - x_0} < \delta\)이므로
    \[
        \frac{f(x)}{g(x)} = \frac{f'(c_x)}{g'(c_x)} \in V
    \]
    가 성립한다. \(V\)가 \(L\)의 임의의 근방이었으므로
    \[
      \lim_{x \to x_0} \frac{f(x)}{g(x)} = L  
    \]
    이 성립한다.
\end{proof}

\begin{cor}
    정리 \ref{thm 8.2.5}에서 \(x_0 \in \pm \infty\)로 바꾸거나\footnote{물론 이 경우에 \(f, g\)의 정의역은 \((M, \infty)\) 또는 \((-\infty, M)\)이 되어야 할 것이다.} (a)를
    \begin{center}
        \(x \to x_0\)일 때 \(\abs{f(x)}, \abs{g(x)} \to \infty\)이다.
    \end{center}
    로 바꾸어도 결론이 성립한다.
\end{cor}
\begin{cor}
    \(f(1/x), g(1/x)\)이나 \(1/f(x), 1/g(x)\)에 정리 \ref{thm 8.2.5}\를 적용하면 된다.
\end{cor}

\begin{ex}
    \quad

    \begin{enum}
        \item \(f(x) = x + \sin x \cos x, g(x) = f(x) e^{\sin x}\)로 정의하면,  임의의 정수 \(k \in \ZZ\)에 대해 \(g'(2k\pi + \pi/2) = 0\)이므로 [\(x > M\)일 때 \(g'(x) \ne 0\)]인 \(M\)을 찾을 수 없다. 따라서 \(\lim_{x \to \infty} f'(x) / g'(x)\)라는 표현이 의미가 없다.
        \item
        \[
        f(x) = \begin{cases}
            x^2 \sin \frac{1}{x} &\textif \ x \ne 0\\
            0 &\textif \  x= 0
        \end{cases}, \quad
        g(x) = x
        \]
        로 정의하면
        \[
          \lim_{x \to 0} \frac{f'(x)}{g'(x)} = \lim_{x \to 0} 2x\sin\frac{1}{x} - \cos\frac{1}{x} \notin \RR \cup \{\pm\infty\}  
        \]
        이지만
        \[
          \lim_{x \to 0} \frac{f(x)}{g(x)}  = \lim_{x \to 0} x \sin \frac{1}{x} = 0
        \]
        이다.
    \end{enum}
\end{ex}

\begin{defn}
    함수 \(F\)가 정의역에서 미분가능하고 \(F' = f\)이면 \(F\)를 \(f\)의 \textbf{원시함수(primitive)}라 한다.
\end{defn}

\begin{thm}[다르부 정리\footnote{Darboux's theorem.}]
    함수 \(F : [a, b] \to \RR\)가 미분가능하고 \(F'(a) < \alpha < F'(b)\)이면 \(F'(c) = \alpha\)인 \(c \in (a, b)\)가 존재한다.
\end{thm}
\begin{proof}
    \(G : x \mapsto F(x) - \alpha x\)라 하면, \(G\)는 어떤 \(c \in [a, b]\)에서 최솟값을 가진다. 그런데 \(G'(a) < 0, G'(a) > 0\)이므로 \(G\)가 최솟값을 가지는 점은 \((a, b)\)에 존재한다. 따라서 \(G'(c) = F'(c) - \alpha = 0\)이므로 \(F'(c) = \alpha\)이다.
\end{proof}

\begin{cor}
    구간 \(I\)에서 정의된 함수 \(f : I \to \RR\)가 원시함수를 가지면 \(f(I)\)도 구간이다.
\end{cor}

\section{테일러 정리}

테일러 다항식은 여러 번 미분가능한 함수를 근사하는 가장 기초적인 방법이다.

\begin{lem} \label{lem 8.3.1}
    양의 정수 \(n\)과 \(n\)차 이하의 다항함수 \(P\)에 대하여, \(x \to a\)일 때 \(P(x) = o(\abs{x - a}^n)\)이면 \(P = 0\)이다.
\end{lem}
\begin{proof}
    \(P(x) = \sum_{k=0}^n a_k (x-a)^k\)로 쓰자. 이때
    \[
    \lim_{x \to 0} \frac{P(x)}{(x-a)^n} = \lim_{x \to 0} \sum_{k=0}^n a_k (x-a)^{k-n} = 0    
    \]
    이므로 각 \(k\)에 대해 \(a_k = 0\)이다.
\end{proof}

\begin{thm}
    구간 \(I \subseteq \RR\)에서 정의된 \(f : I \to \RR\)가 \(I\)에서 \(n\)번 미분가능할 때, \(a \in I\)에 대하여 다항함수 \(T_n\)을 다음과 같이 정의하자.
    \begin{equation} \label{eq8.2}
        T_n(x) := \sum_{k=0}^n \frac{f^{(k)(a)}}{k!} (x - a)^k
    \end{equation}
    이때 \(x \to a\)에서
    \begin{equation}\label{eq8.3}
        f(x) - T_n(x) = o(\abs{x-a}^n)
    \end{equation}
    이며, \(x \to a\)에서 (\ref{eq8.3})\를 만족하는 \(n\)차 이하의 다항함수 \(T_n\)은 유일하다.
\end{thm}
\begin{proof}
    먼저 유일성부터 보이자. \(n\)차 이하의 두 다항함수 \(P, Q\)가 (\ref{eq8.3})\를 만족하면, \(x \to a\)에서 \(P(x) - Q(x) = o(\abs{x-a}^n)\)이다. 따라서 보조정리 \ref{lem 8.3.1}에 의해 \(P = Q\)이다.\\
    이제 (\ref{eq8.2})\와 같이 정의된 \(T_n\)이 (\ref{eq8.3})\를 만족하는 것을 보이면 된다. 각 \(0 \le k \le n\)에 대해 \(f^{(k)}(a) = T_n^{(k)}(a)\)이므로, 로피탈의 법칙에 의해
    \begin{align*}
        \lim_{x \to a} \frac{f(x) - T_n(x)}{(x - a)^n} &= \lim_{x \to a} \frac{f'(x) - T_n'(x)}{n(x-a)^{n-1}}\\
        &= \ldots\\
        &= \lim_{x \to a} \frac{f^{(n-1)}(x) - T_n^{(n-1)}(x)}{n!(x-a)}\\
        &= \frac{1}{n!}[f^{(n)}(a) - T_n^{(n)}(a)] = 0
    \end{align*}
    이다.
\end{proof}

\begin{defn}
    구간 \(I \subseteq \RR\)에서 정의된 \(f : I \to \RR\)가 \(I\)에서 \(n\)번 미분가능할 때, \(a \in I\)에 대하여 (\ref{eq8.2})와 같이 정의된 다항함수 \(T_n\)을 \textbf{\(f\)의 \(a\)에서의 \(n\)차 테일러 다항식(\(n\)-th Taylor polynomial of \(f\) at \(a\))}이라고 하고 \(R_n(x) := f(x) - T_n(x)\)를 \textbf{\(f\)의 \(a\)에서의 \(n\)차 나머지항(\(n\)-th remainder term of \(f\) at \(a\))}이라고 한다.
\end{defn}

조건을 조금 더 추가하면 \(n\)차 나머지항을 직접적으로 쓸 수 있다.

\begin{defn} \label{8.3.4}
    구간 \(I \subseteq \RR\)에서 정의된 함수 \(f : I \to \RR\)에 대하여 다음을 정의한다.
    \begin{enum}
        \item \(f\)가 \(I\)에서 \textbf{\(C^0\)-함수}라는 것은 \(f\)가 \(I\)에서 연속함수라는 것이다.
        \item 음이 아닌 정수 \(n\)에 대해, \(f\)가 \(I\)에서 \textbf{\(C^{n+1}\)-함수}라는 것은 \(f\)가 \(I\)에서 미분가능하고 \(f'\)가 \(I\)에서 \(C^n\)-함수라는 것이다.
        \item \(f\)가 \(I\)에서 \textbf{\(C^\infty\)-함수} 또는 \textbf{매끄러운 함수(smooth function}라는 것은 임의의 음이 아닌 정수 \(n\)에 대해 \(f\)가 \(I\)에서 \(C^n\)-함수라는 것이다.
    \end{enum}
    \(n \in \NN \cup \{\infty\}\)에 대해 \(I\)에서 정의된 \(C^n\)-함수들의 집합을 \(C^n(I)\)로 쓴다.\footnote{\(C^0(I)\)는 \(C(I)\)로 쓰기도 한다.}
\end{defn}

\begin{ex}
    \quad

    \begin{enum}
        \item \(n \ge 0\)에 대하여 함수 \(f : \RR \to \RR\)를
        \[
            f : x \mapsto
            \begin{cases}
                x^{n+1} &\textif x \ge 0\\
                0 &\textif x < 0
            \end{cases}
        \]
        로 정의하면 \(f \in C^n(\RR)\)이지만 \(f \notin C^{n+1}(\RR)\)이다.
        \item 함수 \(f : \RR \to \RR\)를
        \[
            f : x \mapsto 
            \begin{cases}
                e^{-1/x} &\textif x > 0\\
                0 &\textif x \le 0
            \end{cases}
        \]
        로 정의하면 \(f \in C^\infty(I)\)이다. \(x \ne 0\)에서 무한 번 미분가능한 것은 당연하므로, \(x = 0\)만 체크하면 된다.\\
        \textbf{Claim.} 각 \(k \le 0\)에 대하여 다항식 \(P_k\)가 존재하여 \(x > 0\)일 때 \(f^{(k)}(x) = P_k(1/x)e^{-1/x}\)이고, \(f^{(k+1)}(0) = 0\)이다.
        \begin{proof}[Proof of Claim.]
            \(k = 0\)일 때는 \(P_0 = 1\)이다. 한편
            \[
            \lim_{x \searrow 0} \frac{f(x) - f(0)}{x - 0} = \lim_{x \searrow 0} \frac{e^{-1/x}}{x} = \lim_{t \to \infty} te^{-t} = 0 
            \]
            이므로 \(f'(0) = 0\)이다. 이제 어떤 \(k \ge 0\)에 대하여 Claim이 성립한다고 가정하자. 그러면 \(x > 0\)에서
            \[
            f^{(k+1)}(x) = \frac{1}{x^2}\sqbracket{P_k\paren{\frac{1}{x}} - P_k'\paren{\frac{1}{x}}}e^{-1/x}
            \]
            이므로 \(P_{k+1}(x) := x^2(P_k(x) - P_k'(x))\)로 두면 \(f^{(k+1)}(x) = P_{k+1}(1/x)e^{-1/x}\)이다. 또
            \[
            \lim_{x \searrow 0} \frac{f^{(k+1)}(x) - f^{(k+1)}(0)}{x-0} = \lim_{x \searrow 0} \frac{P_{k+1}(1/x)e^{-1/x}}{x} = \lim_{t \to \infty} tP_{k+1}(t)e^{-t} = 0    
            \]
            이므로 \(f^{(k+2)}(x) = 0\)이다. 따라서 모든 \(k \ge 0\)에 대해 Claim이 성립한다.
        \end{proof}
    \end{enum}
    
\end{ex}

\begin{thm}[테일러 정리]
    점 \(a\)를 포함하는 구간 \(I \subseteq \RR\)에서 정의된 실함수 \(f\)가 \(C^n\)-함수이고 \(f^{(n)}\)이 \(\Int I \setminus \{a\}\)에서 미분가능 하다고 하자. \(f\)의 \(a\)에서의 \(n\)차 테일러 다항식과 나머지항을 각각 \(T_n, R_n\)이라 쓰면, 임의의 \(x \in I \setminus \{a\}\)에 대해 다음을 만족시키는 \(c_x\)가 \(a\)와 \(x\) 사이에 존재한다.
    \begin{equation} \label{eq8.4}
        R_n(x) = \frac{f^{(n+1)}(c_x)}{(n+1)!} (x-a)^{n+1}
    \end{equation}
\end{thm}
\begin{proof}
    \(x \in I \setminus \{a\}\)를 고정하자. 일반성을 잃지 않고 \(a < x\)라 할 수 있다. 실수 \(\alpha\)를 다음과 같이 정의하자.
    \[
    \alpha := \frac{R_n(x)}{(x-a)^{n+1}}    
    \]
    그러면 (\ref{eq8.4})\는 다음과 동치이므로, 다음을 만족하는 \(c_x \in (a, x)\)를 찾으면 된다.
    \begin{equation} \label{eq8.5}
        f^{(n+1)}(c_x) = (n+1)! \cdot \alpha
    \end{equation}
    그런데 함수 \(g(t) := R_n(t) - \alpha(t-a)^{n+1}\)에 대하여,
    \[
        g^{(n+1)} = = R_n^{(n+1)}(t) - (n+1)! \cdot \alpha = f^{(n+1)}(t) - (n+1)! \cdot \alpha
    \]
    이므로 \(g^{(n+1)}(c_x) = 0\)을 만족하는 \(c_x \in (a, x)\)를 찾으면 된다. 한편
    \[
      g(a) = g'(a) = \ldots g^{(n)}(a) = 0
    \]
    이고, \(g(x) = 0\)이므로 평균값 정리에 의해 \(g'(x_1) = 0\)인 \(x_1 \in (a, x)\)가 존재한다. 다시 평균값 정리에 의해 \(g''(x_2) = 0\)인 \(x_2 \in (a, x)\)가 존재한다. 이를 반복하여 \(g^{(n+1)}(x_{n+1}) = 0\)인 \(x_{n+1} \in (a, x_n)\)을 찾을 수 있으며, \(c_x := x_{n+1}\)로 두면 증명이 끝난다.
\end{proof}

\begin{defn}
    열린구간 \(I \subseteq \RR\)에서 매끄러운 실함수 \(f\)와 \(a \in I\)에 대하여 다음 급수를 \textbf{\(f\)의 \(a\)에서의 테일러 급수(Taylor series of \(f\) at \(a\))}라고 한다.
    \[
    \sum_{n=0}^\infty \frac{f^{(n)}(a)}{n!} (x-a)^n = f(a) + f'(a)(x-a) + \frac{f''(a)}{2!}(x-a)^1 + \ldots + \frac{f^{(n)}(a)}{n!} (x-a)^n + \ldots
    \]
    \(f\)가 \textbf{\(a\)에서 해석적(analytic at \(f\))}이라는 것은, \(a\)의 적당한 근방 \((a - r, a + r) \subseteq I\)가 존재하여 모든 \(x \in (a-r, a+r)\)에 대해
    \[
    f(x) =     \sum_{n=0}^\infty \frac{f^{(n)}(a)}{n!} (x-a)^n
    \]
    이 성립한다는 것이다. 정의역의 모든 점에서 해석적인 함수를 \textbf{해석함수(analytic function)}라고 한다.
\end{defn}

\begin{ex}
    \quad

    \begin{enum}
        \item 다항함수는 해석함수이다.
        \item 앞의 예시에서와 같이 함수 \(f : \RR \to \RR\)를
        \[
            f : x \mapsto
            \begin{cases}
                e^{-1/x} &\textif x > 0\\
                0 &\textif x \le 0
            \end{cases}
        \]
        로 정의하면 \(f\)는 \(C^\infty\)-함수이지만 0에서 해석적이지 않다. \(f\)의 0에서의 테일러 급수는 항등적으로 0이기 때문이다.
    \end{enum}
\end{ex}

\(f^{(k)}\)들의 값들의 상계를 잡을 잡을 수 있으면 테일러 급수가 원래 함수로 수렴한다.

\begin{thm}
    열린구간 \(I \subseteq \RR\)에서 매끄러운 실함수 \(f\)와 서로 다른 두 점 \(a, x \in I\)에 대하여, \(a\)와 \(x\) 사이의 닫힌구간을 \(J\)라 하자. 어떤 양수 \(M\)이 존재하여 [모든 \(n \ge 0, t \in J\)에 대해 \(\abs{f^{(n)}(t)} \le M\)]이면 \(f(x) = \sum_{n=0}^\infty \frac{f^{(n)}(a)}{n!} (x-a)^n\)이다.
\end{thm}
\begin{proof}
    \(n\)차 나머지항 \(R_n\)이 점 \(x\)에서 0으로 수렴하는 것을 보이는 것과 같다. 각 \(n\)에 대하여, 테일러 정리에 의해 \(c_x \in J\)가 존재하여
    \[
    \abs{R_n(x)} = \abs{\frac{f^{(n+1)}(c_x)}{(n+1)!}(x-a)^{n+1}} \le M \frac{\abs{x-a}^{n+1}}{(n+1)!}
    \]
    이다. \(\abs{x-a}\)는 고정된 값이므로 \(\lim_{n \to \infty} R_n(x) = 0\)이 증명된다.
\end{proof}

\begin{ex}
	대표적인 함수들의 \(x=0\)에서의 테일러 급수와 그것들이 수렴하는 범위이다.
	\begin{align*}
		\sin x &= \sum_{n=0}^\infty \frac{(-1)^n}{(2n+1)!} x^{2n+1} = x - \frac{x^3}{3!} + \frac{x^5}{5!} - \frac{x^7}{7!} + \ldots \quad &(x \in \RR)\\
		\cos x &= \sum_{n=0}^\infty \frac{(-1)^n}{(2n)!} x^{2n} = 1 - \frac{x^2}{2!} + \frac{x^4}{4!} - \frac{x^6}{6!} + \ldots \quad &(x \in \RR)\\
		e^x &= \sum_{n=0}^\infty \frac{x^n}{n!} = 1 + x + \frac{x^2}{2!} + \frac{x^3}{3!} + \ldots \quad &(x \in \RR)\\
		\arctan x &= \sum_{n=0}^\infty \frac{(-1)^n}{2n+1} x^{2n+1} = x - \frac{x^3}{3} + \frac{x^5}{5} - \frac{x^7}{7!} + \ldots \quad &(-1 \le x \le 1)\\
		\log (1+x) &= \sum_{n=1}^\infty \frac{(-1)^{n-1}}{n} x^n = x - \frac{x^2}{2} + \frac{x^3}{3} - \frac{x^4}{4} + \ldots \quad &(-1 < x \le 1)
	\end{align*}
\end{ex}

테일러 급수에 대한 공부는 \ref{sec12.3}절에서 거듭제곱급수를 공부한 후에 계속된다.



\chapter{리만-스틸체스적분}

이 장에서는 고등학교에서 배운 적분의 엄밀한 정의인 리만적분과 그 일반화인 리만-스틸체스적분에 대해 공부한다. 어떤 함수가 리만-스틸체스적분가능하기 위한 충분조건들을 보고, 그 성질을 공부할 것이다. 이로부터 미적분학의 기본정리를 증명할 수 있다. 그리고 곡선의 길이와 유계변동함수에 대해서도 공부할 것이다.

\section{리만-스틸체스적분의 정의 (1)}

\begin{defn}
    유계닫힌구간 \([a, b] \subseteq \RR\) 안의 유한 개의 점 \(x_0, \ldots, x_n\)이
    \[
        x_0 = a < x_1 < \ldots < x_{n-1} < x_n = b
    \]
    를 만족할 때 집합 \(\{x_0, \ldots, x_n\}\)을 \([a, b]\)의 \textbf{분할(partition)}이라고 한다. \([a, b]\)의 모든 분할의 집합을 \(\calP([a, b])\)로 쓴다.
\end{defn}
\begin{notn}
    \quad

    \begin{enum}
        \item \(\{x_0, \ldots, x_n\}\)이 \([a, b]\)의 분할이라고 하면, 앞으로 아무 말이 없으면 \(x_0 = a < x_1 < \ldots < x_{n-1} < x_n = b\)인 것으로 생각한다.
        \item \([a, b]\)의 분할 \(\{x_0, \ldots, x_n\}\)이 주어졌을 때
        \[
            \Delta x_i := x_i - x_{i-1} \quad (i = 1, \ldots, n)
        \]
        으로 정의한다. 또 함수 \(\alpha : [a, b] \to \RR\)에 대하여
        \[
            \alpha_i := \alpha(x_i) \ (i = 0, \ldots, n), \quad \Delta \alpha_i := \alpha_i - \alpha_{i-1} \ (i = 1, \ldots, n)
        \]
        으로 정의한다.
        \item 어떤 구간에 대해 이야기하는지 명백하면 \(\calP([a, b])\)를 간단히 \(\calP\)로 쓰자.
    \end{enum}
\end{notn}

\begin{defn}
    구간 \([a, b]\)에서 정의된 유계 실함수 \(f\)와 단조증가함수 \(\alpha\)가 주어졌다고 하자. 구간 \([a, b]\)의 분할 \(P := \{x_0, \ldots, x_n\}\)에 대하여
    \[
        M_i := \sup_{x \in [x_{i-1}, x_i]} f(x), \quad m_i := \inf_{x \in [x_{i-1}, x_i]} f(x)
    \]
    로 쓰자. 이때 \textbf{상합(upper sum)}과 \textbf{하합(lower sum)}을 각각
    \[
        U_a^b(f, P, \alpha) := \sum_{i=1}^n M_i \Delta \alpha_i, \quad L_a^b(f, P, \alpha) := \sum_{i=1}^n m_i \Delta \alpha_i
    \]
    로 정의한다. \(f\)가 유계이므로 상합과 하합은 각각 유계인 실수로 잘 정의된다.\\
    각 \(x \in [a, b]\)에 대해 \(\alpha(x) = x\)이면 \(\alpha\)를 생략하여 상합과 하합을 각각 \(U_a^b(f, P), L_a^b(f, P)\)로 쓴다.\\
    어떤 구간에 대해 이야기하는지 명확하면 \(a, b\)를 생략하여 \(U(f, P, \alpha)\)와 같이 쓴다.
\end{defn}

\ref{sec9.3}절까지 아무 말이 없으면 \(f\)는 유계 실함수, \(\alpha\)는 단조증가함수이다.

\begin{defn}
    \([a, b]\)의 두 분할 \(P, P'\)에 대하여 \(P \subseteq P'\)이면 \(P'\)가 \(P\)의 \textbf{세분(refinement)}이라고 한다.
\end{defn}

\begin{prop} \label{9.1.5}
    \([a, b]\)의 두 분할 \(P, P'\)에 대하여 \(P'\)가 \(P\)의 세분이면
    \[
        L(f, P, \alpha) \le L(f, P', \alpha) \le U(f, P', \alpha), U(f, P, \alpha)
    \]
    이다.
\end{prop}
\begin{proof}
    두 번째 부등호는 정의에 의해 자명하다. 세 번째 부등호는, \(P'\)의 원소의 개수가 \(P\)의 원소의 개수보다 1개 더 많을 때만 보이면 귀납적으로 증명이 완료된다. \(P = \{x_0, \ldots, x_n\}\)이고 어떤 \(x' \in (x_{k-1}, x_k)\)에 대해 \(P' = P \cup \{x'\}\)이라 하자. 이제 정의에 의해
    \begin{align*}
        &U(f, P, \alpha) - U(f, P', \alpha)\\
        =\,& M_k \cdot \Delta \alpha_k - \sqbracket{\sup_{x \in [x_{k-1}, x']} f(x) \cdot (\alpha(x') - \alpha_{k-1}) + \sup_{x \in [x', x_k]} f(x) \cdot (\alpha_k - \alpha(x'))}\\
        \ge\,& M_k \cdot \Delta \alpha_k - \sqbracket{M_k \cdot (\alpha(x') - \alpha_{k-1}) + M_k \cdot (\alpha_k - \alpha(x'))} = 0
    \end{align*}
    이다. 첫 번째 부등호도 비슷하게 증명된다.
\end{proof}

\begin{cor} \label{9.1.6}
    구간 \([a, b]\)에서 정의된 유계 실함수 \(f\)와 단조증가함수 \(\alpha\)에 대해 다음이 성립한다.
    \[
        - \infty < \sup_{P \in \calP} L(f, P, \alpha) \le \inf_{P \in \calP} U(f, P, \alpha) < \infty
    \]
    즉 하합들의 집합은 위로 유계이고, 상합들의 집합은 아래로 유계이다.
\end{cor}
\begin{proof}
    임의의 두 분할 \(P, Q \in \calP\)에 대하여 \(L(f, P, \alpha) \le U(f, Q, \alpha)\)임을 보이는 것과 같다. 이때 \(P \cup Q\)는 \(P\)와 \(Q\) 각각의 세분이므로, 명제 \ref{9.1.5}에 의해
    \[
        L(f, P, \alpha) \le L(f, P \cup Q, \alpha) \le U(f, P \cup Q, \alpha) \le U(f, Q, \alpha)
    \]
    가 성립한다.
\end{proof}

\begin{defn}
    구간 \([a, b]\)에서 정의된 유계 실함수 \(f\)와 단조증가함수 \(\alpha\)에 대해, \textbf{상적분(upper integral)}과 \textbf{하적분(lower integral)}을 각각
    \[
        \upint{a}{b} f d\alpha := \sup_{P \in \calP} L(f, P, \alpha), \quad \lowint{a}{b} f d\alpha := \inf_{P \in \calP} U(f, P, \alpha)
    \]
    로 정의한다. 물론 따름정리 \ref{9.1.6}에 의해
    \[
        -\infty < \lowint{a}{b} f d\alpha \le \upint{a}{b} f d\alpha < \infty
    \]
    이다. 이때 상적분과 하적분의 값이 같으면 \(f\)가 \([a, b]\)에서 \textbf{\(\alpha\)에 관해 리만-스틸체스적분가능(Riemann-Stieltjes integrable with respect to \(\alpha\))}하다고 하고, \([a, b]\)에서 \(f \in \calR(\alpha)\)로 쓴다. 그리고 그 리만-스틸체스적분의 값을
    \[
        \int_a^b f d\alpha := \upint{a}{b} f d\alpha = \lowint{a}{b} f d\alpha
    \]
    로 정의한다.\\
    각 \(x \in [a, b]\)에 대해 \(\alpha(x) = x\)인 경우에 상적분과 하적분을 각각
    \[
        \upint{a}{b} f := \sup_{P \in \calP} L(f, P), \quad \lowint{a}{b} f := \inf_{P \in \calP} U(f, P)
    \]
    로 정의한다. 이때 상적분과 하적분의 값이 같으면 \(f\)가 \([a, b]\)에서 \textbf{리만적분가능(Riemann integrable)}하다고 하고, \([a, b]\)에서 \(f \in \calR\)로 쓴다. 그리고 그 리만적분의 값을
    \[
        \int_a^b f := \upint{a}{b} f = \lowint{a}{b} f 
    \]
    로 정의한다.
\end{defn}
\begin{notn}
    변수를 나타내고 싶은 경우에 다음과 같은 표기법도 사용한다. (이쪽이 더 익숙하다.)
    \[
        \int_a^b f(x) d\alpha(x) := \int_a^b f d\alpha, \quad \int_a^b f(x) dx := \int_a^b f
    \]
\end{notn}
\begin{rem}
    \([a, b]\)에서 정의된 유계 복소함수 \(f : [a, b] \to \CC\)에 대해서, 실수부분과 허수부분 각각의 리만-스틸체스적분가능성 봄으로써 \(f\)의 리만-스틸체스적분가능성과 그 적분값을 정의할 수 있다. 따라서 실함수에 대해 논하는 것만으로 충분할 것이다.
\end{rem}

\begin{defn}
    \([a, b]\)의 분할 \(P := \{x_0, \ldots, x_n\}\)에 대하여, 각 \(i\)에 대해 \(t_i \in [x_{i-1}, x_i]\)를 택한다. 이때 \textbf{리만-스틸체스합(Riemann-Stieltjes sum)}을
    \[
        S(f, P, \alpha) := \sum_{i=1}^n f(t_i)\Delta \alpha_i
    \]
    로 정의한다.\\
    각 \(x \in [a, b]\)에 대해 \(\alpha(x) = x\)인 경우에 \textbf{리만합(Riemann sum)}을
    \[
        R(f, P) := \sum_{i=1}^n f(t_i) \Delta x_i
    \]
    로 정의한다.
\end{defn}

\begin{prop} \label{9.1.10}
    분할 \(P \in \calP\)에 대하여
    \[
        \sup_{t_1, \ldots, t_n} S(f, P, \alpha) = U(f, P, \alpha), \quad \inf_{t_1, \ldots, t_n} S(f, P, \alpha) = L(f, P, \alpha)
    \]
    이다.
\end{prop}
\begin{proof}
    연습문제로 남긴다.
\end{proof}

\begin{lem} \label{9.1.11}
    \([a, b]\)에서 \(f \in \calR(\alpha)\)라 하고 \(\displaystyle A :=\int_a^b f d\alpha\)로 두자. 그러면 임의의 \(\eps > 0\)에 대해
    \begin{equation} \label{eq9.1}
        A - \eps < L(f, P_0, \alpha) \le U(f, P_0, \alpha) \le A + \eps
    \end{equation}
    를 만족하는 분할 \(P_0 \in \calP\)가 존재한다.
\end{lem}
\begin{proof}
    리만-스틸체스적분가능성의 정의에 의해 다음을 만족하는 두 분할 \(P, Q \in \calP\)가 존재한다.
    \[
        U(f, P, \alpha) < A + \eps, \quad L(f, Q, \alpha) > A - \eps
    \]
    이제 \(P_0 := P \cup Q\)로 두면 원하는 부등식을 얻는다.
\end{proof}

이제 \(f \in \calR(\alpha)\)일 첫 번째 필요충분조건이다.

\begin{thm} \label{9.1.12}
    구간 \([a, b]\)에서 정의된 \(f, \alpha\)에 대하여 다음이 동치이다.
    \begin{enum}
        \item \(f \in \calR(\alpha)\).
        \item 임의의 양수 \(\eps > 0\)에 대해 \(U(f, P, \alpha) - L(f, P, \alpha) < \eps\)를 만족하는 분할 \(P \in \calP\)가 존재한다.
        \item 어떤 실수 \(A \in \RR\)가 존재하여, 임의의 \(\eps > 0\)에 대해
        \[
            P \supseteq P_0 \implies \abs{S(f, P, \alpha) - A} < \eps \quad (P \in \calP)
        \]
        를 만족하는 분할 \(P_0 \in \calP\)가 존재한다.
    \end{enum}
    위의 조건들이 성립할 때, \(\displaystyle \int_a^b f d\alpha = A\)이다.
\end{thm}
\begin{proof}
    \quad\\
    (a)\(\Rightarrow\)(c) \(\displaystyle A := \int_a^b f d\alpha\)로 두면, 보조정리 \ref{9.1.11}에 의해 (\ref{eq9.1})\을 만족하는 분할 \(P_0 \in \calP\)가 존재한다. 이제 \(P \in \calP\)가 \(P_0\)의 세분이면, 명제 \ref{9.1.10}에 의해
    \[
        A - \eps < L(f, P_0, \alpha) \le L(f, P, \alpha) \le S(f, P, \alpha) \le U(f, P, \alpha) \le U(f, P_0, \alpha) \le A + \eps
    \]
    이므로 \(\abs{S(f, P, \alpha) - A} < \eps \)이다.\\
    (c)\(\Rightarrow\)(b) 임의의 양수 \(\eps > 0\)에 대해, 다음을 만족하는 분할 \(P_0 \in \calP\)가 존재한다.
    \[
        A - \frac{\eps}{3} < S(f, P_0, \alpha) < A + \frac{\eps}{3}
    \]
    이제 명제 \ref{9.1.10}에 의해
    \[
        A - \frac{\eps}{3} \le L(f, P_0, \alpha) \le U(f, P_0, \alpha) \le A + \frac{\eps}{3}
    \]
    이므로, \(U(f, P_0, \alpha) - L(f, P_0, \alpha) < \eps\)이다.\\
    (b)\(\Rightarrow\)(a) 임의의 양수 \(\eps > 0\)에 대해
    \[
        \upint{a}{b} f d\alpha - \lowint{a}{b} f d\alpha \le U(f, P, \alpha) - L(f, P, \alpha) < \eps
    \]
    이므로 \(\displaystyle \upint{a}{b} f d\alpha = \lowint{a}{b} f d\alpha\)이다.
\end{proof}

리만적분의 경우에는 조건 하나를 더 추가할 수 있다.

\begin{notn}
    \([a, b]\)의 분할 \(P := \{x_0, \ldots, x_n\}\)에 대하여 \(\norm{P} := \max_{i = 1,\ldots, n} \Delta x_i\)로 정의하자.
\end{notn}

\begin{lem}
    \(f \in \calR(\alpha)\)이면 상수 \(c_1, c_2 \in \RR\)에 대해 \(c_1 f + c_2 \in \calR(\alpha)\)이고
    \[
        \int_a^b (c_1 f + c_2) d\alpha= c_1 \int_a^b f d\alpha + c_2 (b-a)
    \]
    이다.
\end{lem}
\begin{proof}
    연습문제로 남긴다.
\end{proof}

\begin{thm} \label{9.1.15}
    구간 \([a, b]\)에서 정의된 \(f\)에 대하여 다음이 동치이다.
    \begin{enum}
        \item \(f \in \calR\).
        \item 임의의 양수 \(\eps > 0\)에 대해 \(U(f, P) - L(f, P) < \eps\)를 만족하는 분할 \(P \in \calP\)가 존재한다.
        \item 어떤 실수 \(A \in \RR\)가 존재하여, 임의의 \(\eps > 0\)에 대해
        \[
            P \supseteq P_0 \implies \abs{R(f, P) - A} < \eps \quad (P \in \calP)
        \]
        를 만족하는 분할 \(P_0 \in \calP\)가 존재한다.
        \item 어떤 실수 \(A \in \RR\)가 존재하여, 임의의 \(\eps > 0\)에 대해
        \[
            \norm{P} < \delta \implies \abs{R(f, P) - A} < \eps \quad (P \in \calP)
        \]
        를 만족하는 양수 \(\delta > 0\)이 존재한다.
    \end{enum}
    위의 조건들이 성립할 때, \(\displaystyle \int_a^b f d\alpha = A\)이다.
\end{thm}

\begin{proof}
    \quad\\
    (a)\(\Leftrightarrow\)(b)\(\Leftrightarrow\)(c)인 것은 증명하였다.\\
    (d)\(\Rightarrow\)(c) \(\norm{P_0} < \delta\)인 \(P_0 \in \calP\)가 (c)를 만족함은 분명하다.\\
    (a)\(\Rightarrow\)(d) \(\displaystyle A := \int_a^b f d\alpha\)로 두면, 임의의 양수 \(\eps > 0\)에 대해
    \[
        U(f, P_0) < A + \frac{\eps}{2}
    \]
    이도록 하는 분할 \(P_0 := \{x_0, \ldots, x_n\} \in \calP\)가 존재한다. 이제 \(\abs{f} < M\)인 양수 \(M\)에 대하여 \(\delta_1 := \eps / 2nM\)이라 하고 \(\norm{P} < \delta_1\)인 임의의 분할 \(P := \{y_0, \ldots, y_m\} \in \calP\)를 택하고,
    \[
        I := \{i = 1, \ldots, m : (y_{i-1}, y_i) \cap P_0 \ne \varnothing\}, \quad J := \{i = 1, \ldots, m : (y_{i-1}, y_i) \cap P_0 = \varnothing\}
    \]
    라 하자. 이때 \(\abs{I} \le n\)임을 확인할 수 있다.\\
    이제 먼저 \(f \ge 0\)인 경우를 보면,
    \[
        U(f, P) = \sum_{i \in I} M_i \Delta y_i + \sum_{i \in J} M_i \Delta y_i \le nM\delta_1 + U(f, P_0) < A + \eps
    \]
    가 성립한다. 그리고 일반적인 \(f\)에 대해, \(f + c \ge 0\)이도록 하는 상수 \(c \in \RR\)이 존재하여
    \[
        U(f, P) = U(f+c, P) - c(b-a) < \int_a^b (f+c) + \eps - c(b-a) = A + \eps
    \]
    이다. 따라서 \(\norm{P} < \delta_1\)이면 항상 \(U(f, P) < A + \eps\)가 성립한다.\\
    한편 \(L(f, P) = -U(-f, P)\)이므로 위에서와 같은 논리로 [\(\norm{P} < \delta_2\)이면 \(L(f, P) > A - \eps\)]인 \(\delta_2 > 0\)을 찾을 수 있다.\\
    이제 \(\delta := \min \{\delta_1, \delta_2\}\)로 두면 \(\norm{P} < \delta\)일 때
    \[
        A - \eps < L(f, P) \le R(f, P) \le U(f, P) < A + \eps
    \]
    이므로 (d)가 성립한다.
\end{proof}

\begin{ex}
    \quad

    \begin{enum}
        \item 구간 \([0, 1]\)에서 정의된 함수 \(f = \chi_{\QQ}\)를 생각하면, 임의의 분할 \(P\)에 대해 \(U(f, P) = 1, L(f, P) = 0\)이므로 \(f\)는 정리 \ref{9.1.15}의 (b)를 만족하지 않고, 따라서 \([0, 1]\)에서 리만적분가능하지 않다.
        \item 구간 \([-1, 1]\)에서 정의된 함수 \(g = \chi_{\{0\}}\)을 생각하자. 임의의 양수 \(\eps > 0\)에 대하여 분할 \(P \in \calP\)를
        \[
            P = \left\{-1, -\frac{\eps}{3}, \frac{\eps}{3}, 1\right\}
        \]
        로 정의하면
        \[
            U(g, P) - L(g, P) = \frac{2\eps}{3} - 0 < \eps
        \]
        이므로 정리 \ref{9.1.15}의 (b)를 만족하여 \([-1, 1]\)에서 리만적분가능하다.
        \item 구간 \([-1, 1]\)에서 다음과 같이 정의된
        \[
            h : x \mapsto
            \begin{cases}
                0 &\textif \ -1 \le x < 0\\
                1 &\textif \ 0 \le x \le 1
            \end{cases}, \quad
            \alpha : x \mapsto
            \begin{cases}
                0 &\textif \ -1 \le x \le 0\\
                1 &\textif \ 0 < x \le 1
            \end{cases}
        \]
        을 생각하자. 분할 \(P \in \calP\)를
        \[
            P = \{-1, 0, 1\}
        \]
        로 정의하면
        \[
            U(h, P, \alpha) - L(f, P, \alpha) = 1 - 1 = 0
        \]
        이므로, 정리 \ref{9.1.12}의 (b)를 만족하여 \(h\)는 \(\alpha\)에 관해 리만-스틸체스적분가능하다.
    \end{enum}
\end{ex}

\begin{rem}
    \quad

    \begin{enum}
        \item 정리 \ref{9.1.15}의 (d)는
        \[
            \lim_{\norm{P} \to 0} R(f, P) = A
        \]
        으로 쓸 수 있다.
        \item \(\alpha\)가 일반적인 단조증가함수일 때는 정리 \ref{9.1.15}의 (d)와 같은 필요충분조건을 추가할 수 없다. 더 정확히 말해서,
        \[
            \lim_{\norm{P} \to 0} S(f, P, \alpha) = A \implies \int_a^b f d\alpha = A
        \]
        이지만 역은 성립하지 않는다. 위 명제의 증명은 정리 \ref{9.1.15}의 (d)\(\Rightarrow\)(c)의 증명과 완전히 같다. 역의 반례로, 위 예시의 (c)의 \(h, \alpha\)를 생각하자. 어떤 \(\delta > 0\)이 다음을 만족한다고 가정하자.
        \[
            \norm{P} < \delta \implies \abs{S(h, P, \alpha) - 1} < \frac{1}{2} \quad (P \in \calP)
        \]
        그러면 \(\norm{P} < \delta\)인 분할 \(P \in \calP\)를 다음과 같이 잡을 수 있다.
        \[
            P = \left\{x_0=-1, \ldots, x_{k-1} = -\frac{\delta}{3}, x_k = \frac{\delta}{3}, \quad, x_n=1 \right\}
        \]
        이때 각 \(t_i \in [x_{i-1}, x_i]\)에 \(t_k = x_{k-1}\)로 택하면 \(S(h, P, \alpha) = 0\)이므로 모순이다.
    \end{enum}
    
\end{rem}

\section{리만-스틸체스적분과 연속성}

앞에서 시시한 예시만 다뤄도 됐던 이유는 다음 정리 덕분이다.

\begin{thm}
    \(f\)가 \([a, b]\)에서 연속이면 \(f \in \calR(\alpha)\)이다.
\end{thm}
\begin{proof}
    양수 \(\eps > 0\)이 주어졌다고 하고 다음을 만족하는 \(\eta > 0\)을 잡는다.
    \[
        \eta[\alpha(b) - \alpha(a)] < \eps
    \]
    \(f\)는 \([a, b]\)에서 고른연속이므로 다음을 만족하는 \(\delta > 0\)이 존재한다.
    \[
        \abs{x - y} < \delta \implies \abs{f(x) - f(y)} < \eta \quad (x, y \in [a, b])
    \]
    이제 \(\norm{P} < \delta\)인 분할 \(P := \{x_0, \ldots, x_n\} \in \calP\)가 존재하여
    \begin{align*}
        U(f, P, \alpha) - L(f, P, \alpha) = \sum_{i=1}^n (M_i - m_i) \Delta \alpha_i \le \sum_{i=1}^n \eta \Delta \alpha_i = \eta[\alpha(b) - \alpha(a)] < \eps
    \end{align*}
\end{proof}

고등학교에서 적분은 연속함수에 대해서만 정의하였다. 해석개론에서는, ``어느 정도의 불연속성''까지 적분가능성이 유지되는가에 대해 답한다.

\begin{thm}
    \(f\)의 \([a, b]\)에서의 불연속점의 집합 \(D\)가 가산집합이라 하자. 이때 \(\alpha\)가 \(D\)의 각 점에서 연속이면 \(f \in \calR(\alpha)\)이다. 특히 \(f \in \calR\)이다.
\end{thm}
\begin{rem}
    증명의 편의를 위해 \(D\)가 가산무한인 경우만 할 것인데, \(D\)가 유한집합인 경우는 아래 증명에서 몇 부분만 적당히 수정하면 되므로 연습문제로 남긴다.
\end{rem}
\begin{proof}
    \(\alpha\)가 상수함수이면 증명할 것이 없으므로 \(\alpha(a) < \alpha(b)\)라 하고, \(\abs{f} < M\)인 양수 \(M > 0\)을 잡자. 임의의 양수 \(\eps > 0\)이 주어졌을 때, 다음을 만족하는 모든 \(x \in [a, b]\)의 집합을 \(U\)라 하자.
    \begin{center}
        적당한 양수 \(r_x > 0\)이 존재하여 \(\displaystyle \sup_{\abs{t - x} < r_x} f(t) - \inf_{\abs{t - x} < r_x} f(t) < \frac{\eps}{\alpha(b) - \alpha(a)}\)이다.
    \end{center}
    그리고 각 점 \(x \in U\)에 대해 \(I_{x} := (x - r_x, x + r_x) \cap [a, b]\)라 하자. 그러면 \(U = \bigcup_{x \in U} I_x\)이므로, \(U\)는 \([a, b]\)의 열린집합이다. 또 \(f\)가 \(x\)에서 연속이면 \(x \in U\)이므로 \([a, b] = U \cup D\)이다.\\
    한편 가정에 의해 \(D = \{d_i\}_{i \in \ZZ_+}\)로 쓸 수 있다. 각 \(i\)에 대하여 \(\alpha\)가 \(d_i\)에서 연속이므로 다음을 만족하는 양수 \(\delta_i > 0\)이 존재한다.
    \[
        \abs{x - d_i} < \delta_i \implies \abs{\alpha(x) - \alpha(d_i)} < \frac{\eps}{2^i M} \quad (x \in [a, b])
    \]
    각 \(i\)에 대하여 \(D_i := (d_i - \delta_i/2, d_i + \delta/2) \cap [a, b]\)로 정의하면, \(\{U\} \cup \{D_i\}_{i \in \ZZ_+}\)은 \([a, b]\)의 열린덮개이고 \([a, b]\)이 옹골집합이므로 충분히 큰 \(N \in \ZZ_+\)에 대하여 다음이 성립한다.
    \[
        [a, b] = U \cup \bigcup_{i=1}^N D_i
    \]
    이제
    \[
        C := [a, b] \setminus \bigcup_{i=0}^N D_i
    \]
    로 두면 \(C\)는 \([a, b]\)의 닫힌집합이다. 한편 \(D_i\)들이 모두 구간의 형태이므로, \(C\)는 유한 개의 서로소인 닫힌구간 \(C_1, \ldots, C_L \subseteq [a, b]\)의 합집합으로 쓸 수 있다.\\
    또 \(C \subseteq U\)가 옹골집합이므로, 유한 개의 점들 \(x_1, \ldots, x_K\)가 존재하여
    \[
        C \subseteq \bigcup_{k=1}^K I_{x_k}
    \]
    이다. 이제 각 \(l = 1, \ldots, L\)에 대해, 다음을 만족시키는 \(C_l := [a_l, b_l]\)의 분할 \(P_l := \{x_{l0}, \ldots, x_{lj_l}\}\)이 존재한다:
    \begin{center}
        각 \(j = 1, \ldots, j_l\)에 대해, \([x_{l(j-1)}, x_{lj}] \subseteq I_{x_k}\)인 어떤 \(k =1 , \ldots, K\)가 존재한다.
    \end{center}
    이때 다음이 성립한다.
    \begin{align*}
        \sum_{l=1}^L [U_{a_l}^{b_l} (f, P_l, \alpha) - L_{a_l}^{b_l}(f, P_l, \alpha)] &\le \sum_{l=1}^L \sum_{j=1}^{j_l} \frac{\eps}{\alpha(b) - \alpha(a)} [\alpha(x_{lj}) - \alpha(x_{l(j-1)})]\\
        &= \sum_{l=1}^L \frac{\eps}{\alpha(b) - \alpha(a)} [\alpha(b_l) - \alpha(a_l)] \le \eps
    \end{align*}
    다음으로 각 \(i = 1, \ldots, N\)에 대해 구간 \(D_i\)의 양 끝점을 왼쪽부터 \(d_{il}, d_{i2}\)라 하자. \(D_i\)의 닫힘의 분할 \(Q_i := \{x_{il}, d_{i2}\}\)에 대하여 다음이 성립한다.
    \begin{align*}
        \sum_{i=1}^N [U_{d_{i1}}^{d_{i2}}(f, Q_i, \alpha) - U_{d_{i1}}^{d_{i2}}(f, Q_i, \alpha)] &\le \sum_{i=1}^N 2M[\alpha(d_{i2}) - \alpha(d_{i1})]\\
        &\le \sum_{i=1}^N 2M \cdot \frac{\eps}{2^{i-1}M} < 4\eps
    \end{align*}
    마지막으로
    \[
        P := \bigcup_{l=1}^L P_l \cup \bigcup_{i=1}^N Q_i \in \calP([a, b])
    \]
    라 하면
    \begin{align*}
        &U_a^b(f, P, \alpha) - L(f, P, \alpha)\\
        \le\,& \sum_{l=1}^L [U_{a_l}^{b_l} (f, P_l, \alpha) - L_{a_l}^{b_l}(f, P_l, \alpha)] + \sum_{i=1}^N [U_{d_{i1}}^{d_{i2}}(f, Q_i, \alpha) - U_{d_{i1}}^{d_{i2}}(f, Q_i, \alpha)]\\
        <\,&5\eps
    \end{align*}
    이다. 따라서 \(f \in \calR(\alpha)\)이다.
\end{proof}

\begin{cor}
    단조함수 \(f\)의 \([a, b]\)에서의 불연속점의 집합을 \(D\)라 할 때, \(\alpha\)가 \(D\)의 각 점에서 연속이면 \(f \in \calR(\alpha)\)이다. 특히 \(f \in \calR\)이다.
\end{cor}
\begin{proof}
    단조함수의 불연속점의 집합은 가산집합이므로.
\end{proof}

\begin{thm} \label{9.2.4}
    \([a, b]\)에서 \(f \in \calR(\alpha)\)이고 \(m \le f \le M\)이라 하자. 연속함수 \(\phi : [m, M] \to \RR\)에 대하여 \(\phi \circ f \in \calR(\alpha)\)이다.
\end{thm}
\begin{proof}
    \(\phi\)가 유계이므로 \(h := \phi \circ f\)도 유계인 것은 분명하다. 이제 양수 \(\eps > 0\)이 주어졌다고 할 때, \(\phi\)는 \([m, M]\)에서 고른연속이므로 다음을 만족하는 양수 \(\delta \in (0, \eps)\)가 존재한다.
    \[
        \abs{s - t} < \delta \implies \abs{\phi(s) - \phi(t)} < \eps \quad (s, t \in [m, M])
    \]
    한편 \(f \in \calR(\alpha)\)이므로 다음을 만족하는 분할 \(P := \{x_0, \ldots, x_n\} \in \calP\)가 존재한다.
    \[
        U(f, P, \alpha) - L(f, P, \alpha) < \delta^2
    \]
    이제
    \[
        I := \{i = 1, \ldots, n : M_i - m_i < \delta\}, \quad J := \{i = 1, \ldots, n : M_i - m_i \ge \delta\}
    \]
    라 하면
    \[
        \sum_{i \in J} \Delta \alpha_i \le \frac{1}{\delta} \sum_{i \in J} (M_i - m_i) \Delta \alpha_i  \le \frac{1}{\delta} \sum_{i =1}^n (M_i - m_i) \Delta \alpha_i < \delta
    \]
    이다. 그리고
    \[
        M_i' := \sup_{x \in [x_{i-1}, x_i]} h(x), \quad m_i' := \inf_{x \in [x_{i-1}, x_i]} h(x)
    \]
    로 정의하면 \(i \in I\)에 대하여 \(M_i' - m_i' < \eps\)이다. 이제 \(\phi\)가 유계이므로 \(\abs{\phi} < K\)인 양수 \(K > 0\)을 잡으면 
    \begin{align*}
        U(h, P, \alpha) - L(h, P, \alpha) &= \sum_{i \in I} (M'_i - m'_i) \Delta \alpha_i + \sum_{i \in J} (M'_i - m'_i) \Delta \alpha_i\\
        &\le \eps \sum_{i \in I} \Delta \alpha_i + 2K \sum_{i \in J} \Delta \alpha_i\\
        &< \eps[\alpha(b) - \alpha(a)] + 2K\delta\\
        &< \eps[\alpha(b) = \alpha(a) + 2K]
    \end{align*}
    이다. 따라서 \(h \in \calR(\alpha)\)가 증명되었다.
\end{proof}

\section{리만-스틸체스적분의 성질} \label{sec9.3}

이 절에서는 리만-스틸체스적분의 성질을 증명함으로써 리만-스틸체스적분이 우리가 생각하는 대로 잘 작동함을 보일 것이다.

\begin{prop}
    \([a, b]\)에서 \(f, g \in \calR(\alpha)\)이고 \(f \le g\)이면
    \[
        \int_a^b f d\alpha \le \int_a^b g d\alpha
    \]
    이다.
\end{prop}
\begin{proof}
    임의의 분할 \(P \in \calP\)에 대해
    \[
        U(f, P, \alpha) \le U(g, P, \alpha), \quad L(f, P, \alpha) \le L(g, P, \alpha)
    \]
    이므로.
\end{proof}

\begin{lem} \label{9.3.2}
    \([a, b]\)에서 \(f, g \in \calR(\alpha)\)이면 임의의 분할 \(P \in \calP\)에 대해
    \[
        L(f, P, \alpha) + L(g, P, \alpha) \le L(f+g, P, \alpha) \le U(f+g, P, \alpha) \le U(f, P, \alpha) + U(g, P, \alpha)
    \]
    이다.
\end{lem}
\begin{proof}
    두 번째 부등호는 자명하고, 첫 번째 부등호는 세 번째 부등호와 비슷하게 보일 수 있다. 세 번째 부등호는
    \[
        \sup_{x \in [x_{i-1}, x_i]} [f(x) + g(x)] \le \sup_{x \in [x_{i-1}, x_i]} f(x) + \sup_{x \in [x_{i-1}, x_i]} g(x)
    \]
    에서 성립한다.
\end{proof}

\begin{thm} \label{9.3.3}
    \([a, b]\)에서 \(f, g \in \calR(\alpha)\)이고 \(c \in \RR\)이면 \(f + g, cf \in \calR(\alpha)\)이고
    \[
        \int_a^b (f+g) d\alpha = \int_a^b f d\alpha + \int_a^b g d\alpha, \quad \int_a^b cf d\alpha = c \int_a^b f d\alpha
    \]
    이다.
\end{thm}
\begin{proof}
    \(cf \in \calR(\alpha)\)이고 \(\displaystyle \int_a^b cf d\alpha = c \int_a^b f d\alpha\)인 것은 쉽게 확인할 수 있으므로 연습문제로 남긴다. 이제 \(f, g \in \calR(\alpha)\)이므로 임의의 양수 \(\eps > 0\)에 대해 다음을 만족하는 분할 \(P_1, P_2 \in \calP\)가 존재한다.
    \[
        U(f, P_1, \alpha)  < \int_a^b f d\alpha +  \eps, \quad U(g, P_1, \alpha) < \int_a^b f d\alpha + \eps
    \]
    이제 \(P := P_1 \cup P_2 \in \calP\)라 하면 보조정리 \ref{9.3.2}에 의해
    \begin{align*}
        \upint{a}{b} (f+g) d\alpha \le U(f+g, P, \alpha) &\le U(f, P, \alpha) + U(g, P, \alpha)\\
        &\le U(f, P_1, \alpha) + U(g, P_2, \alpha)\\
        &< \int_a^b f d\alpha + \int_a^b g d\alpha + 2\eps
    \end{align*}
    이고 \(\eps > 0\)이 임의의 양수였으므로
    \[
        \upint{a}{b} (f+g) d\alpha \le \int_a^b f d\alpha + \int_a^b g d\alpha
    \]
    이다. 마찬가지로 하합에 대해서도 같은 논리에 의해
    \[
        \lowint{a}{b} (f+g) d\alpha \ge \int_a^b f d\alpha + \int_a^b g d\alpha
    \]
    임을 얻을 수 있으며, 이는 \(f+g \in \calR(\alpha)\)이고
    \[
        \int_a^b (f+g) d\alpha = \int_a^b f d\alpha + \int_a^b g d\alpha
    \]
    임을 의미한다.
\end{proof}

\begin{thm} 
    \([a, b]\)에서 \(f \in \calR(\alpha)\)이면 \(\abs{f} \in \calR(\alpha)\)이고
    \[
        \abs{\int_a^b f d\alpha} \le \int_a^b \abs{f} d \alpha
    \]
    이다.
\end{thm}
\begin{proof}
    Theorem \ref{9.2.4}에서 \(\phi : t \mapsto \abs{t}\)를 생각하면 \(\abs{f} \in \calR(\alpha)\)이다. 그리고 \(-\abs{f} \le f \le \abs{f}\)에서 원하는 부등식을 얻는다.
\end{proof}

\begin{thm} \label{9.3.5}
    \([a, b]\)에서 \(f, g \in \calR(\alpha)\)이면 \(fg \in \calR(\alpha)\)이다.
\end{thm}
\begin{proof}
    \[
        fg = \frac{1}{4}((f+g)^2 - (f-g)^2)
    \]
    이므로, 임의의 \(f \in \calR(\alpha)\)에 대해 \(f^2 \in \calR(\alpha)\)인 것만 보이면 \(fg \in \calR(\alpha)\)인 것이 증명된다. 이는 정리 \ref{9.2.4}에서 \(\phi : t \mapsto t^2\)을 생각하면 된다.
\end{proof}

\begin{lem} \label{9.3.4}
    \(P \in \calP([a, c]), Q \in \calP([c, b])\)이면 \(P \cup Q \in \calP([a, b])\)이고
    \[
        U_a^c(f, P, \alpha) + U_c^b(f, Q, \alpha) = U_a^b(f, P \cup Q, \alpha), \quad L_a^c(f, P, \alpha) + L_c^b(f, Q, \alpha) = L_a^b(f, P \cup Q, \alpha)
    \]
    이다.
\end{lem}
\begin{proof}
    연습문제로 남긴다.
\end{proof}

\begin{thm} \label{9.3.7}
    \(c \in (a, b)\)에 대하여 다음이 동치이다.
    \begin{enum}
        \item \([a, b]\)에서 \(f \in \calR(\alpha)\).
        \item \([a, c]\)와 \([c, b]\) 각각에서 \(f \in \calR(\alpha)\).
    \end{enum}
    위의 조건들이 성립할 때
    \begin{equation} \label{eq9.2}
        \int_a^c f d\alpha + \int_c^b f d\alpha = \int_a^b f d\alpha
    \end{equation}
    이다.
\end{thm}
\begin{proof}
    \quad\\
    \((\Rightarrow)\) 분할 \(P \in \calP([a, b])\)에 대해
    \[
        P_1 := (P \cap [a, c]) \cup \{c\} \in \calP([a, c]), \quad P_2 := (P \cap [c, b]) \cup \{c\} \in \calP([c, b])
    \]
    라 하면, \(P_1 \cup P_2 = P \cup \{c\}\)는 \(P\)의 세분이므로
    \begin{align*}
        & [U_a^c(f, P_1, \alpha) - L_a^c(f, P_1, \alpha)] + [U_c^b(f, P_2, \alpha) - L_c^b(f, P_2, \alpha)] \\
        =\,& U_a^b(f, P \cup \{c\}, \alpha) - L_a^b(f, P \cup \{c\}, \alpha)\\
        \le\,& U_a^b(f, P, \alpha) - L_a^b(f, P, \alpha)
    \end{align*}
    이다. 따라서 \([a, b]\)에서 \(f \in \calR(\alpha)\)이므로 \([a, c]\)와 \([c, b]\) 각각에서도 \(f \in \calR(\alpha)\)이다.\\
    \((\Leftarrow)\) 분할 \(Q \in \calP([a, c]), R \in \calP([c, b])\)에 대해 보조정리 \ref{9.3.4}에 의해
    \[
        U_a^b(f, Q \cup R, \alpha) - L_a^b(f, Q \cup R, \alpha) = [U_a^c(f, Q, \alpha) - L_a^c(f, Q, \alpha)] + [U_c^b(f, R, \alpha) - L_c^b(f, R, \alpha)]
    \]
    이므로, \([a, c]\)와 \([c, b]\) 각각에서 \(f \in \calR(\alpha)\)이면 \([a, b]\)에서 \(f \in \calR(\alpha)\)이다.\\
    이제 (a), (b)가 모두 참이라 하고 (\ref{eq9.2})\를 보이기 위해서는
    \[
        \upint{a}{c} f d\alpha + \upint{c}{b} f d\alpha = \upint{a}{b} f d\alpha
    \]
    인 것만 보이면 된다. 먼저 임의의 분할 \(P \in \calP([a, b])\)에 대하여
    \[
        \upint{a}{c} f d\alpha + \upint{c}{b} f d\alpha \le U_a^c(f, P_1, \alpha) + U_c^b(f, P_2, \alpha) = U_a^b(f, P \cup \{c\}, \alpha) \le U_a^b(f, P, \alpha)
    \]
    이므로
    \[
        \upint{a}{c} f d\alpha + \upint{c}{b} f d\alpha \le \upint{a}{b} f d\alpha
    \]
    를 얻는다. 다음으로 임의의 양수 \(\eps > 0\)에 대해 어떤 분할 \(Q \in \calP([a, c]), R \in \calP([c, b])\)가 존재하여
    \[
        U_a^c(f, Q, \alpha) < \upint{a}{c} f d\alpha + \frac{\eps}{2}, \quad U_c^b(f, R, \alpha) < \upint{c}{b} f d\alpha + \frac{\eps}{2}
    \]
    를 만족한다. 이제 보조정리 \ref{9.3.4}에 의해
    \[
        \upint{a}{b} f d\alpha \le U_a^b(f, Q \cup R, \alpha) = U_a^c(f, Q, \alpha) + U_c^b(f, R, \alpha) < \upint{a}{c} f d\alpha + \upint{c}{b} f d\alpha + \eps
    \]
    이다. \(\eps > 0\)이 임의의 양수였으므로
    \[
        \upint{a}{b} f d\alpha \le \upint{a}{c} f d\alpha + \upint{c}{b} f d\alpha
    \]
    가 성립한다. 따라서 (\ref{eq9.2})\가 성립한다.
\end{proof}

\begin{defn}
    \([a, b]\)에서 \(f \in \calR(\alpha)\)일 때,
    \[
        \int_b^a f d\alpha := - \int_a^b f d\alpha, \quad \int_c^c f d\alpha := 0 \ (c \in [a, b])
    \]
    로 정의하자.
\end{defn}

\begin{prop}
    구간 \([A, B]\)에서 정의된 유계함수 \(f : [A, B] \to \RR\)와 단조증가함수 \(\alpha : [a, b] \to \RR\)에 대하여 \([A, B]\)에서 \(f \in \calR(\alpha)\)이면 임의의 \(a, b, c \in [A, B]\)에 대하여 (\ref{eq9.2})\가 성립한다.
\end{prop}
\begin{proof}
    연습문제로 남긴다.
\end{proof}

\begin{lem}
    \([a, b]\)에서 정의된 단조증가함수 \(\alpha, \beta\)와 \(c \ge 0\), 분할 \(P \in \calP\)에 대하여
    \[
        U(f, P, \alpha + c\beta) = U(f, P, \alpha) + cU(f, P, \beta), \quad L(f, P, \alpha + c\beta) = L(f, P, \alpha) + cL(f, P, \beta)
    \]
    이다.
\end{lem}
\begin{proof}
    연습문제로 남긴다.
\end{proof}

\begin{thm} \label{9.3.11}
    \([a, b]\)에서 정의된 단조증가함수 \(\alpha, \beta\)와 \(c \ge 0\)에 대하여 \(f \in \calR(\alpha), f \in \calR(\beta)\)이면 \(f \in \calR(\alpha + c\beta)\)이고
    \[
        \int_a^b f d(\alpha + c\beta) = \int_a^b f d\alpha + c \int_a^b d\beta
    \]
    이다.
\end{thm}
\begin{proof}
    연습문제로 남긴다.
\end{proof}

리만-스틸체스적분의 기호를 고등학교 때나 미적분학에서 치환적분의 의미로 사용한 적이 많은데, 실제로 관련이 있다.

\begin{thm} \label{9.3.12}
    \(\alpha \in C^1([a, b])\)이고 \(f \in \calR(\alpha)\)이면 \(f \alpha' \in \calR\)이고
    \[
        \int_a^b f d\alpha = \int_a^b f \alpha'
    \]
    이다.
\end{thm}
\begin{proof}
    \(\alpha'\)가 연속함수이므로 \(f\alpha \in \calR\)인 것은 분명하다.\\
    분할 \(P := \{x_0, \ldots, x_n\}\)이 주어졌다고 하고 각 \(t_i \in [x_{i-1}, x_i]\)에 의한 리만-스틸체스합과 리만합을 각각 \(S(f, P, \alpha)\)와 \(R(f\alpha', P)\)라 하자. 그리고 평균값 정리에 의해, 각 \(i\)에 대해
    \[
        \Delta \alpha_i  = \alpha'(s_i) \Delta x_i
    \]
    인 \(s_i \in (x_{i-1}, x_i)\)가 존재한다. 이때
    \begin{align*}
        &\abs{S(f, P, \alpha) - R(f\alpha', P)}\\
        =\,& \abs{\sum_{i=1}^n f(t_i) \Delta\alpha_i - \sum_{i=1}^n f(t_i)\alpha'(t_i) \Delta x_i}\\
        =\,& \abs{\sum_{i=1}^n f(t_i)(\alpha'(s_i) - \alpha'(t_i)) \Delta x_i}\\
        =\,& \sum_{i=1}^n \abs{f(t_i)(\alpha'(s_i) - \alpha'(t_i))} \Delta x_i
    \end{align*}
    이다. \(\abs{f} < M\)인 양수 \(M > 0\)을 잡으면, \(\alpha'\)가 \([a, b]\)에서 고른연속이므로 임의의 양수 \(\eps > 0\)에 대해
    \[
        \abs{s - t} < \delta \implies \abs{\alpha'(s) - \alpha'(t)} < \frac{\eps}{M(b - a)} \quad (s, t \in [a, b])
    \]
    이도록 하는 양수 \(\delta > 0\)이 존재한다. 또 \(\displaystyle A := \int_a^b f d\alpha\)에 대하여,
    \[
        P \supseteq P_0 \implies \abs{S(f, P, \alpha) - A} < \eps \quad (P \in \calP)
    \]
    이도록 하는 분할 \(P_0 \in \calP\)를 \(\norm{P_0} < \delta\)이도록 잡을 수 있다. 이제
    \[
        \abs{R(f\alpha', P_0) - A} \le \abs{R(f\alpha', P_0) - S(f, P_0, \alpha)} + \abs{S(f, P_0, \alpha) - A} < \sum_{i=1}^n \frac{M\eps}{M(b-a)} \cdot \Delta x_i  + \eps = 2\eps
    \]
    이므로, \(\displaystyle \int_a^b f\alpha' = A\)이다.
\end{proof}

\begin{ex}
    \(X\)가 \([a, b]\) 안의 값을 가지는 연속확률변수라 하고\footnote{물론 아직 확률변수가 무엇인지 정의하지도 않았다. ㅋㅋ.}, 연속함수 \(p : [a, b] \to \RR\)가 \(X\)의 확률밀도함수라 하자. 그리고 \(P\)가 \(X\)의 누적분포함수라 하면, \(P' = p\)인 것을 통계학에서 배워서 알고 있다. 이제 연속함수 \(f : [a, b] \to \RR\)에 대하여 확률변수 \(f(X)\)의 기댓값은
    \[
        \int_a^b fp = \int_a^b fP' = \int_a^b fdP
    \]
    가 된다. 이제 \(X\)가 \([a, b]\) 안에서 유한 개의 값만을 가지는 이산확률변수라 하자. 즉, \([a, b]\) 안의 유한 개의 점 \(c_1, \ldots, c_n\)이 있어서
    \[
        P(X = c_i) = p_i \ (i = 1, \ldots, n), \quad \sum_{i=1}^n p_i = 1
    \]
    이다. 이때 \(X\)가 따르는 누적분포함수 \(P\)는
    \[
        P = \sum_{i=1}^n p_i \chi_{[c_i, b]}
    \]
    로 주어지며, \(f(X)\)의 기댓값이
    \[
        \int_a^b f dP = \sum_{i=1}^n f(c_i)p_i
    \]
    임을 확인할 수 있다. 분명 통계학 시간에는 연속확률변수와 이산확률변수의 기댓값을 다르게 쓸 수밖에 없었는데, 리만-스틸체스적분에 의해
    \[
        \int_a^b f dP
    \]
    라는 공통된 표기로 둘을 나타낼 수 있게 되었다.
\end{ex}

\begin{thm} \label{9.3.12}
    \(\phi : [c, d] \to [a, b]\)가 증가하는 전사함수일 때, \([a, b]\)에서 \(f \in \calR(\alpha)\)이면 \([c, d]\)에서 \(f \circ \phi \in \calR(\alpha \circ \phi)\)이고
    \[
        \int_c^d (f \circ \phi)d(\alpha \circ \phi) = \int_a^b f d\alpha
    \]
    이다.
\end{thm}
\begin{proof}
    \([a, b]\)의 분할 \(P := \{x_0, \ldots, x_n\} \in \calP([a, b])\)에 대하여
    \[
        Q := \{\phi^{-1}(x_0), \ldots, \phi^{-1}(x_n)\}
    \]
    으로 정의하면 \(Q \in \calP([c, d])\)이고
    \[
        U_c^d(f \circ \phi, Q, \alpha) = U_a^b(f, P, \alpha), \quad L_c^d(f \circ \phi, Q, \alpha) = L_a^b(f, P, \alpha)
    \]
    이므로.
\end{proof}

\begin{ex}
    정리 \ref{9.3.12}에서 \(\phi : [0, \pi/2] \to [0, 1] : x \mapsto \sin x\)를 생각하면
    \[
        \int_0^{\pi/2} \sin x \cos x dx = \int_0^{\pi /2} \sin x d(\sin x) = \int_0^1 t dt = \frac{1}{2}.
    \]
\end{ex}

\section{미적분학의 기본정리}

이 절에서는 리만적분만 다룬다. \([a, b]\)에서 \(f \in \calR\)이면 임의의 \([a, x] \subseteq [a, b]\)에서 \(f \in \calR\)이므로 함수 \(F : [a, b] \to \RR\)를
\begin{equation} \label{eq9.3}
    F : x \mapsto \int_a^x f (t) dt
\end{equation}
로 정의할 수 있다.

\begin{thm}
    \([a, b]\)에서 리만적분가능한 \(f\)에 대하여 (\ref{eq9.3})\과 같이 정의된 함수 \(F\)는 립시츠 연속이다. 만약 \(f\)가 \(x_0 \in [a, b]\)에서 연속이면 \(F\)는 \(x_0\)에서 미분가능하고 \(F'(x_0) = f(x_0)\)이다.
\end{thm}
\begin{proof}
    \(\abs{f} < M\)인 양수 \(M > 0\)을 잡으면 임의의 \(x, y \in [a, b] \ (x < y)\)에 대해
    \[
        \abs{F(y) - F(x)} \le \int_x^y \abs{f(t)} dt \le M(y - x)
    \]
    이므로 \(F\)는 \([a, b]\)에서 립시츠 연속이다. 한편 \(f\)가 \(x_0\)에서 연속이면 \(x \ne x_0\)에 대해
    \begin{align*}
        \abs{\frac{F(x) - F(x_0)}{x - x_0} - f(x_0)} &= \abs{\frac{1}{x - x_0} \int_{x_0}^x [f(t) - f(x_0)] dt}\\
        &\le \frac{1}{x - x_0} \int_{x_0}^x \abs{f(t) - f(x_0)} dt\\
        &\le \sup_{t \in [x_0, x]} \abs{f(t) - f(x_0)} \xrightarrow[]{x \to x_0} 0
    \end{align*}
    이므로 \(F'(x_0) = f(x_0)\)이다.
\end{proof}

\begin{cor}[적분의 평균값 정리]
    연속함수 \(f : [a, b] \to \RR\)에 대하여
    \[
        \frac{1}{b-a} \int_a^b f(t) dt = f(x_0)
    \]
    인 \(x_0 \in (a, b)\)가 존재한다.
\end{cor}

이제 드디어 미적분학의 기본정리이다.

\begin{thm}[미적분학의 기본정리\footnote{Fundamental theorem of calculus.}]
    \([a, b]\)에서 \(f\)가 리만적분가능하고 원시함수 \(F\)를 가지면
    \[
        \int_a^b f = F(b) - F(a)
    \]
    이다.
\end{thm}
\begin{proof}
    분할 \(P := \{x_0, \ldots, x_n\} \in \calP\)가 주어졌을 때, 평균값 정리에 의해 각 \(i = 1, \ldots, n\)에 대해 다음을 만족하는 \(t_i \in (x_{i-1}, x_i)\)가 존재한다.
    \[
        F(x_i) - F(x_{i-1}) = f(t_i) \Delta x_i
    \]
    따라서
    \[
        F(b) - F(a) = \sum_{i=1}^n [F(x_i) - F(x_{i-1})] = \sum_{i=1}^n f(t_i) \Delta x_i
    \]
    이므로
    \[
        L(f, P) \le F(b) - F(a) \le U(f, P)
    \]
    이다. 그런데 \(f\)가 리만적분가능하고 \(P\)가 임의의 분할이었으므로
    \[
        \int_a^b f(t)dt = F(b) - F(a)
    \]
    를 얻는다.
\end{proof}

다음 두 따름정리를 증명하는 것은 연습문제로 남긴다.

\begin{cor}[치환적분]
    \(g \in C^1([a, b])\)이고 \(g(a)\)와 \(g(b)\) 사이의 닫힌구간에서 정의된 실함수 \(f\)가 연속이면
    \[
        \int_{g(a)}^{g(b)} f(x)dx = \int_a^b f(g(t))g'(t) dt
    \]
    이다.
\end{cor}

\begin{cor}
    \(f, g\)가 \([a, b]\)에서 리만적분가능하고 각각 원시함수 \(F, G\)를 가지면
    \[
        \int_a^b Fg = F(b)G(b) - F(a)G(a) - \int_a^b fG
    \]
    이다.
\end{cor}

\section{유계변동함수}

이제 잠시 다른 이야기를 하려고 한다.

\begin{defn}
    함수 \(f : [a, b] \to \RR\)와 분할 \(P := \{x_0, \ldots, x_n\} \in \calP\)에 대하여, \(P\)에 의한 \(f\)의 \textbf{변동(variation)}을
    \[
        V_a^b(f, P) := \sum_{i=1}^n \abs{f(x_i) - f(x_{i-1})}
    \]
    로 정의한다. \(f\)의 \textbf{전변동(total variation)}을
    \[
        V_a^b(f) := \sup_{P \in \calP} V_a^b(f, P)
    \]
    로 정의하며, \(V_a^b(f) < \infty\)이면 \(f\)를 \textbf{유계변동함수(function of bounded variation)}라 하고 \(f \in BV([a, b])\)라 쓴다.\\
    어떤 구간에 대해 이야기하는지 명확하면 \(a, b\)를 생략하여 \(V(f, P)\) 등과 같이 쓴다.
\end{defn}

\begin{prop} \label{9.5.2}
    \(f \in BV([a, b])\)이면 \(f\)는 \([a, b]\)에서 유계이다.
\end{prop}
\begin{proof}
    임의의 \(x \in (a, b)\)에 대해 분할 \(P := \{a, x, b\} \in \calP\)를 생각하면
    \[
        \abs{f(x)} \le \abs{f(x) - f(a)} + \abs{f(a)} \le V(f) + \abs{f(a)}
    \]
    이다.
\end{proof}

\begin{ex}
    \quad

    \begin{enum}
        \item \(f : [a, b] \to \RR\)가 단조함수이면 \(V(f) = \abs{f(b) - f(a)}\)이므로 \(f \in BV([a, b])\)이다.
        \item \(f : [a, b] \to \RR\)가 립시츠 연속이면 임의의 분할 \(P := \{x_0, \ldots, x_n\} \in \calP\)에 대해
        \[
            V(f, P) = \sum_{i=1}^n \abs{f(x_{i}) - f(x_{i-1})} \le \sum_{i=1}^n M(x_i - x_{i-1}) = M(b - a)
        \]
        이도록 하는 양수 \(M > 0\)이 존재한다. 따라서 \(f \in BV([a, b])\). 그러나 모든 유계변동함수가 립시츠 연속인 것은 아니다. \([0, 1]\)에서 정의된 함수 \(f : x \mapsto \sqrt{x}\)는 단조함수이므로 유계변동함수이지만 립시츠 연속이 아니다.
        \item 특히 \(f\)가 \([a, b]\)에서 미분가능하고 \(f'\)이 유계이면 \(f\)는 립시츠 연속이므로 유계변동함수이다. 그러나 모든 미분가능한 유계변동함수의 도함수가 유계인 것은 아니다. 함수 \(f : [0, 1] \to \RR\)를
        \[
            f : x \mapsto
            \begin{cases}
                x^2 \sin (x^{-3/2}) &\textif \ x > 0\\
                0 &\textif \ x = 0
            \end{cases}
        \]
        으로 정의하면 \(f\) \([0, 1]\)에서 미분가능한 유계변동함수이지만 그 도함수는 유계가 아니다. 자세한 증명은 연습문제로 남긴다. (이와 같은 형태의 반례들을 많이 알고 있는 것이 이 단원에서 중요하다.)
    \end{enum}
\end{ex}

\begin{lem}
    \(P, P' \in \calP\)에 대하여 \(P'\)가 \(P\)의 세분이면 \(V(f, P) \le V(f, P')\)이다.
\end{lem}
\begin{proof}
    연습문제로 남긴다.
\end{proof}

\begin{thm}
    \(f, g \in BV([a, b]), c \in \RR\)에 대하여 \(f + g, cf, \abs{f} \in BV([a, b])\)이고
    \[
        V(f + g) \le V(f) + V(g), \quad V(cf) \le \abs{c}V(f), V(\abs{f}) \le V(f)
    \]
    이다.
\end{thm}
\begin{proof}
    연습문제로 남긴다.
\end{proof}

\begin{thm}
    \(f \in BV([a, b])\)이고 \(m \le f \le M\)일 때, 립시츠 연속함수 \(\phi : [m, M] \to \RR\)에 대하여 \(\phi \circ f \in BV([a, b])\)이다.
\end{thm}
\begin{proof}
    다음을 만족하는 양수 \(K > 0\)이 존재한다.
    \[
        \abs{\phi(s) - \phi(t)} \le K\abs{s - t} \quad (s, t \in [a, b])
    \]
    이제 임의의 \(P := \{x_0, \ldots, x_n\} \in \calP\)에 대해
    \begin{align*}
        V(\phi \circ f, P) = \sum_{i=1}^n \abs{(\phi \circ f)(x_i) - (\phi \circ f)(x_{i-1})} &\le K \sum_{i=1}^n \abs{f(x_i) - f(x_{i-1})}\\
        &= K \cdot V(f, P) \le K \cdot V(f)
    \end{align*}
    이므로 \(V(\phi \circ f) \le K \cdot V(f)\)이다.
\end{proof}

\begin{thm}
    \(f, g \in BV([a, b])\)이면 \(fg \in BV([a, b])\)이고 \(\abs{f} \le M_f, \abs{f} \le M_g\)인 양수 \(M_f, M_g > 0\)에 대하여
    \[
        V(fg) \le M_f V(g) + M_g V(f)
    \]
    이다.
\end{thm}
\begin{proof}
    분할 \(P := \{x_0, \ldots, x_n\} \in \calP\)에 대하여
    \begin{align*}
        V(fg, P) &= \sum_{i=1}^n \abs{f(x_i)g(x_i) - f(x_{i-1})g(x_{i-1})}\\
        &\le \sum_{i=1}^n [\abs{f(x_{i}) [g(x_i) - g(x_{i-1})]} + \abs{g(x_{i-1}) [f(x_i) - f(x_{i-1})]}]\\
        &\le M_f V(g, P) + M_g V(f, P) \le M_f V(g)  + M_g V(f).
    \end{align*}
\end{proof}

\begin{thm}
    \(f : [a, b] \to \RR\)와 \(c \in (a, b)\)에 대하여 다음이 동치이다.
    \begin{enum}
        \item \(f \in BV([a, b])\).
        \item \(f \in BV([a, c])\)이고 \(f \in BV([c, b])\).
    \end{enum}
    위 조건들이 성립할 때 \(V_a^b(f) = V_a^c(f) + V_c^b(f)\)이다.
\end{thm}
\begin{proof}
    연습문제로 남긴다.
\end{proof}

다음은 이 절의 핵심 결론이다.

\begin{thm} \label{9.5.8}
    함수 \(f : [a, b] \to \RR\)에 대하여 다음이 동치이다.
    \begin{enum}
        \item \(f \in BV([a, b])\).
        \item 두 단조증가함수 \(g, h : [a, b] \to \RR\)가 존재하여 \(f = g - h\)이다.
    \end{enum}
\end{thm}
\begin{proof}
    단조함수는 유계변동함수이므로 \((\Rightarrow)\)는 쉽게 확인된다. 이제 (a)를 가정하고, 함수 \(g\)를
    \[
        g : x \mapsto V_a^x(f)
    \]
    로 정의하고 \(h := g - f\)라 하자. 이제 \(g\)와 \(h\) 각각이 단조증가하는 것을 보이면 되는데, 임의의 \(a \le x < y \le b\)에 대해
    \[
        g(y) = V_a^y(f) = V_a^x(f) + V_x^y(f) \ge V_a^x(f) = g(x)
    \]
    이므로 \(g\)는 단조증가함수이다. 또
    \[
        h(y) - h(x) = [V_a^y(f) - V_a^x(f)] - [f(y) - f(x)] = V_x^y(f) - [f(y) - f(x)] \ge 0
    \]
    이므로 \(h\)도 단조증가함수이다.
\end{proof}

\begin{thm} \label{9.5.9}
    \(f \in BV([a, b])\)에 대하여 단조증가함수 \(g, h : [a, b] \to \RR\)가 존재하여 \(f = g - h\)이고, \(f\)가 \(x \in [a, b]\)에서 연속이면 \(g, h\)도 \(x\)에서 연속이다.
\end{thm}
\begin{proof}
    정리 \ref{9.5.8}의 증명에서 정의한 \(g\)가, \(f\)가 연속인 점들에서 연속임을 보이면 충분하다.\\
    먼저 \(f\)가 \(x \in (a, b]\)에서 연속이라 하자. 임의의 양수 \(\eps > 0\)에 대해
    \[
        V_a^{x}(f, P) > g(x) - \frac{\eps}{2}
    \]
    를 만족하는 분할 \(P := \{x_0, \ldots, x_n\} \in \calP([a, x])\)가 존재한다. 그리고 \(f\)가 \(x\)에서 연속이므로, \(\abs{f(t) - f(x)} < \eps/2\)인 \(t \in (x_n, x)\)를 잡을 수 있다. 이제 \(P' := P \cup \{t\} \in \calP([a, x])\)라 하면 \(V_a^x(f, P') \ge V_a^x(f, P)\)이다. 그런데
    \[
        V_a^x(f, P') - \abs{f(t) - f(x)} = \sum_{i=1}^{n-1} \abs{f(x_i) - f(x_{i-1})} + \abs{f(t) - f(x_n)}
    \]
    은 구간 \([a, t]\)에서 분할 \(P' \cap [a, t]\)에 의한 \(f\)의 변동과 같으므로
    \[
        V_a^x(f, P') - \abs{f(t) - f(x)} = V_a^t(f, P' \cap [a, t]) \le g(t)
    \]
    이다. 따라서
    \[
        g(x) - \frac{\eps}{2} < V_a^x(f, P) \le V_a^x(f, P') \le g(t) + \abs{f(t) - f(x)} < g(t) + \frac{\eps}{2}
    \]
    가 성립한다. 이는
    \[
        g(x) - g(t) < \eps
    \]
    임을 의미하며, \(g\)가 단조함수이고 \(\eps > 0\)이 임의의 양수였으므로 이는 \(g(x-) = g(x)\)를 의미한다. 마찬가지로 \(f\)가 \(x \in [a, b)\)에서 연속일 때 \(g(x+) = g(x)\)임을 보일 수 있으며, 따라서 \(f\)가 \(x \in [a, b]\)에서 연속이면 \(g\)도 \(x\)에서 연속이다.
\end{proof}

\section{리만-스틸체스적분의 정의 (2)}

여전히 \(\alpha\)를 단조증가함수에서 유계변동함수로 확장했을 때도 리만-스틸체스적분을 정의할 수 있다.

\begin{defn} \label{9.6.1}
   구간 \([a, b]\)에서 정의된 유계 실함수 \(f\)와 유계변동함수 \(\alpha\)가 주어졌다고 하자. 만약 \(\alpha = \alpha_1 - \alpha_2\)이고 \([a, b]\)에서 \(f \in \calR(\alpha_1)\)이고 \(f \in \calR(\alpha_2)\)이도록 하는 단조증가함수 \(\alpha_1, \alpha_2 : [a, b] \to \RR\)가 존재하면 \(f\)가 \textbf{\(\alpha\)에 관해 리만-스틸체스적분가능}하다고 하고, \([a, b]\)에서 \(f \in \calR(\alpha)\)로 쓴다. 그리고 그 리만-스틸체스적분의 값을
   \[
        \int_a^b f d\alpha := \int_a^b f d\alpha_1 - \int_a^b f d\alpha_2
   \]
   로 정의한다.
\end{defn}

정의 \ref{9.6.1}\이 잘 정의되었음을 보이는 것이 필수적이다.

\begin{prop}
    \(\alpha \in BV([a, b])\)에 대하여 \([a, b]\)에서 정의된 단조증가함수 \(\alpha_1, \alpha_2, \beta_1, \beta_2\)가 존재하여
    \[
        \alpha = \alpha_1 - \alpha_2 = \beta_1 - \beta_2
    \]
    이고 \(f\)가 이들 각각에 대하여 리만-스틸체스적분가능할 때,
    \[
        \int_a^b f d\alpha_1 - \int_a^b f d\alpha_2 = \int_a^b f d\beta_1 - \int_a^b f d\beta_2
    \]
    이다.
\end{prop}
\begin{proof}
    \(\alpha_1 + \beta_2 = \beta_1 + \alpha_2\)이므로 
    \[
        \int_a^b f d\alpha_1 + \int_a^b f d\beta_2 = \int_a^b f d(\alpha_1 + \beta_2) = \int_a^b f d(\beta_1 + \alpha_2) = \int_a^b f d\beta_1 + \int_a^b f d\alpha_2.
    \]
\end{proof}

이 절이 끝날 때까지 아무 말이 없으면 \(f\)는 유계 실함수, \(\alpha\)는 유계변동함수이고 \(\alpha_1, \alpha_2\)는 \(\alpha = \alpha_1 - \alpha_2\)를 만족하는 단조증가함수이다.

\begin{defn}
    \([a, b]\)의 분할 \(P := \{x_0, \ldots, x_n\}\)에 대하여, 각 \(i\)에 대해 \(t_i \in [x_{i-1}, x_i]\)를 택한다. 이때 \textbf{리만-스틸체스합}을
    \[
        S(f, P, \alpha) := S(f, P, \alpha_1) - S(f, P, \alpha_2)
    \]
    로 정의한다.
\end{defn}

\begin{thm}
    구간 \([a, b]\)에서 정의된 \(f, \alpha\)에 대하여 다음이 동치이다.
    \begin{enum}
        \item \(f \in \calR(\alpha)\).
        \item 어떤 실수 \(A \in \RR\)가 존재하여, 임의의 \(\eps > 0\)에 대해
        \begin{equation} \label{eq9.4}
            P \supseteq P_0 \implies \abs{S(f, P, \alpha) - A} < \eps \quad (P \in \calP)
        \end{equation}
        를 만족하는 분할 \(P_0 \in \calP\)가 존재한다.
    \end{enum}
    위의 조건들이 성립할 때, \(\displaystyle \int_a^b f d\alpha = A\)이다.
\end{thm}
\begin{proof}
    \((\Rightarrow)\) 정리 \ref{9.1.12}에 의해 다음을 만족하는 분할 \(P_1, P_2 \in \calP\)와 \(A_1, A_2 \in \RR\)가 존재한다.
    \begin{align*}
        &P \supseteq P_1 \implies \abs{S(f, P, \alpha_1) - A_1} < \frac{\eps}{2} \quad (P \in \calP)\\
        &P \supseteq P_2 \implies \abs{S(f, P, \alpha_2) - A_2} < \frac{\eps}{2} \quad (P \in \calP)
    \end{align*}
    이제 \(P_0 := P_1 \cup P_2 \in \calP, A := A_1 - A_2\)라 하면 \(P_0\)의 임의의 세분 \(P \in \calP\)에 대하여
    \[
        \abs{S(f, P, \alpha) - A} \le \abs{S(f, P, \alpha_1) - A_1} + \abs{S(f, P, \alpha_2) - A_2} < \eps
    \]
    이고, \(\displaystyle \int_a^b f d\alpha = A\)이다.\\
    \((\Leftarrow)\) 함수 \(V : [a, b] \to \RR : x \mapsto V_a^x(\alpha)\)로 정의하면, \(V\)는 단조증가함수이다.\\
    \textbf{Claim.} \(f \in \calR(V)\)이다.
    \begin{proof}[Proof of Claim.]
        주어진 양수 \(\eps > 0\)에 대하여 (\ref{eq9.4})\를 만족하는 분할 \(P_0 \in \calP\)를 잡고, \(V_a^b(\alpha, P_1) > V(b) - \eps\)를 만족하는 분할 \(P_1 \in \calP\)를 잡는다. 그리고 \(P_2 := P_0 \cup P_1 \in \calP\)라 하자. 이제 \(P_2\)의 세분 \(P := \{x_0, \ldots, x_n\}\in \calP\)에 대하여
        \[
            U(f, P, V) - L(f, P, V) = \sum_{i=1}^n (M_i - m_i) (\Delta V_i - \abs{\Delta \alpha_i}) + \sum_{i=1}^n (M_i - m_i) \abs{\Delta \alpha_i}
        \]
        로 쓸 수 있다. 먼저 \(\abs{f} \le M\)인 양수 \(M > 0\)에 대하여
        \begin{align*}
            \sum_{i=1}^n (M_i - m_i) (\Delta V_i - \abs{\Delta \alpha_i}) &\le 2M \sum_{i=1}^n (\Delta V_i - \abs{\Delta \alpha_i})\\
            &\le 2M [V(b) - V_a^b(\alpha, P)] < 2M\eps
        \end{align*}
        이다. 그리고
        \[
            I := \{i = 1, \ldots, n : \Delta \alpha_i \ge 0\}, \quad J := \{i = 1, \ldots, n : \Delta \alpha_i < 0\}
        \]
        이라 하고, 각 \(i \in I\)에 대해
        \[
            f(t_i) - f(s_i) > M_i - m_i - \eps
        \]
        를 만족하는 \(t_i, s_i \in [x_{i-1}, x_i]\)를, 각 \(i \in J\)에 대해
        \[
            f(s_i) - f(t_i) > M_i - m_i - \eps
        \]
        를 만족하는 \(t_i, s_i \in [x_{i-1}, x_i]\)를 택하자. 이제
        \begin{align*}
            \sum_{i=1}^n (M_i - m_i) \abs{\Delta \alpha_i} &= \sum_{i \in I} (M_i - m_i) \abs{\Delta \alpha_i} + \sum_{i \in J} (M_i - m_i) \abs{\Delta \alpha_i}\\
            &< \sum_{i \in I} [f(t_i) - f(s_i) + \eps] \abs{\Delta \alpha_i} + \sum_{i \in J} [f(s_i) - f(t_i) + \eps] \abs{\Delta \alpha_i}\\
            &= \sum_{i=1}^n [f(t_i) - f(s_i)] \Delta \alpha_i + \eps \sum_{i=1}^n \abs{\Delta \alpha_i} < 2\eps + \eps V(b)
        \end{align*}
        이다. 마지막 부등호는 (\ref{eq9.4})에 의한 것이다. 따라서
        \[
            U(f, P, V) - L(f, P, V) < (2M + 2 + V(b))\eps
        \]
        이므로, 정리 \ref{9.1.12}에 의해 \(f \in \calR(V)\)이다.
    \end{proof}
    \(V - \alpha\)도 단조증가함수이고,
    \[
        S(f, P, V - \alpha) = S(f, P, V) - S(f, P, \alpha)
    \]
    이다. \(P\)를 세분하여
    \[
        S(f, P, V - \alpha) = = S(f, P, V) - S(f, P, \alpha) \to \int_a^b f dV - A
    \]
    가 되게 할 수 있다. 따라서 정리 \ref{9.1.12}에 의해 \(f \in \calR(V - \alpha)\)이고, 정의에 의해 \(f \in \calR(\alpha)\)이다.
\end{proof}

\begin{thm}
    \(f\)의 \([a, b]\)에서의 불연속점의 집합 \(D\)가 가산집합이라 하자. 이때 \(\alpha \in BV([a, b])\)가 \(D\)의 각 점에서 연속이면 \(f \in \calR(\alpha)\)이다.
\end{thm}
\begin{proof}
    \(\alpha = \alpha_1 - \alpha_2\)로 썼을 때 \(\alpha_1, \alpha_2\)가 모두 \(D\)의 점들에서 연속이도록 할 수 있으면 된다. 이는 정리 \ref{9.5.9}에 의해 가능하다. 
\end{proof}

정리 \ref{9.2.4}, \ref{9.3.3}, \ref{9.3.5}, \ref{9.3.7}, \ref{9.3.11}, \ref{9.3.12}\가 (필요한 경우 적절한 수정을 가하여) 유계변동함수에 대한 리만-스틸체스적분에서도 그대로 성립함을 보이는 것은 연습문제로 남긴다.

다음 정리는 꽤 그럴듯해 보인다.

\begin{thm} \label{9.6.6}
    유계변동함수 \(f, \alpha : [a, b] \to \RR\)에 대하여 \(f \in \calR(\alpha)\)이면 \(\alpha \in \calR(f)\)이고
    \[
        \int_a^b f d\alpha + \int_a^b \alpha df = f(b)\alpha(b) - f(a)\alpha(a)
    \]
    이다.
\end{thm}
\begin{proof}
    임의의 양수 \(\eps > 0\)에 대하여 다음을 만족하는 분할 \(P_0 \in \calP\)를 잡자.
    \[
        P \supseteq P_0 \implies \abs{S(f, P, \alpha) - \int_a^b f d\alpha} < \eps
    \]
    이제 \(P_0\)의 세분 \(P := \{x_0, \ldots, x_n\} \in \calP\)가 주어졌을 때, 각 \(i = 1, \ldots, n\)에 대하여 \(t_i \in [x_{i-1}, x_i]\)를 임의로 택하고 \(Q := P \cup \{t_1, \ldots, t_n\} \in \calP\)라 하자. 이때 \(t_i \in [x_{i-1}, x_i]\)에 의한 리만-스틸체스합을 \(S(\alpha, P, f)\)라 하면
    \begin{align*}
        &S(\alpha, P, f)\\
        =\,& \sum_{i=1}^n \alpha(t_i)f(x_i) - \sum_{i=1}^n \alpha(t_i)f(x_{i-1})\\
        =\,& \sum_{i=1}^n f(x_i) \alpha(x_i) + \sum_{i=1}^n f(x_i)[\alpha(t_i) - \alpha(x_i)] - \sum_{i=1}^n f(x_{i-1})\alpha(x_{i-1}) + \sum_{i=1}^n f(x_{i-1})[\alpha(x_{i-1}) - \alpha(t_i)]\\
        =\,& \sum_{i=1}^n [f(x_i) \alpha(x_i) - f(x_{i-1})\alpha(x_{i-1})] - \sqbracket{\sum_{i=1}^n f(x_i)[\alpha(x_i) - \alpha(t_i)] + \sum_{i=1}^n f(x_{i-1})[\alpha(t_i) - \alpha(x_{i-1})]}\\
        =\,& f(b)\alpha(b) - f(a)\alpha(a) - S(f, Q, \alpha)
    \end{align*}
    이다. 따라서 \(P\)를 세분하여(따라서 \(Q\)를 세분하여)
    \[
        S(\alpha, P, f) = f(b)\alpha(b) - f(a)\alpha(a) - S(f, Q, \alpha) \to f(b)\alpha(b) - f(a)\alpha(a) - \int_a^b f d\alpha
    \]
    가 되게 할 수 있다.
\end{proof}

\begin{ex}
    \[
        \int_0^\pi \sin \cos x dx  = \int_0^\pi \sin x d(\sin x)
    \]
    이다. \([0, \pi]\)에서 정의된 \(f, \alpha : x \mapsto \sin x\)를 생각하면
    \[
        \int_0^\pi \sin x d(\sin x) + \int_0^\pi \sin x d(\sin x) = 0
    \]
    이므로
    \[
        \int_0^\pi \sin \cos x dx = 0
    \]
    이다.
\end{ex}

\begin{lem}
    \(\alpha \in BV([a, b])\)에 대하여
    \[
        \int_a^b d\alpha = \alpha(b) - \alpha(a)
    \]
    이다.
\end{lem}
\begin{proof}
    연습문제로 남긴다.
\end{proof}

\begin{cor}
    \(f : [a, b] \to \RR\)가 단조증가하고 \(g : [a, b] \to \RR\)가 연속이면, 적당한 \(c \in [a, b]\)에 대하여
    \[
        \int_a^b fg = f(a) \int_a^c g + f(b) \int_c^b g
    \]
    이다.
\end{cor}
\begin{proof}
    \(f(a) = f(b)\)이면 증명할 것이 없으므로 \(f(a) < f(b)\)라 하자. 함수 \(G\)를
    \[
        G : x \mapsto \int_a^x g(t) dt
    \]
    로 정의하자. \(G\)는 \(C^1\)-함수이므로 연속인 유계변동함수이고, 따라서 \(f \in \calR(G)\)이다. 한편 \(f\)는 단조함수이므로 유계변동함수이다. 따라서 정리 \ref{9.6.6}에 의해
    \[
        \int_a^b fg = \int_a^b fdG = f(b)G(b) - f(a)G(a) - \int_a^b G df
    \]
    이다. 이때 \(G\)의 최솟값과 최댓값을 각각 \(m, M\)이라 하면
    \[
        m \le \frac{1}{f(b) - f(a)} \int_a^b G df \le M
    \]
    이다. 따라서 사잇값 정리에 의해
    \[
        G(c) = \frac{1}{f(b) - f(a)} \int_a^b G df
    \]
    인 \(c \in [a, b]\)가 존재한다. 따라서
    \begin{align*}
        \int_a^b fg &= f(b)G(b) - f(a)G(a) - \int_a^b G df\\
        &= f(b)G(b) - f(a)G(a) - G(c)[f(b) - f(a)]\\
        &= f(a) \int_a^c g + f(b) \int_b^c g
    \end{align*}
    이다.
\end{proof}

\section{곡선의 길이}

\begin{defn}
    유계닫힌구간 \([a, b] \subseteq \RR\)에서 \(\RR^d\)로 가는 연속함수 \(\gamma : [a, b] \to \RR^d\)를 \textbf{곡선(curve)}이라고 한다. 분할 \(P := \{x_0, \ldots, x_n\} \in \calP\)에 대하여
    \[
        \Lambda(\alpha, P) := \sum_{i=1}^n \norm{\gamma(x_i) - \gamma(x_{i-1})}
    \]
    로 정의하고, \(\gamma\)의 \textbf{길이(length)}를
    \[
        \Lambda(\gamma) := \sup_{P \in \calP} \Lambda(P, \gamma)
    \]
    로 정의한다. 만약 \(\Lambda(\gamma) < \infty\)이면 \(\gamma\)가 \textbf{길이가 유한한 곡선(rectifiable curve)}이라고 한다.
\end{defn}

\begin{defn}
    곡선 \(\gamma = (\gamma_1, \ldots, \gamma_d) : [a, b] \to \RR^d\)에 대하여
    \[
        \int_a^b \gamma := \paren{\int_a^b \gamma_1, \ldots, \int_a^b \gamma_d} \in\RR^d
    \]
    로 정의하자.
\end{defn}

이 절이 끝날 때까지 \(\gamma = (\gamma_1, \ldots, \gamma_d)\)이다.

\begin{prop}
    곡선 \(\gamma : [a, b] \to \RR^d\)에 대하여 다음이 동치이다.
    \begin{enum}
        \item \(\gamma\)는 길이가 유한한 곡선이다.
        \item 각 \(\gamma_k\)가 유계변동함수이다.
    \end{enum}
\end{prop}

\begin{proof}
    \(k = 1, \ldots, d, i = 1, \ldots, n\)에 대하여
    \[
        \abs{\gamma_k(x_i) - \gamma_k(x_{i-1})} \le \norm{\gamma(x_i) - \gamma(x_{i-1})} \le \sum_{j=1}^d \abs{\gamma_j(x_i) - \gamma_j(x_{i-1})}
    \]
    이므로
    \[
        \sum_{i=1}^n \abs{\gamma_k(x_i) - \gamma_k(x_{i-1})} \le \Lambda(\gamma, P) \le \sum_{i=1}^n\sum_{j=1}^d \abs{\gamma_j(x_i) - \gamma_j(x_{i-1})}
    \]
    이다. 따라서
    \[
        V(\gamma_k, P) \le \Lambda(\gamma, P) \le \sum_{j=1}^d V(\gamma_j, P)
    \]
    이므로, \((\Rightarrow)\)와 \((\Leftarrow)\)는 각각 첫 번째와 두 번째 부등호에 의해 증명된다.
\end{proof}

\begin{defn}
    곡선 \(\gamma : [a, b] \to \RR^d\)가 \textbf{\(C^1\)-곡선(\(C^1\)-curve)}이라는 것은 각 \(\gamma_k\)가 \([a, b]\)에서 \(C^1\)-함수라는 것이다. 이때 \(\gamma' := (\gamma_1', \ldots, \gamma_d')\)로 쓰자.
\end{defn}

\(C^1\)-곡선의 길이에 대해서는 우리가 기대하는 바가 있는데, 실제로 그것이 성립함을 증명할 것이다.

\begin{lem}
    \(C^1\)-곡선 \(\gamma : [a, b] \to \RR^d\)에 대하여
    \[
        \int_a^b \gamma' = \gamma(b) - \gamma(a)
    \]
    이고
    \[
        \norm{\int_a^b \gamma'} \le \int_a^b \norm{\gamma'}
    \]
    이다.
\end{lem}
\begin{proof}
    첫 번째 식은 좌변의 정의와 미적분학의 기본정리에 의해 성립한다. 두 번째 식에서
    \[
        x = (x_1, \ldots, x_d) := \int_a^b \gamma' \in \RR^d
    \]
    로 두자. \(x = 0\)이면 두 번째 식이 바로 증명되므로 \(x \ne 0\)이라 하자. 적분의 선형성에 의해
    \[
        \norm{x}^2 = \inner{x}{\int_a^b \gamma'} = \int_a^b \inner{x}{\gamma'}
    \]
    이고, 코시-슈바르츠 부등식에 의해
    \[
        \abs{\inner{x}{\gamma'}} \le \norm{x}\norm{\gamma'}
    \]
    이다. 즉
    \[
        \norm{x}^2 \le \norm{x} \int_a^b \norm{\gamma'}
    \]
    이고 \(x \ne 0\)이므로
    \[
        \norm{x} \le \int_a^b \norm{\gamma'}
    \]
    를 얻는다.
\end{proof}

\begin{thm}
    \(\gamma : [a, b] \to \RR^d\)가 \(C^1\)-곡선이면 \(\gamma\)는 길이가 유한한 곡선이고
    \[
        \Lambda(\gamma) = \int_a^b \norm{\gamma'}
    \]
    이다.
\end{thm}
\begin{proof}
    분할 \(P := \{x_0, \ldots, x_n\} \in \calP\)가 주어졌다고 하자. 이때
    \[
        \Lambda(\gamma, P) = \sum_{i=1}^n \norm{\int_{x_{i-1}}^{x_i} \gamma'} \le \sum_{i=1}^n \int_{x_{i-1}}^{x_i} \norm{\gamma'} = \int_a^b \norm{\gamma'}
    \]
    이므로
    \[
        \Lambda(\gamma) \le \int_a^b \norm{\gamma'}
    \]
    이다. 이제 반대 방향 부등식을 보이자. 임의의 양수 \(\eps > 0\)이 주어졌다고 하면, \(\gamma'\)이 고른연속이므로 다음을 만족하는 양수 \(\delta > 0\)이 존재한다.
    \[
        \abs{s - t} < \delta \implies \norm{\gamma'(s) - \gamma'(t)} < \eps \quad (s, t \in [a, b])
    \]
    이제 \(\norm{P} < \delta\)이도록 분할 \(P := \{x_0, \ldots, x_n\} \in \calP\)를 택하면, 임의의 \(t \in [x_{i-1}, x_i]\)에 대해 \(\\norm{\gamma'(t)} \le \norm{\gamma'(x_i)} + \eps\)라 할 수 있다. 따라서 각 \(i\)에 대해
    \begin{align*}
        \int_{x_{i-1}}^{x_i} \norm{\gamma'(t)}dt &\le \norm{\gamma'(x_i)}\Delta x_i + \eps \Delta x_i\\
        &\le \norm{\int_{x_{i-1}}^{x_i} \gamma'(t)dt} + \norm{\int_{x_{i-1}}^{x_i} [\gamma'(x_i) - \gamma'(t)] dt} + \eps \Delta x_i\\
        &\le \norm{\gamma(x_i) - \gamma(x_{i-1})} + 2\eps\Delta x_i
    \end{align*}
    이므로
    \[
        \int_a^b \norm{\gamma'} = \sum_{i=1}^n \int_{x_{i-1}}^{x_i} \norm{\gamma'} \le \sum_{i=1}^n \norm{\gamma(x_i) - \gamma(x_{i-1})} + 2\eps(b-a) = \Lambda(\gamma, P) + 2\eps(b-a)
    \]
    이다. \(\eps > 0\)이 임의의 양수였으므로
    \[
        \int_a^b \norm{\gamma'} \le \Lambda(\gamma, P)
    \]
    가 성립하여 원하는 등식이 증명된다.
\end{proof}

\chapter{함수열과 함수공간 (1)}

이제 해석개론의 반환점을 돌았다. 전반부에서 많은 내용을 배우긴 했지만, 본질적으로 고등학교와 미적분학에서 배운 것들을 제대로 증명하는 것에서 크게 벗어난 것은 아니었다. 이제부터 \textbf{진짜 해석개론을 시작한다.} 첫 번째 주인공은 연속함수들의 공간이다. 먼저 고른수렴에 의해 연속성이 보존됨을 보이고, 이로부터 옹골 거리공간에서 정의된 연속 실함수 또는 복소함수들의 공간이 바나흐공간임을 보일 것이다. 아르젤라-아스콜리 정리와 스톤-바이어슈트라스 정리를 증명하는 것이 이 장의 목표이다. 두 정리는 각각 연속함수공간의 옹골집합과 조밀집합에 관한 정리이다. 그리고 나서 정의역을 일반적인 거리공간으로 보았을 때 생각할 수 있는 연속함수공간에 대해 공부할 것이다. 이 장이 끝날 때까지 \(X, Y\)는 각각 거리함수 \(d_X, d_Y\)가 주어진 거리공간이고, \(F = \RR\) 또는 \(\CC\)이다.

\section{연속함수열과 연속함수공간}

\begin{defn}
    \(X\)에서 \(Y\)로 가는 모든 연속함수들의 집합을 \(C(X, Y)\)로 쓴다.
\end{defn}

\begin{prop}
    함수 \(\overline{\rho} : C(X, Y) \times C(X, Y) \to \RR\)를
    \[
    \overline{\rho} : (f, g) \mapsto \sup_{x \in X} \min\{d_Y(f(x), g(x)), 1\}
    \]
    로 정의하면 \(\overline{\rho}\)는 \(C(X, Y)\)의 거리함수이다. 또 옹골 거리공간 \(K\)에 대하여 \(\rho : C(K, Y) \times C(K, Y) \to \RR\)를
    \[
        \rho : (f, g) \mapsto \sup_{x \in X} d_Y(f(x), g(x))
    \]
    로 정의하면 \(\rho\)도 \(C(K, Y)\)의 거리함수이다.
\end{prop}
\begin{proof}
    연습문제로 남긴다.
\end{proof}

앞으로 아무 말이 없으면 \(C(X, Y)\)에는 거리함수 \(\overline{\rho}\)가 주어진 것으로 생각하고, 특히 정의역이 옹골집합인 경우에 한정해서 논의할 때는 거리함수 \(\rho\)가 주어진 것으로 생각할 것이다.

\begin{prop} \label{prop 10.1.2}
    \(C(X, Y)\) 안의 함수열 \((f_n)_{n \in \NN}\)과 \(f \in C(X, Y)\)에 대하여 다음이 동치이다.
    \begin{enum}
        \item \(f_n \to f\).
        \item 임의의 양수 \(\eps > 0\)에 대하여 자연수 \(N\)이 존재하여 다음을 만족한다.
        \[
        n \ge N \implies d_Y(f_n(x), f(x)) < \eps \quad (x \in X)    
        \]
    \end{enum}
\end{prop}
\begin{proof}
    연습문제로 남긴다.
\end{proof}

\begin{thm} \label{thm 10.1.3}
    \(Y\)가 완비 거리공간이면 \(C(X, Y)\)도 완비 거리공간이다.
\end{thm}
\begin{proof}
    \quad\\
    \textbf{Step 1.} 함수열 \((f_n)_{n \in \NN}\)이 \(C(X, Y)\)의 코시수열이라고 하자. 즉 임의의 양수 \(0 < \eps < 1\)에 대하여 [\(m,n \ge N\)이면 \(\overline{\rho}(f_m, f_n) < \eps\)]인 자연수 \(N\)이 존재한다. 이때 각 \(x \in X\)에 대해, \(m, n \ge N\)이면 \(d_Y(f_m(x), f_n(x)) \le \overline{\rho}(f_m, f_n) < \eps\)이므로 \((f_n(x))_{n \in \NN}\)은 \(Y\)의 코시수열이다. \(Y\)가 완비 거리공간이므로 \((f_n(x))_{n \in \NN}\)은 \(Y\) 안의 어떤 값으로 수렴하며 이 값을 \(f(x)\)라 하자.\\
    \textbf{Step 2.} 이제 \(f : X \to Y\)가 연속함수임을 보이자. 먼저 \(x_0 \in X\)를 고정하고, \(0 < \eps < 1\)이 주어졌다고 하자. 이때
    \[
      m, n \ge M \implies d_Y(f_m(x), f_n(x)) < \frac{\eps}{3}   \quad (x \in X)
    \]
    를 만족하는 자연수 \(M\)이 존재한다. 이때 \(m \to \infty\)를 취하면
    \begin{equation} \label{eq10.1}
        n \ge M \implies d_Y(f(x), f_n(x)) \le \frac{\eps}{3} \quad (x \in X)  
    \end{equation}
    를 얻는다. 그런데 \(f_M\)은 연속함수이므로, 다음을 만족하는 양수 \(\delta > 0\)이 존재한다.
    \[
    d_X(x, x_0) < \delta \implies d_Y(f_M(x), f_M(x_0)) < \frac{\eps}{3}
    \]
    따라서 \(d_X(x, x_0) < \delta\)이면
    \begin{align*}
        d_Y(f(x), f(x_0)) &\le d_Y(f(x), f_M(x)) + d_Y(f_M(x), f_M(x_0)) + d_Y(f_M(x_0), f(x_0))\\
        &< \frac{\eps}{3} + \frac{\eps}{3} + \frac{\eps}{3} = \eps
    \end{align*}
    이므로 \(f\)는 \(x_0\)에서 연속이다 .\(x_0 \in X\)가 임의의 점이었으므로 \(f \in C(X, Y)\)이다.\\
    \textbf{Step 3.} 마지막으로 \(C(X, Y)\)에서 \(f_n \to f\)인 것만 보이면 되는데, 이는 임의의 \(0 < \eps < 1\)에 대해 자연수 \(M\)이 존재하여 (\ref{eq10.1})\이 성립하는 것으로부터 나온다.
\end{proof}

\begin{defn}
    \(X\)가 집합이고 \(Y\)가 거리공간이라 하자. \(Y^X\) 안의 함수열\footnote{연속함수열일 필요는 없다.} \((f_n)_{n \in \NN}\)과 \(f : X \to Y\)에 대하여 다음을 정의한다.
    \begin{enum}
        \item 각 점 \(x \in X\)에 대해 \(Y\)에서 \(f_n(x) \to f(x)\)이면 \((f_n)_{n \in \NN}\)이 \(f\)로 \textbf{점별수렴(converge pointwise)}한다고 한다.
        \item 명제 \ref{prop 10.1.2}의 (b)가 성립하면 \((f_n)_{n \in \NN}\)이 \(f\)로 \textbf{고른수렴(converge uniformly)}한다고 한다. 점별수렴과 구분하기 위해 \(f_n \rightrightarrows f\)라고 쓰기도 한다.
    \end{enum}
\end{defn}

\begin{cor}
    거리공간 \(X\)에서 완비 거리공간 \(Y\)로 가는 연속함수열 \((f_n)_{n \in \NN}\)에 대하여 다음이 동치이다.
    \begin{enum}
        \item \((f_n)_{n \in \NN}\)이 어떤 함수 \(f : X \to Y\)로 고른수렴한다.
        \item \((f_n)_{n \in \NN}\)은 \(C(X, Y)\)에서 코시수열이다.
        \item 임의의 양수 \(\eps > 0\)에 대하여 자연수 \(N\)이 존재하여 다음을 만족한다.
        \[
        m, n \ge N \implies d_Y(f_m(x), f_n(x)) < \eps \quad (x \in X)    
        \]
    \end{enum}
    위의 조건들이 성립할 때 \(f\)는 연속함수이다.
\end{cor}
\begin{proof}
    연습문제로 남긴다.
\end{proof}

\begin{ex}
    연속함수열의 점별 극한이 연속함수가 아닌 대표적인 예시는 다음과 같다. \(f_n : [0, 1] \to \RR : x \mapsto x^n\)으로 정의하면, \((f_n)_{n \in \NN}\)은 \(\chi_{\{1\}}\)로 점별수렴하지만 그 극한은 연속함수가 아니다.
\end{ex}


\begin{prop}
    \(F\)-벡터공간 \(V\)와 \(f, g \in C(X, V), c \in F\)에 대하여
    \[
    (f+g) : x \mapsto f(x) + g(x), \quad cf : x \mapsto c(f(x))    
    \]
    로 정의하면 \(C(X, V)\)는 자연스러운 \(F\)-벡터공간의 구조를 가지고, \(\overline{\rho}\)에 의한 완비 거리공간의 구조를 가진다. 또 \(K\)가 옹골 거리공간이고 \(V\)가 노름공간일 때
    \[
    \norm{\cdot}_{\sup} : C(K, V) \to \RR : f \mapsto \sup_{x \in K} \norm{f(x)}
    \]
    로 정의하면 \(\norm{\cdot}_{\sup}\)은 \(C(K, V)\)의 노름이 된다. 따라서 \(C(K, V)\)는 자연스러운 노름공간의 구조를 가지며, 이는 거리공간으로서 \((C(K, V), \rho)\)와 같다.
\end{prop}
\begin{proof}
    연습문제로 남긴다.
\end{proof}

따라서 \(V\)가 바나흐공간이면 \(C(K, V)\)도 바나흐공간이다. 특히 \(C(K, F)\)는 바나흐공간이다.

이제 연속함수공간에서의 급수 판정법을 몇 가지 알아본다.

\begin{thm}[바이어슈트라스 판정법: 연속함수열의 경우]
    완비 노름공간 \(V\)에 대하여 \((f_n)_{n \in \NN}\)이 \(C(X, V)\) 안의 함수열이라고 하자. 각 \(n\)에 대하여 [모든 \(x \in X\)에 대해 \(\norm{f_n(x)} \le M_n\)]인 양수 \(M_n > 0\)이 존재하고 \(\sum_n M_n < \infty\)이면 급수 \(\sum_n f_n\)이 고른수렴한다.
\end{thm}
\begin{proof}
    급수 \(\sum_n f_n\)의 부분합 \((F_n)_{n \in \NN}\)이 \(C(X, V)\)에서 코시수열인 것을 보이는 것과 같다. 먼저 임의의 두 자연수 \(m > n\)과 \(x \in X\)에 대해 다음이 성립하는 것을 관찰하자.
    \[
    \norm{F_m(x) - F_n(x)} \le \sum_{k=n+1}^m \norm{f_k} \le \sum_{k=n+1}^m M_k  
    \]
    \(\sum_n M_n < \infty\)이므로, 임의의 \(\eps > 0\)에 대해 다음을 만족하는 자연수 \(N\)이 존재한다.
    \[
    m > n \ge N \implies \sum_{k=n+1}^m M_k < \eps    
    \]
    따라서 \(m > n \ge N\)이면 임의의 \(x \in X\)에 대해 \(\norm{f_m(x) - f_n(x)} < \eps\)이고, 이는 \(\overline{\rho}(F_m, F_n) < \eps\)임을 의미한다. 따라서 \((F_n)_{n \in \NN}\)은 코시수열이다.
\end{proof}

\begin{defn}
    \(\calF \subseteq C(X, Y)\)가 \textbf{고른유계(uniformly bounded)}라는 것은 어떤 양수 \(M > 0\)이 존재하여 임의의 \(x, y \in X, f \in \calF\)에 대해 \(d_Y(f(x), f(y)) < M\)이라는 것이다.
\end{defn}

물론 옹골 거리공간 \(K\)에 대하여 \(\calF \subseteq C(K, Y)\)가 고른유계인 것은 \(\calF\)가 \(\rho\)에 관하여 유계인 것과 동치이다.

아래 판정법들은 \ref{sec5.8}절에서 배운 판정법들과 본질적으로 다른 것이 없으므로 증명은 생략한다. 아래에서 \(V\)는 모두 바나흐공간이다. 

\begin{thm}[디리클레 판정법: 연속함수열의 경우]
    \(C(X, V)\) 안의 연속함수열 \((f_n)_{n \in \NN}\)과 \(C(X, \RR)\) 안의 연속 실함수열 \((g_n)_{n \in \NN}\)이 다음을 만족한다고 하자.
    \begin{enum}
        \item \(\sum_n f_n\)의 부분합이 고른유계이다.
        \item \((g_n)_{n \in \NN}\)은 0으로 고른수렴한다.
        \item \(\sum_n \abs{g_{n+1} - g_n}\)이 고른수렴한다.
    \end{enum}
    이때 \(\sum_n f_n g_n\)은 고른수렴한다.
\end{thm}

\begin{cor}
    \(C(X, V)\) 안의 연속함수열 \((f_n)_{n \in \NN}\)과 \(C(X, \RR)\) 안의 연속 실함수열 \((g_n)_{n \in \NN}\)이 다음을 만족한다고 하자.
    \begin{enum}
        \item \(\sum_n f_n\)의 부분합이 고른유계이다.
        \item \(g_n\)은 0으로 고른수렴하고 각 \(x \in X\)에 대해 \((g_n(x))_{n \in \NN}\)이 단조감소수열이다.
    \end{enum}
    이때 \(\sum_n f_n g_n\)은 고른수렴한다.
\end{cor}

\begin{thm}[아벨 판정법: 연속함수열의 경우]
    \(C(X, V)\) 안의 연속함수열 \((f_n)_{n \in \NN}\)과 \(C(X, \RR)\) 안의 연속 실함수열 \((g_n)_{n \in \NN}\)이 다음을 만족한다고 하자.
    \begin{enum}
        \item \(\sum_n f_n\)이 고른수렴한다.
        \item \((g_n)_{n \in \NN}\)은 고른유계이고 각 \(x \in X\)에 대해 \((g_n(x))_{n \in \NN}\)이 단조감소수열이다.
    \end{enum}
    이때 \(\sum_n f_n g_n\)은 고른수렴한다.
\end{thm}

\begin{ex}
    0으로 단조감소하면서 수렴하는 수열 \((a_n)_{n \in \NN}\)과 양수 \(\delta \in (0, \pi)\)에 대하여 급수
        \[
        \sum_n a_n e^{inx}    
        \]
        가 \([-\delta, 2\pi - \delta]\)에서 고른수렴함을 보이자. 디리클레 판정법을 이용하면 \(\sum_n e^{inx}\)의 부분합이 \([-\delta, 2\pi - \delta]\)에서 고른유계인 것만 보이면 된다. 임의의 \(x \in [-\delta, 2\pi - \delta]\)에 대하여,
        \[
        \abs{\sum_{k=0}^n e^{ikx}} = \abs{\frac{1 - e^{i(n+1)x}}{1 - e^{ix}}} \le \frac{2}{\abs{1 - e^{ix}}} \le \frac{2}{\abs{1 - e^{i\delta}}} 
        \]
        이므로 \(\sum_n e^{inx}\)의 부분합이 \([-\delta, 2\pi - \delta]\)에서 고른유계이다. 그리고 주어진 급수의 실수부분과 허수부분을 생각하면,
        \[
        \sum_n a_n \cos nx, \quad \sum_n b_n \sin nx    
        \]
        도 \([-\delta, 2\pi - \delta]\)에서 고른수렴한다.
\end{ex}

\section{아르젤라-아스콜리 정리}

하이네-보렐 정리에 의해, 유클리드공간의 임의의 유계닫힌집합은 옹골집합이었다. 그러나 일반적인 노름공간에서 이는 성립하지 않는다. 노름공간 \(C([0, 1], \RR)\)의 부분집합
\[
    \calF := \{f \in C([0, 1], \RR) : \norm{f}_{\sup} \ge 1\}    
\]
는 유계닫힌집합이다. 이때 \(\calF\) 안의 함수열 \((f_n)_{n \in \NN}\)을 다음과 같이 정의하자.
\[
f_n : x \mapsto \begin{cases}
    2^{n+2}(x - 2^{-n-1}) &\textif \ 2^{-n-1} \le x < \frac{2^{-n-1} + 2^{-n}}{2}\\
    -2^{n+2}(x - 2^{-n}) &\textif \ \frac{2^{-n-1} + 2^{-n}}{2} \le x < 2^{-n}\\
    0 &\otw
\end{cases}    
\]
이때 \(m \ne n\)이면 \(\norm{f_m - f_n}_{\sup} = 1\)이므로 \((f_n)_{n \in \NN}\)의 어떤 부분수열도 코시수열이 아니다. 따라서 \(\calF\)는 수열 옹골집합이 아니며, 정리 \ref{thm 6.2.4}에 의해 이는 옹골집합이 아닌 것과 동치이다.

이제 옹골집합에서 \(\RR^d\)로 가는 연속함수들의 공간의 부분집합이 옹골집합일 필요충분조건을 제시한다. 이를 위하여 몇 가지 준비가 필요하다.

\begin{defn}
    \(X\)가 \textbf{완전유계(totally bounded)}라는 것은 임의의 양수 \(\eps > 0\)에 대하여 \(X\)를 반지름이 \(\eps\)인 유한 개의 열린 공들의 합집합으로 쓸 수 있다는 것이다. 즉 \(X\)의 유한 부분집합 \(X_{\eps}\)가 존재하여 \(X = \bigcup_{x \in X_{\eps}} B_X(x, \eps)\)인 것이다.
\end{defn}


\begin{prop} \label{10.2.2}
    옹골 거리공간 \(K\)는 완전유계이다.
\end{prop}
\begin{proof}
    임의의 \(\eps > 0\)에 대해 \(\{B(x, \eps)\}_{x \in K}\)가 \(K\)의 열린덮개이므로 \(\{B(x, \eps)\}_{x \in K}\)의 유한부분덮개가 존재한다. 따라서 정의에 의해 \(K\)는 완전유계이다.
\end{proof}

\begin{defn}
    \(\calF \subseteq C(X, Y)\)가 \textbf{\(x_0 \in X\)에서 동등연속(equicontinuous at \(x_0 \in X\))}이라는 것은 임의의 양수 \(\eps > 0\)에 대하여 양수 \(\delta > 0\)이 존재하여 다음을 만족하는 것이다.
    \[
    d_X(x, x_0) < \delta \implies d_Y(f(x), f(x_0)) < \eps \quad (x \in X, f \in \calF)    
    \]
    \(\calF\)가 \(X\)의 모든 점에서 동등연속이면 \(\calF\)가 \textbf{동등연속(equicontinuous)}이라고 한다. 그리고 임의의 양수 \(\eps > 0\)에 대해서
    \[
    d_X(x, y) < \delta \implies d_Y(f(x), f(y)) < \eps \quad (x, y \in F, f \in \calF)    
    \]
    를 만족하는 양수 \(\delta > 0\)이 존재하면 \(\calF\)는 \textbf{고른동등연속(uniformly equicontinuous)}이라고 한다. 
\end{defn}

고른동등연속이 동등연속을 함축함은 자명하다.

\begin{prop} \label{10.2.4}
    옹골 거리공간 \(K\)에 대하여 \(\calF \subseteq C(K, Y)\)가 동등연속이면 \(\calF\)는 고른동등연속이다.
\end{prop}
\begin{proof}
    임의의 양수 \(\eps > 0\)이 주어졌을 때, 각 \(x \in K\)에 대해 다음을 만족하는 양수 \(\delta_x > 0\)이 존재한다. 
    \[
    d_K(x, y) < \delta_x \implies d_Y(f(x), f(y)) < \frac{\eps}{2} \quad (y \in K, f \in \calF)
    \]
    이때 \(\{B_K(x, \delta_x)\}_{x \in K}\)는 \(K\)의 열린덮개이므로, 이 덮개의 르베그 수 \(\delta > 0\)이 존재한다. 이제 임의의 \(x, y \in K\)에 대해 \(d_X(x, y) < \delta\)이면 \(\diam \{x, y\} < \delta\)이므로 어떤 \(x_0 \in K\)에 대해 \(x, y \in B_K(x_0, \delta_{x_0})\)이다. 따라서 임의의 \(f \in \calF\)에 대해
    \[
    d_Y(f(x), f(y)) \le d_Y(f(x), f(x_0)) + d_Y(f(x_0), f(y)) < \frac{\eps}{2} + \frac{\eps}{2} = \eps    
    \]
    이므로 \(\calF\)는 고른동등연속이다.
\end{proof}

이제 완전유계와 동등연속의 관계를 알아보자.

\begin{prop} \label{10.2.5}
    \(\calF \subseteq C(X, Y)\)가 완전유계이면 동등연속이다.
\end{prop}
\begin{proof}
    임의의 \(0 < \eps < 1, x_0 \in X\)가 주어졌다고 하자. \(\calF\)가 완전유계이므로, \(\calF\)의 유한 부분집합 \(\{f_1, \ldots, f_n\}\)이 존재하여
    \[
    \calF \subseteq \bigcup_{i=1}^n B_{\overline{\rho}}(f_i, \eps/3)    
    \]
    이 성립한다. 각 \(i = 1, \ldots, n\)에 대해 \(f_i\)가 \(x_0\)에서 연속이므로 양수 \(\delta_i > 0\)이 존재하여
    \[
    d_X(x, x_0) < \delta_i \implies d_Y(f_i(x), f_i(x_0)) < \frac{\eps}{3} \quad (x \in X)    
    \]
    가 성립한다. 이때 \(\delta := \min_{1 \le i \le n} \delta_i\)로 두자. 그러면 임의의 \(f \in \calF, x \in B_X(x_0, \delta)\)에 대해, \(f \in B_{\overline{\rho}}(f_i, \eps/3)\)이도록 하는 \(i = 1, \ldots, n\)이 존재한다. 이때
    \begin{align*}
        d_Y(f(x), f(x_0)) &\le d_Y(f(x), f_i(x)) + d_Y(f_i(x), f_i(x_0)) + d_Y(f_i(x_0), f(x_0))\\
        &< \frac{\eps}{3} + \frac{\eps}{3} + \frac{\eps}{3} = \eps
    \end{align*}
    이므로 \(\calF\)는 \(x_0\)에서 동등연속이고, \(x_0 \in X\)가 임의의 점이었으므로 \(\calF\)는 동등연속이다.
\end{proof}

\begin{defn}
    \(\calF \subseteq C(X, Y)\)가 \textbf{점별유계(pointwise bounded)}라는 것은 각 \(x \in X\)에 대해 \(\{f(x) \in Y : f \in \calF\}\)가 \(Y\)에서 유계라는 것이다.
\end{defn}

이제 아르젤라-아스콜리 정리를 진술하고 한쪽 방향을 증명하자.

\begin{thm}[아르젤라-아스콜리 정리] \label{10.2.7}
    옹골 거리공간 \(K\)와 \(\calF \subseteq C(K, \RR^d)\)에 대하여 다음이 동치이다.
    \begin{enum}
        \item \(\calF\)의 \(C(K, \RR^d)\)에서의 닫힘이 옹골집합이다.
        \item \(\calF\)가 고른동등연속이고 점별유계이다.
    \end{enum}
\end{thm}
\begin{proof}
    먼저 (a)\(\Rightarrow\)(b)를 증명한다. \(\overline{\calF}\)가 옹골집합이라고 가정하면 명제 \ref{10.2.2}에 의해 \(\overline{\calF}\)는 완전유계이고, 따라서 명제 \ref{10.2.5}에 의해 동등연속이다. 또 \(K\)가 옹골집합이므로 명제 \ref{10.2.4}에 의해 \(\overline{\calF}\)는 고른동등연속이다. 또 \(\overline{\calF}\)가 옹골집합이므로 \(\overline{\calF}\)는 \(C(K, \RR^d)\)의 노름에 대해 유계이고, 이는 \(\overline{\calF}\)가 점별유계임을 의미한다. 이제 \(\overline{\calF}\)가 고른동등연속이고 점별유계이면 \(\calF \subseteq \overline{\calF}\)는 자명하게 고른동등연속이고 점별유계이다.
\end{proof}

이제 정리 \ref{10.2.7}의 (b)\(\Rightarrow\)(a)를 증명하는 데 필요한 사실들을 살펴본다.

\begin{lem} \label{10.2.8}
    옹골 거리공간 \(K\)에 대하여 \(\calF \subseteq C(K, Y)\)가 동등연속이고 점별유계이면 \(\calF\)의 \(C(K, Y)\)에서의 닫힘도 동등연속이고 점별유계이다.
\end{lem}
\begin{proof}
    임의의 \(\eps >0 , x_0 \in K\)에 대하여 \(\calF\)의 동등연속성에 의해 다음을 만족하는 양수 \(\delta > 0\)이 존재한다.
    \[
    d_K(x, x_0) < \delta \implies d_Y(f(x), f(x_0)) < \frac{\eps}{3}    \quad (x \in K)
    \]
    이제 각 \(g \in \overline{\calF}\)에 대해 \(\rho(f, g) < \eps/3\)인 \(f \in \calF\)가 존재한다. 따라서 \(d_K(x, x_0 ) < \delta \)이면
    \begin{align*}
        d_Y(g(x), g(x_0)) &\le d_Y(g(x), f(x)) + d_Y(f(x), f(x_0)) + d_Y(f(x_0), g(x_0))\\
        &< \frac{\eps}{3} + \frac{\eps}{3} + \frac{\eps}{3} = \eps
    \end{align*}
    이므로 \(\overline{\calF}\)는 \(x_0\)에서 동등연속이고, \(x_0 \in K\)가 임의의 점이었으므로 \(\overline{\calF}\)는 동등연속이다.\\
    한편 각 \(x \in K\)를 고정했을 때 [모든 \(f, f' \in \calF\)에 대해 \(d_Y(f(x), f'(x)) \le M\)]인 양수 \(M\)이 존재한다. 이제 임의의 \(g, g' \in \overline{\calF}\)에 대해 \(\rho(f, g), \rho(f', g') < 1\)인 \(f, f' \in \calF\)가 존재하는데, 이때
    \[
    d_Y(g(x), g'(x)) \le d_Y(g(x), f(x)) + d_Y(f(x), f'(x)) + d_Y(f'(x), g'(x)) < M+2    
    \]
    이므로 \(\{g(x) \in Y : x \in K\}\)도 \(Y\)에서 유계이다. 따라서 \(\overline{\calF}\)는 점별유계이다.
\end{proof}

\begin{lem} \label{10.2.9}
    가산 거리공간 \(D\)에 대하여 \(\calF \subseteq C(D, \RR^d)\)가 점별유계이면 \(\calF\) 안의 임의의 함수열은 \(D\)의 각 점에서 점별수렴하는 부분수열을 가진다.
\end{lem}
\begin{proof}
    \(D = \{x_n\}_{n \in \NN}\)으로 쓰고, \((f_n)_{n \in \NN}\)이 \(\calF\) 안의 함수열이라 하자. \(\RR^d\) 안의 수열 \((f_n(x_0))_{n \in \NN}\)은 유계수열이므로 수렴하는 부분수열을 가지는데, 이를 \((f_{0n}(x_0))_{n \in \NN}\)으로 쓰자. \((f_{0n}(x_1))_{n \in \NN}\)도 \(\RR^d\)의 유계수열이므로 수렴하는 부분수열을 가지며, 이를 \((f_{1n}(x_1))_{n \in \NN}\)이라 쓰자. 이를 반복하여, 각 \(k \in \NN\)에 대해 \((f_n)_{n \in \NN}\)의 부분수열 \((f_{kn})_{n \in \NN}\)을 얻는다. 그런데 각 \(k\)에 대해 \((f_{(k+1)n})_{n \in \NN}\)은 \((f_{kn})_{n \in \NN}\)의 부분수열이고, \((f_{kn})_{n \in \NN}\)은 \(x_k\)에서 점별수렴하므로 \((f_{kn})_{n \in \NN}\)은 \(x_0, \ldots, x_k\)에서 점별수렴한다. 이제 \(g_k := f_{kk}\)로 두면 \((g_k)_{k \in \NN}\)은 \(D\)에서 점별수렴한다.
\end{proof}

\begin{prop} \label{10.2.10}
    옹골 거리공간 \(K\)는 가산 조밀집합 \(D \subseteq K\)를 가진다. 
\end{prop}
\begin{proof}
    \(K\)는 완전유계이므로 각 \(n \in \ZZ_+\)에 대해 \(K\)의 유한 부분집합 \(D_n\)이 존재하여 
    \[
    K = \bigcup_{x \in D_n} B(x, 1/n)    
    \]
    이다. \(D := \bigcup_n D_n\)으로 두면 \(D\)는 가산집합이다. 또 임의의 \(x \in K, n \in \ZZ_+\)에 대해 \(d(x, y) < 1/n\)인 \(y \in D_n \subseteq D\)를 찾을 수 있으므로 \(D\)는 \(K\)에서 조밀하다.
\end{proof}

\begin{lem} \label{10.2.11}
    옹골 거리공간 \(K\)와 완비 거리공간 \(Y\)에 대하여 \(\calF \subseteq \calF\)가 동등연속이라고 하자. \(\calF\) 안의 함수열 \((f_n)_{n \in \NN}\)이 \(K\)의 조밀집합 \(D\)의 각 점에서 점별수렴하면 \((f_n)_{n \in \NN}\)은 \(C(K, Y)\)에서 수렴한다.
\end{lem}
\begin{proof}
    \(Y\)가 완비 거리공간이므로 \(C(K, Y)\)가 완비 거리공간이고 따라서 \((f_n)_{n \in \NN}\)이 코시수열임을 보이면 된다. \(K\)가 옹골집합이므로 \(\calF\)는 고른동등연속이고, 임의의 \(\eps > 0\)에 대해 다음을 만족하는 양수 \(\delta > 0\)이 존재한다.
    \[
    d_K(x, y) < \delta \implies d_Y(f(x), f(y)) < \frac{\eps}{3} \quad (x, y \in K, f \in \calF)    
    \]
    이때 \(\{B_K(x, \delta)\}_{x \in D}\)가 \(K\)의 열린덮개이므로, \(D\)의 유한 부분집합 \(\{x_1, \ldots, x_k\}\)이 존재하여
    \[
    K = \bigcup_{i=1}^k B_K(x_i, \delta)    
    \]
    이다. 그리고 각 \(i = 1, \ldots, k\)에 대해 \((f_n(x_i))_{n \in \NN}\)이 \(Y\)에서 수렴하는 수열이므로 코시수열이다. 따라서 다음을 만족하는 자연수 \(N\)이 존재한다.
    \[
    m, n \ge N \implies d_Y(f_m(x_i), f_n(x_i)) < \frac{\eps}{3} \quad (i = 1, \ldots, n)
    \]
    이제 각 \(x \in X\)에 대해 \(x \in B(x_i, \delta)\)인 \(i = 1, \ldots, k\)가 존재하고, \(m, n \ge N\)이면
    \begin{align*}
        d_Y(f_m(x), f_n(x)) &\le d_Y(f_m(x), f_m(x_i)) + d_Y(f_m(x_i), f_n(x_i)) + d_Y(f_n(x_i), f_n(x))\\
        &< \frac{\eps}{3} + \frac{\eps}{3} + \frac{\eps}{3} = \eps
    \end{align*}
    에서 \(\rho(f_m, f_n) \ge \eps\)이다. 따라서 \((f_n)_{n \in \NN}\)이 \(C(K, Y)\)의 코시수열이므로 수렴한다.
\end{proof}

이제 정리 \ref{10.2.7}의 (b)\(\Rightarrow\)(a)를 증명할 준비가 되었다.
\begin{proof}[Proof of Theorem \ref*{10.2.7}.]
    (b)를 가정하면 보조정리 \ref{10.2.8}\과 명제 \ref{10.2.4}에 의해 \(\overline{\calF}\)도 고른동등연속이고 점별유계이다. 이제 \(\overline{\calF}\)가 옹골집합임을 보이는데, 이는 \(\overline{\calF}\)가 수열 옹골집합임을 보이는 것과 동치이다. \((f_n)_{n \in \NN}\)이 \(\overline{\calF}\) 안의 임의의 함수열이라 하자. 명제 \ref{10.2.10}에 의해 \(K\)는 가산 조밀집합 \(D \subseteq K\)를 가지는데, 보조정리 \ref{10.2.9}에 의해 \((f_n)_{n \in \NN}\)은 \(D\)의 각 점에서 점별수렴하는 부분수열 \((g_k)_{k \in \NN}\)을 가진다. 이제 \(\RR^d\)가 완비 거리공간이므로 보조정리 \ref{10.2.11}\을 적용하면 \((g_k)_{k \in \NN}\)은 \(C(K, \RR^d)\)에서 수렴한다. 그런데 \(\overline{\calF}\)가 닫힌집합이므로 \((g_k)_{k \in \NN}\)은 \(\overline{\calF}\) 안에서 수렴한다. 따라서 \((f_n)_{n \in \NN}\)은 \(\overline{\calF}\) 안에서 수렴하는 부분수열을 가지므로, \(\overline{\calF}\)는 수열 옹골집합이고 따라서 옹골집합이다.
\end{proof}

\begin{cor}
    옹골 거리공간 \(K\)와 \(\calF \subseteq C(K, \RR^d)\)에 대하여 다음이 동치이다.
    \begin{enum}
        \item \(\calF\)는 옹골집합이다.
        \item \(\calF\)는 고른동등연속이고 점별유계이고 \(C(K, \RR^d)\)에서 닫힌집합이다.
    \end{enum}
\end{cor}


\section{스톤-바이어슈트라스 정리}

이번 절에서는 \(C(K, \RR)\) 또는 \(C(K, \CC)\)의 조밀집합을 찾는다. 먼저 \(K\)가 실수의 유계닫힌구간인 경우부터 보자.

\begin{lem} \label{10.3.1}
    \(n \ge 1, x \in [0, 1]\)에 대해 \((1 - x^2)^n \ge 1 - nx^2\)이다.
\end{lem}
\begin{proof}
    \(t \in [0, 1]\)에 대해 \(f(t) := (1-t)^n +nt - 1 \ge 0\)임을 보이는 것과 같다. \(f(0) = 0\)이고 \(f'(t) = -n(1-t)^{n-1} + n = n[1 - (1-t)^{n-1}] \ge 0\)이므로 \([0, 1]\)에서 \(f \ge 0\)이다.
\end{proof}

\begin{lem} \label{10.3.2}
    각 \(n \in \ZZ_+\)에 대해 함수 \(Q_n : [0, 1] \to \RR\)를
    \[
    Q_n : x \mapsto c_n (1-x^2)^n, \quad c_n = \sqbracket{\int_{-1}^1 (1-x^2)^n dx}^{-1}
    \]
    로 정의하자. 그러면 임의의 \(0 < \delta < 1\)에 대해 \(\delta \le \abs{x} \le 1\)이면 \(Q_n(x) \le \sqrt{n}(1-\delta^2)^n\)이다.
\end{lem}
\begin{proof}
    보조정리 \ref{10.3.1}에 의해,
    \[
        \frac{1}{c_n} = \int_{-1}^1 (1-x^2)^n \ge 2 \int_0^{1/\sqrt{n}} (1-nx^2) dx = \frac{4}{3\sqrt{n}} \ge \frac{1}{\sqrt{n}}    
    \]
    이므로 \(c_n \le \sqrt{n}\)이다. 따라서 \(\delta \le \abs{x} \le 1\)이면 \(Q_n(x) \le c_n (1 - \delta^2)^n \le \sqrt{n}(1-\delta^2)^n\)이다.
\end{proof}

\begin{thm}[바이어슈트라스 정리]
    유계닫힌구간 \(I := [a, b] \subseteq \RR\)에서 정의된 실수계수 다항함수들의 집합은 \(C(I, \RR)\)에서 조밀하다.
\end{thm}
\begin{proof}
    일반성을 잃지 않고 \(I = [0, 1]\)이라 가정해도 무방하다. 이제 \(f \in C(I, \RR)\)에 대하여, \(f\)로 고른수렴하는 다항함수열 \((P_n)_{n \in \ZZ_+}\)을 찾는다. 그런데 \(\tilde{f}(x) = f(x) - f(0) - [f(1) - f(0)]x \)로 쓰면 \(\tilde{f}(0) = \tilde{f}(1) = 0\)이고, \(\tilde{f}\)로 수렴하는 다항함수열이 존재하면 \(f\)로 수렴하는 다항함수열도 존재한다. 따라서 처음부터 \(f(0) = f(1) = 0\)이었다고 가정할 수 있다. 그리고 \(f\)의 정의역을 실수 전체로 확장하여 \([0, 1]\) 밖에서는 0으로 정의하자. 이때 \(f\)는 \(\RR\) 전체에서 고른연속이다.\\
    보조정리 \ref{10.3.2}에서와 같이 각 \(n \in \ZZ_+\)에 대해 다항함수 \(Q_n\)을 정의하고, \(P_n : [0, 1] \to \RR\)를
    \[
    P_n : x \mapsto \int_0^1 f(t) Q_n(x - t) dt  
    \]
    로 정의하면 \(P_n\)은 다항함수이다. 그런데 \(f\)의 함숫값이 \([0, 1]\) 밖에서 0이므로
    \[
    P_n(x) = \int_{-x}^{1-x} f(x+t)Q_n(t) dt = \int_{-1}^1 f(x+t) Q_n(t) dt
    \]
    이다. 임의의 양수 \(\eps > 0\)에 대하여, \(f\)가 \(\RR\)에서 고른연속이므로 다음을 만족하는 양수 \(0 < \delta < 1\)이 존재한다.
    \[
    \abs{x - y} < \delta \implies \abs{f(x) - f(y)} < \frac{\eps}{2} \quad (x, y \in \RR)
    \]
    \(M := \norm{f}_{\sup}\)으로 두면 임의의 \(x \in [0, 1]\)에 대해
    \begin{align*}
        \abs{P_n(x) - f(x)} &= \abs{\int_{-1}^1 (f(x+t) - f(x))Q_n(t) dt}\\
        &\le \int_{-1}^1 \abs{f(x+t)-f(t)} Q_n(t) dt\\
        &= \int_{\delta \le \abs{t} \le 1} \abs{f(x+t)-f(t)} Q_n(t) dt + \int_{\abs{t} \le \delta} \abs{f(x+t)-f(t)} Q_n(t) dt\\
        &\le 2M \int_{\delta \le \abs{t} \le 1} Q_n(t) dt + \frac{\eps}{2} \int_{\abs{t} \le \delta} Q_n(t) dt\\
        &\le 2M\sqrt{n}(1-\delta^2)^n + \frac{\eps}{2}
    \end{align*}
    이다. \(n \to \infty\)일 때 \(2M\sqrt{n}(1-\delta^2)^n \to 0\)이므로 [\(n \ge N\)이면 \(2M\sqrt{n}(1 - \delta^2)^2 < \eps/2\)]인 자연수 \(N\)이 존재한다. 따라서 \(n \ge N\)이면 \(\norm{P_n - f}_{\sup} \le \eps\)이므로 다항함수열 \((P_n)_{n \in \ZZ_+}\)이 \(f\)로 \(C(I, \RR)\)에서 수렴한다. 따라서 실수계수 다항함수들의 집합은 \(C(I, \RR)\)에서 조밀하다.
\end{proof}

이제 바이어슈트라스 정리를 일반화해보자. \(F = \RR\) 또는 \(\CC\).

\begin{defn}
    \(F\)-벡터공간 \(\calA\)에 대하여, \(\calA\) 위의 이항 연산 \(\cdot : \calA \to \calA \times \calA\)\footnote{\(\cdot\)은 표기에서 주로 생략한다.}가 정의되어 다음 (A1)-(A5)를 만족할 때 \((\calA, \cdot)\) 또는 간단히 \(\calA\)를 \textbf{\(F\)-가환대수(commutative algebra over \(F\))} 또는 \textbf{\(F\)-대수(algebra over \(F\))}라고 부른다. 그리고 \(\cdot\)을 곱셈(multiplication)이라고 부른다.
    \begin{enumerate}[label=(A\arabic*), leftmargin=2\parindent]
        \item 임의의 \(x, y, z \in \calA\)에 대해 \((xy)z = x(yz)\).
        \item \(1 \in \calA\)가 존재하여 [임의의 \(x \in \calA\)에 대해 \(1x = x1 = x\)].
        \item 임의의 \(x, y, z \in \calA\)에 대해 \(x(y+z) = xy+xz\)이고 \((x+y)z = xz + yz\).
        \item 임의의 \(x, y \in \calA, c \in F\)에 대해 \(c(xy) = (cx)y = x(cy)\).
        \item 임의의 \(x, y \in \calA\)에 대해 \(xy = yx\).
    \end{enumerate}
\end{defn}

\begin{ex}
    \quad

    \begin{enum}
        \item 체 \(F\)는 자기 자신 위의 대수이다.
        \item \(C(X, F)\)의 두 원소 \(f, g \in C(K, F)\)에 대하여 \(fg \in C(X, F)\)를 \(fg : x \mapsto f(x)g(x)\)로 정의하면 \(C(X, F)\)는 \(F\)-대수의 구조를 가진다.
        \item \(\RR\)의 유계닫힌구간에서 정의된 모든 실수계수 다항함수의 집합에 자연스럽게 덧셈, 스칼라곱, 곱셈을 정의하면 이는 \(\RR\)-대수의 구조를 가진다.
    \end{enum}
\end{ex}

\begin{prop}
    옹골 거리공간 \(K\)에 대하여 \(\calA \subseteq C(K, F)\)가 \(C(K, F)\)의 부분대수라고 하자. 그러면 \(\calA\)의 \(C(K, F)\)에서의 닫힘도 \(C(K, F)\)의 부분대수이다.
\end{prop}
\begin{proof}
    임의의 \(f, g \in \overline{\calA}, c \in F\)에 대해 \(f+g, cf, fg \in \overline{\calA}\)인 것을 보이면 된다. \(f, g\)로 각각 수렴하는 \(\calA\) 안의 함수열 \((f_n)_{n \in \NN}, (g_n)_{n \in \NN}\)을 잡고, 임의의 \(\eps > 0\)이 주어졌다고 하자. 이때 [\(n \ge N\)이면 \(\norm{f_n - f}_{\sup}, \norm{g_n - g}_{\sup} < \eps\)]인 자연수 \(N\)이 존재한다. 이제 \(n \ge N\)이면
    \begin{align*}
        \norm{(f_n + g_n) - (f + g)}_{\sup} &\le \norm{f_n - f}_{\sup} + {g_n - g}_{\sup} < 2\eps\\
        \norm{cf_n - cf}_{\sup} &= \abs{c}\norm{f_n - f}_{\sup} < c\eps\\
        \norm{f_n g_n - fg}_{\sup} &\le \norm{f_n(g_n - g) + g(f_n - f)}_{\sup}\\
        &\le \norm{f_n}_{\sup} \norm{g_n - g}_{\sup} + \norm{g}_{\sup}\norm{f_n - f}_{\sup}\\
        &< \eps(\norm{f}_{\sup} + \eps) + \eps \norm{g}_{\sup}
    \end{align*}
    이므로 \(C(K, F)\)에서 \(f_n + g_n \to f + g, cf_n \to cf, f_n g_n \to fg\)이다. 그런데 \(\calA\)가 덧셈, 스칼라곱, 곱셈에 대해 닫혀있으므로 \(f_n + g_n, cf_n, f_n g_n \in \calA\)이고, 따라서 \(f+g, cf, fg \in \overline{\calA}\)이다.
\end{proof}

\begin{prop} \label{10.3.6}
    옹골 거리공간 \(K\)에 대하여 \(\calA \subseteq C(K, \RR)\)가 \(C(K, \RR)\)의 닫힌 부분대수라고 할 때, 다음이 성립한다.
    \begin{enum}
        \item \(f \in \calA\)이면 \(\abs{f} \in \calA\)이다.
        \item \(f_1, \ldots, f_n \in \calA\)이면 \(\max_{1 \le i \le n} f_i, \min_{1 \le i \le n} f_i \in \calA\)이다.
    \end{enum}
\end{prop}
\begin{proof}
    먼저 (a)가 성립한다고 가정하고 (b)를 보이겠다. \(n = 2\)일 때만 보이면 귀납적으로 나머지 \(n\)에 대해서도 성립한다. 이때
    \[
    \max\{f_1, f_2\} = \frac{f_1 + f_2}{2} + \frac{\abs{f_1 - f_2}}{2}, \quad \min\{f_1, f_2\} = \frac{f_1 + f_2}{2} - \frac{\abs{f_1 - f_2}}{2}    
    \]
    이므로 \(\max\{f_1, f_2\}, \min\{f_1, f_2\} \in \calA\)이다. 따라서 (a)만 보이면 된다.\\
    임의의 양수 \(\eps > 0\)이 주어졌다고 하자. \(M := \norm{f}_{\sup}\)으로 두면 바이어슈트라스 정리에 의해 다음을 만족하는 다항함수 \(\sum_{k=0}^m a_k t^k\)가 존재한다.
    \[
        \abs{\sum_{k=0}^m a_k t^k - \abs{t}} < \eps \quad (t \in [-M, M])    
    \]
    이제 \(g := \sum_{k=0}^m c_k f^k \in \calA\)로 두면 \(\norm{g - \abs{f}}_{\sup} \le \eps\)이다. \(\calA\)가 \(C(K, \RR)\)의 닫힌집합이고 \(\eps > 0\)이 임의의 양수였으므로 \(\abs{f} \in \overline{\calA} = \calA\)이다.
\end{proof}

\begin{defn}
    부분대수 \(\calA \subseteq C(X, F)\)가 \textbf{\(X\)의 점들을 분리한다(separates points in \(X\))}는 것은, 임의의 서로 다른 두 점 \(x_1, x_2 \in X\)에 대하여 어떤 \(f \in \calA\)가 존재하여 \(f(x_1) \ne f(x_2)\)가 성립하는 것이다.
\end{defn}

위 정의의 직접적인 응용은 다음과 같다.

\begin{prop} \label{10.3.8}
    부분대수 \(\calA \subseteq C(X, F)\)가 \(X\)의 점들을 분리한다고 하자. 이때 임의의 서로 다른 두 점 \(x_1, x_2 \in X\)와 \(c_1, c_2 \in F\)에 대하여 어떤 \(f \in \calA\)가 존재하여 다음을 만족한다.
    \[
    f(x_1) = c_1, \quad f(x_2) = c_2    
    \]
\end{prop}
\begin{proof}
    가정에 의해 \(g(x_1) \ne g(x_2)\)를 만족하는 \(g \in \calA\)가 존재한다. 이때
    \[
    f := \frac{c_1 [g - g(x_2)]}{g(x_1) - g(x_2)} + \frac{c_2[g - g(x_1)]}{g(x_2) - g(x_1)} \in \calA
    \]
    가 주어진 조건을 만족한다.
\end{proof}

이제 옹골 거리공간 \(K\)에 대하여 \(\calA \subseteq C(K, \RR)\)가 조밀한 부분대수일 충분조건을 쓸 수 있다.

\begin{thm}[스톤-바이어슈트라스 정리: \(F = \RR\)인 경우]
    옹골 거리공간 \(K\)에 대하여 부분대수 \(\calA \subseteq C(K, \RR)\)가 \(K\)의 점들을 분리하면 \(\calA\)는 \(C(K, \RR)\)에서 조밀하다.
\end{thm}
\begin{proof}
    먼저 다음 주장을 가정하면 정리가 증명됨을 보이자.\\
    \textbf{Claim.} 임의의 \(f \in C(K, \RR), x \in K, \eps > 0\)에 대하여 [\(g_x(x) = f(x)\)이고 각 \(y \in K\)에 대해 \(g_x(y) > f(y) - \eps\)]을 만족하는 \(g_x \in \overline{\calA}\)가 존재한다.\\
    Claim을 가정하면, 각 \(x \in K\)에 대해 \(g_x - f\)가 연속함수이므로 [\(y \in U_x\)이면 \(g_x(y) < f(y) + \eps\)]인 \(x\)의 근방 \(U_x \subseteq K\)가 존재한다. \(\{U_x\}_{x \in K}\)가 \(K\)의 열린덮개이므로 \(K\)의 유한 부분집합 \(\{x_1, \ldots, x_n\}\)이 존재하여
    \[
    K = \bigcup_{i=1}^n U_{x_i}    
    \]
    로 쓸 수 있다. 이제 \(h := \min_{1 \le i \le n} g_{x_i}\)로 두면 명제 \ref{10.3.6}에 의해 \(h \in \overline{\calA}\)이고, \(g_x\)의 정의에 의해서 임의의 \(y \in K\)에 대해 \(h(y) > f(y) - \eps\)이다. 한편 임의의 \(y \in K\)에 대해 \(y \in V_{x_i}\)인 \(i = 1, \ldots, n\)이 존재하므로 \(h(y) \le g_{x_i}(y) < f(y) + \eps\)이다. 따라서 \(\norm{f - h}_{\sup} < \eps\)이다. 그런데 \(\eps > 0\)이 임의의 양수였으므로 \(f\)는 \(\overline{\calA}\)의 닫힘에 포함되는데 이는 \(\overline{\calA}\) 자신이다. 따라서 \(C(K, \RR) = \overline{\calA}\)가 증명된다. 이제 Claim을 보이는 것만 남았다.
    \begin{proof}[Proof of Claim.]
        \(\calA\)가 \(K\)의 점들을 분리하면 \(\calA\)를 포함하는 \(\overline{\calA}\)도 \(K\)의 점들을 분리한다. 따라서 임의의 \(t \in K\)에 대해, 명제 \ref{10.3.8}에 의해 다음을 만족하는 \(k_t \in \overline{\calA}\)가 존재한다.
        \[
            k_t(x) = f(x), \quad k_t(t) = f(t)
        \] 
        \(h_t - f\)가 연속함수이므로 각 \(t \in K\)에 대해 [\(y \in V_t\)이면 \(k_t(y) > f(y) - \eps\)]인 \(t\)의 근방 \(V_t \subseteq K\)가 존재한다. \(\{V_t\}_{t \in K}\)가 \(K\)의 열린덮개이므로 \(K\)의 유한 부분집합 \(\{t_1, \ldots, t_m\}\)이 존재하여
        \[
        K = \bigcup_{j=1}^m V_{t_j}    
        \]
        로 쓸 수 있다. 이제 \(g_x := \max_{1 \le j \le m} h_{t_j}\)로 두면 명제 \ref{10.3.6}에 의해 \(g_x \in \overline{\calA}\)이다. \(g_x(x) = f(x)\)인 것은 자명하다. 마지막으로 각 \(y \in K\)에 대해 \(y \in V_{t_j}\)인 \(j = 1, \ldots, m\)이 존재하므로 \(g_x(y) \ge k_{t_j}(y) > f(y) - \eps\)가 성립한다.
    \end{proof}
\end{proof}

\begin{ex}
    유계닫힌구간 \(I := [a, b] \subseteq \RR\)에서 정의된 실수계수 다항함수들의 집합 \(P\)는 \(C(I, \RR)\)의 부분대수 구조를 가진다. 이때 다항식들은 \(I\)의 점들을 분리하므로 \(C(I, \RR)\)에서 조밀하다. 따라서 스톤-바이어슈트라스 정리는 바이어슈트라스 정리의 일반화이다.
\end{ex}

이제 \(F = \CC\)인 경우를 살펴보자. 지금까지의 논의는 명제 \ref{10.3.6}의 (a)에 강하게 의존했는데 이를 증명하면서 \(F = \RR\)임이 이용되었다. 따라서 \(F = \CC\)일 때는 다른 조건이 추가되어야 함을 알 수 있다.

\begin{defn}
    부분대수 \(\calA \subseteq C(X, F)\)가 \textbf{자기 수반(self-adjoint)}이라는 것은, 임의의 \(f \in \calA\)에 대하여 \(\overline{f} : x \mapsto \overline{f}(x)\)도 \(\calA\)의 원소라는 것이다.
\end{defn}

\begin{thm}[스톤-바이어슈트라스 정리: \(F = \CC\)인 경우] \label{10.3.11}
    옹골 거리공간 \(K\)에 대하여 부분대수 \(\calA \subseteq C(K, \CC)\)가 자기 수반이고 \(K\)의 점들을 분리하면 \(\calA\)는 \(C(K, \CC)\)에서 조밀하다.
\end{thm}
\begin{proof}
    \(\calA\)의 모든 실함수들의 집합을 \(\calA_{\RR}\)로 쓰면 \(\calA_{\RR}\)는 \(C(K, \RR)\)의 부분대수이다. 이때 다음 주장을 가정하면 정리가 증명됨을 보이자.\\
    \textbf{Claim.} \(\calA_{\RR}\)는 \(K\)의 점들을 분리한다.\\
    Claim을 가정하면, 스톤-바이어슈트라스 정리의 \(F = \RR\)인 경우에 의해 \(\calA_{\RR}\)가 \(C(K, \RR)\)에서 조밀하다. 이제 임의의 \(f \in C(K, \CC)\)에 대해 \(f = u + iv\)로 쓰자(\(u, v\)는 실함수). 이때 \(u, v\)로 각각 수렴하는 \(\calA_{\RR}\)의 함수열 \((u_n)_{n \in \NN}, (v_n)_{n \in \NN}\)이 존재한다. 이때 \(u_n + iv_n \to f\)이고, 각 \(n\)에 대해 \(u_n + iv_n \in \calA\)이므로 \(f\)로 수렴하는 \(\calA\) 안의 함수열이 존재한다. 따라서 \(f \in \overline{\calA}\)이므로 \(\calA\)가 \(C(K, \CC)\)에서 조밀하다. 이제 Claim만 보이면 된다.
    \begin{proof}[Proof of Claim.]
        \(f = u + iv \in \calA\)에 대해 \(u - iv \in \calA\)이므로 \(u \in \calA\), 즉 \(u \in \calA_{\RR}\)이다. 이제 서로 다른 두 점 \(x_1, x_2 \in K\)에 대하여 \(\calA\)가 \(K\)의 점들을 분리하므로 \(f \in \calA\)가 존재하여 \(f(x_1) = 0, f(x_2) = 1\)이게 할 수 있다. 실함수 \(u, v\)에 대해 \(f = u +iv\)로 썼을 때, \(\calA\)가 자기 수반이므로 \(u - iv\)도 \(\calA\)의 함수이다. 따라서 \(u \in \calA\), 즉 \(u \in \calA_{\RR}\)이고 \(u(x_1) = 0 \ne 1 = u(x_2)\)이다. 따라서 \(\calA_{\RR}\)는 \(K\)의 점들을 분리한다.
    \end{proof}
\end{proof}

\begin{ex}
    \(I := [-\pi, \pi]\)라 두자. 유한 개의 \(n\)에 대해서만 \(c_n \ne 0\)인 복소수열 \((c_n)_{n \in \ZZ}\)에 대하여, 다음 꼴로 나타나는 \(I\)에서 정의된 복소함수를 \textbf{삼각다항식(trigonometric polynomial)}이라고 한다.
    \[
    \sum_{n=-\infty}^\infty c_n e^{inx} = c_0 + \sum_{n=1}^\infty [(c_n + c_{-n})\cos nx + i(c_n - c_{-n})\sin nx]
    \]
    \(I\)에서 정의된 모든 삼각다항식들의 집합을 \(\calT \subseteq C(I, \CC)\)로 쓸 때, \(\calT\)는 \(C(I, \CC)\)의 부분대수이고 \(I\)의 점들을 분리한다. 마지막으로 임의의 \(f = \sum_{n=-\infty}^\infty c_n e^{in\cdot} \in \calT\)에 대해,
    \[
    \overline{f} = \sum_{n=-\infty}^\infty \overline{c_{n}} e^{-in\cdot}= \sum_{n=-\infty}^\infty \overline{c_{-n}} e^{in\cdot}
    \]
    이므로 \(\calT\)는 자기 수반이다. 따라서 \(\calT\)는 \(C(I, \CC)\)에서 조밀하다.
\end{ex}

\section{\(C_b(X), C_0(X), C_c(X)\) 공간}

이번 절에서는 정의역이 옹골집합이라는 가정 없이 일반적인 거리공간인 경우를 보겠다.

\begin{defn}
    \(X\)에서 \(F\)로 가는 모든 유계 연속함수들의 집합을 \(C_b(X, F)\)로 쓰자(\(b\)는 물론 bounded에서 온 것이다).
\end{defn}

\begin{prop}
    \(C_b(X, F)\)는 \(C(X, F)\)의 부분공간으로서 \(F\)-벡터공간의 구조를 가지며,
    \[
    \norm{\cdot}_{\sup} : C_b(X, F) \mapsto \RR : f \mapsto \sup_{x \in X} \abs{f(x)}    
    \]
    는 \(C_b(X, F)\)의 노름이 된다. 따라서 \(C_b(X, F)\)는 자연스러운 노름공간의 구조를 가진다.
\end{prop}
\begin{proof}
    연습문제로 남긴다.
\end{proof}

\begin{thm} \label{10.4.3}
    \(C_b(X, F)\)는 바나흐공간이다.
\end{thm}
\begin{proof}
    \((f_n)_{n \in \NN}\)이 \(C_b(X, F)\) 안의 코시수열이라 하자. 그러면 \((f_n)_{n \in \NN}\)은 \(C(X, F)\)의 코시수열이므로 어떤 \(f \in C(X, F)\)로 수렴한다. 이제 \(f\)의 상이 \(F\)에서 유계인 것을 보이면 된다. 자연수 \(N\)이 존재하여 다음을 만족한다.
    \[
    n \ge N \implies \abs{f_n(x) - f(x)} < 1 \quad (x \in X)    
    \]
    이때 \(f_N\)의 상은 \(F\)에서 유계이므로, \(M := \diam f_N(X) < \infty\)라 하자. 그러면 \(\diam f(X) \le M+2\)이므로 \(f\)의 상도 유계이고, 따라서 \(f \in C_b(X, Y)\)이다. 즉 \(C_b(X, F)\)가 완비 거리공간이다.
\end{proof}

\begin{defn}
    함수 \(f : X \to F\)에 대하여 집합 \(\{x \in X : f(x) \ne 0\}\)의 \(X\)에서의 닫힘을 \textbf{\(X\)의 지지(support of \(f\))}라 하고 \(\text{supp} f\)로 쓴다. \(\text{supp} f \subseteq X\)가 옹골집합이면 \(f\)가 \textbf{옹골 지지(compact support)}를 가진다고 한다.
\end{defn}

\begin{defn}
    \(C_b(X, F)\)의 두 부분집합을 다음과 같이 쓰기로 하자.
    \begin{enum}
        \item 임의의 \(\eps > 0\)에 대해
        \[
        x \notin K \implies \abs{f(x)} < \eps \quad (x \in X)    
        \]
        를 만족하는 옹골집합 \(K \subseteq X\)가 존재하는 모든 \(f \in C(X, F)\)의 집합을 \(C_0(X, F)\)로 쓰자.
        \item 옹골 지지를 가지는 모든 \(f \in C(X, F)\)들의 집합을 \(C_c(X, F)\)로 쓰자(\(c\)는 물론 compact에서 온 것이다).
    \end{enum}
\end{defn}

\begin{ex}
    \(X = \RR^d\)일 때, \(f \in C_0(\RR^d, F)\)일 필요충분조건은 \(\lim_{\norm{x} \to \infty} \abs{f(x)} = 0\)인 것이다. 여기서 물론 \(\lim_{\norm{x} \to \infty} \abs{f(x)} = 0\)은 임의의 \(\eps > 0\)에 대해 다음을 만족하는 양수 \(M > 0\)이 존재함을 의미한다.
    \[
    \norm{x} > M \implies \abs{f(x)} < \eps \quad(x \in \RR^d)    
    \]
\end{ex}

\begin{prop} \label{10.4.6}
    \(C_c(X, F) \subseteq C_0(X, F) \subseteq C_b(X, F) \subseteq C(X, F)\)이고, 각각은 \(C(X, F)\)에서 물려받은 \(F\)-벡터공간의 구조를 가진다. 그리고 \(C_b(X, F)\)가 노름공간이므로 \(C_c(X, F)\)와 \(C_0(X, F)\)도 노름공간이다.
\end{prop}
\begin{proof}
    연습문제로 남긴다.
\end{proof}

\begin{notn} \label{10.4.7}
    여기까지 공부한 독자들은 \(F = \RR\)일 때와 \(\CC\)일 때 대부분의 경우 별 차이가 없음을 느꼈을 것이다. 따라서 아무 말 없이 \(C(X), C_c(X), C_0(X), C_b(X)\)라고 쓰면 \(F = \CC\)인 것으로 생각하고, \(F = \RR\)인 경우는 따로 언급할 것이다. 만약 둘을 동시에 써야 하면 \(C(X, F)\) 등의 표기법을 부활시킬 것이다.
\end{notn}

\begin{lem}
    완비 거리공간 \(X\)와 \(Y \subseteq X\)에 대해 다음이 동치이다.
    \begin{enum}
        \item \(Y\)는 완비 거리공간이다.
        \item \(Y\)는 \(X\)의 닫힌집합이다.
    \end{enum}
\end{lem}
\begin{proof}
    연습문제로 남긴다.
\end{proof}

\begin{thm}
    \(C_0(X)\)는 바나흐공간이다.
\end{thm}
\begin{proof}
    \(C_0(X)\)가 \(C_b(X)\)의 닫힌집합인 것만 보이면 된다. \(C_0(X)\) 안의 함수열 \((f_n)_{n \in \NN}\)이 어떤 \(f \in C_b(X)\)로 수렴한다고 하자. 그러면 즉 다음을 만족하는 자연수 \(N\)이 존재한다.
    \[
    n \ge N \implies \norm{f_n - f}_{\sup} < \frac{\eps}{2}    
    \]
    이때 \(f_N \in C_0(X)\)이므로, 다음을 만족하는 옹골집합 \(K \subseteq X\)가 존재한다.
    \[
    x \notin K \implies \abs{f_N(x)} < \frac{\eps}{2} \quad (x \in X)
    \]
    이제 \(x \in X \setminus K\)이면
    \[
    \abs{f(x)} \le \abs{f_N(x)} + \abs{f(x) - f_N(x)} < \frac{\eps}{2} + \frac{\eps}{2} = \eps
    \]
    이다. 따라서 \(f \in C_0(X)\)이므로 \(C_0(X)\)는 닫힌집합이다.
\end{proof}

이제 \(C_c(X)\)가 \(C_0(X)\)에서 조밀함을 보인다. 먼저 \(X\)가 옹골 거리공간이 아니면 \(C_c(X)\)는 일반적으로 바나흐공간이 아니다. \(X = \RR\)일 때, 다음과 같이 정의된
\[
f : \RR \to \RR : x \mapsto \frac{1}{x^2 + 1}    
\] 
은 \(C_0(\RR)\)의 원소이지만 \(C_c(\RR)\)의 원소는 아니다. 이제 각 \(n \in \ZZ_+\)에 대해 함수 \(f_n : \RR \to \RR\)를
\[
f_n : x \mapsto \begin{cases}
    f(x) &\textif \ \abs{x} \le n\\
    -f(n)[x - (n+1)] &\textif \ n < x \le n+1\\
    f(n)[x - (-n-1)] &\textif \ -n-1 \le x < -n\\
    0 &\otw
\end{cases}    
\]
로 정의하면, \((f_n)_{n \in \ZZ_+}\)은 \(C_c(\RR)\) 안의 함수열이지만 \(C_0(\RR)\)에서 \(f\)로 수렴한다. 따라서 \(C_c(\RR)\)는 \(C_0(\RR)\)의 닫힌집합이 아니고, 따라서 완비 거리공간이 아니다.

\begin{lem}[우리손 보조정리]
    \(A, B \subseteq X\)가 서로소인 닫힌집합일 때 \(\phi \vert_A = 1, \phi \vert_B = 0\)을 만족하는 연속함수 \(\phi : X \to [0, 1]\)이 존재한다.
\end{lem}
\begin{rem}
    원래 우리손 보조정리는 \(X\)가 정규 공간일 때 성립하는데, 일반적인 증명은 굉장히 길고 전혀 직관적이지 않다. \(X\)가 특수하게 거리공간이면 굉장히 간단한 증명이 있다.
\end{rem}
\begin{proof}
    다음과 같이 \(\phi\)를 정의하면 증명이 끝난다.
    \[
    \phi : x \mapsto \frac{d(x, B)}{d(x, A) + d(x, B)}    
    \]
\end{proof}

이제 이 절의 마지막 정리를 진술한다.

\begin{thm} \label{10.4.11}
    \(C_c(X)\)는 \(C_0(X)\)에서 조밀하다.
\end{thm}
\begin{proof}
    임의의 \(f \in C_0(X), \eps > 0\)이 주어졌다고 하자. 이때
    \[
    E_1 := \{x \in X : \abs{f(x)} \ge \eps\}, \quad E_2 := \{x \in X : \abs{f(x)} \le \eps/2\}    
    \]
    로 정의하면 \(E_1, E_2\)는 \(X\)의 서로소인 닫힌집합이다. 우리손 보조정리에 의해 \(\phi_{E_1} = 1, \phi_{E_2} = 0\)인 연속함수 \(\phi : X \to [0, 1]\)이 존재한다. 이제 \(g := \phi f\)라 두면 \(\norm{f - g}_{\sup} \le \eps\)이다. 따라서 \(g \in C_c(X)\)인 것만 보이면 된다. 이때 \(g \in C_0(X)\)이므로 다음을 만족하는 옹골집합 \(K\)가 존재한다.
    \[
    \abs{f(x)} \ge \frac{\eps}{2} \implies x \in K \quad (x \in X)    
    \]
    따라서
    \[
    \text{supp} g \subseteq \overline{\{x \in X : \abs{f(x)} > \eps / 2\}} \subseteq \overline{K} = K
    \]
    이므로 \(\text{supp} g \subseteq X\)도 옹골집합이다.
\end{proof}



\chapter{함수열과 함수공간 (2)}

연속함수공간에 이어 다음으로 공부할 함수공간은 수렴하는 수열 공간이다. 먼저 수렴하는 실수열 또는 복소함수들의 공간이 완비 거리공간임을 보이고, 다음으로 이중수열 또는 이중급수에서 두 극한 또는 무한합의 순서를 바꿀 수 있는 조건에 대해 공부할 것이다. 마지막으로 수렴하는 수열 공간의 부분공간 또는 더 넓은 공간에 대해 공부할 것이다. 이 장이 끝날 때까지 \(X\)는 거리함수 \(d\)가 주어진 거리공간이고, \(F = \RR\) 또는 \(\CC\)이다.

\section{수렴하는 수열 공간}

\begin{defn}
    \(X\) 안의 모든 수렴하는 수열들의 집합을 \(c(X)\)로 쓰자.
\end{defn}

\begin{thm}
    함수 \(\rho : c(X) \times c(X) \to \RR\)를
    \[
    \rho : (x, y) \mapsto \sup_{n \in \NN} d(x(n), y(n))
    \]
    으로 정의하면 \(\rho\)는 \(c(X)\)의 거리함수이다.
\end{thm}
\begin{proof}
    연습문제로 남긴다.
\end{proof}

\begin{prop}
    \(c(X)\) 안의 함수열 \((x_n)_{n \in \NN}\)과 \(x \in c(X)\)에 대하여 다음이 동치이다.
    \begin{enum}
        \item \(x_n \to x\).
        \item 임의의 양수 \(\eps > 0\)에 대하여 자연수 \(N\)이 존재하여 다음을 만족한다.
        \[
        n \ge N \implies d(x_n(m), x(m)) < \eps \quad (m \in \NN)    
        \]
    \end{enum}
\end{prop}
\begin{proof}
    연습문제로 남긴다.
\end{proof}

\begin{prop}
    \(x \in c(X)\)에 대하여 \((x(n))_{n \in \NN}\)의 \(X\)에서의 극한을 \(L(x)\)라 할 때, \(L : x \mapsto L(x)\)는 립시츠 연속함수이다.
\end{prop}
\begin{proof}
    서로 다른 두 함수 \(x, y \in c(X)\)를 고정하고, \(r := \rho(x, y)\)라 하자. 임의의 양수 \(\eps > 0\)에 대해 다음을 만족하는 자연수 \(N\)이 존재한다.
    \[
    n \ge N \implies d(x(n), L(x)) < \eps, d(y(n), L(y)) < \eps    
    \]
    이제
    \[
    d(L(x), L(y)) \le d(L(x), x(N)) + d(x(N), y(N)) + d(y(N), L(y)) < \rho(x, y) + 2\eps
    \]
    인데 \(\eps > 0\)이 임의의 양수였으므로 \(d(L(x), L(y)) \le \rho(x, y)\)이다. 따라서 \(L\)은 립시츠 연속이다.
\end{proof}

이 장에서 별 말이 없으면 \(L : c(X) \to X\)는 수열의 극한값을 주는 함수로 생각한다.

\begin{thm}
    \(X\)가 완비 거리공간이면 \(c(X)\)도 완비 거리공간이다.
\end{thm}
\begin{proof}
    \quad\\
    \textbf{Step 1.} 함수열 \((x_n)_{n \in \NN}\)이 코시수열이라고 하자. 즉 임의의 \(\eps > 0\)에 대해 [\(m, n \ge N\)이면 \(\rho(x_m, x_n) < \eps\)]인 자연수 \(N\)이 존재한다. 이때 각 \(k \in \NN\)에 대해, \(m, n \ge N\)이면 \(d(x_m(k), x_n(k)) \le \rho(x_m, x_n) < \eps\)이므로 \((x_n(k))_{n \in \NN}\)은 \(X\)의 코시수열이다. \(X\)가 완비 거리공간이므로 \((x_n(k))_{k \in \NN}\)은 \(X\) 안의 어떤 값으로 수렴하며 이 값을 \(x(k)\)라 하자.\\
    \textbf{Step 2.} 이제 수열 \(x : \NN \to X\)가 \(X\) 안에서 수렴함을 보이자. 임의의 양수 \(\eps > 0\)에 대하여, 다음을 만족하는 자연수 \(M_1\)이 존재한다.
    \[
        m, n \ge M_1 \implies d(x_m(k), x_n(k)) < \frac{\eps}{3} \quad (k \in \NN)
    \]
    이때 \(m \to \infty\)를 취하면
    \begin{equation} \label{eq11.1}
        n \ge M_1 \implies d(f(k), f_n(k)) \le \frac{\eps}{3} \quad (k \in \NN)
    \end{equation}
    을 얻는다. 그런데 \(L\)이 립시츠 연속이므로 \((L(x_n))_{n \in \NN}\)도 \(X\)에서 코시수열이고, \(X\)의 완비성에 의해 \((L(x_n))_{n \in \NN}\)은 어떤 \(\alpha \in X\)로 수렴한다. 즉 어떤 자연수 \(M_2\)가 존재하여 다음을 만족한다.
    \[
      n \ge M_2 \implies d(L(x_n), \alpha) < \frac{\eps}{3}  
    \]
    \(M := \max\{M_1, M_2\}\)라 두면, 다음을 만족하는 자연수 \(K\)가 존재한다.
    \[
    k \ge K \implies d(x_M(k), L(x_M)) < \frac{\eps}{3} \quad    
    \]
    따라서 \(k \ge K\)이면
    \[
    d(x(k), \alpha) \le d(x(k), x_M(k)) + d(x_M(k), L(x_M)) + d(L(x_M), \alpha) < \frac{\eps}{3} + \frac{\eps}{3} + \frac{\eps}{3} = \eps
    \]
    이다. 따라서 \((x(k))_{k \in \NN}\)은 \(X\)에서 \(\alpha\)로 수렴한다.\\
    \textbf{Step 3.} 마지막으로 \(c(X)\)에서 \(f_n \to f\)인 것만 보이면 되는데, 이는 임의의 \(\eps > 0\)에 대해 자연수 \(M_1\)이 존재하여 (\ref{eq11.1})\이 성립하는 것으로부터 나온다.
\end{proof}

\begin{cor}
    완비 거리공간 \(X\)에 대하여 \((x_n)\)이 \(c(X)\) 안의 함수열일 때 다음이 동치이다.
    \begin{enum}
        \item \((x_n)_{n \in \NN}\)이 어떤 수열 \(x : \NN \to X\)로 고른수렴한다.
        \item \((x_n)_{n \in \NN}\)은 \(c(X)\)에서 코시수열이다.
        \item 임의의 양수 \(\eps > 0\)에 대하여 자연수 \(N\)이 존재하여 다음을 만족한다.
        \[
        m, n \ge N \implies d(x_m(k), x_n(k)) < \eps \quad (k \in \NN)    
        \]
    \end{enum}
    위의 조건들이 성립할 때 \(x \in c(X)\)이다.
\end{cor}
\begin{proof}
    연습문제로 남긴다.
\end{proof}

\begin{prop}
    노름공간 \(V\)에 대하여 \(c(V)\)는 자연스러운 \(F\)-벡터공간의 구조를 가진다. 또
    \[
    \norm{\cdot}_{\sup} : c(V) \to \RR : x \mapsto \sup_{n \in \NN} x(n)    
    \]
    으로 정의하면 \(\norm{\cdot}_{\sup}\)은 \(c(V)\)의 노름이 된다. 따라서 \(c(V)\)는 자연스러운 노름공간의 구조를 가지며, 이는 거리공간으로서 \((c(V), \rho)\)와 같다.
\end{prop}
\begin{proof}
    연습문제로 남긴다.
\end{proof}

\begin{thm}[바이어슈트라스 판정법: 수렴하는 수열들의 함수열의 경우]
    바나흐공간 \(V\)에 대하여 \((x_n)_{n \in \NN}\)이 \(c(X)\) 안의 함수열이라고 하자. 각 \(n\)에 대하여 \(\norm{x_n}_{\sup} \le M_n\)인 양수 \(M_n > 0\)이 존재하고 \(\sum_n M_n < \infty\)이면 급수 \(\sum_n x_n\)이 고른수렴한다.
\end{thm}
\begin{proof}
    연습문제로 남긴다.
\end{proof}

\section{이중수열과 이중급수}

\begin{defn}
    집합 \(X\)에 대하여 함수 \(a : \NN \times \NN \to X\)를 \(X\) 안의 \textbf{이중수열(double sequence)}이라고 하고 \((a_{mn})_{m, n \in \NN} = a, a_{mn} = a(m, n)\)으로 표기한다.
\end{defn}

다음 정리는 이중수열에서 두 극한의 순서를 바꿀 수 있는 충분조건을 말해준다.

\begin{thm}
    완비 거리공간 \(X\) 안의 이중수열 \((a_{mn})_{m,n \in \NN}\)에 대하여 다음이 성립한다고 하자.
    \begin{enum}
        \item 각 \(m \in \NN\)에 대하여 \(X\) 안의 수열 \((a_{mn})_{n \in \NN}\)이 어떤 \(b_m \in X\)로 수렴한다.
        \item 각 \(n \in \NN\)에 대하여 \(X\) 안의 수열 \((a_{mn})_{m \in \NN}\)이 어떤 \(c_n \in X\)로 수렴한다.
        \item 임의의 양수 \(\eps > 0\)에 대하여 다음을 만족하는 자연수 \(N\)이 존재한다.
        \[
            m \ge N \implies d(a_{mn}, c_n) < \eps \quad (n \in \NN)     
        \]
    \end{enum}
    이때
    \begin{equation} \label{eq11.2}
        \lim_m b_m = \lim_n c_n = \lim_{n} a_{nn}
    \end{equation}
    이다. 즉,
    \[
    \lim_m \lim_n a_{mn} = \lim_n \lim_m a_{mn} = \lim_{n} a_{nn}
    \]
    이다.
\end{thm}
\begin{proof}
    각 \(m \in \NN\)에 대해 \(X\) 안의 수열 \(x_m : n \mapsto a_{mn}\)을 생각하자. (a)에 의해 각 \(x_m \in c(X)\)이고, (c)에 의해 \(c \in c(X)\)이고 함수열 \((x_m)_{m \in \NN}\)이 \(c\)로 \(c(X)\)에서 수렴한다. 이제 \(L\)이 연속함수이므로,
    \[
    \lim_m L(x_m) = L(c)   
    \]
    이다. 이를 다시 쓰면 \(\lim_m b_m = \lim_n c_n\)이므로 (\ref{eq11.2})의 첫 번째 등호가 증명되었다. 이제 (\ref{eq11.2})의 두 번째 등호를 보이자. 양수 \(\eps > 0\)이 주어졌다고 하고, [\(m \ge N_1\)이면 \(\norm{x_m - c}_{\sup} < \eps / 2\)]인 자연수 \(N_1\)을 잡는다. 또 [\(n \ge N_2\)이면 \(d(c(n), L(c)) < \eps/2\)]인 자연수 \(N_2\)를 잡는다. 이제 \(n \ge \max\{N_1, N_2\}\)이면
    \[
    d(a_{nn}, L(c)) = d(x_n(n), L(c)) \le d(x_n(n), c(n)) + d(c(n), L(c)) < \frac{\eps}{2} + \frac{\eps}{2} = \eps
    \]
    이다. 따라서 \(\lim_n a_{nn} \to L(c)\)도 성립한다.
\end{proof}

\begin{ex}
    이중수열 \((a_{mn})_{m,n \in \NN}\)을 다음과 같이 정의하자.
    \[
    a_{mn} =    
    \begin{cases}
        1 &\textif \ n \ge m\\
        0 &\textif \ n < m
    \end{cases}
    \]
    이때 각 \(m \in \NN\)에 대해 \(\lim_n a_{mn} = 1\)이고 각 \(n \in \NN\)에 대해 \(\lim_m a_{mn} = 0\)인데 두 극한 모두 고른수렴이 아니다. 그리고
    \[
    \lim_m \lim_n a_{mn} = 1 \ne 0 = \lim_n \lim_m a_{mn}    
    \]
    이다.
\end{ex}

\begin{defn}
    노름공간 \(V\) 안의 이중수열 \((a_{mn})_{m, n \in \NN}\)이 주어졌다고 하자. 각 \(m \in \NN\)에 대하여 \(\sum_n a_{mn}\)이 어떤 \(b_m \in V\)로 수렴할 때, \(\sum_m \sum_n a_{mn} := \sum_m b_m\)을 \textbf{이중급수(double series)}라고 한다.
\end{defn}

이중급수에서 두 무한합의 순서를 바꿀 수 있는 충분조건을 알아보자. 먼저 각 항이 음이 아닌 이중급수의 경우에는 상황이 간단하다.

\begin{thm}
    \(\RR\) 안의 이중수열 \((a_{mn})_{m,n\in \NN}\)의 각 항이 0 이상일 때
    \begin{equation} \label{eq11.3}
        \sum_m \sum_n a_{mn} = \sum_n \sum_m a_{mn} = \lim_{N \to \infty} \sum_{0 \le m, n \le N} a_{mn}    
    \end{equation}
    이 성립한다(\(\infty\)도 허용).
\end{thm}
\begin{proof}
    이중수열 \((s_{mn})_{m,n \in \NN}\)을 다음과 같이 정의하자.
    \begin{equation} \label{eq11.4}
        s_{mn} := \sum_{i=0}^m \sum_{j=0}^n a_{mn}  
    \end{equation}
    이때 (\ref{eq11.3})\을 보이는 것은 다음을 보이는 것과 같다.
    \begin{equation} \label{eq11.5}
        \lim_m \lim_n s_{mn} \stackrel{?}{=} \lim_{n} \lim_m s_{mn} \stackrel{?}{=} \lim_{n} s_{nn}
    \end{equation}
    이때 \((s_{mn})_{m, n \in \NN}\)은 \(m\) 또는 \(n\)을 고정시켰을 때 각각 \(n\) 또는 \(m\)에 대한 단조증가수열이다. 따라서 임의의 두 자연수 \(m \ge n\)에 대해 \(s_{mn} \le s_{nn}\)이므로, 양변에 \(n \to \infty\)를 취하면
    \[
        \lim_n s_{mn} \le \lim_n s_{nn}    
    \]
    을 얻고, 다시 양변에 \(m \to \infty\)를 취하면
    \[
        \lim_m \lim_n s_{mn} \le \lim_n s_{nn}
    \]
    을 얻는다. 한편 임의의 \(n\)에 대해 \(s_{nn} \le \lim_m s_{mn}\)이므로 양변에 \(n \to \infty\)를 취했을 때
    \[
        \lim_m \lim_n s_{mn} \le \lim_n s_{nn} \le \lim_n \lim_n s_{mn}
    \]
    을 얻는다. 반대 방향 부등식도 마찬가지 방법으로 보일 수 있으므로 (\ref{eq11.5})\가 증명되었다.
\end{proof}

\begin{thm}
    바나흐공간 \(V\) 안의 이중수열 \((a_{mn})_{m,n\in \NN}\)에 대하여
    \[
    \sum_{m} \sum_n \norm{a_{mn}} = \sum_n \sum_m \norm{a_{mn}} < \infty    
    \]
    이면
    \begin{equation} \label{eq11.3}
        \sum_m \sum_n a_{mn} = \sum_n \sum_m a_{mn} = \lim_{N \to \infty} \sum_{0 \le m, n \le N} a_{mn}    
    \end{equation}
    이 성립한다.
\end{thm}
\begin{proof}
    (\ref{eq11.4})\와 같이 이중수열 \((s_{mn})_{n \in \NN}\)을 정의하였을 때 (\ref{eq11.5})\가 성립하는 것을 보이는 것과 같다. 바이어슈트라스 판정법에 의해 각 \(m \in \NN\)에 대해 \((s_{mn})_{n \in \NN}\)이 어떤 \(b_m \in V\)로 수렴하고, 각 \(n \in \NN\)에 대해 \((s_{mn})_{m \in \NN}\)이 어떤 \(c_n \in V\)로 수렴한다. 이제 각 \(m \in \NN\)에 대해 \(x_m : n \to s_{mn}\)으로 정의하면 \(\lim_n x_m = b_m\)이므로 \(x_m \in c(V)\)이다. 다시 바이어슈트라스 판정법에 의해 \(\sum_n c_n\)이 수렴하므로 \(c_n \to 0\)이고 특히 \(c \in c(V)\)이다. 이제 \(x_m \to c\)인 것만 보이면 되는데, \((x_m)_{m \in \NN}\)이 \(c\)로 점별수렴함은 알고 있으므로 \((x_m)_{m \in \NN}\)이 코시수열인 것만 보이면 된다.\\
    임의의 두 자연수 \(m_1 < m_2\)에 대해
    \[
    \norm{x_{m_2} - x_{m_1}} \le \sum_{m=m_1+1}^{m_2} \norm{x_{m+1} - x_m} \le \sum_{m=m_1+1}^{m_2} \sum_n \norm{a_{(m+1)n}}
    \]
    이고, \(\sum_m \sum_n \norm{a_{(m+1)n}} < \infty\)이므로 \((x_m)_{m \in \NN}\)이 코시수열이다.
\end{proof}

\begin{ex}
    이중수열 \((a_{mn})_{m,n \in \NN}\)을 다음과 같이 정의하자.
    \[
    a_{mn} =    
    \begin{cases}
        1 &\textif \ n = m\\
        -1 &\textif \ n = m+1\\
        0 &\otw
    \end{cases}
    \]
    이때 각 \(m \in \NN\)에 대해 \(\sum_n a_{mn} = 0\)이고 각 \(n \in \NN\)에 대해 \(n = 0\)일 때는 \(\sum_m a_{mn} = 1\), 나머지 경우는 \(\sum_{mn} = 0\)인데 두 극한 모두 고른수렴이 아니다. 그리고
    \[
        \lim_m \lim_n a_{mn} = 0 \ne 1 = \lim_n \lim_m a_{mn} 
    \]
    이다.
\end{ex}

\begin{defn} \label{11.2.6}
    \(\CC\) 안의 이중수열 \((a_{mn})_{m, n \in \NN}\)에 대하여 다음 두 경우 중 하나가 성립한다고 하자.
    \begin{enum}
        \item \((a_{mn})_{m, n \in \NN}\)의 각 항이 음이 아닌 실수이다.
        \item \(\sum_m \sum_n \abs{a_{mn}} < \infty\)이다.
    \end{enum}
    이때 (\ref{eq11.3})\이 성립하며, 이 값을 \(\sum_{m,n} a_{mn}\)으로 정의한다.
\end{defn}

\begin{defn}
    \(\ZZ\)에서 정의된 복소수열 \((x_n)_{n \in \ZZ}\)는 다음과 같은 이중수열 \((a_{mn})_{m, n \in \NN}\)과 동일시할 수 있다.
    \[
        a_{mn} :=
        \begin{cases}
            x_0 &\textif \ m = n = 0\\
            x_n &\textif \ n > m = 0\\
            x_{-m} &\textif \ m > n = 0\\
            0 &\otw
        \end{cases}
    \]
    만약 \((a_{mn})_{m, n \in \NN}\)이 (\ref{11.2.6})의 (a) 또는 (b)를 만족할 때,
    \[
    \sum_{n = -\infty}^{\infty} x_n := \sum_{m,n} a_{mn}    
    \]
    으로 쓴다.
\end{defn}

\section{\(\ell^\infty, c_{0}, c_{00}\) 공간}

\begin{defn}
    \(F\) 안의 모든 유계 수열들의 집합을 \(\ell^\infty(F)\)로 쓰자.
\end{defn}

\begin{prop}
    \(\ell^\infty(F)\)는 \(F^\NN\)의 부분공간으로서 \(F\)-벡터공간의 구조를 가지며,
    \[
    \norm{\cdot}_{\sup} : \ell^\infty(F) \to \RR : x \mapsto \sup_{n \in \NN} \abs{x(n)}    
    \]
    은 \(\ell^\infty(F)\)의 노름이 된다. 따라서 \(\ell^\infty(F)\)는 자연스러운 노름공간의 구조를 가진다.
\end{prop}
\begin{proof}
    연습문제로 남긴다.
\end{proof}

\begin{thm}
    \(\ell^\infty(F)\)는 바나흐공간이다.
\end{thm}
\begin{proof}
    연습문제로 남긴다(정리 \ref{10.4.3}의 증명과 거의 같다).
\end{proof}

\begin{defn}
    \(c(F)\)의 두 부분집합을 다음과 같이 쓰기로 하자.
    \begin{enum}
        \item 0으로 수렴하는 모든 \(x \in c(F)\)들의 집합을 \(c_0(F)\)로 쓰자. 즉 \(c_0(F) = \{x \in c(F) : L(x) = 0\}\)이다.
        \item {[\(n \ge N\)이면 \(x(n) = 0\)]}인 자연수 \(N\)이 존재하는 \(x \in c(F)\)들의 집합을 \(c_{00}(F)\)로 쓰자.
    \end{enum}
\end{defn}

\begin{prop} \label{prop}
    \(c_{00}(F) \subseteq c_{0}(F) \subseteq c(F) \subseteq \ell^\infty(F)\)이고, 각각은 \(\ell^\infty\)에서 물려받은 \(F\)-벡터공간의 구조를 가진다. 그리고 \(\ell^\infty\)가 노름공간이므로 \(c_{00}(F), c_0(F), c(F)\)도 노름공간이다. 물론 \(c(F)\)에 원래 정의된 노름공간의 구조와 \(\ell^\infty(F)\)에서 물려받은 구조는 같다.
\end{prop}
\begin{proof}
    연습문제로 남긴다.
\end{proof}

\begin{notn}
    표기법 \ref{10.4.7}에서와 같이 \(F\)를 생략하여 \(c_{00}, c_0, c, \ell^\infty\)의 표기를 사용하겠다.
\end{notn}

\begin{thm}
    \(c_0\)는 바나흐공간이다.
\end{thm}
\begin{proof}
    \(c_0 = L^{-1}(0)\)이므로 \(c_0\)는 \(c\)의 닫힌집합이고 따라서 바나흐공간이다.
\end{proof}

\begin{thm} \label{11.3.8}
    \(c_{00}\)는 \(c_0\)에서 조밀하다.
\end{thm}
\begin{proof}
    임의의 \(x \in c_0, \eps > 0\)이 주어졌을 때, [\(n \ge N\)이면 \(\abs{x(n)} < \eps\)]인 자연수 \(N\)이 존재한다. 이제 \(y \in c_{00}\)를
    \[
    y(n) :=
    \begin{cases}
        x(n) & \textif \ n < N\\
        0 & \otw
    \end{cases}    
    \]
    로 정의하면 \(\norm{x - y}_{\sup} \le \eps\)이다. 따라서 \(c_{00}\)는 \(c_0\)에서 조밀하다.
\end{proof}



\chapter{함수열과 함수공간 (3)}

이제 미분가능함수열과 적분가능함수열에 대해 알아본다. 각각이 고른수렴할 때 극한과 미분/적분의 순서를 바꿀 수 있는지를 알아볼 것인데, 결론부터 말하면 미분가능함수열에서는 고른수렴보다도 강한 조건이 필요하고 고른수렴하는 적분가능함수열에서는 극한과 적분의 순서를 바꿀 수 있다. 이를 바탕으로 미적분학에서부터 미뤄온 거듭제곱급수의 미분과 적분에 관한 증명을 다룬다.

\section{미분가능함수열}

고른수렴과 미분가능함수열의 관계를 알아보자.

\begin{thm} \label{12.1.1}
    \(I := [a, b] \subseteq \RR\)에서 \(\RR\)로 가는 미분가능함수열 \((f_n)_{n \in \NN}\)이 어떤 점 \(x_0 \in I\)에 대해 \((f_n(x_0))_{n \in \NN}\)이 실수에서 수렴하고, 함수열 \((f_n')_{n \in \NN}\)이 어떤 \(g : I \to \RR\)로 고른수렴한다고 하자. 이때 함수열 \((f_n)_{n \in \NN}\)은 어떤 미분가능한 \(f : I \to \RR\)로 고른수렴하고, \(f' = g\)이다.
\end{thm}

\begin{proof}
    \quad\\
    \textbf{Claim 1.} 임의의 \(\eps > 0\)에 대해 다음을 만족하는 자연수 \(N\)이 존재한다.
    \[
    m, n \ge N \implies \abs{(f_m(x) - f_n(x)) - (f_m(y) - f_n(y))} \le \eps \abs{x - y} \quad (x, y \in I)
    \]
    \begin{proof}[Proof of Claim 1.]
        \((f'_n)_{n \in \NN}\)이 고른수렴하므로 임의의 \(\eps > 0\)에 대해 다음을 만족하는 자연수 \(N\)이 존재한다.
        \[
        m, n \ge N \implies \abs{f'_m(x) - f'_n(x)} < \eps \quad (x \in I)    
        \]
        이제 \(m, n \ge N\)이면, 평균값 정리에 의해 \(x\)와 \(y\) 사이의 어떤 \(z\)가 존재하여
        \[
            \abs{(f_m(x) - f_n(x)) - (f_m(y) - f_n(y))} = \abs{f_m'(z) - f_n'(z)}\abs{x-y} \le \eps\abs{x - y} \quad (x, y \in I)
        \]
        이다.
    \end{proof}
    \noindent\textbf{Claim 2.} \((f_n)_{n \in \NN}\)은 어떤 \(f : I \to \RR\)로 고른수렴한다.
    \begin{proof}[Proof of Claim 2.]
        연속함수열 \((f_n)_{n \in \NN}\)이 \(C(I)\)에서 코시수열인 것만 보이면 된다. 임의의 \(\eps > 0\)에 대하여 Claim 1에 의해 다음을 만족하는 자연수 \(N_1\)이 존재한다.
        \[
        m, n \ge N_1 \implies \abs{(f_m(x) - f_n(x)) - (f_m(x_0) - f_n(x_0))} \le \frac{\eps}{2(b-a)}\abs{x-x_0} \le \frac{\eps}{2} \quad (x \in I)
        \]
        그리고 \((f_n(x_0))_{n \in \NN}\)이 수렴하는 실수열이므로 다음을 만족하는 자연수 \(N_2\)가 존재한다.
        \[
        m, n \ge N_2 \implies \abs{f_m(x_0) - f_n(x_0)} < \frac{\eps}{2}    
        \]
        따라서 \(N := \max\{N_1, N_2\}\)로 두면 \(m, n \ge N, x \in I\)에 대하여
        \[
        \abs{f_m(x) - f_n(x)} \le \abs{(f_m(x) - f_n(x)) - (f_m(x_0) - f_n(x_0))} + \abs{f_m(x_0) - f_n(x_0)} < \eps
        \]
        이므로 \((f_n)_{n \in \NN}\)이 \(C(I)\)에서 코시수열이다.
    \end{proof}
    \noindent\textbf{Claim 3.}이제 \(x \in I\)를 고정하고, \(y \in I \setminus \{x\}\)에 대해
    \[
    h_n(y) := \frac{f_n(y) - f_n(x)}{y-x}, \quad h(y) := \frac{f(y) - f(x)}{y - x}    
    \]
    로 정의하자. 그러면 \((h_n)_{n \in \NN}\)은 \(I \setminus \{x\}\)에서 \(h\)로 고른수렴한다.
    \begin{proof}[Proof of Claim 3.]
        임의의 \(\eps > 0\)에 대하여, Claim 1에 의해 어떤 자연수 \(N\)이 존재하여 \(m, n \ge N\)이면 다음이 성립한다.
        \[
         \abs{(f_m(x) - f_n(x)) - (f_m(y) - f_n(y))} \le \eps \abs{x - y} \quad (x, y \in I)
        \]
        이제 \(m \to \infty\)를 취하면 다음을 얻는다.
        \[
        \abs{(f(x) - f_n(y)) - (f(y) - f_n(y))} \le \eps \abs{x - y} \quad (x, y \in I)  
        \]
        이제 \(y \in I \setminus \{x\}\)일 때 양변을 \(\abs{x - y}\)로 나누면
        \[
        \abs{h(y) - h_n(y)} \le \eps \quad (y \in I \setminus \{x\})    
        \]
        를 얻는다.
    \end{proof}
    \noindent\((h_n)_{n \in \NN}\)이 각각 \(h\)로 고른수렴하고 \((f'_n(x))_{n \in \NN}\)가 \(g(x)\)로 수렴하므로 다음을 만족하는 자연수 \(N\)이 존재한다.
    \[
        n \ge N \implies \abs{h_n(y) - h(y)} < \frac{\eps}{3}, \abs{f'_n(x) - g(x)} < \frac{\eps}{3} \quad (y \in I \setminus \{x\})
    \]
    그리고 \(f'_N\)이 \(x\)에서 미분가능하므로 다음을 만족하는 양수 \(\delta > 0\)이 존재한다.
    \[
    0 < \abs{y - x} < \delta \implies \abs{h_N(y) - f_n'(x)} < \frac{\eps}{3} \quad (y \in I \setminus \{x\})
    \]
    따라서 \(0 < \abs{y - x} < \delta\)이면
    \[
        \abs{h(y) - g(x)} \le \abs{h(y) - h_N(y)} + \abs{h_N(y) - f_n'(x)} + \abs{f_n'(x) - g(x)} < \eps
    \]
    이고 이는 \(f'(x) = g(x)\)임을 의미한다. \(x \in I\)가 임의의 점이었으므로 \(f\)는 \(I\)에서 미분가능하고 \(f' = g\)이다.
\end{proof}

\begin{cor} \label{12.1.2}
    유계닫힌구간 \(I \subseteq \RR\)에서 \(\RR\)로 가는 미분가능함수열 \((f_n)_{n \in \NN}\)에 대하여 \((f_n)_{n \in \NN}\)과 \((f_n')_{n \in \NN}\)이 각각 \(I\)에서 \(g_0, g_1\)고른수렴하면 \(g_0\)은 \(I\)에서 미분가능하고 \(g_0' = g_1\)이다.
\end{cor}

\begin{cor}
    유계닫힌구간  \(I \subseteq \RR\)에서 \(\RR\)로 가는 미분가능함수열 \((f_n)_{n \in \NN}\)에 대하여 급수 \(\sum_n f_n\)과 \(\sum_n f'_n\)이 각각 \(I\)에서 고른수렴하면 \(\sum_n f_n\)은 \(I\)에서 미분가능하고
    \[
    \sqbracket{\sum_n f_n}' = \sum_n f'_n    
    \]
    이다.
\end{cor}

\begin{cor}
    유계닫힌구간 \(I \subseteq \RR\)에서 \(\RR\)로 가는 \(C^k\)-함수열 \((f_n)_{n \in \NN}\)에 대하여(\(k\)는 양의 정수), 각 \(0 \le i \le k\)에 대해 \((f_n^{(i)})_{n \in \NN}\)이 \(I\)에서 \(g_i\)로 고른수렴하면 \(g_0\)은 \(I\)에서 \(C^k\)-함수이고 각 \(i\)에 대해 \(g_0^{(i)} = g_i\)이다.
\end{cor}

\begin{cor}
    유계닫힌구간 \(I \subseteq \RR\)와 양의 정수 \(k\)에 대해 \(C^k(I)\)는 자연스러운 \(\RR\)-벡터공간의 구조를 가진다. 이때 \(\norm{\cdot}_{C^k} : C^k(I) \to \RR \)를 다음과 같이 정의하면
    \[
    \norm{\cdot}_{C^k} : f \mapsto \max_{0 \le i \le k} \norm{f^{(i)}}_{\sup}    
    \] 
    \(\norm{\cdot}_{C^k}\)는 \(C^k(I)\)의 노름이 되고, 노름공간 \((C^k(I), \norm{\cdot}_{C^k})\)는 바나흐공간이다.
\end{cor}

\begin{ex}
    따름정리 \ref{12.1.2}에서 \((f_n)_{n \in \NN}\)이 어떤 미분가능한 함수 \(f\)로 고른수렴해도 \((f_n')_{n \in \NN}\)이 고른수렴하지 않으면 \((f_n')_{n \in \NN}\)의 점별 극한과 \(f'\)이 다를 수 있다. 각 \(n \in \ZZ_+\)에 대해, \(I := [-1, 1]\)에서 다음과 같이 정의된 함수
    \[
    f_n(x) = \frac{x}{1+nx^2}    
    \]
    을 생각하자. 이때 \(\norm{f_n}_{\sup} = 1/2\sqrt{n} \to 0\)이므로 \((f_n)_{n \in \NN}\)은 상수함수 \(f = 0\)으로 고른수렴한다. 이제 \(f_n'\)을 계산하면
    \[
    f_n'(x) = \frac{1-nx^2}{(1+nx^2)^2}    
    \]
    이므로 \((f'_n(x))_{n \in \NN}\)는 \(x = 0\)일 때 1로, \(x \ne 0\)일 때 0으로 수렴한다. 따라서 \(f'\)와 \((f_n')_{n \in \NN}\)의 점별 극한이 일치하지 않는다.
\end{ex}

\section{적분가능함수열} \label{sec12.2}

고른수렴과 적분가능함수열의 관계를 알아보자.

\begin{thm}
    단조증가함수 \(\alpha : I := [a, b] \to \RR\)에 대하여 \((f_n)_{n \in \NN}\)이 \(\calR(\alpha)\) 안의 함수열이라고 하자. \((f_n)_{n \in \NN}\)이 어떤 \(f : I \to \RR\)로 고른수렴하면 \(f \in \calR(\alpha)\)이고 \(\int_I f_n d \alpha \to \int_I f d \alpha\)이다.
\end{thm}
\begin{proof}
    각 \(n \in \NN\)에 대하여 \(\eps_n := \sup_{x \in I} \abs{f_n(x) - f(x)}\)로 쓰면
    \[
    f_n (x) - \eps_n \le f(x) \le f_n(x) + \eps_n \quad (x \in I)    
    \]
    이므로 상적분과 하적분의 정의에 의해
    \[
    \int_I (f_n - \eps_n) d\alpha \le \lowint{I}{} f d\alpha \le \upint{I}{} f d\alpha \le \int_a^b (f_n + \eps) d\alpha    
    \]
    이다. 따라서
    \[
    0 \le \upint{I}{} f d\alpha - \lowint{I}{} f d\alpha  \le 2\eps_n [\alpha(b) - \alpha(a)] \to 0
    \]
    이므로 \(f \in \calR(\alpha)\)이다. 또
    \[
    \abs{\int_I f d\alpha - \int_I f_n d\alpha} \le \eps_n[\alpha(b) - \alpha(a)] \to 0    
    \]
    이므로 \(\int_I f_n d \alpha \to \int_I f d \alpha\)이다.
\end{proof}

\begin{cor}
    유계닫힌구간 \(I \subseteq \RR\)와 단조증가함수 \(\alpha : I  \to \RR\)에 대하여 \((f_n)_{n \in \NN}\)이 \(\calR(\alpha)\) 안의 함수열이라고 하자. 급수 \(\sum_n f_n\)이 어떤 \(f : I \to \RR\)로 고른수렴하면 \(\sum_n f_n \in \calR(\alpha)\)이고
    \[
    \int_I \sum_n f_n d\alpha = \sum_n \int_I f_n d\alpha
    \]
    이다.
\end{cor}

\begin{ex}
    \quad

    \begin{enum}
        \item 적분가능함수열 \((f_n)_{n \in \NN}\)이 점별수렴하지만 고른수렴하지 않으면, 일반적으로 \mbox{\((f_n)_{n \in \NN}\)}의 점별 극한의 적분가능성도 보장되지 않는다. \(I := [0, 1]\)에 포함되는 유리수 점들의 집합은 가산집합이므로 \(\{x_n\}_{n \in \NN}\)으로 쓸 수 있다. 이때 각 \(n\)에 대해 \(f_n := \chi_{\{x_0, \ldots, x_n\}}\)으로 두면 각 \(f_n\)은 유한 개의 점들에서만 불연속이므로 리만적분가능하다. 그러나 \((f_n)_{n \in \NN}\)의 점별 극한인 \(\chi_{\QQ \cap [0, 1]}\)은 리만적분가능하지 않다.
        \item 적분가능함수열 \((f_n)_{n \in \NN}\)이 \(I\)에서 점별수렴하지만 고른수렴하지 않으면, 점별 극한의 적분가능성을 가정하더라도 일반적으로 \(f_n\)의 적분이 점별 극한의 적분으로 수렴하는 것이 보장되지 않는다. \(I := [0, 1]\)에서 각 \(n \in \ZZ_+\)에 대해 \(f_n = n\chi_{(0, 1/n)}\)으로 정의하면 \(\int_0^1 f_n = 1\)이다. 한편 \(f_n\)은 상수함수 0으로 점별수렴한다. 따라서 \(f_n\)의 적분이 점별 극한의 적분으로 수렴하지 않는다.
    \end{enum}
\end{ex}

\section{거듭제곱급수} \label{sec12.3}

\begin{defn}
    복소수열 \((a_n)_{n \in \NN}\)과 \(x, x_0 \in \RR\)에 대하여 \(\sum_n a_n (x - x_0)^n\) 꼴의 급수를 \textbf{거듭제곱급수(power series)}라고 한다.
\end{defn}

\begin{thm} \label{12.3.2}
    복소수열 \((a_n)_{n \in \NN}\)에 대하여
    \[
    \lambda := \limsup_n \abs{a_n}^{1/n}, \quad R := \begin{cases}
        \frac{1}{\lambda} &\textif \ \lambda > 0\\
        \infty &\textif \ \lambda = 0
    \end{cases}
    \]
    라 하자. 이때 거듭제곱급수
    \begin{equation} \label{eq12.1}
        \sum_n a_n x^n
    \end{equation}
    은 \(\abs{x} < R\)일 때 절대수렴하고 \(\abs{x} > R\)일 때 발산한다. 그리고 임의의 \(0 < M < R\)에 대하여 (\ref{eq12.1})\은 \([-M, M]\)에서 고른수렴한다.
\end{thm}
\begin{proof}
    \[
    \limsup_n \abs{a_n x^n}^{1/n} = \lambda\abs{x}    
    \]
    이므로 근판정법에 의해 (\ref{eq12.1})\은 \(\abs{x} < R\)에서 절대수렴하고 \(\abs{x} > R\)에서 발산한다. 한편 \(x \in [-M, M]\)에서
    \[
    \abs{a_n x^n} \le \abs{a_n} M^n  
    \]
    이고
    \[
    \limsup_n \abs{a_n M^n}^{1/n} = \lambda M < 1    
    \]
    이므로 근판정법에 의해 \(\sum_n \abs{a_n}M^n < \infty\)이다. 따라서 바이어슈트라스 판정법에 의해 (\ref{eq12.1})\은 \([-M, M]\)에서 고른수렴한다.
\end{proof}

위와 같이 정의된 \(R \in [0, \infty]\)를 거듭제곱급수 (\ref{eq12.1})의 \textbf{수렴반경(radius of convergence)}라 한다. 수렴반경이 0인 경우는 의미가 없으므로, 이 절이 끝날 때까지 특별한 말이 없으면 (\ref{eq12.1})의 수렴반경이 \(R \in (0, \infty]\)라고 가정하고 \((-R, R)\)에서 (\ref{eq12.1})의 점별 극한을 \(f\)라 하자.

\begin{cor} \label{12.3.3}
    거듭제곱급수 (\ref{eq12.1})\과 그 점별 극한 \(f\)에 대하여 다음이 성립한다.
    \begin{enum}
        \item \(f\)는 \((-R, R)\)에서 미분가능하고, 임의의 \(x \in (-R, R)\)에 대해
        \begin{equation} \label{eq12.2}
            f'(x) = \sum_n na_n x^{n-1}
        \end{equation}
        이다. 그리고 (\ref{eq12.2})의 우변의 수렴반경도 \(R\)이다.
        \item 임의의 \(x \in (-R, R)\)에 대해
        \begin{equation} \label{eq12.3}
            \int_0^x f(t)dt = \sum_n \frac{a_n}{n+1}x^{n+1}
        \end{equation}
        이다. 그리고 (\ref{eq12.3})의 우변의 수렴반경도 \(R\)이다. 
    \end{enum}
\end{cor}

\begin{proof}
    \quad

    \begin{enum}
        \item \(x_0 \in (-R, R)\)를 고정하고, \(\abs{x_0} < M < R\)인 양수 \(M > 0\)을 잡는다. 그러면 정리 \ref{12.3.2}에 의해 (\ref{eq12.1})\은 \([-M, M]\)에서 고른수렴한다. 한편
        \[
            \limsup_n \abs{n a_n x^{n-1}}^{1/n} = \limsup_n \abs{a_n}^{1/n}
        \]
        이므로 (\ref{eq12.2})의 우변의 수렴반경도 \(R\)이다. 다시 정리 \ref{12.3.2}에 의해 (\ref{eq12.2})의 우변이 \([-M, M]\)에서 고른수렴하므로, \(f\)는 \([-M, M]\)에서 미분가능하고 특히 \(x_0 \in [-M, M]\)에서
        \[
        f'(x_0) = \sum_n na_n x_0^{n-1}    
        \]
        이다. 그런데 \(x_0 \in (-R, R)\)가 임의의 점이었으므로 임의의 \(x \in (-R, R)\)에서 (\ref{eq12.2})\가 성립한다.
        \item (\ref{eq12.3})의 우변의 수렴반경을 \(R' \in [0, \infty]\)라 하면 (a)에 의해 (\ref{eq12.3})의 우변은 \((-R', R')\)에서 미분가능하고 그 도함수는
        \begin{equation} \label{eq12.4}
            \sqbracket{\sum_n \frac{a_n}{n+1}x^{n+1}}' = \sum_n a_n x^n \quad (x \in (-R', R'))
        \end{equation}
        이다. 다시 (a)에 의해 (\ref{eq12.3})의 우변의 수렴반경도 \(R'\)이므로, \(R' = R\)이고 임의의 \(t \in (R, R)\)에 대해
        \[
            f(t) = \sqbracket{\sum_n \frac{a_n}{n+1}t^{n+1}}'
        \]
        이다. 양변을 0부터 \(x \in (-R, R)\)까지 리만적분하면 (\ref{eq12.3})\을 얻는다.
    \end{enum}
\end{proof}

\begin{cor}
    거듭제곱급수 (\ref{eq12.1})\과 그 점별 극한 \(f\)에 대하여 \(f \in C^\infty((-R, R))\)이고 특히 \(f\)는 0에서 해석적이다.
\end{cor}
\begin{proof}
    따름정리 \ref{12.3.3}\을 반복적으로 적용하면, 임의의 \(k \in \NN\)에 대해 \(f^{(k)}\)가 존재하고
    \[
    f^{(k)}(x) = \sum_n k! \binom{n}{k} a_n x^{n-k} \quad (x \in (-R, R))
    \]
    이다. 따라서 \(f \in C^\infty((-R, R))\)이다. 특히에서
    \[
    f^{(k)}(0) = k! a_k   
    \]
    이므로 0에서 \(f\)의 테일러 급수는
    \[
    \sum_k \frac{f^{(k)}(0)}{k!} x^k = \sum_k a_k x^k    
    \]
    인데, 이것이 \((-R, R)\)에서 \(f\)로 점별수렴하므로 \(f\)는 0에서 해석적이다.
\end{proof}

거듭제곱급수 (\ref{eq12.1})의 수렴반경이 양수일 때 \(f\)가 0에서 해석적임은 지금까지 배운 내용으로부터 비교적 쉽게 도출된다. 그런데 0이 아닌 다른 점에서도 해석적임을 증명할 수 있다.

\begin{thm}
    거듭제곱급수 (\ref{eq12.1})\과 그 점별 극한 \(f\)가 주어졌다고 하자. 이때 임의의 \(x_0 \in (-R, R)\)에 대하여 다음이 성립한다.
    \[
    f(x) = \sum_n \frac{f^{(n)}}{n!} (x - x_0)^n \quad (\abs{x - x_0} < R - \abs{x_0})    
    \]
\end{thm}
\begin{proof}
    \(\abs{x - x_0} < R - \abs{x_0}\)인 \(x\)에 대하여
    \begin{equation} \label{eq12.5}
        f(x) = \sum_n a_n ((x - x_0) + x_0)^n = \sum_n \sum_m \binom{n}{m} a_n (x-x_0)^m x_0^{n-m}
    \end{equation}
    이다. 이때 두 급수의 순서를 바꾸기 위해
    \begin{equation} \label{eq12.6}
        \sum_n \sum_m \binom{n}{m} \abs{a_n (x-x_0)^m x_0^{n-m}} = \sum_n \abs{a_n} (\abs{x - x_0} + \abs{x_0})^n   
    \end{equation}
    이 유한한지 확인하자.
    \[
    \limsup_n [\abs{a_n} (\abs{x - x_0} + \abs{x_0})^n ]^{1/n} = \limsup_n \abs{a_n}^{1/n} \cdot (\abs{x - x_0} + \abs{x_0})    
    \]
    에서, \(0 < R < \infty\)이면 \(\abs{x - x_0} < R - \abs{x_0}\)에서
    \[
        \limsup_n \abs{a_n}^{1/n} \cdot (\abs{x - x_0} + \abs{x_0}) < \limsup_n \abs{a_n}^{1/n} \cdot R = 1
    \]
    이고 \(R = \infty\)이면
    \[
        \limsup_n \abs{a_n}^{1/n} \cdot (\abs{x - x_0} + \abs{x_0}) = 0
    \]
    이다. 어느 경우든 (\ref{eq12.6})의 급수는 유한하므로 (\ref{eq12.5})의 두 급수의 순서를 바꿀 수 있다. 따라서
    \begin{align*}
        f(x) &= \sum_m \sum_n \binom{n}{m} a_n (x-x_0)^m x_0^{n-m}\\
        &= \sum_m \frac{(x - x_0)^m}{m!} \sum_n m! \binom{n}{m} a_n x_0^{n-m}\\
        &= \sum_m \frac{f^{(m)}(x_0)}{m!}(x-x_0)^m
    \end{align*}
    이다.
\end{proof}

\begin{cor}
    거듭제곱급수 (\ref{eq12.1})\과 그 점별 극한 \(f\)에 대하여 \(f\)는 \((-R, R)\)에서 해석함수이다.
\end{cor}

이제 두 거듭제곱급수의 곱에 대해 살펴보자. \ref{sec5.9}절에서 코시곱이 급수의 곱의 역할을 한다는 것을 배웠는데, 거듭제곱급수에 대해서도 마찬가지이다.

\begin{thm}
    두 거듭제곱급수 \(\sum_n a_n x^n, \sum_n b_n x^n\)의 수렴반경이 각각 \(R_1, R_2 \in (0, \infty]\)이고 각각의 수렴반경 안에서 점별 극한을 \(f, g\)라 하자. 두 급수 \(\sum_n a_n, \sum_n b_n\)의 코시곱을 \(\sum_n c_n\)이라 하면 \(\abs{x} < R := \min \{R_1, R_2\}\)에 대하여 거듭제곱급수 \(\sum_n c_n x^n\)은 \(f(x)g(x)\)로 수렴한다.
\end{thm}
\begin{proof}
    \[
        \sum_{k=0}^n (a_k x^k) (b_{n-k} x^{n-k}) = \sum_{k=0}^n a_k b_{n-k} x^n = c_n x^n
    \]
    이므로 두 거듭제곱급수 \(\sum_n a_n x^n\)과 \(\sum_n b_n x^n\)의 코시곱은 \(\sum_n c_n x^n\)이다. 이제 각 \(x \in (-R, R)\)에 대하여 급수 \(\sum_n a_n x^n\)과 \(\sum_n b_n x^n\)이 각각 \(f(x), g(x)\)로 절대수렴하므로 정리 \ref{5.9.2}에 의해 \(\sum_n c_n x^n = f(x)g(x)\)이다.
\end{proof}

지금까지 수렴반경의 끝점에 대해서는 아무 이야기도 하지 않았는데, 아벨 정리는 거듭제곱급수의 수렴반경에서의 오른쪽 끝점이 수렴할 경우에 그것이 어디로 수렴하는지를 가르쳐 준다.

\begin{thm}[아벨 정리] \label{12.3.8}
    거듭제곱급수 (\ref{eq12.1})의 수렴반경이 \(0 < R < \infty\)이고 \((-R, R)\)에서의 점별 극한을 \(f\)라 하자. 만약
    \[
    \sum_n a_n R^n    
    \]
    이 수렴하면
    \[
    \lim_{x \nearrow R} f(x) = \sum_n a_n R^n    
    \]
    이다.
\end{thm}
\begin{proof}
    일반성을 잃지 않고 \(R = 1\)이라 두어도 된다. \((-1, 1)\)에서 다음 급수가 수렴함을 알고 있다.
    \[
    \sum_n x^n = \frac{1}{1-x}    
    \]
    급수 \(\sum_n a_n\)의 부분합을 \((s_n)_{n \in \NN}\)으로 쓰자. 이때 \((-1, 1)\)에서 (\ref{eq12.1})\과 \(\sum_n x^n\)의 코시곱을 생각하면
    \[
    \frac{f(x)}{1-x} = \sum_n s_n x^n
    \]
    이고, 따라서
    \[
    f(x) = (1-x) \sum_n s_n x^n    
    \]
    이다. 가정에 의해 \(s := \sum_n a_n\)이라 두면
    \[
    f(x) - s = (1-x) \sum_n s_n x^n  - s = (1-x) \sum_n (s_n - s) x^n   
    \]
    이다. 이제 양수 \(\eps > 0\)이 주어졌을 때, \(s_n \to s\)이므로 [\(n \ge N\)이면 \(\abs{s_n - s} < \eps\)]인 자연수 \(N\)이 존재한다. 그리고
    \[
    M := \max_{0 \le n \le N} \abs{s_n - s}    
    \]
    로 두자. 이제
    \begin{align*}
        \abs{f(x) - s} &\le (1-x) \sum_{n=0}^N \abs{s_n - s}\abs{x}^n + \abs{(1-x) \sum_{n=N+1}^\infty (s_n - s) x^n}\\
        &\le (1-x)M(N+1) + \eps(1-x)\cdot \frac{x^{N+1}}{1-x}\\
        &< (1-x)M(N+1) + \eps
    \end{align*}
    이고, \(x \nearrow 1\)을 취하면
    \[
    \limsup_{x \nearrow 1} \abs{f(x) - s} \le \eps    
    \]
    를 얻는다. \(\eps > 0\)이 임의의 양수였으므로 \(x \nearrow 1\)일 때 \(f(x) \to s\)이다.
\end{proof}

\begin{ex}
    \(\log (1+x)\)의 0에서의 테일러 급수는
    \[
    \sum_{n=1}^\infty \frac{(-1)^{n-1}}{n} x^n = x - \frac{x^2}{2} + \frac{x^3}{3} - \frac{x^4}{4}    
    \]
    이고 그 수렴반경은 1이다. 그런데 \(x = 1\)일 때 이 급수는 교대급수판정법에 의해 수렴하므로, 아벨 정리에 의해
    \[
    \sum_{n=1}^\infty \frac{(-1)^{n-1}}{n} = \lim_{x \nearrow 1} \log x = \log 2    
    \]
    이다.
\end{ex}

이제 정리 \ref{5.9.3}\을 드디어 증명할 수 있다.

\begin{proof}[Proof of Theorem \ref*{5.9.3}]
    급수 \(\sum_n a_n, \sum_n b_n\)이 수렴하므로, 아벨 판정법에 의해 거듭제곱급수 \(\sum_n a_n x^n, \sum_n b_n x^n\)은 \(x \in (0, 1)\)에 대해서는 수렴한다. 따라서 각각의 수렴반경이 적어도 1 이상이므로 거듭제곱급수 \(\sum_n a_n x^n, \sum_n b_n x^n\)은 \(\abs{x} < 1\)에서 수렴한다. \((-1, 1)\)에서 각각의 점별 극한을 \(f, g\)라 하면 아벨 정리에 의해 다음이 성립한다.
    \[
    \lim_{x \nearrow 1} f(x) = A, \quad \lim_{x \nearrow 1} g(x) = B
    \]
    이제 \(\sum_n a_n x^n, \sum_n b_n x^n\)의 코시곱을 생각하면
    \[
    f(x)g(x) = \sum_n c_n x^n \quad (\abs{x} < 1)    
    \]
    이고, \(\sum_n c_n\)이 \(C\)로 수렴하므로 다시 아벨 정리에 의해
    \[
      C = \lim_{x \nearrow 1} f(x)g(x) = AB 
    \]
    이다.
\end{proof}



\chapter{함수열과 함수공간 (4)}

마지막으로 다룰 함수공간은 \(\mathcal{R}^p\) 공간과 \(\ell^p\) 공간이다. 이를 위해 먼저 미적분학 등에서 엄밀하게 정의하지 않고 사용한 (그러나 매우 유용한) 특이적분을 정의할 것이다. 이는 리만적분을 유계닫힌구간이 아닌 다른 구간으로 확장한 것이다. 다음으로 \(\mathcal{R}^p\) 공간과 \(\ell^p\) 공간의 이론에서 아주 핵심적인 부등식들인 옌센 부등식, 영 부등식, 횔더 부등식, 민코프스키 부등식을 증명할 것이다. 마지막으로 이를 바탕으로 \(\mathcal{R}^p\) 공간과 \(\ell^p\) 공간을 정의하고 이들 공간의 특징을 알아볼 것이다. 두 공간의 이론은 매우 흡사한데, 실해석학에서 르베그 적분을 공부하고 나면 그 이유를 알게 된다.

\section{특이적분}

이 절이 끝날 때까지 아무 말이 없으면 \(I \subseteq \RR\)는 실수의 구간이다.

\begin{defn} \label{13.1.1}
    음이 아닌 값을 가지는 실함수 \(f : I \to \RR_{\ge 0}\)가, 임의의 유계닫힌구간 \(J \subseteq I\)에서 리만적분가능하다고 하자. 이때
    \[
    \int_I f := \sup_{\substack{J \subseteq I \\ J \text{는 유계닫힌구간}}} \int_J f
    \]
    로 정의하고, \(\int_I f\)를 \textbf{특이적분(improper integral)}이라고 한다.
\end{defn}

그런데 이는 우리가 아는 정의가 아니다. 우리에게 익숙한 정의는 다음과 같다.

\begin{prop} \label{13.1.2}
    음이 아닌 값을 가지는 실함수 \(f : I \to \RR_{\ge 0}\)가 임의의 유계닫힌구간 \(J \subseteq I\)에서 리만적분가능할 때 다음이 성립한다.
    \begin{enum}
        \item \(I = [a, \infty)\)의 꼴일 때,
        \[
        \int_I f = \lim_{R \to \infty} \int_a^R f.
        \]
        \item \(I = (a, b]\)의 꼴일 때,
        \[
            \int_I f = \lim_{\eps \searrow 0} \int_{a+\eps}^b f.
        \]
    \end{enum}
\end{prop}
\begin{proof}
    나머지 경우들도 비슷하게 증명 가능하므로, \(I = [a, \infty)\)의 경우만 보이고 나머지는 연습문제로 남긴다. \(\alpha := \int_I f\)라 하자. 이때 \(R \mapsto \int_a^R f\)는 단조증가함수이므로 \(\beta := \lim_{R \to \infty} \int_a^R f \in \RR \cup \{\infty\}\)이고, 단조함수의 극한의 성질에 의해
    \[
    \beta = \sup_R \int_a^R f \le \alpha   
    \]
    이다. 한편 임의의 \(M < \alpha\)에 대해
    \[
    \int_J f > M    
    \]
    인 유계닫힌구간 \(J \subseteq I\)가 존재하는데, 충분히 큰 \(R\)에 대하여 \(J \subseteq [a, R]\)이다. 따라서
    \[
    M < \int_J f \le \int_a^R f \le \beta
    \]
    이다. \(M < \alpha\)가 임의의 값이었으므로 \(\alpha \le \beta\)이다. 따라서 \(\alpha = \beta\)가 성립한다.
\end{proof}

\(I\)가 나머지 형태의 구간들일 때도 위와 마찬가지의 등식이 성립한다. 그리고 이 장에서 앞으로 등장할 \(I\) 위에서 정의된 함수들은 임의의 유계닫힌구간 \(J \subseteq I\)에서 리만적분가능하다고 가정할 것이다.

한편 다음 성질들이 성립하지 않는다면 제대로 된 적분의 정의라고 할 수 없을 것이다.

\begin{prop} \label{13.1.3}
    함수 \(f, g : I \to \RR_{\ge 0}\)에 대하여 \(f \le g\)이면 \(\int_I f \le \int_I g\).
\end{prop}
\begin{proof}
    임의의 유계닫힌구간 \(J \subseteq I\)에 대하여
    \[
    \int_J f \le \int_J g \le \int_I g    
    \]
    이므로, \(J\)에 관한 상한을 취하면 \(\int_I f \le \int_I g\).
\end{proof}

\begin{cor}
    함수 \(f, g : I \to \RR_{\ge 0}\)에 대하여 \(f \le g\)이고 \(\int_I g < \infty\)이면 \(\int_I f < \infty\).
\end{cor}

\begin{thm}
    \(f, g : I \to \RR_{\ge 0}, c \ge 0\)에 대하여 다음이 성립한다.
    \begin{enum}
        \item \(\int_I (f+g) = \int_I f + \int_I g\).
        \item \(\int_I cf = c \int_I f\).
    \end{enum}
\end{thm}
\begin{proof}
    (b)는 정의에 의해 자명하므로 (a)만 보이자. 명제 \ref{13.1.3}에 의해, \(\int_I f = \infty\)이거나 \(\int_I g = \infty\)이면 (a)의 좌변도 \(\infty\)이므로 성립한다. 따라서 \(\int_I f < \infty, \int_I g < \infty\)라고 가정하자. 임의의 유계닫힌구간 \(J \subseteq I\)에 대하여
    \[
    \int_J (f+g) = \int_J f + \int_J g \le \int_I f + \int_I g    
    \]
    이므로
    \[
    \int_I (f+g) \le \int_I f + \int_I g    
    \]
    이다. 한편 임의의 \(\eps > 0\)에 대해 다음을 만족하는 유계닫힌구간 \(J \subseteq I\)가 존재한다.
    \[
    \int_J f > \int_I f - \frac{\eps}{2}, \quad \int_J f > \int_I f - \frac{\eps}{2}    
    \]
    이제
    \[
    \int_I (f+g) \ge \int_J (f+g) = \int_J f + \int_J g > \int_I f + \int_I g - \eps    
    \]
    이고, \(\eps > 0\)이 임의의 양수였으므로 \(\int_I (f+g) \ge \int_I f + \int_I g\)이다. 따라서 (a)가 증명되었다.
\end{proof}

\begin{thm}[적분판정법]
    함수 \(f : \RR_{\ge 0} \to \RR_{\ge 0}\)이 단조감소함수이고 \(\lim_{x \to \infty}f(x) = 0\)일 때, \(\sum_n f(n) < \infty\)일 필요충분조건은 \(\int_0^{\infty} f < \infty\)인 것이다.
\end{thm}
\begin{proof}
    \(\sum_n f(n) \le f(0) + \int_0^\infty f(x) dx = f(0) + \sum_n f(n)\).
\end{proof}

이제 실함수 또는 복소함수의 특이적분을 정의한다. 먼저 다음을 관찰하자.

\begin{prop} \label{13.1.7}
    \quad

    \begin{enum}
        \item \(f : I \to \RR\)에 대하여
        \[
        \int_I \abs{f} < \infty \iff \int_I f^+ < \infty, \int_I f^- < \infty.    
        \]
        \item \(f : I \to \CC\)에 대하여
        \[
            \int_I \abs{f} < \infty \iff \int_I \abs{\mathrm{Re}f} < \infty, \int_I \abs{\mathrm{Im}f} < \infty.
        \]
    \end{enum}
\end{prop}
\begin{proof}
    \quad

    \begin{enum}
        \item \(f^+, f^- \le \abs{f} = f^+ + f^-\).
        \item \(\abs{\mathrm{Re}f}, \abs{\mathrm{Im}f} \le \abs{f} \le \abs{\mathrm{Re}f} + \abs{\mathrm{Im}f}\).
    \end{enum}
\end{proof}

\begin{defn}
    \(\int_I \abs{f} < \infty  \)를 만족하는 모든 복소함수 \(f : I \to \CC\)의 집합을 \(\calR^1(I)\)라 쓰자. 그리고 \(f \in \calR^1(I)\)에 대하여
    \[
    \int_I f := \sqbracket{\int_I (\mathrm{Re}f)^+ - \int_I (\mathrm{Re}f)^-} + i \sqbracket{\int_I (\mathrm{Im}f)^+ - \int_I (\mathrm{Im}f)^-}
    \]
    로 정의한다. 이것이 잘 정의되는 것은 물론 명제 \ref{13.1.7}에 의해서이다.
\end{defn}

\begin{rem}
    \(f \in \calR^1(I)\)가 실함수이면 \(\mathrm{Re}f = f, \mathrm{Im}f = 0\)이므로
    \[
    \int_I f = \int_I f^+ - \int_I f^-
    \]
    이다. 따라서 일반적인 \(f \in \calR^1(I)\)에 대해
    \[
    \int_I f = \int_I \mathrm{Re}f + i\int_I \mathrm{Im}f    
    \]
    로 쓸 수 있다.
\end{rem}

\begin{thm}
    \(f, g \in \calR^1(I), c \in \CC\)에 대하여 다음이 성립한다.
    \begin{enum}
        \item \(f + g \in \calR^1(I)\)이고
        \begin{equation} \label{eq13.1}
            \int_I (f+g) = \int_I f + \int_I g.
        \end{equation}
        \item \(cf \in \calR^1(I)\)이고
        \begin{equation} \label{eq13.2}
            \int_I cf = c \int_I f.
        \end{equation}
    \end{enum}
\end{thm}
\begin{proof}
    \[
    \int_I \abs{f+g} \le \int_I (\abs{f}+\abs{g}) = \int_I \abs{f} + \int_I \abs{g} < \infty, \quad \int_I \abs{cf} = \int_I \abs{c}\abs{f} = \abs{c} \int_I \abs{f} < \infty  
    \]
    이므로 \(f+g, cf \in \calR^1(I)\)이다.\\
    먼저 \(f, g \in \calR^1(I)\)가 실함수라고 하고, \(h := f+g\)로 두자. 이때
    \[
    h^+ - h^- = (f^+ - f^-) + (g^+ - g^-)    
    \]
    이므로
    \[
    h^+ + f^- + g^- = h^- + f^+ + g^+
    \]
    이다. 음이 아닌 값을 가지는 실함수에 대해서는 적분과 덧셈이 교환되는 것을 알고 있으므로
    \[
    \int_I h^+ + \int_I f^- + \int_I g^- = \int_I h^- + \int_I f^+ + \int_I g^+  
    \]
    이다. 각 항이 유한하므로
    \begin{align*}
        \int_I h &= \int_I h^+ - \int_I h^-\\
        &= \int_I f^+ - \int_I f^- + \int_I g^+ - \int_I g^-\\
        &= \int_I f + \int_I g
    \end{align*}
    가 성립한다. 따라서 실함수 \(f, g \in \calR^1(I)\)에 대해 (\ref{eq13.1})\이 증명된다. 이제 일반적인 \(f, g \in \calR^1(I)\)에 대해,
    \begin{align*}
        \int_I (f+g) &= \int_I \mathrm{Re}(f+g) + i \int_I \mathrm{Im}(f+g)\\
        &= \int_I (\mathrm{Re}f + \mathrm{Re}g) + i \int_I (\mathrm{Im}f + \mathrm{Im}g)\\
        &= \int_I \mathrm{Re}f + i \int_I \mathrm{Im}f + \int_I \mathrm{Re}g +  + i \int_I \mathrm{Im}g\\
        &= \int_I f + \int_I g
    \end{align*}
    이므로 (\ref{eq13.1})\이 증명된다.\\
    다음으로 (\ref{eq13.2})\를 증명하는데, 먼저 \(c \in \RR\)인 경우에 (\ref{eq13.2})\가 성립하는 것은 정의에 의해 분명하므로 연습문제로 남긴다. 이제 \(c = i\)인 경우를 보면,
    \[
    \int_I if = \int_I (-\mathrm{Im}f + i\mathrm{Re}f) = -\int_I \mathrm{Im}f + i \int_I \mathrm{Re}f = i \sqbracket{\int_I \mathrm{Re}f + i \int_I \mathrm{Im}f} = i \int_I f
    \]
    이다. 따라서 \(c = ib \ (b \in \RR)\)일 때 (\ref{eq13.2})\가 성립한다. 마지막으로 임의의 \(c \in \CC\)에 대해 \(c = a + ib \ (a, b \in \RR)\)로 쓰면,
    \[
        \int_I cf = \int_I (af + ibf) = a\int_I f + ib \int_I f = c \int_I f    
    \]
    이므로 (\ref{eq13.2})\가 성립한다.
\end{proof}

\begin{cor}
    \(\calR^1(I)\)는 \(\CC\)-벡터공간의 구조를 가진다.
\end{cor}

\begin{prop}
    함수 \(f, g \in \calR^1(I)\)가 실함수이고 \(f \le g\)이면 \(\int_I f \le \int_I g\).
\end{prop}
\begin{proof}
    \(\int_I g - \int_I f = \int_I (g - f) \ge 0\).
\end{proof}

\begin{thm}
    \(f \in \calR^1(I)\)에 대하여 \(\abs{\int_I f} \le \int_I \abs{f}\)이다.
\end{thm}
\begin{proof}
    \(r \ge 0, \theta \in \RR\)에 대해 \(\int_I f = re^{i\theta}\)로 쓸 수 있다. 이때
    \[
        \abs{\int_I f} = \int_I e^{-i\theta} f \in \RR
    \]
    이므로
    \[
        \int_I e^{-i\theta} f = \int_I \mathrm{Re}(e^{-i\theta}f)
    \]
    이다. 따라서
    \[
        \abs{\int_I f} = \int_I \mathrm{Re}(e^{-i\theta}f) \le \int_I \abs{e^{-i\theta}f} = \int_I \abs{f}
    \]
    이다.
\end{proof}

명제 \ref{13.1.2}에서와 같은 이야기를 복소함수에 대해서도 할 수 있을 것이다.

\begin{prop} \label{13.1.13}
    \(f \in \calR^1(I)\)에 대하여 다음이 성립한다.
    \begin{enum}
        \item \(I = [a, \infty)\)의 꼴일 때,
        \begin{equation} \label{eq13.3}
            \int_I f = \lim_{R \to \infty} \int_a^R f.
        \end{equation}
        \item \(I = (a, b]\)의 꼴일 때,
        \begin{equation} \label{eq13.4}
            \int_I f = \lim_{\eps \searrow 0} \int_{a+\eps}^b f.
        \end{equation}
    \end{enum}
\end{prop}
\begin{proof}
    연습문제로 남긴다.
\end{proof}

이제 \ref{sec12.2}절에서처럼 특이적분과 극한의 순서를 바꿀 수 있는 충분조건에 대해 알아보자.

\begin{thm} \label{13.1.14}
    \(\calR^1(I)\) 안의 함수열 \((f_n)_{n \in \NN}\)과 \(f, g \in \calR^1(I)\)에 대해 다음이 성립한다고 하자.
    \begin{enum}
        \item 각 \(n \in \NN\)에 대해 \(\abs{f_n} \le g\).
        \item 임의의 유계닫힌구간 \(J \subseteq I\)에서 \((f_n)_{n \in \NN}\)이 \(f\)로 고른수렴한다.
    \end{enum}
    이때
    \[
    \lim_{n} \int_I f_n = \int_I f    
    \]
    가 성립한다.
\end{thm}
\begin{proof}
    먼저 각 \(x \in I\)에 대해 \(f_n(x) \to f(x)\)이므로 \(\abs{f(x)} \le g(x)\)이다. 이제 양수 \(\eps > 0\)이 주어졌다고 하자. \(g \in \calR^1(I)\)이므로 다음을 만족하는 유계닫힌구간 \(J := [a, b] \subseteq I\)가 존재한다.
    \[
    \int_{I \setminus J} g < \frac{\eps}{2}    
    \]
    이제 \(J\)에서 \((f_n)_{n \in \NN}\)이 \(f\)로 고른수렴하므로, 다음을 만족하는 자연수 \(N\)이 존재한다.
    \[
    n \ge N \implies \abs{f_n(x) - f(x)} < \frac{\eps}{2(b-a)} \quad (x \in J)    
    \]
    이제 \(n \ge N\)이면
    \begin{align*}
        \abs{\int_I f_n - \int_I f} &\le \int_I \abs{f_n - f}\\
        &= \int_J \abs{f_n - f} + \int_{I \setminus J} \abs{f_n - f}\\
        &< (b-a) \cdot \frac{\eps}{2(b-a)} + \int_{I \setminus J} 2g < \eps
    \end{align*}
    이다. 따라서
    \[
    \int_I f_n \to \int_I f    
    \]
    가 성립한다.
\end{proof}

한편, \(f \notin \calR^1(I)\)이어도 (\ref{eq13.3})이나 (\ref{eq13.4})의 우변의 극한이 수렴할 수는 있다. 가령 \(I = [0, \infty)\)에서 정의된 함수
\[
f : x \mapsto \frac{\sin x}{x}
\]
을 생각하자(물론 \(f(0) = 0\)이다). 양의 정수 \(N\)에 대해
\begin{align*}
    \int_0^{\pi N} \abs{\frac{\sin x}{x}} dx &= \sum_{n=1}^N \int_{\pi(n-1)}^{\pi n} \abs{\frac{\sin x}{x}} dx\\
    &\ge \sum_{n=1}^N \frac{1}{\pi n} \int_{\pi(n-1)}^{\pi n} \abs{\sin x} dx\\
    &= \sum_{n=1}^N \frac{2}{\pi n} \xrightarrow[]{N \to \infty} \infty
\end{align*}
이므로
\[
\int_0^\infty \abs{\frac{\sin x}{x}} dx = \infty
\]
이다. 따라서 \(f \notin \calR^1(I)\)이다. 그러나
\[
\lim_{R \to \infty} \int_0^{R} \frac{\sin x}{x} dx = \frac{\pi}{2}
\]
임이 알려져 있다.\footnote{이를 복소해석학의 도구를 이용하지 않고 계산하는 것은 굉장히 힘든 일이다. 따라서 이 노트에서는 생략한다.} 따라서 편의를 위해,
\[
\int_0^\infty \frac{\sin x}{x} dx := \lim_{R \to \infty} \int_0^{R} \frac{\sin x}{x} dx = \frac{\pi}{2}    
\]
로 정의한다. 마찬가지로 \(I\)가 \((a, b]\)와 같은 형태일 때에도 (\ref{eq13.4})의 우변의 극한이 수렴한다면 그것을 \(\int_I f\)로 정의한다. 하지만 우리의 story에서 중요한 것들은 아니다.

\section{여러 가지 부등식}

\begin{defn}
    열린구간 \((a, b) \subseteq \RR\)에서 정의된 실함수 \(\phi : (a, b) \to \RR\)가 \textbf{볼록함수(convex function)}라는 것은, 임의의 \(x, y \in (a, b), \lambda \in (0, 1)\)에 대해
    \[
        \phi(\lambda x + (1-\lambda)y) \le \lambda \phi(x) + (1-\lambda)\phi(y)    
    \]
    라는 것이다.
\end{defn}

\begin{ex}
    최고차항이 계수가 양수인 이차함수, \(x \mapsto e^x\), \(x \mapsto - \log x\) 등은 각각의 정의역에서 볼록함수이다.
\end{ex}

다음은 볼록함수의 중요한 성질들이다.

\begin{prop} \label{13.2.2}
    볼록함수 \(\phi : (a, b) \subseteq \RR \to \RR\)에 대하여, 세 실수 \(x, y, z \in I\)가 \(x < y < z\)를 만족하면
    \[
    \frac{\phi(y) - \phi(x)}{y - x} \le   \frac{\phi(z) - \phi(x)}{z - x}  \le \frac{\phi(z) - \phi(y)}{z - y}
    \]
    가 성립한다.
\end{prop}
\begin{proof}
    연습문제로 남긴다.
\end{proof}

\begin{prop}
    \(\phi : (a, b) \subseteq \RR \to \RR\)가 볼록함수라고 하자. 이때 임의의 \(t \in (a, b)\)에 대하여 어떤 \(\beta \in \RR\)가 존재하여, 모든 \(u \in (a, b)\)에 대해
    \begin{equation} \label{eq13.5}
        \phi(x) \ge \beta(x-t) + \phi(t)
    \end{equation}
    가 성립한다.
\end{prop}
\begin{proof}
    \(t \in (a, b)\)를 고정하자. \(t\)와 \(b\) 사이의 점 \(t'\)를 고정하면, 명제 \ref{13.2.2}에 의해 임의의 \(s \in (a, t)\)에 대해
    \[
    \frac{\phi(t) - \phi(s)}{t - s}  \le \frac{\phi(t') - \phi(t)}{t' - t}   
    \]
    이다. 따라서
    \[
    \beta := \sup_{s \in (a, t)} \frac{\phi(t) - \phi(s)}{t - s} < \infty  
    \]
    이다. 이제 임의의 \(x \in (a, b)\)에 대해 (\ref{eq13.5})\가 성립하는 것을 보인다. \(x = t\)일 때는 자명하게 성립한다. \(x < t\)일 때는 \(\beta\)의 정의에 의해
    \[
    \frac{\phi(t) - \phi(x)}{t - x} \le \beta    
    \]
    이므로 (\ref{eq13.5})\가 성립한다. 또 \(x > t\)를 고정하면 임의의 \(s < t\)에 대해
    \[
        \frac{\phi(t) - \phi(s)}{t - s} \le \frac{\phi(x) - \phi(t)}{x - t}
    \]
    이고, 양변에 \(s\)에 관한 상한을 취하면
    \[
    \beta \le \frac{\phi(x) - \phi(t)}{x - t}    
    \]
    를 얻는다. 따라서 모든 \(x \in (a, b)\)에 대해 (\ref{eq13.5})\가 성립한다.
\end{proof}

\begin{thm}
    볼록함수는 연속함수이다.
\end{thm}
\begin{proof}
    연습문제로 남긴다.
\end{proof}

이제 첫 번째 부등식을 증명할 준비가 되었다.

\begin{thm}[옌센 부등식\footnote{Jensen's inequality.}: 적분의 경우]
    \(p : I \to \RR\)가 음이 아닌 값만을 가지고 \(\int_{I} p = 1\)이라 하자. 그리고 \(\phi : (a, b) \subseteq \RR \to \RR\)가 볼록함수라 하자. 이때
    \begin{enum}
        \item \(fp \in \calR^1(I)\)
        \item 모든 \(x \in I\)에 대해 \(a < f(x) < b\)
    \end{enum}
    를 만족하는 실함수 \(f : I \to \RR\)에 대하여, 다음이 성립한다.
    \begin{equation} \label{eq13.6}
        \phi \paren{\int_I fp} \le \int_I (\phi \circ f)p
    \end{equation}
\end{thm}
\begin{proof}
    먼저 (\ref{eq13.6})의 좌변이 잘 정의되려면 \(t := \int_I fp \in (a, b)\)를 보여야 한다. 먼저 \(a = -\infty\)이면 \(t > a\)인 것은 자명하므로 \(a > -\infty\)라 하면,
    \[
        t - a = \int_I (f - a)p \ge 0
    \]
    이다. 따라서 \(t > a\)이고, 비슷한 방법으로 \(t < b\)임을 보일 수 있다.\\
    이제 모든 \(x \in (a, b)\)에 대해 (\ref{eq13.5})\를 만족하는 \(\beta \in \RR\)를 잡자. 즉 임의의 \(y \in I\)에 대해
    \[
    (\phi \circ f)(y) \ge \beta(f(y)-t) + \phi(t)    
    \]
    가 성립한다. 양변에 \(p(y)\)를 곱하고 식을 정리하면
    \[
        (\phi \circ f)(y)p(y) \ge \beta f(y)p(y) - \beta t p(y) + \phi(t)p(y)
    \] 
    를 얻는다. 이제 양변을 \(y\)에 관하여 \(I\)에서 적분하면
    \[
        \int_I (\phi \circ f)p \ge \beta \int_I fp - \beta t + \phi(t) = \phi(t) = \phi \paren{\int_I fp}
    \]
    이므로 (\ref{eq13.6})\이 증명되었다.
\end{proof}

위 증명에서 적분 기호를 모두 무한합으로 바꾸어도 증명은 크게 변하지 않는다.

\begin{thm}[옌센 부등식: 무한합의 경우]
    음이 아닌 값만을 가지는 실수열 \((p_n)_{n \in \NN}\)이 \(\sum_n p_n = 1\)을 만족한다고 하자. 그리고 \(\phi : (a, b) \subseteq \RR \to \RR\)가 볼록함수라 하자. 이때
    \begin{enum}
        \item \(\sum_n \abs{x_n p_n} < \infty\)
        \item 모든 \(n \in \NN\)에 대해 \(a < x_n < b\)
    \end{enum}
    를 만족하는 실수열 \((x_n)_{n \in \NN}\)에 대하여, 다음이 성립한다.
    \[
        \phi \paren{\sum_n x_n p_n} \le \sum_n \phi(x_n)p_n  
    \]
\end{thm}

\begin{rem}
    앞으로 등장할 횔더 부등식과 민코프스키 부등식에는 각각 적분 버전과 무한합 버전이 있다. 적분과 무한합이 대충 비슷하다는 느낌은 가지고 있겠지만, 해석개론 수준에서 둘의 연관성을 엄밀하게 설명하기는 쉽지 않다. 르베그 적분을 배우고 나면 무한합을 적분으로 이해할 수 있는 방법을 알게 된다. 따라서 그 이후에는 무한합 버전은 쓰지 않아도 된다.
\end{rem}

\begin{defn}
    \(p, q \in [1, \infty]\)가 \(1/p + 1/q = 1\)을 만족하면 \(p, q\) 각각이 서로의 \textbf{켤레 지수(conjugate exponent)}라고 한다.
\end{defn}
\begin{ex}
    2는 자기 자신의 켤레 지수이다. 또 1과 \(\infty\)는 서로의 켤레 지수이다.
\end{ex}

\begin{thm}[영 부등식\footnote{Young's inequality.}]
    \(p, q \in (1, \infty)\)가 켤레 지수일 때, 음이 아닌 실수 \(x, y \ge 0\)에 대하여
    \[
    xy \le \frac{x^p}{p} + \frac{y^q}{q}    
    \]
    가 성립한다.
\end{thm}
\begin{proof}
    \(x = 0\)이거나 \(y = 0\)이면 자명하게 성립하므로 둘 다 양수라고 하자.
    \[
    t := p \log x, \quad s:= q \log y
    \]
    로 정의하고, 볼록함수 \(\phi : x \mapsto e^x\)를 생각하면, 옌센 부등식에 의해
    \[
    xy = \phi \paren{\frac{t}{p} + \frac{s}{q}} \le \frac{\phi(t)}{p} + \frac{\phi(s)}{q} = \frac{x^p}{p} + \frac{y^q}{q}   
    \]
    이다.
\end{proof}

\begin{thm}[횔더 부등식\footnote{H\"older's inequality.}: 적분의 경우]
    \(p, q \in (1, \infty)\)가 켤레 지수일 때, 두 함수 \(f, g : I \to \CC\)에 대하여
    \begin{equation} \label{eq13.7}
        \int_I \abs{fg} \le \sqbracket{\int_I \abs{f}^p}^{1/p}\sqbracket{\int_I \abs{g}^q}^{1/q}
    \end{equation}
    가 성립한다.
\end{thm}
\begin{proof}
    \[
    A := \sqbracket{\int_I \abs{f}^p}^{1/p} \in [0, \infty], \quad B := \sqbracket{\int_I \abs{g}^q}^{1/q} \in [0, \infty]
    \]
    라 하자.\\
    \textbf{Case 1.} \(A, B \in (0, \infty)\)인 경우\\
    \(F := \abs{f} / A, G := \abs{g} / B\)로 정의하면, (\ref{eq13.7})\을 보이는 것은 \(\int_I FG \le 1\)을 보이는 것과 같다. 이때
    \[
    \int_I F^p = \int_I G^q = 1  
    \]
    이고, 영 부등식에 의해
    \[
        \int_I FG \le \int_I \paren{\frac{F^p}{p} + \frac{G^q}{q}} = \frac{1}{p} + \frac{1}{q} = 1
    \]
    이다.\\
    \textbf{Case 2.} \(A = \infty\) 또는 \(B = \infty\)이면 (\ref{eq13.7})의 우변이 \(\infty\)이므로 자명하게 성립한다.\\
    \textbf{Case 3.} \(A = 0\) 또는 \(B = 0\)이면 (\ref{eq13.7})의 좌변이 0이므로 자명하게 성립한다.
\end{proof}

\begin{thm}[횔더 부등식: 무한합의 경우]
    \(p, q \in (1, \infty)\)가 켤레 지수일 때, 복소수열 \((x_n)_{n \in \NN}, (y_n)_{n \in \NN}\)에 대하여
    \[
    \sum_n \abs{x_n y_n} \le \sqbracket{\sum_n \abs{x_n}^p}^{1/p}\sqbracket{\sum_n \abs{y_n}^p}^{1/q}  
    \]
    가 성립한다.
\end{thm}

\begin{thm}[민코프스키 부등식\footnote{Minkowski's inequality.}: 적분의 경우]
    \(p \in [1, \infty)\)와 \(f, g : I \to \CC\)에 대하여
    \begin{equation} \label{eq13.8}
        \sqbracket{\int_I \abs{f+g}^p}^{1/p} \le \sqbracket{\int_I \abs{f}^p}^{1/p} + \sqbracket{\int_I \abs{g}^p}^{1/p}
    \end{equation}
    가 성립한다.
\end{thm}
\begin{proof}
    \(p = 1\)일 때는 자명하므로 \(p \in (1, \infty)\)라 하고, \(q \in (1, \infty)\)를 \(p\)의 켤레 지수라 하자. 그리고 (\ref{eq13.8})의 좌변과 우변을 각각 \(A, B \in [0, \infty]\)라 하자.\\
    \textbf{Case 1.} \(A \in (0, \infty)\)인 경우\\
    횔더 부등식에 의해
    \begin{align*}
        A^p = \int_I \abs{f+g}^p &\le \int_I \abs{f}\abs{f+g}^{p-1} + \int_I \abs{g}\abs{f+g}^{p-1}\\
        &\le \sqbracket{\int_I \abs{f}^p}^{1/p} \sqbracket{\int_I \abs{f+g}^{(p-1)q}}^{1/q} + \sqbracket{\int_I \abs{g}^p}^{1/p} \sqbracket{\int_I \abs{f+g}^{(p-1)q}}^{1/q}\\
        &= B \cdot A^{p/q}
    \end{align*}
    이고, 양변을 \(A^{p/q}\)로 나누면 \(A \le B\)를 얻는다.\\
    \textbf{Case 2.} \(A = 0\)이면 자명하게 성립한다.\\
    \textbf{Case 3.} \(A = \infty\)인 경우\\
    함수 \(t \mapsto t^p\)는 \((0, \infty)\)에서 볼록함수이므로 옌센 부등식에 의해
    \[
    \paren{\frac{\abs{f}+\abs{g}}{2}}^p \le \frac{\abs{f}^p + \abs{g}^p}{2}
    \]
    가 성립한다. 따라서
    \[
    \abs{f+g}^p \le (\abs{f} + \abs{g})^p \le 2^{p-1}(\abs{f}^p + \abs{g}^p)
    \]
    를 얻고, 위 부등식을 \(I\)에서 적분하면
    \[
        \int_I \abs{f}^p + \int_I \abs{g}^p = \infty
    \]
    를 얻는다. 즉 \(\int_I \abs{f}^p = \infty\) 또는 \(\int_I \abs{g}^p = \infty\)이므로 \(B = \infty\)이다.
\end{proof}

\begin{thm}[민코프스키 부등식: 무한합의 경우]
    \(p \in [1, \infty)\)와 복소수열 \((x_n)_{n \in \NN}, (y_n)_{n \in \NN}\) 대하여
    \[
        \sqbracket{\sum_n \abs{x_n + y_n}^p}^{1/p} \le \sqbracket{\sum_n \abs{x_n}^p}^{1/p} + \sqbracket{\sum_n \abs{y_n}^p}^{1/p}    
    \]
    가 성립한다.
\end{thm}

\section{\(\mathcal{R}^p\) 공간}

\begin{defn}
    함수 \(f : I \to \CC\)와 \(p \in [1, \infty]\)에 대하여 다음과 같이 \(\norm{f}_p \in \RR\)를 정의한다.
    \begin{enum}
        \item \(p < \infty\)이면
        \[
            \norm{f}_p := \sqbracket{\int_I \abs{f}^p}^{1/p}.
        \]
        \item \(\int_I (\abs{f} - c)^+ = 0\)을 만족하는 모든 \(c \in [0, \infty]\)들의 집합을 \(E_f\)라 하자. 이때 \(\norm{f}_{\infty} := \inf E_f\)로 정의한다.
    \end{enum}
\end{defn}
\begin{rem}
    \quad

    \begin{enum}
        \item 일반적으로 \(\norm{f}_{\infty} < \infty\)이면 \(\norm{f}_{\infty} = \min E_f\)가 성립하는데, 이를 사실로 받아들이겠다. 즉,
        \[
        \int_I (\abs{f} - \norm{f}_{\infty})^+ = 0    
        \]
        이다. 그리고
        \[
        \sup_{x \in I} [\abs{f} - (\abs{f} - \norm{f}_{\infty})^+](x) = \norm{f}_{\infty}
        \]
        도 성립한다.
        \item \(f \in C_b(I)\)이면 \(\norm{f}_{\sup} = \norm{f}_{\infty}\)이다.
    \end{enum}
\end{rem}

(a)로부터 다음 명제가 성립한다.

\begin{prop} \label{13.3.2}
    \(f : I \to \CC\)에 대하여, 어떤 함수 \(g : I \to \CC\)가 존재하여
    \[
    \int_I g = 0, \quad \sup_{x \in I} [\abs{f} + g](x) = \norm{f}_{\infty}    
    \]
    이다.
\end{prop}

이제 횔더 부등식과 민코프스키 부등식을 좀 더 간결한 형태로 쓸 수 있다.

\begin{thm}[횔더 부등식: 적분의 경우]
    \(p, q \in [1, \infty]\)가 켤레 지수일 때, 두 함수 \(f, g : I \to \CC\)에 대하여
    \[
    \norm{fg}_1 \le \norm{f}_p \norm{g}_q    
    \]
    가 성립한다.
\end{thm}
\begin{proof}
    \(p = 1, q = \infty\)일 때만 추가로 보이면 된다. 명제 \ref{13.3.2}에 의해
    \[
    \int_I h = 0, \quad \sup_{x \in I} [\abs{g} + h](x) = \norm{g}_{\infty}       
    \]
    인 \(h : I \to \CC\)가 존재한다. 이제
    \[
        \norm{fg}_1 = \int_I \abs{fg} = \int_I \abs{f}(\abs{g} + h) \le \int_I \abs{f} \norm{g}_{\infty} = \norm{f}_1 \norm{g}_{\infty}
    \]
    이므로 \(p = 1\)일 때도 횔더 부등식이 성립한다.
\end{proof}

\begin{thm}[민코프스키 부등식: 적분의 경우]
    \(p \in [1, \infty]\)와 \(f, g : I \to \CC\)에 대하여
    \[
    \norm{f+g}_p \le \norm{f}_p + \norm{g}_p    
    \]
    가 성립한다.
\end{thm}
\begin{proof}
    \(p = \infty\)일 때 추가적인 증명은 연습문제로 남긴다.
\end{proof}

이제 \(\calR^p\) 공간을 정의할 수 있다.

\begin{defn}
    \(p \in [1, \infty]\)에 대하여 \(\norm{f}_p < \infty\)를 만족하는 모든 \(f : I \to \CC\)의 집합을 \(\calR^p(I)\)라고 쓰자.
\end{defn}

\begin{cor}
    \(\calR^p(I)\)는 자연스러운 \(\CC\)-벡터공간의 구조를 가진다.
\end{cor}
\begin{proof}
    덧셈과 스칼라배에 대해 닫혀 있는 것만 보이면 된다. 스칼라배에 대해 닫혀 있는 것은 분명하고, 덧셈에 대해 닫혀 있는 것은 민코프스키 부등식에 의해서 성립한다.
\end{proof}

이제 \(\calR^p(I)\)에 노름공간의 구조를 주려고 한다. 당연히 \(\norm{\cdot}_p\)가 노름이 되기를 기대한다. 그런데 문제가 있다. \(\norm{f}_p = 0\)이라고 해서 \(f = 0\)이라는 보장이 없다. 가장 쉽게, 한 점에서만 1이고 나머지 점들에서 모두 0인 함수 \(f\)는 0이 아니지만 \(\norm{f}_p = 0\)이다. 즉 약간의 수정이 필요하다.

\begin{defn}
    \(f, g \in \calR^p(I)\)에 대하여
    \[
    f \sim g  \iff \int_I (f - g) = 0
    \]
    으로 정의한다.
\end{defn}

\begin{prop}
    \(\sim\)은 \(\calR^p(I)\)의 동치관계이다.
\end{prop}
\begin{proof}
    연습문제로 남긴다.
\end{proof}

\begin{prop}
    \(\calR^p(I) / \sim\)은 자연스럽게 유도된 \(\CC\)-벡터공간의 구조를 가진다.
\end{prop}
\begin{proof}
    \(f, g \in \calR^p(I), c \in \CC\)에 대해
    \[
        [f] + [g] := [f+g], \quad c[f] := [cf]
    \]
    의 연산이 유도된다. 이들이 잘 정의됨을 보이고 따라서 \(\CC\)-벡터공간이 됨을 보이는 것은 연습문제로 남긴다.
\end{proof}

\begin{defn}
    \textbf{표기법을 남용하여}, 앞으로 \(\calR^p(I) := \calR^p(I) / \sim\)으로 쓰겠다. 그리고 \(f \in \calR^p(I)\)에 대해 \(f := [f]\)로 쓴다.
\end{defn}

\begin{thm}
    \(\calR^p(I)\)는 \(\norm{\cdot}_p\)에 의한 노름공간의 구조를 가진다.
\end{thm}
\begin{proof}
    지금까지의 논의에 의해 거의 당연하다.
\end{proof}

\(p = 2\)일 때는 이야기를 더 할 수 있다.

\begin{thm}
    \(f, g \in \calR^2(I)\)에 대해
    \[
    \inner{f}{g} := \int_I f \overline{g}   
    \]
    로 정의하면 \(\inner{\cdot}{\cdot}\)은 \(\calR^2(I)\)의 내적이고, \(\inner{\cdot}{\cdot}\)에 의해 유도되는 노름은 \(\norm{\cdot}_2\)와 같다. 따라서 \(\calR^2(I)\)는 \(\inner{\cdot}{\cdot}\)에 의한 내적공간의 구조를 가진다.
\end{thm}
\begin{proof}
    우선 위와 같이 정의된 내적이 잘 정의되었는지부터 확인해야 하자. 횔더 부등식에 의해\footnote{따라서 코시-슈바르츠 부등식은 횔더 부등식의 특수한 경우이다.}
    \[
    \abs{\inner{f}{g}} \le \int_I \abs{f \overline{g}} \le \norm{f}_2 \norm{g}_2 < \infty    
    \]
    이므로 \(\inner{\cdot}{\cdot}\)은 잘 정의된 함수이다. 이제 \(\inner{\cdot}{\cdot}\)이 내적의 공리를 만족함은 쉽게 확인할 수 있으며, 특히
    \[
    \inner{f}{f} = \int_I \abs{f}^2 = \norm{f}_2^2    
    \]
    이다.
\end{proof}

이제 옹골 지지를 가지는 연속 복소함수들의 공간 \(C_c(I)\)를 생각하자. 각 \(p\)에 대해 \(\calR^p(I)\)가 \(C_c(I)\)를 포함하는 것은 분명하다. 우리의 주장은 \(p < \infty\)일 때 \(C_c(I)\)가 \(\calR^p(I)\)에서 조밀하다는 것이다. 이를 위해 몇 가지 준비가 필요하다.

\begin{defn}
    구간 \(I_1, \ldots, I_n \subseteq I\)와 \(c_1, \ldots, c_n \in \CC\)에 대하여
    \[
    s = \sum_{k=1}^n c_k \chi_{I_k}
    \]
    로 쓸 수 있는 함수 \(s\)를 \textbf{계단함수(step function)}라고 한다.
\end{defn}

\begin{rem}
    계단함수의 정의에서 \(I_k\)들이 서로소인 구간이라고 해도 동치인 정의를 얻는다(동치임을 보이는 것은 연습문제로 남긴다). 상황에 따라 둘 중 편한 쪽을 택하면 된다.
\end{rem}

\begin{lem} \label{13.1.14}
    함수 \(f : [a, b] \subseteq \RR \to \RR\)가 리만적분가능하다고 하자. 이때 임의의 양수 \(\eps > 0\)에 대하여 다음을 만족하는 계단함수 \(l, u : [a, b] \to \RR\)가 존재한다.
    \[
        l \le f \le u, \quad \abs{l}, \abs{u} \le \sup_{x \in I} \abs{f(x)}, \quad \int_a^b (u - l) < \eps      
    \]
\end{lem}
\begin{proof}
    \(f\)가 리만적분가능하므로, 구간 \([a, b]\)의 적당한 분할 \(\{x_0, \ldots, x_n\}\)이 존재하여
    \[
    U(f, P) - L(f, P) < \eps    
    \]
    를 만족한다. 이제
    \[
    I_1 := [x_0, x_1], \quad I_k := (x_{k-1}, x_k] \ (2 \ge k \ge n), \quad m_k := \inf_{t \in \overline{I_k}} f(t), \quad M_k := \sup_{t \in \overline{I_k}} f(t)
    \]
    로 정의하고
    \[
    l := \sum_{k=1}^n m_k \chi_{I_k}, \quad u := \sum_{k=1}^n M_k \chi_{I_k}    
    \]
    로 정의하면, \(l\)과 \(u\)는 계단함수이다. \(l \le f \le u\)이고 \(\abs{l}, \abs{u} \le \sup_{x \in I} \abs{f(x)}\)인 것은 쉽게 확인할 수 있고,
    \[
    L(f, P) = \int_a^b l, \quad U(f, P) = \int_a^b u
    \]
    이므로 \(\int_a^b (u - l) < \eps\)도 성립한다.
\end{proof}

\begin{lem} \label{13.3.15}
    \(p \in [1, \infty)\)이고 \(f \in \calR^p([a, b])\)라 하자. 이때 임의의 양수 \(\eps > 0\)에 대하여
    \[
        \norm{f - s}_p < \eps, \quad \abs{s} \le \abs{f}
    \]
    를 만족하는 계단함수 \(s : [a, b] \to \CC\)가 존재한다.
\end{lem}
\begin{proof}
    \quad\\
    \textbf{Step 1.} \(f\)가 실함수인 경우\\
    \(f\)는 유계이므로 \(\abs{f} \le M\)인 어떤 양수 \(M > 0\)이 존재한다. 보조정리 \ref{13.1.14}에 의해, 다음을 만족하는 계단함수 \(l, u : [a, b] \to \RR\)를 찾을 수 있다.
    \[
        l \le f \le u, \quad \abs{l}, \abs{u} \le \sup_{x \in I} \abs{f(x)}, \quad \int_a^b (u - l) < \frac{\eps^p}{(2M)^{p-1}}
    \]
    이제
    \[
    \norm{f - l}_p^p = \int_a^b (f - l)^p \le \int_a^b (u - l)^{p-1}(u - l) \le (2M)^{p-1} \int_a^b (u - l) < \eps^p    
    \]
    이므로 \(\norm{f - l}_p < \eps\)이다.\\
    \textbf{Step 2.} 일반적인 경우\\
    Step 1에 의해, 다음을 만족하는 계단함수 \(l_1, l_2 : [a, b] \to \RR\)가 존재한다.
    \[
    \norm{\mathrm{Re}f - l_1}_p < \frac{\eps}{2}, \quad \norm{\mathrm{Im}f - l_2}_p < \frac{\eps}{2}    
    \]
    이제 \(s := l_1 + il_2\)로 두면 \(s\)도 계단함수이고,
    \[
    \norm{f - s}_p \le \norm{\mathrm{Re}f - l_1}_p +  \norm{\mathrm{Im}f - l_2}_p < \eps
    \]
    이다. 또
    \[
        \abs{s} = \sqrt{\abs{\mathrm{Re}s}^2 + \abs{\mathrm{Im}s}^2} \le \sqrt{\abs{\mathrm{Re}f}^2 + \abs{\mathrm{Im}f}^2} = \abs{f}
    \]
    이다.
\end{proof}

\begin{lem} \label{13.3.16}
    \(p \in [1, \infty)\)이고 \(s : [a, b] \subseteq \RR \to \CC\)가 계단함수라고 하자. 이때 임의의 양수 \(\eps > 0\)에 대해
    \[
    \norm{g - s}_p < \eps, \quad \abs{g} \le \abs{s}    
    \]
    를 만족하는 \(g \in C([a, b])\)가 존재한다.
\end{lem}
\begin{proof}
    \quad\\
    \textbf{Step 1.} \(s\)가 실함수인 경우\\
    서로소인 구간들 \(I_1, \ldots, I_k\)에 대해 \(s = \sum_{k=1}^n c_k \chi_{I_k}\)로 쓰고, \(I_k\)의 왼쪽 끝점과 오른쪽 끝점을 각각 \(a_k, b_k\)라 하자. 이때 양수 \(\delta > 0\)을 다음을 만족하도록 고를 수 있다.
    \[
    0 < \delta < \min \left\{\frac{b_1 - a_1}{2}, \ldots, \frac{b_n - a_n}{2}, \frac{\eps^p}{2 \sum_{k=1}^n \abs{c_k}^p }\right\}    
    \]
    이제 각 \(k = 1, \ldots, n\)에 대해 \(g_k : [a, b] \to \RR\)를 다음과 같이 정의하자.
    \[
    g_k : x \mapsto
    \begin{cases}
        0 &\textif \ x \notin I_k\\
        1 &\textif \ a_k + \delta \le x \le b_k - \delta\\
        (1/\delta)(x - a_k) &\textif \ a_k \le x \le a_k + \delta\\
        -(1-\delta)(x - b_k) &\textif \ b_k - \delta \le x \le b_k
    \end{cases}    
    \]
    이때 \(\int_a^b \abs{g_k - \chi_{I_k}}^p < 2\delta\)이다. 이제 \(g := \sum_{k=1}^n c_k g_k\)로 정의하면 \(g\)는 연속함수이고,
    \[
    \norm{g - s}_p^p = \int_a^b \abs{\sum_{k=1}^n c_k (g_k - \chi_{I_k})}^p \le \int_a^b \sum_{k=1}^n \abs{c_k}^p \abs{g_k - \chi_{I_k}}^p <  \sum_{k=1}^n \abs{c_k}^p \cdot 2\delta < \eps^p
    \]
    이므로 \(\norm{g - s}_p < \eps\)이고 \(\abs{g} \le \abs{s}\)이다.\\
    \textbf{Step 2.} 일반적인 경우\\
    Step 1에 의해, 다음을 만족하는 연속 실함수 \(g_1, g_2 : [a, b] \to \RR\)가 존재한다.
    \[
    \norm{g_1 - \mathrm{Re}s}_p < \frac{\eps}{2}, \quad  \norm{g_2 - \mathrm{Im}s}_p < \frac{\eps}{2}   
    \]
    이제 \(g := g_1 + ig_2\)로 두면
    \[
    \norm{g - s}_p \le \norm{g_1 - \mathrm{Re}s}_p +  \norm{g_2 - \mathrm{Im}s}_p < \eps
    \]
    이고,
    \[
    \abs{g} = \sqrt{\abs{g_1}^2 + \abs{g_2}^2} \le \sqrt{\abs{\mathrm{Re}s}^2 + \abs{\mathrm{Im}s}^2} = \abs{s}
    \]
    이다.
\end{proof}

\begin{prop} \label{13.3.17}
    \(p \in [1, \infty)\)이고 \(f \in \calR^p([a, b])\)일 때 임의의 양수 \(\eps > 0\)에 대하여
    \[
        \norm{f - g}_p < \eps, \quad \abs{g} \le \abs{f}
    \]
    를 만족하는 연속함수 \(g : [a, b] \to \CC\)가 존재한다.
\end{prop}
\begin{proof}
    보조정리 \ref{13.3.15}\와 \ref{13.3.16}의 직접적인 결과이다.
\end{proof}

이제 우리의 결론이다.

\begin{thm}
    \(p \in [1, \infty)\)일 때 \(C_c(I)\)는 \(\calR^p(I)\)에서 조밀하다.
\end{thm}
\begin{proof}
    임의의 \(f \in \calR^p(I), \eps > 0\)이 주어졌다고 하자. 특이적분의 정의에 의해, 다음을 만족하는 유계닫힌구간 \([a, b] \subseteq I\)가 존재한다.
    \[
    \int_a^b \abs{f}^p > \int_I \abs{f}^p - \frac{\eps^p}{4}
    \]
    명제 \ref{13.3.17}에 의해, 다음을 만족하는 연속함수 \(g \in C([a, b])\)가 존재한다.
    \[
    \int_a^b \norm{f - g}^p < \frac{\eps^p}{4}, \quad \abs{g(x)} \le \abs{f(x)} \ (x \in [a, b])
    \]
    한편 \([a, b]\)에서 \(\abs{f} \le M\)인 양수 \(M > 0\)이 존재한다. 양수 \(\delta > 0\)을 다음을 만족하도록 고르자.
    \[
    0 < \delta < \min \left\{\frac{b -a }{2}, \frac{\eps^p}{4 \cdot (2M)^p}\right\}    
    \]
    그러면 \(J := [a + \delta, b - \delta]\)도 유계닫힌구간이고, 특히
    \[
    \int_J \norm{f - g}^p \le \int_a^b \norm{f - g}^p < \frac{\eps^p}{4}    
    \]
    이다. \(J\)와 \(I \setminus (a + \delta/2, b - \delta/2)\)는 서로소인 닫힌집합이므로, 우리손 보조정리를 이용하여
    \[
        \phi \vert_J = 1, \phi \vert_{I \setminus (a + \delta/2, b - \delta/2)} = 0
    \]
    을 만족하는 연속함수 \(\phi : I \to [0, 1]\)을 잡자. 그리고 함수 \(h : I \to \CC\)를 다음과 같이 정의한다.
    \[
    h : x \mapsto
    \begin{cases}
        \phi(x)g(x) &\textif \ x \in [a, b]\\
        0 &\textif \ x \in I \setminus [a, b]
    \end{cases}    
    \]
    이때 \(h\)는 연속함수이고, \(\mathrm{supp}h \subseteq [a, b]\)이므로 \(h \in C_c(I)\)이다. 이제
    \begin{align*}
        \norm{f - h}_p^p &= \int_{I \setminus [a, b]} \abs{f - h}^p + \int_{[a, b] \setminus J} \abs{f - h}^p + \int_J \abs{f - h}^p\\
        &= \int_{I \setminus [a, b]} \abs{f}^p + \int_{[a, b] \setminus J} \abs{f - \phi g}^p + \int_J \abs{f - g}^p\\
        &< \frac{\eps^p}{4} + 2\delta \cdot (2M)^p + \frac{\eps^p}{4}\\
        &< \eps^p
    \end{align*}
    이므로 \(\norm{f - h}_p < \eps\)이다. 따라서 \(C_c(I)\)는 \(\calR^p(I)\)에서 조밀하다.
\end{proof}

그러면 \(p = \infty\)일 때는 어떻게 될까? 사실 이것은 이전에 증명하였다. \(C_0(I) \subseteq \calR^\infty(I)\)이고, 명제 \ref{10.4.6}에서 주어진 \(C_0(I)\)의 노름은 \(\calR^\infty(I)\)에서 물려받은 노름과 완전히 같기 때문에, 정리 \ref{10.4.11}\이 그대로 성립한다. 즉 \(\calR^\infty(I)\)에서 \(C_c(I)\)의 닫힘은 \(C_0(I)\)이다.

여기까지 오면서 언급하지 않은 것이 하나 있는데, 바로 \(\calR^p(I)\)가 바나흐공간인지 여부이다. 안타깝게도 \(\calR^p(I)\)는 바나흐공간이 아니다. \(\calR^1([0, 1])\) 안의 함수열 \((f_n)_{n \in \ZZ_+}\)을
\[
    f_n : x \mapsto \min \{n, x^{-1/2}\}
\]
으로 정의하자. \(m > n \ge N\)에 대하여
\[
\int_0^1 \abs{f_m - f_n} \le \int_0^{-1/N^2} x^{-1/2} dx = 2N^{1/2} \xrightarrow[]{N \to \infty} 0    
\]
이므로 \((f_n)_{n \in \ZZ_+}\)은 코시수열이지만 \(\calR^1([0, 1])\) 안에서 수렴하지 않는다. 따라서 \(\calR^1([0, 1])\)은 바나흐공간이 아니다. 이를 보완하기 위한 새로운 적분(르베그적분)이 필요해질 것이다.

마지막으로 \(\calR^p\) 공간들 사이의 포함관계에 대해 언급하고 이 절을 마친다. 일반적으로 \(\calR^p\) 공간들 사이에는 아무런 포함관계가 없는데, 이를 보이는 것은 연습문제로 남긴다. 한편 \(I\)가 유계구간인 경우에는 포함관계를 얻는다. 먼저 다음 부등식을 보자. 이는 횔더 부등식의 직접적인 따름정리이다.

\begin{cor} \label{13.3.19}
    \(\mu(I) := \int_I 1 \in (0, \infty]\)로 쓰자. \(1 \le p < q \le \infty\)와 임의의 함수 \(f : I \to \CC\)에 대하여
    \[
    \norm{f}_p \le \mu(I)^{1/p - 1/q} \norm{f}_q    
    \]
    이다. 
\end{cor}
\begin{proof}
    \(q = \infty\)일 때 \((q-p)/q = q/(q-p) = 1\)로 쓰면, \(q/(q-p)\)와 \(q/p\)는 켤레 지수이다. 따라서 횔더 부등식에 의해
    \[
    \norm{\abs{f}^p}_1 \le \norm{1}_{q/(q-p)} \norm{\abs{f}^p}_{q/p}   
    \]
    이다. 양변을 다시 쓰면
    \[
    \int_I \abs{f}^p \le \mu(I)^{(q-p)/q} \sqbracket{\int_I \abs{f}^q}^{p/q}    
    \]
    이고, 양변에 \(1/p\)승을 취하면 원하는 부등식을 얻는다.
\end{proof}

이로부터 그럴듯한 다음 정리를 얻는다.

\begin{thm}
    \(I\)가 유계구간이면 \(1 \le p < q \le \infty\)일 때 \(\calR^q(I) \subseteq \calR^p(I)\)이고,
    \[
        \iota : (\calR^q(I), \norm{\cdot}_q) \to (\calR^p(I), \norm{\cdot}_p) : f \mapsto f    
    \]
    는 립시츠 연속이다.
\end{thm}
\begin{proof}
    따름정리 \ref{13.3.19}에 의해, 어떤 상수 \(0 < M < \infty\)가 존재하여 임의의 \(f \in \calR^q(I)\)에 대해 \(\norm{f}_p \le M \norm{f}_q < \infty\)이므로 \(f \in \calR^p(I)\)이며, \(\iota : (\calR^q(I), \norm{\cdot}_q) \to (\calR^p(I), \norm{\cdot}_p)\)가 립시츠 연속이다.
\end{proof}

\section{\(\ell^p\) 공간} \label{sec13.4}

지난 절에서 했던 내용들을 거의 그대로 반복할 것이다. 전에도 말했지만, 르베그적분을 공부하고 나면 수열의 무한합과 함수의 적분을 한꺼번에 이해할 수 있게 된다.

\begin{defn}
    복소수열 \(x = (x_k)_{k \in \NN} = (x(k))_{k \in \NN}\)과 \(p \in [1, \infty]\)에 대하여 다음과 같이 \(\norm{x}_{p} \in \RR\)를 정의한다.
    \begin{enum}
        \item \(p < \infty\)이면
        \[
        \norm{x}_p := \sqbracket{\sum_k \abs{x(k)}^p}^{1/p}.
        \]
        \item \(\norm{x}_{\infty} := \sup_{k \in \NN} \abs{x(k)}\).
    \end{enum}
\end{defn}
\begin{rem}
    \(\norm{\cdot}_{\infty}\)와 \(\norm{\cdot}_{\sup}\)은 완전히 같다.
\end{rem}

마찬가지로 횔더 부등식과 민코프스키 부등식을 좀 더 간결한 형태로 쓸 수 있다. 증명은 생략한다.

\begin{thm}[횔더 부등식: 무한합의 경우]
    \(p, q \in [1, \infty]\)가 켤레 지수일 때, 두 복소수열 \(x, y\)에 대하여
    \[
    \norm{xy}_{1} \le \norm{x}_p \norm{y}_q    
    \]
    가 성립한다.
\end{thm}

\begin{thm}[민코프스키 부등식: 무한합의 경우]
    \(p \in [1, \infty]\)와 두 복소수열 \(x, y\)에 대하여
    \[
    \norm{x + y}_p \le \norm{x}_p + \norm{y}_p    
    \]
    가 성립한다.
\end{thm}

\begin{cor}
    \(F = \RR\) 또는 \(\CC\). \(1 \le p \le \infty\)에 대해 함수 \(\norm{\cdot}_p : F^d \to \RR\)를 다음과 같이 정의하자.
    \[
        \norm{(x_1, \ldots, x_d)}_p :=
        \begin{cases}
            \displaystyle
            \sqbracket{\sum_{i=1}^d \abs{x_i}^p}^{1/p} &\textif \ 1 \le p < \infty\\
            \displaystyle
            \max_{1 \le i \le d} \abs{x_i} &\textif \ p = \infty
        \end{cases}
    \]
    이때 \(\norm{\cdot}_p\)는 \(F^d\)의 노름이 된다.
\end{cor}

이제 \(\ell^p\) 공간을 정의할 수 있다.

\begin{defn}
    \(p \in [1, \infty]\)에 대하여 \(\norm{x}_p < \infty\)를 만족하는 모든 복소수열 \(x\)의 집합을 \(\ell^p\)라고 쓰자.
\end{defn}

\begin{cor}
    \(\ell^p\)는 자연스러운 \(\CC\)-벡터공간의 구조를 가진다.
\end{cor}
\begin{proof}
    민코프스키 부등식에 의해 \(\ell^p\)는 덧셈에 의해 닫혀 있다. 나머지는 거의 당연하다.
\end{proof}

\(\calR^p(I)\)에서와는 다르게 \(\ell^p\)는 그 자체가 노름공간의 구조를 가진다.

\begin{thm}
    \(\ell^p\)는 \(\norm{\cdot}_p\)에 의한 노름공간의 구조를 가진다.
\end{thm}
\begin{proof}
    지금까지의 논의에 의해 거의 당연하다.
\end{proof}

여기서도 마찬가지로 \(p = 2\)일 때는 이야기를 더 할 수 있다.

\begin{thm}
    \(x, y \in \calR^2(I)\)에 대해
    \[
    \inner{x}{y} := \sum_n x(k) \overline{y(k)}    
    \]
    으로 정의하면 \(\inner{\cdot}{\cdot}\)은 \(\ell^2\)의 내적이고, \(\inner{\cdot}{\cdot}\)에 의해 유도되는 노름은 \(\norm{\cdot}_2\)와 같다. 따라서 \(\ell^2\)는 \(\inner{\cdot}{\cdot}\)에 의한 내적공간의 구조를 가진다.
\end{thm}
\begin{proof}
    거의 당연하다. 자세한 증명은 연습문제로 남긴다.
\end{proof}

이제 \(\ell^p\)가 바나흐공간임을 보이자.

\begin{thm}
    \(p \in [1, \infty]\)에 대하여 \(\ell^p\)는 바나흐공간이다. 특히 \(\ell^2\)는 힐베르트공간이다.
\end{thm}
\begin{proof}
    \((x_n)_{n \in \NN}\)이 \(\ell^p\)의 코시수열이라고 하자. 임의의 양수 \(\eps > 0\)에 대하여, 다음을 만족하는 자연수 \(N\)이 존재한다.
    \[
    m, n \ge N \implies \norm{x_m - x_n}_p < \eps
    \]
    그런데 각 \(k \in \NN\)에 대해 \(\abs{x_m(k) - x_n(k)} \le \norm{x_m(k) - x_n(k)}_p\)이므로 다음이 성립한다.
    \[
    m, n \ge N \implies \abs{x_m(k) - x_n(k)} < \eps    \quad (k \in \NN)
    \]
    즉 각 \(k \in \NN\)에 대해 \((x_n(k))_{n \in \NN}\)이 복소수 안의 코시수열이고, 따라서 \(x(k) := \lim_n x_n(k)\)가 존재한다. 이제 \(x_n \to x\)임을 보이자. 먼저 임의의 \(m, n \ge N\)에 대해 다음이 성립한다.
    \[
    \sum_{k=0}^L \abs{x_m(k) - x_n(k)}^p \le \norm{x_m - x_n}_p^p < \eps^p \quad (L \in \NN)
    \]
    양변에 \(m \to \infty\)를 취하면, 임의의 \(n \ge N\)에 대해 다음이 성립한다.
    \[
        \sum_{k=0}^L \abs{x(k) - x_n(k)}^p \le \eps^p \quad (L \in \NN)
    \]
    마지막으로 \(L \to \infty\)를 취하면, 임의의 \(n \ge N\)에 대해 다음이 성립한다.
    \[
        \sum_{k=0}^\infty \abs{x(k) - x_n(k)}^p = \abs{x - x_n}_p^p \le \eps^p
    \]
    따라서 \(n \ge N\)일 때 \(\norm{x - x_n}_p \le \eps\)이고, \(x \in \ell^p\)이다. \(\eps > 0\)이 임의의 양수였으므로 \((x_n)_{n \in \NN}\)이 \(\ell^p\)에서 \(x\)로 수렴하고, 따라서 \((x_n)_{n \in \NN}\)이 수렴한다. 즉 \(\ell^p\)는 바나흐공간이다.
\end{proof}

이제 \(p \in [1, \infty)\)일 때 \(\ell^p\)의 조밀한 부분집합을 찾자. \(\calR^p(I)\)에서와 다르게 이쪽은 상대적으로 쉽다.

\begin{thm}
    \(p \in [1, \infty)\)일 때 \(c_{00}\)는 \(\ell^p\)에서 조밀하다.
\end{thm}
\begin{proof}
    임의의 \(x \in \ell^p, \eps > 0\)에 대하여 다음을 만족하는 자연수 \(N\)이 존재한다.
    \[
        \sum_{k=N+1}^\infty \abs{x(k)} < \eps^p
    \]
    이제 수열 \(y \in c_{00}\)을 다음과 같이 정의하면
    \[
        y : k \mapsto \begin{cases}
            x(k) &\textif \ k \le N\\
            0 &\textif \ k > N
        \end{cases}
    \]
    \(\norm{x - y}_p < \eps\)임은 분명하다. 따라서 \(c_{00}\)는 \(\ell^p\)에서 조밀하다.
\end{proof}

\(p = \infty\)일 때는 어떻게 되는가? 이것은 이미 정리 \ref{11.3.8}에서 증명하였다. 즉 \(\ell^\infty\)에서 \(c_{00}\)의 닫힘은 \(c_0\)이다.

\(\ell^1\) 안에서 코시곱은 곱셈과 같은 역할을 한다.

\begin{thm} \label{13.4.11}
    \(x, y \in \ell^1\)에 대하여 두 급수 \(\sum_n x(n), \sum_n y(n)\)의 코시곱을 \(\sum_n (x * y)(n)\)이라 쓰면 \(x * y \in \ell^1\)이고 \(\norm{x * y}_1 \le \norm{x}_1 \norm{y}_1\)이다.
\end{thm}
\begin{proof}
    정의를 그대로 쓰면 된다.
    \begin{align*}
        \norm{x * y}_1 &= \sum_{n=0}^\infty \abs{\sum_{k=0}^n x(k) y(n-k)}\\
        &\le \sum_{n=0}^\infty \sum_{k=0}^n \abs{x(k) y(n-k)}\\
        &= \sum_{k=0}^\infty \sum_{n=k}^\infty \abs{x(k) y(n-k)}\\
        &= \sum_{k=0}^\infty \abs{x(k)} \norm{y}_1 = \norm{x}_1\norm{y}_1.
    \end{align*}
\end{proof}

마지막으로 \(\ell^p\) 공간들 사이의 포함관계에 대해 언급하고 이 절을 마친다.

\begin{lem}
    \(1 \le p < q \le \infty\)와 임의의 복소수열 \(x\)에 대하여 \(\norm{x}_q \le \norm{x}_p\)이다.
\end{lem}
\begin{proof}
    \(p \in [1, \infty]\)를 고정하자. \(\norm{x}_p = 0\)이면 \(x = 0\)이므로 \(\norm{x}_q = \norm{x}_p = 0\)이고, \(\norm{x}_p = \infty\)이면 더 증명할 것이 없다. 따라서 \(0 < \norm{x}_p < \infty\)라 하자. 이때 임의의 \(k \in \NN\)에 대해 \(\abs{x(k)} \le \norm{x}_p\)이므로 \(\norm{x}_{\infty} \le \norm{x}_p\)인 것은 분명하다. 이제 \(p  < q < \infty\)인 경우만을 생각하자. \(c := \norm{x}_{\infty} \in (0, \infty)\)로 두고 \(y := x / c\)라 하면 모든 \(k \in \NN\)에 대해 \(\abs{y(k)}^p \le 1\)이다. 이제
    \[
    \norm{y}_q^q \le \sum_{k} \abs{y(k)}^q \le \sum_{k} \abs{y(k)}^p = \norm{y}_p^p = 1   
    \]
    이고, 따라서 \(\norm{y}_q \le 1 = \norm{y}_p\)이다. \(x = cy\)였으므로 \(\norm{x}_q \le \norm{x}_p\)를 얻는다.
\end{proof}

\begin{thm}
    \(1 \le p < q \le \infty\)일 때 \(\ell^p \subseteq \ell^q\)이고,
    \[
        \iota : (\ell^p, \norm{\cdot}_p) \to (\ell^q, \norm{\cdot}_q) : x \mapsto x
    \]
    는 립시츠 연속이다.
\end{thm}
\begin{proof}
    연습문제로 남긴다.
\end{proof}



\chapter{적분으로 정의된 함수}

이번 장에서는 리만적분으로 정의된 함수의 연속성과 미분가능성에 대해 공부한다. 이 역시 적분기호와 극한의 순서를 바꿀 수 있는지에 대한 질문의 연장선상에 있다. 이를 이용하여 연속함수에 대한 푸비니 정리를 증명한다. 그리고 이에 대한 응용으로 감마 함수와 합성곱에 대하여 공부한다.

\section{적분으로 정의된 함수의 연속성과 미분}

경험적으로 다음 정리가 성립하는 것을 알고 있는데, 정말로 그렇다.

\begin{thm} \label{14.1.1}
    연속함수 \(f : [a, b] \times [c, d] \subseteq \RR^2 \to \RR\)에 대하여 함수 \(\phi : [a, b] \to \RR\)를
    \begin{equation} \label{eq14.1}
        \phi : x \mapsto \int_c^d f(x, t) dt   
    \end{equation}
    로 정의하면, 다음이 성립한다.
    \begin{enum}
        \item \(\phi\)도 \([a, b]\)에서 연속이다.
        \item \(f\)의 첫 번째 변수에 대한 편도함수 \(D_1f\)가 \((a, b)\times [c, d]\)에서 존재하고 연속이면 \(\phi\)도 \((a, b)\)에서 미분가능하고, 임의의 \(x \in (a, b)\)에 대해
        \begin{equation} \label{eq14.2}
            \phi'(x) = \int_c^d D_1f(x, t)dt 
        \end{equation}
        이다.
    \end{enum}
\end{thm}
\begin{proof}
    \quad

    \begin{enum}
        \item \(f\)가 옹골집합 \([a, b] \times [c, d]\)에서 연속이므로 고른연속이다. 따라서 양수 \(\eps > 0\)이 주어졌을 때 임의의 \(x_1, x_2 \in [a, b], t_1, t_2 \in [c, d]\)에 대해 다음을 만족하는 양수 \(\delta > 0\)이 존재한다.
        \[
        \norm{(x_1, t_1) - (x_2, t_2)} < \delta \implies \abs{f(x_1, t_1) - f(x_2, t_2)} < \frac{\eps}{d - c}
        \]
        이제 \(x, y \in [a, b]\)가 \(\abs{x - y} < \delta\)를 만족하면, 임의의 \(t \in [c, d]\)에 대해 \(\norm{(x, t) - (y, t)} < \delta\)이므로
        \[
        \abs{\phi(x) - \phi(y)} \le \int_c^d \abs{f(x, t) - f(y, t)} dt < \int_c^d \frac{\eps}{d - c} = \eps    
        \]
        이다. 따라서 \(\phi\)는 \([a, b]\)에서 연속함수이다.
        \item \((a, b)\)에 포함되는 임의의 유계닫힌구간 \([u, v]\)를 잡자. 그러면 \(D_1f\)는 옹골집합 \([u, v] \times [c, d]\)에서 연속이므로 고른연속이다. 따라서 양수 \(\eps > 0\)이 주어졌을 때 임의의 \(x_1, x_2 \in [a, b], t_1, t_2 \in [c, d]\)에 대해 다음을 만족하는 양수 \(\delta > 0\)이 존재한다.
        \[
        \norm{(x_1, t_1) - (x_2, t_2)} < \delta \implies \norm{D_1f(x_1, t_1) - D_1f(x_2, t_2)} < \frac{\eps}{d-c}    
        \]
        이때 \(x \in (u, v), 0 \le \abs{h} < \delta\)에 대해
        \begin{align*}
            \frac{\phi(x+h) - \phi(x)}{h} - \int_c^d D_1(x, t)dt &= \frac{1}{h} \int_c^d \int_x^{x+h} D_1f(x', t) dx' dt - \int_c^d D_1(x, t)dt\\
            &= \frac{1}{h} \int_c^d \int_x^{x+h} [D_1f(x', t) - D_1(x, t)] dx' dt
        \end{align*}
        이고, \(\norm{(x', t) - (x, t)} \le h < \delta\)이므로
        \begin{align*}
            \abs{\frac{\phi(x+h) - \phi(x)}{h}} &\le \frac{1}{\abs{h}} \int_c^d \int_x^{x+h} \abs{D_1f(x', t) - D_1(x, t)} dx'dt\\
            &< \frac{1}{\abs{h}} \int_c^d \int_x^{x+h} \frac{\eps}{d-c} dx'dt = \eps   
        \end{align*}
        이다. 따라서 정의에 의해 \(x \in (u, v)\)에 대해 (\ref{eq14.2})\가 성립한다. 그런데 \([u, v] \subseteq (a, b)\)가 임의의 유계닫힌구간이었으므로, 임의의 \(x \in (a, b)\)에 대해 (\ref{eq14.2})\가 성립한다.
    \end{enum}
\end{proof}


\begin{ex}
    함수 \(f, g : \RR \to \RR\)를 다음과 같이 정의하자.
    \[
    f :x \mapsto \sqbracket{\int_0^x e^{-t^2} dt}^2, \quad g : x \mapsto \int_0^1 \frac{e^{-x^2(t^2+1)}}{t^2+1} dt
    \]
    이때 함수
    \[
    (x, t) \mapsto \frac{e^{-x^2(t^2+1)}}{t^2+1}    
    \]
    의 첫 번째 변수에 대한 편도함수가
    \[
    (x, t) \mapsto -2x e^{-x^2(t^2+1)}    
    \]
    로 주어지고 이는 \(\RR \times [0, 1]\)에서 연속이므로, \(g\)가 실수 전체에서 미분가능하고
    \[
    g'(x) = \int_0^1 (-2x e^{-x^2(t^2+1)}) dt = -2e^{-x^2} \int_0^1 xe^{-(xt)^2} dt = -2e^{-x^2} \int_0^x e^{-u^2} du
    \]
    이다. 이제 함수 \(f + g\)의 도함수를 계산하면, 임의의 \(x \in \RR\)에 대해
    \[
    f'(x) + g'(x) = 2e^{-x^2} \int_0^x e^{-t^2}dt -2e^{-x^2} \int_0^x e^{-u^2} du = 0    
    \]
    이므로 \(f + g\)는 상수함수이다. 한편
    \[
    f(0) + g(0) = \int_0^1 \frac{dt}{t^2 + 1} = \frac{\pi}{4}    
    \]
    이므로 \(f+g = \pi/4\)를 얻는다. 그런데
    \[
    \abs{g(x)} = \int_0^1 \frac{e^{-x^2(t^2+1)}}{t^2+1} dt \le \int_0^1 e^{-x^2} dt = e^{-x^2} dx \xrightarrow[]{x \to \infty} 0
    \]
    이므로
    \[
    \lim_{x \to \infty} f(x) = \lim_{x \to \infty} \int_0^x \sqbracket{\int_0^x e^{-t^2} dt}^2 = \frac{\pi}{4}    
    \]
    이고, 따라서
    \[
    \int_0^{\infty} e^{-t^2} dt = \frac{\sqrt{\pi}}{2}    
    \]
    이다.
\end{ex}

\begin{thm}[푸비니 정리: 연속함수의 리만적분]
    연속함수 \(f : [a, b] \times [c, d] \subseteq \RR^2 \to \RR\)에 대하여
    \[
    \int_a^b \int_c^d f(x, y) dydx = \int_c^d \int_a^b f(x,y)dxdy    
    \]
    이다.
\end{thm}
\begin{proof}
    \(x \in \int_c^d f(x, y) dy\)가 \([a, b]\)에서 연속이므로 주어진 식의 좌변이 잘 정의되고, 마찬가지로 우변도 잘 정의된다. 양수 \(\eps > 0\)이 주어졌을 때, \(f\)는 정의역에서 고른연속이므로 다음을 만족하는 양수 \(\delta > 0\)이 존재한다.
    \[
    \norm{(x_1, y_1) - (x_2, y_2)} < \delta \implies \abs{f(x_1, y_1) - f(x_2, y_2)} < \eps \quad (x_1, x_2 \in [a, b], y_1, y_2 \in [c, d])
    \]
    이제 \(P_1 = \{x_0, \ldots, x_m\}, P_2 = \{y_0, \ldots, y_n\}\)이 \(\norm{P_1}, \norm{P_2} < \delta/2\)이도록 하는 \([a, b], [c, d]\) 각각의 분할이라고 하자. 그리고 \(1 \le i \le m, 1 \le i \le j\)에 대하여
    \[
    I_i := \begin{cases}
        [x_{i-1}, x_i) &\textif \ i < m\\
        [x_{i-1}, x_i] &\textif \ i = m
    \end{cases}, \quad
    J_j := \begin{cases}
        [y_{j-1}, y_j) &\textif \ j < n\\
        [y_{j-1}, y_j] &\textif \ j = n
    \end{cases}, \quad Q_{ij} := I_i \times J_j
    \]
    로 정의하고 함수 \(g : [a, b] \times [c, d]  \to \RR\)를
    \[
    g : (x, y) \mapsto \inf_{(s, t) \in Q_{ij}} f(s, t) \quad ((x, y) \in Q_{ij})
    \]
    로 정의하자. 이때
    \[
        L := \int_a^b \int_c^d g(x, y) dydx = \int_c^d \int_a^b g(x,y)dxdy
    \]
    이고, \(\abs{f - g} < \eps\)이므로
    \[
        0 \le \int_a^b \int_c^d f(x, y) dy dx - L = \int_a^b \int_c^d {f(x, y) - g(x, y)} dydx < (b-a)(d-c)\eps
    \]
    이고, 마찬가지 방법으로
    \[
        0 \le {\int_c^d \int_a^b f(x, y) dxdy - L} < (b-a)(d-c)\eps
    \]
    이다. 그러므로
    \[
        \abs{\int_a^b \int_c^d f(x, y) dy dx - \int_c^d \int_a^b f(x, y) dxdy} < (b-a)(d-c)\eps
    \]
    이다. \(\eps > 0\)이 임의의 양수였으므로 원하는 등식을 얻는다.
\end{proof}

이제 \(t\)의 범위가 유계닫힌구간이 아닌 경우를 보자. 

\begin{ex}
    고전적인 반례 하나를 소개하겠다. \(g : \RR_{\ge 0} \to \RR\)가 연속함수이고, \(\int_0^{\infty} g = 1\)이라 하자. 이때 함수 \(f : [0, 1] \times [0, \infty) \to \RR\)를
    \[
    f : (x, t) \mapsto x g(xt)    
    \]
    로 정의하고, 함수 \(\phi : [0, 1] \to \RR\)를
    \[
    \phi : x \mapsto \int_0^\infty f(x, t) dt
    \]
    라 하자. 그러면
    \[
    \phi(0) = \int_0^{\infty} f(0, t) dt = 0    
    \]
    이지만 \(t > 0\)에 대하여
    \[
    \phi(t) = \int_0^\infty x g(xt) dt = \int_0^\infty g(u) du = 1
    \]
    이므로 \(\phi\)는 0에서 연속이 아니다.
\end{ex}

이제 논의를 간단히 하기 위해 \([c, d)\) 꼴의 구간만 볼 것인데, 나머지 경우도 똑같은 방법으로 증명할 수 있다.

\begin{thm} \label{14.1.3}
    연속함수 \(f : [a, b] \times [c, d) \to \RR\)에 대하여 [임의의 \(x \in [a, b], t \in [c, d)\)에 대해 \(\abs{f(x, t)} \le g(t)\)]를 만족하는 함수 \(g \in \calR^1([c, d))\)가 존재한다고 하자. 이때 (\ref{eq14.1})\과 같이 정의된 함수 \(\phi : [a, b] \to \RR\)는 잘 정의된 연속함수이다.
\end{thm}
\begin{proof}
    각 \(x \in [a, b]\)에 대해
    \[
       \abs{\phi(x)}  \le \int_c^d \abs{f(x, t)} dt \le \int_c^d g(t) dt < \infty
    \]
    이므로 \(\phi\)는 잘 정의된다. 이제 \([c, d)\) 안에서 \(d\)로 수렴하는 수열 \((d_n)_{n \in \NN}\)을 잡고, 각 \(n \in \NN\)에 대해 \(\phi_n : [a, b] \to \RR\)를
    \begin{equation} \label{eq14.3}
        \phi_n : x \mapsto \int_c^{d_n} f(x, t) dt  
    \end{equation}
    라 하자. 그러면 정리 \ref{14.1.1}에 의해 각 \(\phi_n\)은 연속함수이다. 우리의 주장은 \((\phi_n)_{n \in \NN}\)이 \(\phi\)로 고른수렴한다는 것이다.\\
    임의의 양수 \(\eps > 0\)에 대해, 다음을 만족하는 자연수 \(N\)이 존재한다.
    \[
    n \ge N \implies \int_{d_n}^d g < \eps    
    \]
    이제 \(n \ge N\)이면 임의의 \(x \in [a, b]\)에 대해
    \[
    \abs{\phi_n(x) - \phi(x)} \le \int_{d_n}^d \abs{f(x, t)} dt \le \int_{d_n}^d g(t) dt < \eps
    \]
    이므로, \((\phi_n)_{n \in \NN}\)이 \(\phi\)로 고른수렴한다. 따라서 \(\phi\)는 연속함수이다.
\end{proof}

\begin{thm} \label{14.1.4}
    연속함수 \(f : [a, b] \times [c, d)\)에 대하여 \(f\)의 첫 번째 변수에 대한 편도함수 \(D_1f\)가 \((a, b) \times [c, d)\)에서 존재하고 다음을 만족한다고 하자.
    \begin{enum}
        \item \(\int_c^d f(x_0, t) dt\)가 수렴하는 \(x_0 \in (a, b)\)가 존재한다.
        \item {[임의의 \(x \in (a, b), t \in [c, d)\)에 대해 \(\abs{D_1f(x, t)} \le h(t)\)]}를 만족하는 함수 \(h \in \calR^1([c, d))\)가 존재한다.
    \end{enum}
    이때 (\ref{eq14.1})\과 같이 정의된 함수 \(\phi : [a, b] \to \RR\)는 \([a, b]\)에서 잘 정의된 연속함수이고 \((a, b)\)에서 미분가능하며 임의의 \(x \in (a, b)\)에 대해 (\ref{eq14.2})\가 성립한다.
\end{thm}
\begin{proof}
    \([c, d)\) 안에서 \(d\)로 수렴하는 수열 \((d_n)_{n \in \NN}\)을 잡고, 각 \(n \in \NN\)에 대해 \(\phi_n : [a, b] \to \RR\)를 (\ref{eq14.3})\과 같이 정의하자. 그리고 \(x_0 \in [u, v] \subseteq (a, b)\)인 유계닫힌구간 \([u, v]\)를 잡자. 이제 \([u, v]\)에서 정리 \ref{12.1.1}\을 쓰려고 한다. 각 \(\phi_n\)은 \([u, v]\)에서 미분가능하고 그 도함수가
    \[
    \phi_n'(x) = \int_c^{d_n} D_1f(x, t) dt \quad (x \in [u, v])    
    \]
    로 주어진다. 또 가정에 의해 \((\phi_n(x_0))_{n \in \NN}\)이 어떤 실수로 수렴한다. 마지막으로 \((\phi_n')_{n \in \NN}\)이 \([u, v]\)에서 고른수렴함을 보이면 된다. 양수 \(\eps > 0\)에 대해 다음을 만족하는 자연수 \(N\)이 존재한다.
    \[
    m, n \ge N \implies \abs{\int_{d_n}^{d_m} h} < \eps     
    \]
    이제 \(m, n \ge N, x \in [u, v]\)에 대해
    \[
    \abs{\phi_m'(x) - \phi_n'(x)} \le \abs{\int_{d_n}^{d_m} \abs{D_1f(x, t)} dt} \le \abs{\int_{d_n}^{d_m} h} < \eps    
    \]
    이므로 \((\phi_n')_{n \in \NN}\)이 \([u, v]\)에서 고른수렴하며, 그 극한은
    \[
        x \mapsto \int_c^d D_1f(x, t) dt
    \]
    로 주어진다. 따라서 정리 \ref{12.1.1}에 의해, (\ref{eq14.1})\과 같이 정의된 함수 \(\phi : [u, v] \to \RR\)는 연속함수이고 (\ref{eq14.2})\가 임의의 \(x \in [u, v]\)에 대해 성립한다. 그런데 \([u, v] \subseteq (a, b)\)가 임의의 유계닫힌구간이었으므로, \(\phi\)는 \([a, b]\)에서 연속이고 \((a, b)\)에서 도함수가 (\ref{eq14.2})\로 주어진다.
\end{proof}

바로 다음 절에서 이것에 대한 예시인 감마 함수를 보기로 하자.

\section{감마 함수}

감마함수는 팩토리얼의 해석적 확장이다.

\begin{defn}
    \textbf{감마 함수(gamma function)} \(\Gamma : \RR_{+} \to \RR\)를
    \[
    \Gamma : x \mapsto \int_0^{\infty} e^{-t} t^{x-1} dt
    \]
    로 정의하자.
\end{defn}

\begin{prop}
    감마 함수는 잘 정의된 매끄러운 함수이고, 각 \(n \in \NN\)과 \(x > 0\)에 대하여
    \begin{equation} \label{eq14.4}
        \Gamma^{(n)}(x) = \int_0^\infty e^{-t}t^{x-1}(\log t)^n dt
    \end{equation}
    으로 주어진다.
\end{prop}
\begin{proof}
    유계닫힌구간 \([a, b] \subseteq \RR_+\)를 고정하고 \(f(x, t) := e^{-t}t^{x-1}\)이라 쓰자. 먼저 \(0 < t \le 1\)일 때
    \[
        f(x, t) = e^{-t}t^{x-1} \le t^{a-1}
    \]
    이고, 어떤 양수 \(M > 0\)이 존재하여 \(t > 1\)일 때
    \[
        f(x, t) = e^{-t}t^{x-1} \le M e^{-t/2}
    \]
    이다. 따라서 함수 \(g : \RR_+ \to \RR\)를
    \[
    g : t \mapsto
    \begin{cases}
        t^{a-1} &\textif \ 0 < t \le 1\\
        Me^{-t/2} &\textif \ t > 1
    \end{cases}   
    \]
    로 정의하면 \(g \in \calR^1(\RR_+)\)이고 \(f(x, t) \le g(t)\)이므로 정리 \ref{14.1.3}에 의해 \(\Gamma\)는 \([a, b]\)에서 잘 정의된 연속함수이다. 한편 \(f\)의 첫 번째 변수에 대한 편도함수
    \[
    D_1f : (x, t) \mapsto e^{-t}t^{x-1} \log t    
    \]
    가 \((a, b) \times \RR_+\)에서 연속이다. 양수 \(M' > 0\)이 존재하여 \(0 < t \le 1\)일 때
    \[
        \abs{D_1f(x, t)} = e^{-t} t^{x-1} \abs{\log t} \le t^{a-1} \abs{\log t} \le t^{a-1} \cdot M' t^{-a/2} = M't^{a/2-1}
    \]
    이고, 양수 \(M'' > 0\)이 존재하여 \(t > 1\)일 때
    \[
        \abs{D_1f(x, t)} = e^{-t} t^{x-1} \log t \le e^{-t} \cdot M''e^{t/2} = M''e^{-t/2}
    \]
    이다. 따라서 함수 \(h : \RR_+ \to \RR\)를
    \[
    h : t \mapsto
    \begin{cases}
        M't^{a/2-1} &\textif \ 0 < t \le 1\\
        M''e^{-t/2} &\textif \ t > 1
    \end{cases}   
    \]
    로 정의하면 \(h \in \calR^1(\RR_+)\)이고 \(\abs{D_1f(x, t)} \le h(t)\)이므로 정리 \ref{14.1.4}에 의해 \(\Gamma\)는 \((a, b)\)에서 미분가능하고 (\ref{eq14.4})\가 \(n = 1, x \in (a, b)\)에 대해 성립한다. 귀납적으로 모든 자연수 \(n\)에 대해 (\ref{eq14.4})\가 \(x \in (a, b)\)에 대해 성립함을 보일 수 있다. 이제 \([a, b] \subseteq \RR_+\)가 임의의 유계닫힌구간이었으므로 \(\Gamma \in C^\infty(\RR_+)\)이고 임의의 \(x > 0\)에 대해 (\ref{eq14.4})\가 성립한다.
\end{proof}

구간 \(I \subseteq \RR\)에서 정의되고 양의 값만을 가지는 함수 \(f : I \to \RR_+\)에 대하여, 함수 \(\log f : I \to \RR\)가 볼록함수이면 \(f\)를 \textbf{로그 볼록함수(log convex function)}이라고 한다. 로그 볼록은 볼록보다 강한 조건이다. 즉 \(f : I \to \RR_+\)가 로그 볼록함수이면 볼록함수인데, 그 역은 일반적으로 성립하지 않는다. 증명 및 반례는 연습문제로 남긴다.

\begin{prop} \label{14.2.3}
    감마 함수 \(\Gamma\)는 다음과 같은 성질을 만족한다.
    \begin{enum}
        \item \(\Gamma(1) = 1\).
        \item 모든 \(x > 0\)에 대해 \(\Gamma(x+1) = x\Gamma(x)\).
        \item \(\Gamma\)는 로그 볼록함수이다.
    \end{enum}
\end{prop}
\begin{proof}
    \quad

    \begin{enum}
        \item \(\Gamma(1) = \int_0^\infty e^{-t} dt = 1\).
        \item 
        \[
        \Gamma(x+1) = \int_0^\infty e^{-t}t^x dt = \sqbracket{-e^{-t} t^x}_{t=0}^{t=\infty} + \int_0^\infty x e^{-t} t^{x-1} dt = x \Gamma(x).
        \]
        \item \(x, y > 0, \lambda \in (0, 1)\)에 대하여
        \begin{align*}
            \Gamma(\lambda x + (1-\lambda)y) &= \int_0^\infty e^{-t} t^{\lambda x + (1-\lambda)y-1} dt\\
            &= \int_0^\infty (e^{-t} t^{x-1})^\lambda (e^{-t} t^{y-1})^{1-\lambda} dt
        \end{align*}
        이다. 이제 \(p = 1/\lambda, q = 1/(1-\lambda)\)로 두고 횔더 부등식을 적용하면
        \[
            \Gamma(\lambda x + (1-\lambda)y) \le \sqbracket{\int_0^\infty e^{-t} t^{x-1} dt}^{\lambda} \sqbracket{\int_0^\infty e^{-t} t^{y-1} dt}^{1-\lambda}
        \]
        를 얻는다. 양변에 \(\log\)를 취하면
        \[
        \log \Gamma (\lambda x + (1-\lambda)y) \le \lambda \log \Gamma(x) + (1-\lambda) \log \Gamma(y)
        \]
        를 얻는다.
    \end{enum}
\end{proof}

\begin{cor}
    \(n \in \ZZ_+\)에 대해 \(\Gamma(n) = (n-1)!\)이다.
\end{cor}
\begin{proof}
    명제 \ref{14.2.3}의 (a)와 (b)에 의해
\end{proof}

감마 함수의 놀라운 점은 명제 \ref{14.2.3}의 역이 성립한다는 사실이다.

\begin{thm} \label{14.2.5}
    함수 \(F : \RR_+ \to \RR_+\)가 다음과 같은 성질을 만족한다고 하자.
    \begin{enum}
        \item \(F(1) = 1\).
        \item 모든 \(x > 0\)에 대해 \(F(x+1) = xF(x)\).
        \item \(F\)는 로그 볼록함수이다.
    \end{enum}
    이때 \(\RR_+\)에서 \(F = \Gamma\)이다.
\end{thm}
\begin{proof}
    (a)와 (b)에 의해, 구간 \((0, 1)\)에서 \(F = \Gamma\)인 것만 보이면 충분하다. 또 \(\Gamma\)가 (a)-(c)를 만족하는 것을 알고 있으므로 구간 \((0, 1)\)에서 \(F\)의 유일성을 보이면 충분하다. \(G = \log F\)로 두면, \(G\)가 볼록함수이므로 각 \(n \in \ZZ_+, x \in (0, 1)\)에 대해 다음이 성립한다.
    \[
    \frac{G(n+1) - G(n)}{1} \le \frac{G(n+1+x) - G(n+1)}{x} \le \frac{G(n+2)-G(n+1)}{1}
    \]
    이때 각 변을 정리하면
    \[
    \log n \le \frac{1}{x}\sqbracket{\log F(x) + \log \prod_{k=0}^n (k+x) - \log n!} \le \log (n+1)    
    \]
    이다. 각 변에 \(x\)를 곱하고 \(x \log n\)을 빼면,
    \[
    0 \le \log F(x) - \log \frac{n! \cdot n^x}{\prod_{k=0}^n (k+x)} \le x \log \frac{n+1}{n}
    \]
    이다. 그런데
    \[
    \lim_{n \to \infty} x \log \frac{n+1}{n}= 0
    \]
    이므로
    \[
    \log F(x) =  \lim_{n \to \infty} \log \frac{n! \cdot n^x}{\prod_{k=0}^n (k+x)}  
    \]
    이고, \(t \mapsto e^t\)는 연속함수이므로
    \[
    F(x) = \lim_{n \to \infty} \frac{n! \cdot n^x}{\prod_{k=0}^n (k+x)}       
    \]
    이다. 따라서 (a)-(c)를 만족하는 \(F\)가 구간 \((0, 1)\)에서 유일함이 증명되었다.
\end{proof}

\begin{cor}
    \(x \in (0, 1)\)에 대하여
    \[
    \Gamma(x) =   \lim_{n \to \infty} \frac{n! \cdot n^x}{\prod_{k=0}^n (k+x)}      
    \]
    이다.
\end{cor}

\begin{defn}
    \textbf{베타 함수(beta function)} \(B : \RR_+ \times \RR_+ \to \RR\)를
    \[
    B : (x, y) \mapsto \int_0^1 t^{x-1} (1-t)^{y-1} dt    
    \]
    로 정의하자.
\end{defn}

\begin{thm}
    \(x, y > 0\)에 대하여
    \[
    B(x, y) = \frac{\Gamma(x)\Gamma(y)}{\Gamma(x+y)}    
    \]
    이다.
\end{thm}
\begin{proof}
    \(y > 0\)을 고정하고, 함수
    \[
    F : \RR_+ \to \RR_+ : x \mapsto \frac{B(x, y) \Gamma(x+y)}{\Gamma(y)}    
    \]
    가 정리 \ref{14.2.5}의 (a)-(c)를 만족함을 보이면 된다.
    \begin{enum}
        \item
        \[
        F(1) = \frac{B(1, y) \Gamma(1+y)}{\Gamma(y)} = y \int_0^1 (1-t)^{y-1} dt = 1.    
        \]
        \item
        \begin{align*}
            B(x+1, y) &= \int_0^1 t^x(1-t)^{y-1} dt\\
            &= \int_0^1 \paren{\frac{t}{1-t}}^x (1-t)^{x+y-1} dt\\
            &= \sqbracket{-\paren{\frac{t}{1-t}}^x \paren{\frac{(1-t)^{x+y}}{x+y}}}_0^1 + \int_0^1 x \paren{\frac{t}{1-t}}^{x-1} \frac{1}{(1-t)^2} \cdot \frac{(1-t)^{x+y}}{x+y} dt\\
            &= \frac{x}{x+y} \int_0^1 t^{x-1} (1-t)^{y-1} dt = \frac{x}{x+y} B(x, y)
        \end{align*}
        이므로
        \[
        F(x+1) = \frac{B(x+1, y) \Gamma(x+1+y)}{\Gamma(y)} = \frac{x}{x+y} \frac{B(x, y) \cdot (x+y)\Gamma(x+y)}{\Gamma(y)} = x F(x)
        \]
        이다.
        \item \(\log F(x) = \log B(x, y) + \log \Gamma(x+y) - \log \Gamma(y)\)인데, \(x \mapsto + \log \Gamma(x+y)\)는 볼록함수이고 \(\log \Gamma(y)\)는 \(x\)에 관하여 상수이므로 \(\log F\)가 볼록함수임을 보이기 위해 \(x \mapsto \log B(x, y)\)가 볼록함수임을 보이면 충분하다. \(u, v > 0, \lambda \in (0, 1)\)에 대해
        \begin{align*}
            B(\lambda u + (1-\lambda)v, y) &= \int_0^1 t^{\lambda u + (1-\lambda)v-1} (1-t)^{y-1} dt\\
            &= \int_0^1 (t^{u-1} (1-t)^{y-1})^\lambda (t^{v-1} (1-t)^{y-1})^{1-\lambda} dt
        \end{align*}
        이다. \(p = 1/\lambda, q = 1/(1-\lambda)\)로 두고 횔더 부등식을 적용하면
        \[
            B(\lambda u + (1-\lambda)v, y) \le \sqbracket{\int_0^1 t^{u-1} (1-t)^{y-1} dt}^\lambda \sqbracket{\int_0^1 t^{v-1} (1-t)^{y-1} dt}^{1-\lambda}
        \]
        이고, 양변에 \(\log\)를 취하면
        \[
        \log B(\lambda u + (1-\lambda)v, y) \le \lambda \log B(u, y) + (1-\lambda) B(v, y)  
        \]
        를 얻는다.
    \end{enum}
\end{proof}

\(x = y = 1/2\)인 경우에 위 정리를 생각하면
\begin{align*}
    \Gamma(1/2)^2 &= B(1/2, 1/2)\\
    &= \int_0^1 \frac{dt}{\sqrt{t(1-t)}}\\
    &= \int_0^{\pi/2} \frac{2\sin\theta\cos\theta d\theta}{\sqrt{\sin^2\theta(1-\sin^2\theta)}}\\
    &= \int_0^{\pi/2} 2 d\theta = \pi
\end{align*}
이므로 \(\Gamma(1/2) = \sqrt{\pi}\)이다.

\begin{cor}
    \(x > 0\)에 대하여
    \[
    \Gamma(x) = \frac{2^{x-1}}{\sqrt{\pi}} \Gamma \paren{\frac{x}{2}} \Gamma \paren{\frac{x+1}{2}}    
    \]
    이다.
\end{cor}
\begin{proof}
    주어진 식의 우변을 \(F(x)\)라 할 때 \(F\)가 정리 \ref{14.2.5}의 (a)-(c)를 만족하는 것만 보이면 된다.
    \begin{enum}
        \item
        \[
        F(1) = \frac{1}{\sqrt{\pi}} \Gamma \paren{\frac{1}{2}} \Gamma (1) = 1.  
        \]
        \item
        \[
        F(x+1) = \frac{2^{x}}{\sqrt{\pi}} \Gamma\paren{\frac{x+1}{2}} \Gamma \paren{\frac{x}{2} + 1} = \frac{2^{x}}{\sqrt{\pi}} \cdot \frac{x}{2} \Gamma \paren{\frac{x}{2}} \Gamma \paren{\frac{x+1}{2}} = xF(x). 
        \]
        \item \(\log F(x) = (\log 2)(x-1) - \log \sqrt{\pi} + \log \Gamma (x/2) + \log \Gamma((x+1)/2)\)인데, 우변이 \(x\)에 관한 볼록함수들의 합이므로 좌변도 \(x\)에 관한 볼록함수이다.
    \end{enum}
\end{proof}

\begin{ex}
    감마 함수의 정의
    \[
    \Gamma(x) = \int_0^\infty e^{-t} t^{x-1} dt \quad (x > 0)    
    \]
    에서 \(t = s^2 / 2\)로 치환하면
    \[
        \Gamma(x) = \frac{1}{2^{x-1}} \int_0^\infty s^{2x-1} e^{-s^2/2} ds
    \]
    를 얻는다. 따라서 자연수 \(n \in \NN\)에 대해
    \begin{equation} \label{eq14.5}
        \Gamma\paren{n + \frac{1}{2}} = \frac{1}{2^{n-1/2}} \int_0^\infty s^{2n} e^{-s^2/2} ds = \frac{1}{2^{n +1/2}} \int_{\RR} s^{2n} e^{-s^2/2} ds 
    \end{equation}
    이다. 이제 \(n = 0\)을 대입하면
    \[
    \frac{1}{\sqrt{2}} \int_{\RR} e^{-s^2/2} ds = \sqrt{\pi}    
    \]
    이므로,
    \[
    f(s) = \frac{1}{\sqrt{2\pi}} e^{-s^2/2} \quad (s \in \RR)   
    \]
    로 정의하였을 때
    \[
    \int_{-\infty}^\infty f(s) ds = 1    
    \]
    이다. 이러한 \(f\)가 바로 1차원 표준정규분포의 확률밀도함수이다. (\ref{eq14.5})\를 이용하면, 1차원 표준정규분포를 따르는 확률변수 \(Z\)와 양의 정수 \(n\)에 대하여 \(\mathrm{E}(Z^{2n})\)의 값이
    \begin{align*}
        \mathrm{E}(Z^{2n}) = \int_{\RR} s^{2n} \cdot \frac{1}{\sqrt{2\pi}} e^{-s^2/2} ds = \frac{2^n}{\sqrt{\pi}} \Gamma \paren{n + \frac{1}{2}} = \prod_{k=0}^{n-1} (2k+1)
    \end{align*}
    로 계산된다. 특히 \(n = 1\)일 때 \(\mathrm{E}(Z^2) = 1\)이므로, \(\mathrm{Var}(Z) = \mathrm{E}(Z^2) - (\mathrm{E}(Z))^2 = 1\)인 것도 증명된다.
\end{ex}

이제 미적분학에서 증명하지 않고 사용한 스털링 근사(Stirling's approximation)를 증명하자.

\begin{thm}[스털링 근사]
    \[
        \lim_{x \to \infty} \frac{\Gamma(x+1)}{(x/e)^x \sqrt{2\pi x}} = 1
    \]
    이다. 특히
    \[
        \lim_{n \to \infty} \frac{n!}{(n/e)^n \sqrt{2\pi n}} = 1
    \]
    이다.
\end{thm}
\begin{proof}
    \(x > 0\)을 고정하고, 감마 함수의 정의에서 \(t = x(1+u)\)로 치환하자. 그러면
    \[
        \Gamma(x) = \int_{-1}^\infty e^{-x(1+u)} [x(1+u)]^{x-1} xdu = x^x e^{-x} \int_{-1}^\infty [(1+u)e^{-u}]^x du
    \]
    이므로
    \begin{equation} \label{eq14.6}
        \Gamma(x+1) = x\Gamma(x) = x(x/e)^x \int_{-1}^\infty [(1+u)e^{-u}]^x du  
    \end{equation}
    이다. 한편 \((-1, \infty)\)에서 
    \[
        h(u) = \frac{2}{u^2} [u - \log (1+u)]
    \]
    로 연속적으로 정의하면(물론 \(h(0) = 1\)) \(h\)는 \((-1, \infty)\)에서 단조감소한다. 이제 (\ref{eq14.6})에서 \(u = s\sqrt{2/x}\)로 치환하면,
    \begin{align*}
        \sqbracket{(1 + s\sqrt{2/x}) e^{-s\sqrt{2/x}}}^x &= \exp \sqbracket{-sx\sqrt{2/x} + x \log (1 + s\sqrt{2/x})}\\
        &= \exp \sqbracket{-s^2 \paren{\frac{2}{s \sqrt{2/x}} - \frac{2}{(s \sqrt{2/x})^2} \log (1 + s\sqrt{2/x})}}\\
        &= \exp[-s^2 h(s \sqrt{2/x})]
    \end{align*}
    이므로
    \[
    \frac{\Gamma(x+1)}{(x/e)^x \sqrt{2x}} = \int_\RR \psi_x(s) ds, \quad \psi_x(s) = \begin{cases}
        \exp[-s^2 h(s \sqrt{2/x})] &\textif \ s > -\sqrt{x/2}\\
        0 &\textif \ s \le -\sqrt{x/2}
    \end{cases}   
    \]
    이다. 이제
    \[
    \lim_{x \to \infty} \int_\RR \psi_x(s) ds = \sqrt{\pi}    
    \]
    임을 보이면 증명이 끝난다. 그런데 각 \(s \in \RR\)에 대해
    \[
    \lim_{x \to \infty} \psi_x(s) = e^{-s^2}    
    \]
    이고 \(\int_{\RR} e^{-s^2} ds = \sqrt{\pi}\)이므로 이는 적분과 극한의 순서를 바꿀 수 있는지에 관한 문제이다. \(\psi(s) = e^{-s^2}\)로 두고 정리 \ref{13.1.14}\를 이용하자.\\
    \textbf{Claim 1.} 임의의 \(R > 0\)에 대해 \([-R, R]\)에서 \(x \to \infty\)일 때 \(\psi_x\)는 \(\psi\)로 고른수렴한다.
    \begin{proof}[Proof of Claim 1.]
        먼저 \(x \ge 2R^2\)이면 \(- \sqrt{x/2} \le -R\)이므로 \(s \in [-R, R]\)에 대하여 \(\psi_x(s) = \exp[-s^2 h(s \sqrt{2/x})]\)이다. 이제 \(h\)가 단조감소한다는 사실을 이용하면 \(x \ge 2R^2, s \in [-R, R]\)에서
        \begin{align*}
            \abs{\psi_x(s) - \psi(s)} &= \exp(-s^2) (\exp[s^2(1 - h(s\sqrt{x/2}))] -1)\\
            &\le \exp[R^2(1 - h(R/\sqrt{2/x}))] - 1 \xrightarrow[]{x \to \infty} 0
        \end{align*}
        이다. 따라서 \(x \to \infty\)일 때 \(\psi_x\)는 \(\psi\)로 \([-R, R]\)에서 고른수렴한다.
    \end{proof}
    \noindent\textbf{Claim 2.} [모든 \(x > 1\)에 대해 \(\psi_x \le g\)]인 \(g \in \calR^1(\RR)\)가 존재한다.
    \begin{proof}
        \(s < 0\)일 때 \(h(s \sqrt{2/x}) > h(0) = 1\)이므로 \(\psi_x(s) < e^{-s^2}\)이다. 그리고 \(s \ge 0, x > 1\)일 때 \(\psi_x(s) \le \psi_1(s)\)도 성립한다. 이제
        \[
        g(s) :=
        \begin{cases}
            e^{-s^2} &\textif \ s < 0\\
            \psi_1(s) &\textif \ s \ge 0
        \end{cases}    
        \]
        으로 두고 \(g \in \calR^1(\RR)\)인 것만 보이면 된다. \(\int_{-\infty}^0 e^{-s^2} ds < \infty\)인 것은 알고 있으므로
        \[
        \int_0^\infty \psi_1(s) ds = \int_0^\infty \exp[-s^2h(s\sqrt{2})] ds  < \infty    
        \]
        인 것을 보이면 충분하다. 그런데 \(s \ge 0\)에서
        \[
            -s^2h(s\sqrt{2}) = -s \sqrt{2} + \log (1 + s\sqrt{2})
        \]
        이고, 어떤 양수 \(M > 0\)이 존재하여 \(s > M\)이면
        \[
            \log (1 + s\sqrt{2}) < \frac{\sqrt{2}}{2}s
        \]
        이도록 할 수 있다. 이제
        \[
        \int_0^\infty \psi_1(s) ds < \int_0^M \exp[-s^2h(s\sqrt{2})] ds + \int_M^\infty \exp(-s/\sqrt{2}) ds < \infty
        \]
        이므로, \(g \in \calR^1(\RR)\)이다.
    \end{proof}
    이제 정리 \ref{13.1.14}에 의해
    \[
        \lim_{x \to \infty} \int_\RR \psi_x(s) ds = \int_{\RR} e^{-s^2} ds = \sqrt{\pi}
    \]
    가 증명된다.
\end{proof}

\section{합성곱}

\begin{defn}
    함수 \(f, g : \RR \to \CC\)에 대하여,
    \[
    f * g : \RR \to \CC : x \mapsto \int_\RR f(y)g(x-y) dy    
    \]
    가 잘 정의될 때 \(f * g\)를 \(f\)와 \(g\)의 \textbf{합성곱(convolution)}이라 한다.
\end{defn}

\begin{prop}
    함수 \(f, g, h : \RR \to \CC\)에 대하여 다음이 성립한다. (단, 아래의 합성곱들이 모두 잘 정의되는 경우에.)
    \begin{enum}
        \item \(f * g = g * f\).
        \item \(c \in \CC\)에 대하여 \((f + cg) * h = f * h + c(g * h)\).
    \end{enum}
\end{prop}
\begin{proof}
    연습문제로 남긴다.
\end{proof}

이 절이 끝날 때까지 아무 말이 없으면 \(p, q \in [1, \infty]\)는 켤레 지수이다.

\begin{prop}
    \(f \in \calR^p(\RR), g \in \calR^q(\RR)\)에 대하여 \(f * g \in \calR^\infty(\RR)\)이고 \(\norm{f * g}_\infty \le \norm{f}_p \norm{g}_q\).
\end{prop}
\begin{proof}
    임의의 \(x \in \RR\)에 대하여, 횔더 부등식에 의해
    \[
    \abs{(f * g)(x)} \le \int_\RR \abs{f(y) g(x -y)}dy \le \norm{f}_p \norm{g(x - \cdot)}_q = \norm{f}_p \norm{g}_q
    \]
    이므로 \(\norm{f * g}_\infty \le \norm{f}_p \norm{g}_q\)이다.
\end{proof}

\begin{cor}
    \(f \in \calR^p(\RR)\)를 고정했을 때
    \[
    \calR^q(\RR) \to \calR^\infty(\RR) : g \mapsto f * g    
    \]
    는 립시츠 연속이다. 마찬가지로 \(g \in \calR^q(\RR)\)를 고정했을 때
    \[
    \calR^p(\RR) \to \calR^\infty(\RR) : f \mapsto f * g
    \]
    도 립시츠 연속이다.
\end{cor}

합성곱과 연속성은 깊은 관련이 있다.

\begin{prop}
    \(f \in C_c(\RR)\)에 대하여 다음이 성립한다.
    \begin{enum}
        \item \(g \in \calR^\infty(\RR)\)이면 \(f * g\)는 실수 전체에서 고른연속이다.
        \item \(g \in \calR^\infty(\RR)\)가 옹골 지지를 가지면 \(f * g \in C_c(\RR)\)이다.
    \end{enum}
\end{prop}
\begin{proof}
    \quad

    \begin{enum}
        \item
        \(\mathrm{supp} f \subseteq [-R, R]\)인 양수 \(R > 0\)을 잡자. 이때
        \begin{align*}
            \abs{(f*g)(x+h) - (f*g)(x)} &\le \int_{\RR} \abs{f(x+t-y) - f(x-y)} \abs{g(y)} dy\\
            &\le \norm{g}_{\infty} \int_{\RR} \abs{f(x+t-y) - f(x-y)} dy    
        \end{align*}
        이다. \(f\)가 고른연속이므로 임의의 \(\eps > 0\)에 대해 다음을 만족하는 \(0 < \delta < R\)가 존재한다.
        \[
        \abs{h} < \delta \implies \abs{f(x+h) - f(x)} < \eps \quad (x \in \RR)    
        \]
        이제 \(\abs{h} < \delta\)이면 임의의 \(x \in \RR\)에 대해
        \begin{align*}
            \abs{(f*g)(x+h) - (f*g)(x)} &\le \norm{g}_{\infty} \int_{\RR} \abs{f(x+t-y) - f(x-y)} dy    \\
            &= \norm{g}_{\infty} \int_{x-3R}^{x+3R} \abs{f(x+t-y) - f(x-y)} dy\\
            &< 6R\norm{g}_{\infty}\eps
        \end{align*}
        이므로 \(f*g\)가 고른연속이다.
        \item 양수 \(R > 0\)에 대해 \(\mathrm{supp}f, \mathrm{supp}g \subseteq [-R, R]\)라 하자. 이때
        \[
        (f*g)(x) = \int_{\RR} f(y) g(x - y) dy = \int_{-R}^R f(y) g(x - y) dy    
        \]
        이므로, \(\abs{x} > 2R\)이면 \((f*g)(x) = 0\)이다. 따라서 \(f*g \in C_c(\RR)\)이다.
    \end{enum}
\end{proof}

\begin{thm} \label{14.3.6}
    \(f \in \calR^p(\RR), g \in \calR^q(\RR)\)에 대하여 다음이 성립한다.
    \begin{enum}
        \item \(p \in (1, \infty)\)이면 \(f * g \in C_0(\RR)\)이다.
        \item \(p = 1\)이고 \(g \in C_0(\RR)\)이면 \(f * g \in C_0(\RR)\)이다.
    \end{enum}
\end{thm}
\begin{proof}
    \quad

    \begin{enum}
        \item 먼저 \(h_1 \in C_c(\RR)\)를 고정하면,
        \[
            (C_c(\RR), \norm{\cdot}_q) \to (C_c(\RR), \norm{\cdot}_\infty) : h_2 \to h_1 * h_2
        \]
        는 립시츠 연속이므로 고른연속이다. 그런데 \(C_c(\RR)\)이 \(\calR^q(\RR)\)에서 조밀하므로, 정리 \ref{7.4.7}에 의해 \(\calR^q(\RR)\)로의 유일한 연속적 확장이 존재한다. 또 \(C_c(\RR)\)의 \(\calR^\infty(\RR)\)에서의 닫힘은 \(C_0(\RR)\)이므로, 연속함수의 성질에 의해
        \[
            \calR^q(\RR) \to C_0(\RR) : g \mapsto h_1 * g
        \]
        가 잘 정의된다. 이제 반대로 \(g \in \calR^q(\RR)\)를 고정하면
        \[
            (C_c(\RR), \norm{\cdot}_p) \to C_0(\RR) : h_1 \mapsto h_1 * g
        \]
        가 립시츠 연속이므로 고른연속이고, 같은 방법으로 연속적 확장을 통해
        \[
            \calR^p(\RR) \to C_0(\RR) : f \mapsto f * g
        \]
        가 잘 정의됨을 알 수 있다.
    \end{enum}
\end{proof}

정리 \ref{14.3.6}의 (b)에서 \(g\)가 연속함수가 아니어도, 옹골 지지를 가지면 같은 결론을 얻는다.

\begin{thm} \label{14.3.7}
    \(f \in \calR^1(\RR), g \in \calR^\infty(\RR)\)에 대하여 다음이 성립한다.
    \begin{enum}
        \item \(f * g \in C_b(\RR)\).
        \item \(g\)가 옹골 지지를 가지면 \(f * g \in C_0(\RR)\)이다.
    \end{enum}
\end{thm}
\begin{proof}
    \quad

    \begin{enum}
        \item \(h \in C_c(\RR), g \in \calR^\infty(\RR)\)에 대하여 \(h * g\)가 연속함수이고 \(h * g \in \calR^\infty(\RR)\)이므로 \(h * g \in C_b(\RR)\)이다. 즉 \(g \in \calR^\infty(\RR)\)을 고정했을 때
        \[
        (C_c(\RR), \norm{\cdot}_1) \to C_b(\RR) : h \mapsto h * g
        \]
        는 립시츠 연속이므로 고른연속이다. 그런데 \(C_b(\RR)\)는 이미 \(\calR^\infty(\RR)\)의 닫힌집합이므로, 연속적 확장을 통해 정의역을 \(\calR^1(\RR)\)로 확장하여도
        \[
        \calR^1(\RR) \to C_b(\RR) : f \mapsto f * g    
        \]
        가 잘 정의된다.
        \item 옹골 지지를 가지는 \(g \in \calR^\infty(\RR)\)를 고정하면
        \[
        (C_c(\RR), \norm{\cdot}_1) \to (C_c(\RR), \norm{\cdot}_\infty) : h \mapsto h * g
        \]
        는 립시츠 연속이므로 고른연속이다. 따라서 연속적 확장인
        \[
            \calR^1(\RR) \to C_0(\RR) : f \mapsto f * g
        \]
        가 잘 정의된다.
    \end{enum}
\end{proof}

\begin{thm} \label{14.3.8}
    \(f \in C_b(\RR)\)이고, \(\calR^1(\RR)\) 안의 함수열 \((g_n)_{n \in \NN}\)이 다음을 만족한다고 하자.
    \begin{enum}
        \item 실수열 \((\norm{g_n}_1)_{n \in \NN}\)이 유계이다.
        \item 각 \(n \in \NN\)에 대해 \(\int_{\RR} g_n = 1\).
        \item 임의의 양수 \(\delta > 0\)에 대해
        \[
        \lim_{n} \int_{\abs{x} \ge \delta} \abs{g_n} \to 0    
        \]
        이다.
    \end{enum}
    이때 \((f * g_n)_{n \in \NN}\)이 \(f\)로 점별수렴한다. 만약 \(f \in C_b(\RR)\)가 고른연속이면 \((f * g_n)_{n \in \NN}\)이 \(f\)로 고른수렴한다.
\end{thm}
\begin{proof}
    (a)에 의해, [각 \(n\)에 대해 \(\norm{g_n}_1 \le M\)]인 양수 \(M > 0\)이 존재한다. \(x \in \RR\)를 고정하고, 양수 \(\eps > 0\)이 주어졌다고 하자. \(f\)가 연속이므로 다음을 만족하는 양수 \(\delta > 0\)이 존재한다.
    \[
    \abs{x - y} < \delta \implies \abs{f(x) - f(y)} < \frac{\eps}{M} \quad  (y \in \RR)
    \]
    이제 \(\int_\RR g_n = 1\)이므로
    \begin{align*}
        \abs{(f * g_n)(x) - f(x)} &= \abs{\int_\RR [f(x-y) - f(x)]g_n(y) dy}\\
        &\le \int_{\abs{y} < \delta} \abs{f(x-y) - f(x)}\abs{g_n(y)} dy + \int_{\abs{y} \ge \delta} \abs{f(x-y) - f(x)}\abs{g_n(y)} dy\\
        &< \eps + 2\norm{f}_{\infty} \int_{\abs{y} \ge \delta} \abs{g_n(y)} dy
    \end{align*}
    이고, \(n \to \infty\)를 취하면
    \[
    \limsup_n \abs{(f * g_n)(x) - f(x)} \le \eps
    \]
    를 얻는다. \(\eps > 0\)이 임의의 양수였으므로 \(((f * g_n)(x))_{n \in \NN}\)이 \(f(x)\)로 수렴한다. 따라서 \((f * g_n)_{n \in \NN}\)이 \(f\)로 점별수렴한다.\\
    이제 \(f\)가 실수 전체에서 고른연속이라 하자. 그러면 양수 \(\eps > 0\)에 대해 다음을 만족하는 양수 \(\delta > 0\)이 존재한다.
    \[
    \abs{x - y} < \delta \implies \abs{f(x) - f(y)} < \frac{\eps}{M} \quad  (x, y \in \RR)
    \]
    이제 위와 같은 방법으로, 임의의 \(x \in \RR\)에 대해
    \[
        \abs{(f * g_n)(x) - f(x)} \le \eps + 2\norm{f}_{\infty} \int_{\abs{y} \ge \delta} \abs{g_n(y)} dy
    \]
    임을 얻는다. 따라서 \((f * g_n)_{n \in \NN}\)이 \(f\)로 고른수렴한다.
\end{proof}

이제 두 \(f, g \in \calR^1(\RR)\)에 대하여 \(f * g\)를 생각해보자.
\begin{align*}
    \norm{f*g}_1 &= \int_{\RR} \abs{\int_{\RR} f(y) g(x - y) dy}dx\\
    &\le \int_{\RR} \int_{\RR} \abs{f(y) g(x - y)} dy dx\\
    &\stackrel{?}{=} \int_{\RR} \int_{\RR} \abs{f(y) g(x - y)} dxdy\\
    &= \int_{\RR} \abs{f(y)} \norm{g}_1 dy = \norm{f}_1 \norm{g}_1
\end{align*}
위 식의 세 번째 줄의 \(\stackrel{?}{=}\)가 성립한다면, 즉 두 특이적분의 순서를 바꿀 수 있다면 \(f * g \in \calR^1(\RR)\)이고 \(\norm{f*g}_1 \le \norm{f}_1 \norm{g}_1\)인데, 실제로 가능하다. 증명은 실해석학에서 한다. (참고: 정리 \ref{13.4.11}\을 다시 읽어보라.)

이제 미분가능성으로 관심을 돌리자. \(k \in \NN \cup \{\infty\}\)에 대하여
\[
C_c^k(\RR) := C_c(\RR) \cap C^k(\RR), \quad C_0^k(\RR) := C_0(\RR) \cap C^k(\RR)
\]
로 쓰자.

\begin{thm}
    \(f \in \calR^p(\RR), g \in C_c^k(\RR)\)에 대하여 \(f * g \in C_0^k(\RR)\)이고, 각 \(i = 0, \ldots, k\)에 대해
    \begin{equation} \label{eq14.7}
        (f*g)^{(i)} = f * g^{(i)}
    \end{equation}
    이다. 특히 \(f\)가 옹골 지지를 가지면 \(f * g \in C_c^k(\RR)\)이다.
\end{thm}
\begin{proof}
    \(g \in \calR^q(\RR)\)이므로 \(f * g \in C_0(\RR)\)이고, \(f\)가 옹골 지지를 가지면 \(f * g \in C_c(\RR)\)인 것도 당연하다. 이제 \(f * g \in C^k(\RR)\)인 것만 보이면 된다. 어떤 \(0 \le i < k\)에 대해 \(f*g \in C_c^i(\RR)\)이고 (\ref{eq14.7})\이 성립한다고 하자.임의의 양수 \(\eps > 0\)이 주어졌을 때, \(g^{(i+1)} \in C_c(\RR)\)가 고른연속이므로 다음을 만족하는 양수 \(\delta > 0\)이 존재한다.
    \[
    \abs{x-y} < \delta \implies \abs{g^{(i+1)}(x) - g^{(i+1)}(y)} < \eps \quad (x, y \in \RR)
    \]
    이제 임의의 \(x \in \RR, 0 < \abs{h} < \delta\)에 대해, \(x\)와 \(x+h\) 사이의 어떤 실수 \(z\)가 존재하여
    \[
    \abs{\frac{1}{h}(g^{(i)}(x+h) - g^{(i)}(x)) - g^{(i+1)}(x)} = \abs{g^{(i+1)}(z) - g^{(i+1)}(x)} < \eps
    \]
    이다. 따라서 \(0 < \abs{h} < \delta\)이면, 가정에 의해
    \begin{align*}
        &\abs{\frac{(f*g)^{(i)}(x+h) - (f*g)^{(i)}(x)}{h} - (f * g^{(i+1)})(x)}\\
        =\,& \abs{\frac{(f * g^{(i)})(x+h) - (f * g^{(i)})(x)}{h} - (f * g^{(i+1)})(x)}\\
        \le\,& \int_{\RR} \abs{\frac{1}{h}(g^{(i)}(x+h-y) - g^{(i)}(x-y)) - g^{(i+1)}(x-y)} \abs{f(y)} dy\\
        \le\,& \norm{\frac{1}{h}(g^{(i)}(x+h-\cdot) - g^{(i)}(x-\cdot)) - g^{(i+1)}(x-\cdot)}_{q} \norm{f}_p
    \end{align*}
    이다. 그런데 어떤 양수 \(R > 0\)이 존재하여
    \[
    \abs{y} > R \implies g^{(i)}(x+h-y) = g^{(i)}(x-y) = g^{(i+1)}(x-y) = 0 \quad (y \in \RR)
    \]
    이게 할 수 있다. \(q = \infty\)이면
    \[
        \norm{\frac{1}{h}(g^{(i)}(x+h-\cdot) - g^{(i)}(x-\cdot)) - g^{(i+1)}(x-\cdot)}_{q} \le \eps
    \]
    이고, \(q \in (1, \infty)\)이면
    \begin{align*}
        &\norm{\frac{1}{h}(g^{(i)}(x+h-\cdot) - g^{(i)}(x-\cdot)) - g^{(i+1)}(x-\cdot)}_{q}^q\\
        =\,& \int_{-R}^R \abs{\frac{1}{h}(g^{(i)}(x+h-y) - g^{(i)}(x-y)) - g^{(i+1)}(x-y)}^q dy\\
        \le\,& 2R\eps^q
    \end{align*}
    이다. 따라서 간단하게 어느 경우든
    \[
        \norm{\frac{1}{h}(g^{(i)}(x+h-\cdot) - g^{(i)}(x-\cdot)) - g^{(i+1)}(x-\cdot)}_{q} \le (2R)^{1/q} \eps
    \]
    로 쓸 수 있으며, 이는 \(0 < \abs{h} < \delta\)일 때 모든 \(x \in \RR\)에 대하여
    \[
        \abs{\frac{(f*g)^{(i)}(x+h) - (f*g)^{(i)}(x)}{h} - (f * g^{(i+1)})(x)} \le (2R)^{1/q} \norm{f}_p \eps
    \]
    임을 의미한다. \((2R)^{1/q} \norm{f}_p\)는 \(\eps\)에 의존하지 않는 양수이므로,
    \[
        (f*g)^{(i+1)} = f * g^{(i+1)}
    \]
    이다. 따라서 귀납법에 의해 각 \(i = 0, \ldots, k\)에 대해 \(f*g \in C_c^i(\RR)\)이고 (\ref{eq14.7})\이 성립한다.
\end{proof}

\begin{cor} \label{14.3.10}
    \(f \in \calR^p(\RR), g \in C_c^\infty(\RR)\)에 대하여 \(f * g \in C_0^\infty(\RR)\)이고, 각 \(i \in \NN\)에 대해 (\ref{eq14.7})\이 성립한다. 특히 \(f\)가 옹골 지지를 가지면 \(f * g \in C_c^\infty(\RR)\)이다.
\end{cor}

\begin{cor}
    \(f \in \calR^1(\RR) \cap C_b(\RR)\)에 대하여, \(f\)로 점별수렴하는 \(C_0^\infty(\RR)\) 안의 함수열이 존재한다. \(f\)가 고른연속이라는 가정을 추가하면, \(f\)로 고른수렴하는 \(C_0^\infty(\RR)\) 안의 함수열이 존재한다.
\end{cor}

\begin{proof}
    함수 \(\psi : \RR \to \RR\)를
    \[
        \psi : x \mapsto
        \begin{cases}
            e^{-1/x} &\textif x > 0\\
            0 &\textif x \le 0
        \end{cases}
    \]
    로 정의하면 \(\psi \in C^\infty(\RR)\)인 것을 정의 \ref{8.3.4} 아래의 예시의 (b)에서 증명하였다. 이제
    \[
        g : \RR \to \RR : x \mapsto c\psi(1+x)\psi(1-x), \quad c := \frac{1}{\int_\RR \psi(1+x)\psi(1-x) dx}
    \]
    로 정의하면 \(g \in C_c^\infty(\RR)\)이고 \(\int_\RR g = 1\)이다. 그리고 각 \(n \in \ZZ_+\)에 대하여
    \[
        g_n : \RR \to \RR : x \mapsto n g(nx)
    \]
    로 정의하면 \(g_n \in C_c^\infty(\RR)\)이므로 따름정리 \ref{14.3.10}에 의해 \((f * g_n)_{n \in \ZZ_+}\)은 \(C_0^\infty(\RR)\) 안의 함수열이다. 한편
    \[
    \mathrm{supp} g_n = [-1/n, 1/n], \quad \norm{g_n}_1 = \int_\RR g_n = \int_\RR g = 1
    \]
    이다. 따라서 임의의 \(\delta > 0\)에 대해, \(n < 1/\delta\)이면 \(\abs{x} \ge \delta\)에서 \(g_n(x) = 0\)이므로
    \[
    \lim_{n} \int_{\abs{x} \ge \delta} \abs{g_n} = 0
    \]
    이다. 따라서 정리 \ref{14.3.8}에 의해, \(C_0^\infty(\RR)\) 안의 함수열 \((f * g_n)_{n \in \ZZ_+}\)은 \(f\)로 점별수렴한다. 그리고 \(f\)가 고른연속이라는 가정을 추가하면 \((f * g_n)_{n \in \ZZ_+}\)은 \(f\)로 고른수렴한다.
\end{proof}



\chapter{푸리에 급수}

이 장에서는 독자들이 학부 2학년 수준 선형대수학을 알고 있다고 가정하고 서술할 것이다. 먼저 \(\calR^2([-\pi, \pi])\) 공간에서 푸리에 급수의 의미를 설명하고, 그 결론으로 파르스발의 등식을 제시한다. 다음으로 연속함수의 푸리에 급수가 어떻게 수렴하는지를 볼 것이고, 마지막으로 푸리에 급수와 미분\(\cdot\)적분의 관계를 봄으로써 다양한 급수를 계산할 수 있게 될 것이다.

\section{파르스발의 등식}


\begin{defn}
    내적공간 \(V\)의 부분집합 \(S := \{v_i\}_{i \in I}\)가 \textbf{정규직교집합(orthonormal subset)}이라는 것은
    \[
    \inner{v_i}{v_j} = \delta_{ij} \quad(i, j \in I)    
    \]
    라는 것이다. 정규직교집합 \(S\)에 대하여 \(\gen{S}\)가 \(V\)에서 조밀하면 \(S\)를 \textbf{정규직교기저(orthonormal basis)}라고 한다.
\end{defn}

푸리에 급수에서 우리의 관심사는 \(V\)가 가산 정규직교기저를 가지는 경우이다. 먼저 새로운 정의가 필요하다.

\begin{defn}
    노름공간 \(V, W\)에 대하여 선형사상 \(L : V \to W\)가 \textbf{등장사상(isometry)}이라는 것은 임의의 \(v \in V\)에 대해 \(\norm{Lv} = \norm{v}\)라는 것이다.
\end{defn}

등장사상이 립시츠 연속임은 자명하다. 그리고 등장사상 \(L : V \to W\)에 대해
\[
    v \in \ker L \iff \norm{Lv} = 0 \iff \norm{v} = 0 \iff v = 0
\]
이므로 \(L\)은 단사이다.

\begin{prop}
    \(F\)-내적공간 \(V, W\)와 선형사상 \(L : V \to W\)에 대하여 다음이 동치이다.
    \begin{enum}
        \item \(L\)은 등장사상이다.
        \item \(L\)은 내적을 보존한다. 즉, 임의의 \(v, w \in V\)에 대해 \(\inner{Lv}{Lw}= \inner{v}{w}\)이다.
    \end{enum}
\end{prop}
\begin{proof}
    (b)\(\Rightarrow\)(a)는 자명하므로 (a)\(\Rightarrow\)(b)만 보이면 된다. 먼저 \(F = \RR\)일 때는
    \[
        \inner{L(v+w)}{L(v+w)} = \inner{v+w}{v+w}
    \]
    을 전개하여 \(\inner{Lv}{Lw} = \inner{v}{w}\)을 얻을 수 있으므로 증명이 끝난다. 한편 \(F = \CC\)일 때는 같은 식으로부터
    \[
        \mathrm{Re}\inner{Lv}{Lw} = \mathrm{Re}\inner{v}{w}
    \]
    를 얻는다. 이제
    \[
        \inner{L(v+iw)}{L(v+iw)} = \inner{v+iw}{v+iw}
    \]
    를 전개하면
    \[
        \mathrm{Im}\inner{Lv}{Lw} = \mathrm{Im}\inner{v}{w}
    \]
    를 얻는다. 따라서 증명 끝.
\end{proof}

\begin{thm} \label{15.1.4}
    내적공간 \(V\)의 유한 정규직교집합 \(S := \{v_1, \ldots, v_n\}\)이 주어졌을 때, 임의의 \(w \in V\)에 대하여 어떤 \(v \in \gen{S}\)가 유일하게 존재하여
    \begin{equation} \label{eq15.1}
        \norm{w - v} \le \norm{w - v'} \quad (v' \in \gen{S})
    \end{equation}
    이고, 그러한 \(v\)는
    \begin{equation} \label{eq15.2}
        v = \sum_{k=1}^n \inner{w}{v_k} v_k
    \end{equation}
    이다.
\end{thm}
\begin{proof}
    (\ref{eq15.2})\와 같이 \(v \in \gen{S}\)를 정의하면 \(w - v \in \gen{S}^\perp\)임을 확인할 수 있다. 따라서 임의의 \(v' \in \gen{S}\)에 대해
    \[
        \norm{w - v'}^2 = \norm{w - v}^2 + \norm{v - v'}^2 \ge \norm{w - v}^2
    \]
    이고, 등호는 \(v' = v\)일 때만 성립한다.
\end{proof}

\begin{cor} \label{15.1.5}
    내적공간 \(V\)의 가산 정규직교집합 \(\{v_n\}_{n \in \NN}\)이 주어졌을 때, 임의의 \(v \in V\)에 대하여
    \[
        \sum_n \abs{\inner{v}{v_n}}^2 \le \norm{v}^2
    \]
    이다.
\end{cor}
\begin{proof}
    정리 \ref{15.1.4}에 의해, 임의의 \(n\)에 대해
    \[
        \sum_{k=0}^n \abs{\inner{v}{v_k}}^2 \le \norm{v}^2
    \]
    이다.
\end{proof}

이제부터 아무 말이 없으면 \(V\)는 내적공간이고 \(S := \{v_n\}_{n \in \NN}\)은 \(V\)의 가산 정규직교집합이다.

따름정리 \ref{15.1.5}에 의해, 내적공간 \(V\)의 원소 \(v \in V\)에 대해 수열
    \[
        \hat{v} : n \mapsto \inner{v}{v_n}
    \]
는 \(\ell^2\)의 원소이다. 즉 함수 \(\phi : V \to \ell^2 : v \mapsto \hat{v}\)가 잘 정의된다. 이때 \(\phi\)는 립시츠 연속인 선형사상이다.

이제 우리의 결론이다.

\begin{thm}
    \(S\)가 \(V\)의 정규직교기저이면 다음이 성립한다.
    \begin{enum}
        \item 함수 \(\phi : V \to \ell^2\)는 등장사상이다.
        \item 임의의 \(v \in V\)에 대해
        \[
            \norm{v}^2 = \sum_n \abs{\inner{v}{v_n}}^2
        \]
        이다.
        \item 임의의 \(v \in V\)에 대해
        \begin{equation} \label{eq15.3}
            v = \sum_n \inner{v}{v_n} v_n
        \end{equation}
        이다.
        \item 임의의 \(v, w \in V\)에 대해 \(\inner{v}{w} = \inner{\phi(v)}{\phi(w)}\)이다.
    \end{enum}
    
\end{thm}
\begin{proof}
    \quad

    \begin{enum}
        \item 먼저 \(w \in \gen{S}\)에 대하여, 어떤 \(n\)이 존재해서
        \[
            w = \sum_{k=0}^n c_k v_k
        \]
        로 쓸 수 있다. 이때
        \[
            \norm{w}^2 = \sum_{k=0}^n \abs{c_k}^2 = \norm{\phi(w)}_2^2
        \]
        이다. 이제 임의의 \(v \in V\)에 대하여, \(v\)로 수렴하는 \(\gen{S}\) 안의 수열 \((w_n)_{n \in \NN}\)이 존재한다. 노름과 \(\phi\)가 연속함수이므로
        \[
            \norm{\phi(v)}_2 = \lim_{n} \norm{\phi(w_n)}_2 = \lim_n \norm{w_n} = \lim_n \norm{v}
        \]
        이다. 따라서 함수 \(\phi\)는 \(V\) 전체에서 등장사상이다.
        \item (a)를 다시 쓴 것이다.
        \item 우변의 부분합을 \((w_n)_{n \in \NN}\)이라 하면,
        \[
            \norm{v - w_n}^2 = \norm{\phi(v) - \phi(w_n)}_2^2 = \sum_{k > n} \abs{\inner{v}{v_k}}^2 \xrightarrow[]{n \to \infty} 0
        \]
        이다.
        \item (a)와 동치이다.
    \end{enum}
    
\end{proof}

\begin{defn}
    \(S\)가 \(V\)의 정규직교기저일 때, 각 \(n\)에 대해 \(\inner{v}{v_n}\)을 \(v\)의 \textbf{\(n\)번째 푸리에 계수(\(n\)-th Fourier coefficient)}라고 한다. 그리고 (\ref{eq15.3})의 우변의 급수를 \textbf{푸리에 급수(Fourier series)}라고 한다.
\end{defn}

\begin{prop}
    푸리에 계수는 유일하다. 즉
    \[
        v = \sum_{n} \sum_n c_n v_n = \sum_n \sum_n c_n' v_n
    \]
    이면 \(c_n = c'_n\)이다.
\end{prop}
\begin{proof}
    \(v_n\)과 내적하였을 때 같은 값이 나와야 하므로, \(c_n = c'_n\).
\end{proof}

이제 본격적으로 \(V = \calR^2([-\pi, \pi])\)인 경우를 보자. \(V\)에는 다음과 같은 내적이 주어져 있다고 생각하자.
\[
    \inner{f}{g} = \frac{1}{2\pi} \int_{-\pi}^\pi f\overline{g} \quad (f, g \in \calR^2(I))
\]
그리고 \(n \in \ZZ\)에 대하여 \(u_n : x \mapsto e^{inx}\)로 정의하자. 이때 \(\gen{\{u_n\}_{n \in \ZZ}}\)는 삼각다항식들의 집합이며, \(\calT\)로 쓰기로 한다.

\begin{prop}
    \(\{u_n\}_{n \in \ZZ}\)는 \(\calR^2([-\pi, \pi])\)의 정규직교집합이다.
\end{prop}
\begin{proof}
    \(n, m \in \ZZ\)에 대하여
    \[
        \inner{u_n}{u_m} = \frac{1}{2\pi}\int_{-\pi}^\pi e^{inx} e^{-imx} dx = \frac{1}{2\pi}\int_{-\pi}^\pi e^{i(n-m)x} dx
    \]
    이다. 따라서 \(n \ne m\)이면 \(\inner{u_n}{u_m}  = 0\), \(n = m\)이면 \(\inner{u_n}{u_m} = 1\)을 얻는다.
\end{proof}

우리의 주장은 \(\{u_n\}_{n \in \ZZ}\)가 정규직교기저까지 된다는 것이다.

\begin{thm}
    \(\{u_n\}_{n \in \ZZ}\)는 \(\calR^2([-\pi, \pi])\)의 정규직교기저이다.
\end{thm}

\begin{proof}
    \(f \in \calR^2([-\pi, \pi]), \eps > 0\)이 주어졌다고 하자. \(C_c([-\pi, \pi]) = C([-\pi, \pi])\)가 \( \calR^2([-\pi, \pi])\)에서 조밀하므로 다음을 만족하는 연속함수 \(g\)가 존재한다.
    \[
        \norm{f - g}_2 < \frac{\eps}{2}
    \]
    그런데 정리 \ref{10.3.11} 아래의 예시에서, \(\calT = \gen{\{u_n\}_{n \in \ZZ}}\)가 \(C([-\pi, \pi])\)에서 조밀함을 보였다. 따라서 다음을 만족하는 삼각다항식 \(T \in \calT\)가 존재한다.
    \[
        \norm{g - T}_{\infty} < \frac{\eps}{2}
    \]
    이때
    \[
        \norm{g - T}_2^2 = \frac{1}{2\pi} \int_{-\pi}^\pi \abs{g - T}^2 < \frac{\eps^2}{4}
    \]
    이므로 \(\norm{g - T}_2 < \eps/2\)이고, 따라서 \(\norm{f - T}_2 < \eps\)이다. 즉 \(\calT\)가 \(\calR^2([-\pi, \pi])\)에서 조밀하므로 \(\{u_n\}_{n \in \ZZ}\)는 \(\calR^2([-\pi, \pi])\)의 정규직교기저이다.
\end{proof}

\begin{cor}[파르스발의 등식\footnote{Parseval's identity.}]
    \(f \in \calR^2([-\pi, \pi])\)에 대하여 
    \[
        \int_{-\pi}^\pi \abs{f(x)}^2 dx = 2\pi \sum_{n \in \ZZ} \abs{\hat{f}(n)}^2 = \sum_{n \in \ZZ} \abs{\int_{-\pi}^\pi f(x) e^{-inx} dx}^2
    \]
    이다.
\end{cor}

\begin{cor} \label{15.1.11}
    \(f \in \calR^2([-\pi, \pi]), n \in \NN\)에 대하여
    \[
        s_n(f) : x \mapsto s_n(f ; x) := \sum_{k=-n}^n \hat{f}(k) e^{ikx} \quad (x \in [-\pi, \pi])
    \]
    로 정의하면 \(\calR^2([-\pi, \pi])\)에서 \((s_n(f))_{n \in \NN}\)이 \(f\)로 수렴한다.
\end{cor}

\begin{ex}
    \(f :x \to x\)의 푸리에 계수를 계산하면 \(\hat{f}(0) = 0\)이고, \(n \ne 0\)일 때
    \[
        \hat{f}(n) = \frac{1}{2\pi} \int_{-\pi}^\pi xe^{-inx} dx = \frac{1}{2\pi} \sqbracket{x \paren{\frac{1}{-in} e^{inx}}}_{-\pi}^\pi = \frac{(-1)^{n-1}}{in}
    \]
    이다. 따라서 \(f\)의 푸리에 급수는
    \[
        f(x) = \sum_{n =1}^\infty \frac{2(-1)^{n-1}}{n} \sin nx 
    \]
    로 주어진다. 한편
    \[
        \frac{2\pi^3}{3} = \int_{-\pi}^\pi \abs{f(x)}^2 dx = 2\pi \sum_{n \in \ZZ} \abs{\hat{f}(n)}^2 = \sum_{n=1}^\infty \frac{4\pi}{n^2}
    \]
    이므로
    \[
        \sum_{n=1}^\infty \frac{1}{n^2} = \frac{\pi^2}{6}
    \]
    이다.
\end{ex}

푸리에 계수의 정의
\[
    \hat{f}(n) = \frac{1}{2\pi} \int_{-\pi}^\pi f(x) e^{-inx} dx
\]
를 보면, \(f \in \calR^1([-\pi, \pi])\)이기만 해도 푸리에 계수를 정의하는 것은 가능하다. 따라서 푸리에 계수와 급수의 정의를 \(\calR^1([-\pi, \pi])\)로 확장할 수 있다. 단, 이때는 아직 수렴성에 관해 아는 것이 없으므로 등호 대신
\[
    f(x) \sim \sum_{n \in \ZZ} \hat{f}(n) e^{inx}
\]
로 쓰자.

다음은 푸리에 계수와 급수의 기본적인 성질들이다.

\begin{prop}
    \(f \in \calR^1([-\pi, \pi])\)에 대하여 다음이 성립한다.
    \begin{enum}
        \item \(f\)의 푸리에 급수는
        \[
            f(x) \sim \sum_{n \in \ZZ} \hat{f}(n) e^{inx} = \hat{f}(0) + \sum_{n=1}^\infty [(\hat{f}(n) + \hat{f}(-n)) \cos nx + i(\hat{f}(n) - \hat{f}(-n)) \sin nx]
        \]
        로 주어진다.
        \item \(f\)가 실함수이면 각 \(n\)에 대해 \(\hat{f}(-n) = \overline{\hat{f}(n)}\)이다.
        \item \(f\)가 짝함수이면 각 \(n\)에 대해 \(\hat{f}(-n) = \hat{f}(n)\)이고 그 푸리에 급수는
        \[
            f(x) \sim \hat{f}(0) + \sum_{n=1}^\infty 2\hat{f}(n) \cos nx
        \]
        로 주어진다.
        \item \(f\)가 홀함수이면 각 \(n\)에 대해 \(\hat{f}(-n) = -\hat{f}(n)\)이고 그 푸리에 급수는
        \[
            f(x) \sim \sum_{n=1}^\infty 2i\hat{f}(n) \sin nx
        \]
        로 주어진다.
    \end{enum}
\end{prop}
\begin{proof}
    \quad

    \begin{enum}
        \item \(e^{inx}\)의 실수부와 허수부를 전개해 보면 된다.
        \item \[
            \hat{f}(-n) = \frac{1}{2\pi} \int_{-\pi}^\pi f(x) e^{inx} dx = \frac{1}{2\pi} \int_{-\pi}^\pi \overline{f(x) e^{-inx}}dx = \hat{f}(n).
        \]
        \item
        \[
            \hat{f}(-n) = \frac{1}{2\pi} \int_{-\pi}^\pi f(x) e^{inx} dx = \frac{1}{2\pi} \int_{-\pi}^\pi f(x) \cos nx dx = \frac{1}{2\pi} \int_{-\pi}^\pi f(x) e^{-inx} dx = \hat{f}(n).
        \]
        \item (c)와 유사하다.
    \end{enum}
\end{proof}

\section{푸리에 급수의 점별수렴}

\(\calR^2(I)\)에서의 푸리에 급수에서 꽤 그럴듯한 결론인 파르스발의 등식을 얻었지만, 이것은 점별수렴에 관해서는 아무런 이야기를 하지 못한다. 이번 절에서는 푸리에 급수가 점별수렴할 충분조건에 대해 공부한다. 

다음 보조정리는 \(\calR^1([-\pi, \pi])\)에서의 푸리에 계수를 공부하는 데 결정적인 역할을 한다.

\begin{lem}[리만-르베그 보조정리]
    구간 \(I \subseteq \RR\)와 \(f \in \calR^1(I)\)에 대하여
    \[
        \lim_{\abs{\alpha} \to \infty} \int_I f(t) e^{i\alpha t} dt = 0
    \]
    이다.
\end{lem}
\begin{proof}
    양수 \(\eps > 0\)이 주어졌을 때, \(\norm{f - s}_1 < \eps\)이도록 하는 계단함수 \(s = \sum_{k=1}^n c_k \chi_{I_k}\)가 존재한다(단 \(J_k\)들은 모두 유계구간). 이때
    \begin{align*}
        \abs{\int_I f(t) e^{i\alpha t} dt} &\le \abs{\int_I s(t) e^{i\alpha t} dt} + \int_I \abs{f(t) - s(t)} dt\\
        &< \abs{\sum_{k=1}^n c_k \int_{I_k} e^{i\alpha t} dt} + \eps\\
        &\le \sum_{k=1}^n \abs{c_k} \frac{2}{\abs{\alpha}} + \eps
    \end{align*}
    이므로
    \[
        \limsup_{\abs{\alpha} \to \infty} \abs{\int_I f(t) e^{i\alpha t} dt} \le \eps
    \]
    이다. \(\eps > 0\)이 임의의 양수였으므로
    \[
        \lim_{\abs{\alpha} \to \infty} \int_I f(t) e^{i\alpha t} dt = 0
    \]
    이다.
\end{proof}

\begin{cor}
    구간 \(I \subseteq \RR\)와 \(f \in \calR^1(I)\)가 주어졌을 때, 각 실수 \(\beta \in \RR\)에 대하여
    \[
        \lim_{\abs{\alpha} \to \infty} \int_I f(t) \sin(\alpha t  + \beta) dt = 0
    \]
    이다.
\end{cor}

\(f \in \calR^1([-\pi, \pi])\)에 대하여
\[
    s_n(f; x) = \sum_{k=-n}^n \frac{1}{2\pi} \int_{-\pi}^\pi f(t) e^{-ikt} dt e^{ikx} = \sum_{k=-n}^n \frac{1}{2\pi} \int_{-\pi}^\pi e^{ik(x-t)}dt \quad (x \in [-\pi, \pi])
\]
이므로
\[
    D_n : t \mapsto \sum_{k=-n}^n e^{ikt}
\]
로 정의하면
\[
    s_n(f; x) = \frac{1}{2\pi} \int_{-\pi}^\pi f(t) D_n(x-t) dt
\]
이다. 이러한 \(D_n\)을 \textbf{디리클레 핵(Dirichlet kernel)}이라고 하는데, 디리클레 핵의 몇 가지 성질은 다음과 같다.

\begin{prop} \label{15.2.3}
    \(t \in \RR, n \in \NN\)에 대해 다음이 성립한다.
    \begin{enum}
        \item \(D_n(-t) = D_n(t)\)이고 \(D_n(t+2\pi) = D_n(t)\).
        \item \(\displaystyle \frac{1}{2\pi} \int_{-\pi}^\pi D_n = 1\).
        \item \(t \notin 2\pi\ZZ\)일 때 \(\displaystyle D_n(t) = \frac{\sin [(n + 1/2)t]}{\sin (t/2)}\).
    \end{enum} 
\end{prop}
\begin{proof}
    (a)와 (b)는 연습문제로 남긴다. (c)는
    \[
        D_n(t) = \sum_{k=-n}^n e^{ikt} = \frac{e^{-int}(e^{i(2n+1)t} - 1)}{e^{it} - 1} = e^{-int} \cdot \frac{e^{i(n+1/2)t} \cdot 2i \sin [(n+1/2)t]}{e^{it/2} \cdot 2i \sin (t/2)} = \frac{\sin [(n + 1/2)t]}{\sin (t/2)}
    \]
    에서 성립한다.
\end{proof}

이제부터 ``주기함수 \(f \in \calR^1([-\pi, \pi])\)''라고 하면, \(f\)가 주기 \(2\pi\)를 가지도록 정의역이 실수 전체로 확장되었다고 생각하자. 그러면
\begin{align*}
    s_n(f; x) &= \frac{1}{2\pi} \int_{-\pi}^\pi f(t) D_n(x-t) dt\\
    &= \frac{1}{2\pi} \int_{x-\pi}^{x+\pi} f(x-t) D_n(t) dt\\
    &= \frac{1}{2\pi} \int_{-\pi}^{\pi} f(x-t) D_n(t) dt
\end{align*}
를 얻는다.

\begin{defn}
    거리공간 \(X, Y\)에 대하여 함수 \(f : X \to Y\)가 \(x_0 \in X\)에서 \textbf{립시츠 조건(Lipschitz condition)}을 만족한다는 것은, 어떤 양수 \(\delta, M > 0\)이 존재하여
    \[
        d_X(x, x_0) < \delta \implies d_Y(f(x), f(x_0)) \le M d_X(x, x_0) \quad (x \in X)
    \]
    라는 것이다.
\end{defn}

이제 \(s_n(f ; x) \to f(x)\)일 충분조건이다.

\begin{thm}
    주기함수 \(f \in \calR^1([-\pi, \pi])\)가 \(x \in \RR\)에서 립시츠 조건을 만족하면
    \[
        \lim_{n} s_n(f; x) = f(x)
    \]
    이다.
\end{thm}
\begin{proof}
    \(x\)를 고정하고 함수 \(g\)를 \(g(0) := 0\),
    \[
        g(t) := \frac{f(x - t) - f(x)}{\sin (t/2)} \quad (t \ne 0)
    \]
    으로 정의하자.\\
    \textbf{Claim.} \(g \in \calR^1([-\pi, \pi])\)이다.
    \begin{proof}
        \(f\)가 \(x\)에서 립시츠 조건을 만족하므로, 어떤 양수 \(\delta, M > 0\)이 존재하여 \(\abs{t} < \delta\)이면
        \[
            \abs{f(x + t) - f(x)} \le M\abs{t}
        \]
        이다. 이제
        \[
            \int_{\delta \le \abs{t} \le \pi} \abs{\frac{f(x - t) - f(x)}{\sin (t/2)}} dt \le \frac{1}{\sin(\delta/2)} \paren{2\pi\abs{f(x)} + \int_{-\pi}^\pi \abs{f(t)} dt} < \infty
        \]
        이고, \(\abs{t} < \delta\)일 때는
        \[
            \abs{\frac{f(x - t) - f(x)}{\sin (t/2)}} \le M \abs{\frac{t}{\sin (t/2)}} < \infty
        \]
        이므로 \(g\)가 유계이다. 따라서 \(g \in \calR^1([-\pi, \pi])\)이다.
    \end{proof}
    이제 명제 \ref{15.2.3}의 (b), (c)를 생각하면
    \[
        s_n(f; x) - f(x) = \frac{1}{2\pi} \int_{-\pi}^{\pi} [f(x-t) - f(x)] D_n(t) dt = \frac{1}{2\pi} \int_{-\pi}^{\pi} g(t) \sin[(n+1/2)t] dt
    \]
    인데, 리만-르베그 보조정리에 의해
    \[
        s_n(f; x) - f(x) = \frac{1}{2\pi} \int_{-\pi}^{\pi} g(t) \sin[(n+1/2)t] dt \xrightarrow[]{n \to \infty} 0
    \]
    이다.
\end{proof}

\begin{cor}
    주기함수 \(f \in \calR^1([-\pi, \pi])\)가 \(x \in \RR\)에서 미분가능하면 \(s_n(f; x) \to f(x)\)이다.
\end{cor}
\begin{proof}
    \(f\)가 \(x\)에서 미분가능하면 \(x\)에서 립시츠 조건을 만족한다.
\end{proof}

한편 \(D_n\)의 대칭성에 의해
\[
    s_n(f; x) = \frac{1}{\pi} \int_0^\pi \frac{f(x+t) + f(x-t)}{2} D_n(t) dt
\]
로 쓸 수 있다. 이제 각 \(n \in \ZZ_+\)에 대해
\[
    K_n := \frac{1}{n} \sum_{k=0}^{n-1} D_k
\]
로 정의하면
\[
    \sigma_n(f) : x \mapsto \sigma_n(f; x) := \sum_{k=0}^{n-1} s_n(f; x) = \frac{1}{\pi} \int_0^\pi \frac{f(x+t) + f(x-t)}{2} K_n(t) dt
\]
이다. 이러한 \(K_n\)을 \textbf{페예르 핵(Fej\'er kernel)}이라고 한다. 페예르 핵의 몇 가지 성질은 다음과 같다.

\begin{prop} \label{15.2.7}
    \(t \in \RR, n \in \ZZ_+\)에 대해 다음이 성립한다.
    \begin{enum}
        \item \(K_n(-t) = K_n(t)\)이고 \(K_n(t+2\pi) = K_n(t)\).
        \item \(\displaystyle \frac{1}{\pi} \int_0^\pi K_n = 1\).
        \item  \(t \notin 2\pi\ZZ\)에 대하여 \(\displaystyle K_n(t) = \frac{1}{n} \cdot \frac{\sin^2(nt/2)}{\sin^2(t/2)}\)이고, 따라서 \(K_n \ge 0\).
        \item 임의의 \(\delta \in (0, \pi)\)에 대하여, \((K_n)_{n \in \ZZ_+}\)은 \([\delta, \pi]\)에서 0으로 고른수렴한다.
    \end{enum}
\end{prop}
\begin{proof}
    \quad

    \begin{enum}
        \item \(D_n\)의 대칭성과 주기성으로부터 나온다.
        \item
        \[
            \frac{1}{2\pi} \int_{-\pi}^\pi K_n = \frac{1}{2\pi} \cdot \frac{1}{n} \sum_{k=0}^{n-1} D_k = 1
        \]
        이고, \(K_n\)이 짝함수이므로
        \[
            \frac{1}{\pi} \int_0^\pi K_n = 1
        \]
        이다.
        \item \(\displaystyle D_k(t) = \frac{\sin [(k + 1/2)t]}{\sin (t/2)}\)이므로 (c)를 보이는 것은
        \[
            \sum_{k=0}^{n-1} \sin [(k + 1/2)t] = \frac{\sin^2(nt/2)}{\sin(t/2)}
        \]
        를 보이는 것과 같다. 이때
        \[
            \sum_{k=1}^{n} e^{i(k-1/2)t} = e^{-it/2} \cdot \frac{e^{it}(e^{int} - 1)}{e^{it}-1} = e^{it/2} \cdot \frac{e^{int/2} \cdot 2i \sin(nt/2)}{e^{it/2} \cdot 2i \sin(t/2)} = e^{int/2} \cdot \frac{\sin(nt/2)}{\sin(t/2)}
        \]
        인데, 양변의 허수부분을 취하면
        \[
            \sum_{k=0}^{n-1} \sin [(k + 1/2)t] = \frac{\sin^2(nt/2)}{\sin(t/2)}
        \]
        를 얻는다.
        \item \(\delta \in (0, \pi)\)를 고정하면, 임의의 \(t \in [\delta, \pi]\)에 대해
        \[
            K_n(t) = \frac{1}{n} \cdot \frac{\sin^2(nt/2)}{\sin^2(t/2)} \le \frac{1}{n} \cdot \frac{2}{\sin^2(\delta/2)} \xrightarrow[]{n \to \infty} 0
        \]
        이다.
    \end{enum}
\end{proof}

\begin{defn}
    \(f : \RR \to \CC\)가 주기 \(2\pi\)를 가지는 연속함수이면 \(f\)를 \textbf{연속주기함수}라고 한다.
\end{defn}

\begin{thm} \label{15.2.9}
    주기함수 \(f \in \calR^1([\pi, \pi])\)와 \(x \in \RR\)에 대하여 다음이 성립한다.
    \begin{enum}
        \item \(f(x+)\)와 \(f(x-)\)가 \(\CC\) 안에서 존재하면
        \[
            \lim_{n \to \infty} \sigma_n(f; x) = \frac{f(x+) + f(x-)}{2}
        \]
        이다. 따라서 \(f\)가 \(x\)에서 연속이면, \(\sigma_n(f ; x) \to f(x)\)이다.
        \item \(f\)가 연속주기함수이면 \(\sigma_n(f)\)는 \(f\)로 실수 전체에서 고른수렴한다.
    \end{enum}
\end{thm}

\begin{proof}
    \quad

    \begin{enum}
        \item
        \[
            A := \frac{f(x+) + f(x-)}{2}, \quad g_x : t \mapsto \frac{f(x+t) + f(x-t)}{2} - A
        \]
        로 정의하자. 이때 \(g_x \in \calR^1([-\pi, \pi])\)이고
        \[
            \lim_{t \searrow 0} g_x(t) = 0
        \]
        이고, 그러면 명제 \ref{15.2.7}의 (b)로부터
        \[
            \sigma_n(f; x) - A = \frac{1}{\pi} \int_0^\pi \frac{f(x+t) + f(x-t)}{2} K_n(t) dt - A = \frac{1}{\pi} \int_0^\pi g_x(t) K_n(t) dt
        \]
        이다. 임의의 양수 \(\eps > 0\)이 주어졌다고 하면, 어떤 양수 \(\delta_x \in (0, \pi)\)가 존재하여 
        \[
            0 < t < \delta_x \implies \abs{g_x(t)} < \eps
        \]
        이 성립한다. 이제
        \[
            \abs{\sigma_n(f; x) - A} \le \frac{1}{\pi} \int_0^{\delta_x} \abs{g_x(t)}K_n(t) dt + \frac{1}{\pi} \int_{\delta_x}^{\pi} \abs{g_x(t)}K_n(t) dt
        \]
        로 쓸 수 있는데, 먼저
        \[
            \frac{1}{\pi} \int_0^{\delta_x} \abs{g_x(t)}K_n(t) dt <  \frac{\eps}{\pi} \int_0^1 K_n(t) dt < \eps
        \]
        이다. 다음으로 명제 \ref{15.2.7}의 (d)로부터, 다음을 만족하는 자연수 \(N_x\)가 존재한다.
        \[
            n \ge N_x \implies \abs{K_n(t)} < \eps \quad (t \in [\delta_x, \pi])
        \]
        이제 \(n \ge N_x\)이면
        \[
            \frac{1}{\pi} \int_{\delta_x}^{\pi} \abs{g_x(t)}K_n(t) dt \le \frac{\eps}{\pi} \norm{g}_1
        \]
        이다. 따라서
        \[
            n \ge N_x \implies \abs{\sigma_n(f; x) - A} < \paren{1 + \frac{\norm{g_x}_1}{\pi}}\eps
        \]
        이고, \(\norm{g_x}_1\)은 \(\eps\)와 무관한 양수이므로 \(\sigma_n(f; x) \to A\)가 증명되었다.
        \item \(f\)가 연속주기함수이면 고른연속임은 쉽게 확인할 수 있다. 따라서 임의의 양수 \(\eps > 0\)에 대해 다음을 만족하는 양수 \(\delta > 0\)이 존재한다.
        \[
            \abs{x - y} < \delta \implies \abs{f(x) - f(y)} < \eps \quad (x, y \in \RR)
        \]
        이때 \(0 < \abs{t} < \delta\)이면, 임의의 \(x \in \RR\)에 대해
        \[
            \abs{g_x(t)} = \abs{\frac{f(x+t) + f(x-t)}{2} - f(x)} \le \abs{\frac{f(x+t) - f(x)}{t}} + \abs{\frac{f(x-t) - f(x)}{2}} < \eps
        \]
        이다. 즉 (a)에서 모든 \(x \in \RR\)에 대해 \(\delta_x = \delta\)로 둘 수 있다. 한편 (a)에서 \(N_x\)는 \(\delta_x\)에만 의존하였으므로, 모든 \(x \in \RR\)에 대해 \(N_x = N\)으로 둘 수 있다. 이제 \(n \ge N\)이면 임의의 \(x \in \RR\)에 대해
        \[
            \abs{\sigma_n(f; x) - f(x)} < \paren{1 + \frac{\norm{g_x}_1}{\pi}}\eps
        \]
        이므로, \(\sup_{x \in \RR} \norm{g_x}_1 < \infty\)인 것만 보이면 고른수렴이 증명된다. 이는
        \[
            \int_{-\pi}^\pi \abs{\frac{f(x+t) + f(x-t)}{2} - f(x)} dt \le 2\pi \sup_{x \in [-\pi, \pi]} \abs{f(x)} + \int_{-\pi}^\pi \abs{f(t)} dt < \infty
        \]
        에서 성립한다.
    \end{enum}
\end{proof}

\begin{cor}
    연속주기함수 \(f : \RR \to \CC\)와 \(x \in \RR\)에 대하여 \((s_n(f; x))_{n \in \NN}\)이 수렴하면 그 극한값은 \(f(x)\)이다.
\end{cor}
\begin{proof}
    따름정리 \ref{2.4.8} 아래의 예시에서 증명한 바에 의해, \((s_n(f; x))_{n \in \NN}\)이 어떤 \(s\)로 수렴하면 \((\sigma_n(f; x))_{n \in \ZZ_+}\)도 \(s\)로 수렴한다. 그런데 정리 \ref{15.2.9}에 의해 \((\sigma_n(f; x))_{n \in \ZZ_+}\)는 \(f(x)\)로 수렴하므로 \(s = f(x)\)이다.
\end{proof}

\section{푸리에 급수와 미분$\cdot$적분}

먼저 푸리에 급수와 적분의 관계를 알아보자.

\begin{thm} \label{15.3.1}
    \(f \in \calR^2([-\pi, \pi])\)와 \(a \in [-\pi, \pi]\)에 대하여
    \[
        F(x) := \int_a^x f(t) dt, \quad S_n(f; x) := \int_a^x s_n(f ; t) dt
    \]
    로 정의하면  \((S_n(f))_{n \in \NN}\)은 \(F\)로 고른수렴한다.
\end{thm}
\begin{proof}
    임의의 \(x \in [-\pi, \pi]\)에 대하여
    \begin{align*}
        \abs{S_n(f; x) - F(x)} &\le \int_{-\pi}^\pi \abs{s_n(f; t) - f(t)} dt\\
        &= 2\pi \inner{\abs{s_n(f) - f}}{1}\\
        &\le 2\pi \norm{s_n(f) - f}_2 \xrightarrow[]{n \to \infty} 0
    \end{align*}
    이므로 \((S_n(f))_{n \in \NN}\)가 \(F\)로 고른수렴한다.
\end{proof}

\begin{ex}
    따름정리 \ref{15.1.11} 아래의 예시의 계산으로부터, \(f : x \mapsto x\)에 대하여
    \[
        \hat{f}(0) =  0, \quad \hat{f}(n) = \frac{(-1)^{n-1}}{in} \ (n \ne 0)
    \]
    임을 확인하였다. 따라서
    \[
        s_n(f; x) = 2 \sum_{k=1}^n \frac{(-1)^{k-1}}{k} \sin kx \quad (x \in [-\pi, \pi])
    \]
    로 주어진다. 정리 \ref{15.3.1}에서 \(a = -\pi\)인 경우를 생각하면
    \[
        F(x) = \frac{x^2}{2} - \frac{\pi^2}{2}, \quad S_n(f, x) = -2 \sum_{k=1}^n \frac{1}{k^2} + 2 \sum_{k=1}^n \frac{(-1)^k}{k^2} \cos kx  \quad (x \in [-\pi, \pi])
    \]
    이므로,
    \[
        \frac{x^2}{2} - \frac{\pi^2}{2} = -2 \sum_{n=1}^\infty \frac{1}{n^2} + 2 \sum_{n=1}^\infty \frac{(-1)^n}{n^2} \cos nx  \quad (x \in [-\pi, \pi])
    \]
    이고 우변은 \([-\pi, \pi]\)에서 고른수렴한다. 이를 다시 쓰면
    \[
        \frac{x^2}{2} = \frac{\pi^2}{6} + 2 \sum_{n=1}^\infty \frac{(-1)^n}{n^2} \cos nx  \quad (x \in [-\pi, \pi])
    \]
    이다. 그런데 이미 우변이 푸리에 급수의 형태이므로, 푸리에 계수의 유일성에 의해 \(g : x \mapsto x^2/2\)에 대해
    \[
        \hat{g}(0) = \frac{\pi^2}{6}, \quad \hat{g}(n) = \frac{(-1)^n}{n^2} \ (n \ne 0)
    \]
    임을 알 수 있다. 이제
    \[
        \norm{g}_2^2 = \frac{1}{2\pi} \int_{-\pi}^\pi \frac{x^4}{4} dx = \frac{\pi^4}{20}
    \]
    이고
    \[
        \sum_{n \in \ZZ} \abs{\hat{g}(n)}^2 = \frac{\pi^4}{36} + 2 \sum_{n=1}^\infty \frac{1}{n^4}
    \]
    이므로 파르스발의 등식에 의해
    \[
        \sum_{n=1}^\infty \frac{1}{n^4} = \frac{1}{2} \paren{\frac{\pi^4}{20} - \frac{\pi^4}{36}} = \frac{\pi^4}{90}
    \]
    이다. 이와 유사한 방법을 계속 반복하면 임의의 양의 짝수 \(2m\)에 대해 \(\sum_{n=1}^\infty 1/n^{2m}\)의 값을 정확히 구할 수 있다.
\end{ex}

이제 푸리에 급수와 미분가능성의 관계를 보자.

\begin{defn}
    \(f : \RR \to \CC\)가 주기 \(2\pi\)를 가진다고 하자. \(k \in \NN \cup \{\infty\}\)에 대하여 \(f\)가 \(C^k\)-함수이면 \(f\)를 \textbf{\(C^k\)-주기함수}라 하고, \(f\)가 해석함수이면 \(f\)를 \textbf{해석주기함수}라 한다.
\end{defn}

\begin{lem}
    \(f\)가 \(C^1\)-주기함수이면 \(n \in \ZZ\)에 대해 \(\widehat{f'}(n) = in \hat{f}(n)\)이다.
    \begin{proof}
        \begin{align*}
            \widehat{f'}(n) = \frac{1}{2\pi} \int_{-\pi}^\pi f'(x) e^{-inx} dx &= \frac{1}{2\pi} \paren{\sqbracket{f(x) e^{-inx}}_{-\pi}^\pi - \int_{-\pi}^\pi f(x)(-ine^{-inx})}\\
            &= \frac{in}{2\pi} \int_{-\pi}^\pi f(x) e^{inx} dx = in \hat{f}(n).
        \end{align*}
    \end{proof}
\end{lem}

\begin{thm} \label{15.3.4}
    \(k \in \NN\)에 대하여 \(f\)가 \(C^k\)-주기함수이면 \(\abs{n} \to \infty\)일 때 \(\hat{f}(n) = o(\abs{n}^{-k})\)이다.
\end{thm}
\begin{proof}
    \(k = 0\)일 때 성립함은 당연하다. 이제 귀납적으로, 어떤 \(k = l\ge 0\)에 대해 \(\hat{f}(n) = o(\abs{n}^{-k})\)라고 성립한다고 하자. \(f\)가 \(C^{l+1}\)-주기함수이면 가정에 의해 \(\abs{n} \to \infty\)일 때 \(\widehat{f'}(n) = in \hat{f}(n) = o(\abs{n}^{-l})\)인데, 이는 \(\hat{f}(n) = o(\abs{n}^{-l-1})\)이므로 \(k = l+1\)일 때도 \(\hat{f}(n) = o(\abs{n}^{-k})\)가 성립한다. 따라서 모든 \(k\)에 대해 \(\hat{f}(n) = o(\abs{n}^{-k})\)이다.
\end{proof}

\begin{defn}
    복소수열 \((c_n)_{n \in \ZZ}\)가 \textbf{빠른감소수열(rapidly decaying sequence)}이라는 것은, 임의의 \(k \ge 0\)에 대해 \(\abs{n} \to \infty\)일 때 \(c_n = o(\abs{n}^{-k})\)라는 것이다.
\end{defn}

\begin{prop}
    \((c_n)_{n \in \ZZ}\)가 빠른감소수열이면 \(\sum_{n \in \ZZ} \abs{c_n} < \infty\).
\end{prop}
\begin{proof}
    어떤 자연수 \(N\)이 존재하여, \(\abs{n} \ge N\)이면 \(\abs{c_n} < 1/\abs{n}^2\)이도록 할 수 있기 때문이다.
\end{proof}

\begin{thm}
    \(C^\infty\)-주기함수들의 공간에서 빠른감소수열들의 공간으로 가는 함수 \(f \mapsto \hat{f}\)는 잘 정의된 전단사함수이다.
\end{thm}
\begin{proof}
    잘 정의된 단사함수임은 분명하므로 연습문제로 남기고, 전사인 것만 보이자.\\
    빠른감소수열 \((c_n)_{n \in \ZZ}\)가 주어졌을 때, 함수
    \[
        f : x \mapsto \sum_{n \in \ZZ} c_n e^{inx}
    \]
    가 잘 정의된 \(C^\infty\)-함수임을 보이자. 각 \(n \in \NN\)에 대해
    \[
        s_n : x \mapsto \sum_{k=-n}^n c_k e^{ikx}
    \]
    로 쓰자. 이때 \(\sum_{n \in \ZZ} \abs{c_n} < \infty\)이므로 \((s_n)_{n \in \NN}\)은 실수 전체에서 고른수렴한다. 한편
    \[
        s_n(x) = \sum_{k=-n}^n ikc_k e^{ikx} \quad (x \in \RR)
    \]
    인데 \((inc_n)_{n \in \ZZ}\)도 빠른감소수열이므로 같은 논리로 \((s_n')_{n \in \NN}\)은 실수 전체에서 고른수렴한다. 따라서 \((s_n)_{n \in \NN}\)이 \(f\)로 고른수렴하고,
    \[
        f'(x) = \sum_{n \in \ZZ} in c_n e^{inx} \quad (x \in \RR)
    \]
    로 주어진다. 이를 반복하면, 모든 \(m \in \NN\)에 대해 \(f^{(m)}\)이 존재함을 보일 수 있다. 따라서 \(f\)는 \(C^\infty\)-주기함수이다. \(f\)의 정의로부터 \(\hat{f}(n) = c_n\)임은 쉽게 알 수 있다.
\end{proof}

이제 \(f\)가 해석주기함수인 경우에 \(\hat{f}\)가 얼마나 빠르게 감소하는지 알아보자. 이를 위해 준비가 많이 필요하다.

\begin{lem} \label{15.3.8}
    \(n \in \NN\)에 대하여, \(\abs{x} < 1\)일 때
    \begin{equation} \label{eq15.5}
        \frac{1}{(1-x)^{n+1}} = \sum_{k=0}^\infty \binom{n+k}{k} x^k
    \end{equation}
    가 성립한다.
\end{lem}
\begin{proof}
    \(n = 0\)일 때 성립함은 당연하다. 이제 \(n = m\)에서 \(\abs{x} < 1\)일 때 (\ref{eq15.5})\가 성립한다고 가정하자. 다음 두 급수
    \[
        \sum_k 1, \quad \sum_k \binom{m+k}{k}
    \]
    의 코시곱을 \(\sum_k c_k\)라 쓰면,
    \[
        c_k = \sum_{l=0}^k \binom{m+l}{l} = \binom{m+1+k}{k}
    \]
    를 얻는다. 따라서 두 거듭제곱급수
    \[
        \frac{1}{1-x} = \sum_k x^k, \quad \frac{1}{(1-x)^{m+1}} = \sum_k \binom{m+k}{k} x^k
    \]
    의 공통 수렴반경인 \(\abs{x} < 1\)에서
    \[
        \frac{1}{(1-x)^{m+2}} = \sum_{k=0}^\infty \binom{m+1+k}{k} x^k
    \]
    가 성립한다고 할 수 있다. 따라서 \(n = m+1\)일 때 (\ref{eq15.5})\가 성립하며, 수학적 귀납법에 의해 모든 \(n \in \NN\)에 대해 \(\abs{x} < 1\)에서 (\ref{eq15.5})\가 성립한다.
\end{proof}

\begin{lem} \label{15.3.9}
    해석주기함수 \(f : \RR \to \CC\)가 주어졌을 때, 어떤 양수 \(C, R > 0\)이 존재하여 임의의 \(n \in \NN\)에 대하여
    \[
        \norm{f^{(n)}}_{\sup} \le C n! R^n
    \]
    이 성립한다.
\end{lem}
\begin{proof}
    \(f\)의 주기성에 의해,
    \[
        \sup_{\abs{x} \le \pi} \abs{f^{(n)}(x)} \le C n! R^n \quad (n \in \NN)
    \]
    을 만족하는 \(C, R > 0\)을 찾으면 충분하다.\\
    \(c \in [-\pi, \pi]\)를 고정했을 때, \(c\)에서 \(f\)의 테일러급수의 수렴반경이 양수이므로 어떤 양수 \(M_c > 0\)이 존재하여
    \[
        \abs{\frac{f^{(n)}(c)}{n!}}^{1/n} \le M \quad (n \ge 1)
    \]
    을 만족한다. 이제 양수 \(\delta_c := 1/(2M_c)\)를 고정하면, 임의의 \(x \in (c - \delta_c, c + \delta_c)\)에 대해
    \[
        f^{(n)}(x) = \sum_k \frac{f^{(n+k)}(c)}{k!} (x  - c)^k = n! \sum_k \frac{f^{(n+k)}(c)}{(n+k)!} \binom{n+k}{k} (x  - c)^k
    \]
    이므로 보조정리 \ref{15.3.8}에 의해
    \begin{align*}
        \abs{f^{(n)}(x)} &\le n! \sum_{k} M_c^{n+k}\binom{n+k}{k} \delta_c^k\\
        &= n! M_c^n \sum_k \binom{n+k}{k} \paren{\frac{1}{2}}^k\\
        &= n! M_c^n \cdot 2^{n+1} = 2n! (2M_c)^n
    \end{align*}
    이다. 이제 \([-\pi, \pi]\)의 열린덮개 \(\{(c - \delta_c, c + \delta_c)\}_{c \in [-\pi, \pi]}\)를 생각하면, 유한 개의 \(c_1, \ldots, c_n \in [-\pi, \pi]\)가 존재하여 \(\{(c_i - \delta_{c_i}, c_i + \delta_{c_i})\}_{i=1, \ldots, n}\)가 유한 부분덮개가 된다. 이제
    \[
        C := 2, \quad R := \max_{1 \le i \le n} 2M_{c_i}
    \]
    로 두면 임의의 \(x \in [-\pi, \pi]\)에 대해
    \[
        \abs{f^{(n)}(x)} \le Cn!R^n
    \]
    이 성립한다.
\end{proof}

\begin{lem} \label{15.3.10}
    \([0, \infty]\) 안의 값을 가지는 수열 \((c_n)_{n \in \NN}\)이 어떤 고정된 양수 \(R > 0\)에 대하여
    \[
        c_n \ge \frac{1}{k!} \paren{\frac{n}{R}}^k \quad (n, k \in \NN)
    \]
    이면
    \[
        c_n \ge \frac{1}{R+1} \exp \paren{\frac{n}{R+1}} \quad (n \in \NN)
    \]
    이다.
\end{lem}
\begin{proof}
    \begin{align*}
        \frac{1}{R+1} \exp \paren{\frac{n}{R+1}} &= \frac{1}{R+1} \sum_{k=0}^\infty \frac{1}{k!} \paren{\frac{n}{R+1}}^k\\
        &= \sum_{k=0}^\infty \frac{1}{k!} \paren{\frac{n}{R}}^k \cdot \frac{R^k}{(R+1)^{k+1}}\\
        &\le c_n \sum_{k=0}^\infty \frac{R^k}{(R+1)^{k+1}} = c_n.
    \end{align*}
\end{proof}

이로부터 다음과 같은 결론을 얻는다.

\begin{thm}
    \(f : \RR \to \CC\)가 해석주기함수이면 적당한 \( 0 < r < 1\)이 존재하여, \(\abs{n} \to \infty\)일 때 \(\hat{f}(n) = O(r^{\abs{n}})\)이다.
\end{thm}

\begin{proof}
    보조정리 \ref{15.3.9}에 의해 다음을 만족하는 \(C, R > 0\)이 존재한다.
    \[
        \norm{f^{(k)}}_{\sup} \le C k! R^k \quad (k \in \NN)
    \]
    이때 각 \(k\)에 대해 \(\widehat{f^{(k)}}(n) = (in)^k \hat{f}(n)\)이므로
    \[
        \abs{\hat{f}(n)} = \frac{1}{n^k} \abs{\widehat{f^{(k)}}(n)} \le \frac{1}{2\pi} \int_{-\pi}^\pi \abs{f^{(k)}(t)} dt \le Ck! \paren{\frac{R}{\abs{n}}}^k \quad (n \in \ZZ, k \in \NN)
    \]
    이다. 이제 \(c_n := 1/\abs{\hat{f}(n)}\)으로 두고(단 \(1/0\)은 \(\infty\)로 정의) 보조정리 \ref{15.3.10}을 적용하면
    \[
        \abs{\hat{f}(n)} \le C(R+1) \exp \paren{-\frac{\abs{n}}{R+1}} \quad (n \in \ZZ)
    \]
    을 얻는다. 여기서
    \[
        r := \exp \paren{-\frac{1}{R+1}}
    \]
    로 두면 원하는 부등식을 얻는다.
\end{proof}


\chapter*{References}
\addcontentsline{toc}{chapter}{References}
\begin{enumerate}[label={[\arabic*]}]
    \item\label{ref1} 김성기, 김도한, 계승혁. \emph{해석개론}. 서울대학교출판문화원. 2011.
    \item\label{ref2} W. Rudin. \emph{Principles of Mathematical Analysis}. McGraw-Hill. 1976.
    \item\label{ref3} 이인석. \emph{선형대수와 군}. 서울대학교출판문화원. 2015.
    \item\label{ref4} J. R. Munkres, \emph{Topology}. Prentice Hall. 2000.
\end{enumerate}

\end{document}