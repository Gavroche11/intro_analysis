\documentclass[12pt]{article}

\usepackage[a4paper,left=20mm,right=20mm,top=40mm,bottom=40mm]{geometry}
\usepackage[utf8]{inputenc}
\usepackage{amsmath, amsthm, amsfonts, kotex, enumitem, setspace, hyperref}
\setstretch{1.4}


\theoremstyle{definition}
\newtheorem{thm}{Theorem}[section]
\newtheorem{cor}[thm]{Corollary}
\newtheorem{lem}[thm]{Lemma}
\newtheorem{prop}[thm]{Proposition}
\newtheorem{defn}[thm]{Definition}
\newtheorem{notn}[thm]{Notation}
\newtheorem{obs}[thm]{Observation}
\newtheorem*{rem}{Remark}
\newtheorem*{ex}{Example}

\def\NN{\mathbb{N}}
\def\ZZ{\mathbb{Z}}
\def\QQ{\mathbb{Q}}
\def\RR{\mathbb{R}}
\def\CC{\mathbb{C}}
\def\eps{\epsilon}

\newcommand{\abs}[1]{\left\vert#1\right\vert}
\newcommand{\norm}[1]{\left\vert\left\vert#1\right\vert\right\vert}
\newcommand{\inner}[2]{\langle #1, #2\rangle}

\title{해석개론 이야기}
\author{포만한 수학하는원태인}

\begin{document}
\maketitle

\tableofcontents
\newpage

원래 다른 사이트에서 올리던 글을, 사정이 생겨 여기서 계속 이어 올리게 되었습니다.

1년 동안 공부한 해석개론의 내용을 [\ref{ref1}]\과 [\ref{ref2}]의 내용을 중심으로 복습하는 김에, {\LaTeX} 연습도 좀 하고, (정말 낮은 확률로) 여기에 관심 있는 독자가 존재한다면 재미있는 내용을 전달하기 위해서 쓰는 글입니다. 저도 일개 학부생이니 오류나 오타가 있을 수 있으며, 같은 대상을 지칭하는 영어와 한국어 용어도 혼용하고 있고, 한번에 쓰는 글이 아니므로 서술 방식과 문체도 글 내부에서 상당히 다를 수 있습니다.

\section{Completeness of \(\RR\)}

해석개론의 출발점은 실수 집합 \(\RR\)의 성질을 규명하는 것입니다. 이를 위해서 몇 가지 정의가 필요합니다.

\subsection{Field}

	\begin{defn}
		집합 \(F\)에 이항 연산 \(+\)와 \(\cdot\)이 정의되어\footnote{집합 \(X\)에 이항 연산 \(*\)이 정의되었다는 것은 함수 \(*: X \times X \rightarrow X\)가 잘 정의되었다는 뜻. 다시 말해서, 임의의 \(x, y \in F\)에 대해 \(*(x, y) = x * y\)가 \(X\) 안의 유일한 원소로 결정된다는 뜻이다.} 있을 때, 다음 (F1)-(F9)를 만족하면 \((F, +, \cdot)\) 또는 간단하게 \(F\)를 \textbf{field(체)}라고 한다. \(+\)와 \(\cdot\)을 각각 덧셈, 곱셈이라고 한다.
		\begin{enumerate} [label=(F\arabic*), leftmargin=2\parindent]
			\item
			\(x + (y+z) = (x+y)+z\) for \(x, y, z \in F\).
			\item
			\(0 \in F\)가 존재하여 [\(x+0=0+x=x\) for \(x \in F\)].
			\item
			\(x \in F\)에 대하여 \(-x \in F\)가 존재하여 \mbox{\(x+(-x)=(-x)+x=0\).}
			\item
			\(x+y=y+x\) for \(x, y \in F\).
			\item
			\(x(yz)=(xy)z\) for \(x, y, z \in F\).
			\item
			\(1 \in F \backslash \{0\}\)이 존재하여 [\(x\cdot 1=1 \cdot x=x\) for \(x \in F\)].
			\item
			\(x \in F \backslash \{0\}\)에 대하여 \(x^{-1} \in F\)가 존재하여 \(xx^{-1}=x^{-1}x=1\).
			\item
			\(xy=yx\) for \(x, y \in F\).
			\item
			\(x(y+z)=xy+xz, (x+y)z = xz+yz\) for \(x, y, z \in F\).
		\end{enumerate}
	\end{defn}
	
다음이 성립하는 것을 관찰할 수 있습니다.

	\begin{prop} \label{field addition}
		Field \(F\)에 대하여 다음이 성립한다(\(x, y, z \in F\)).
		\begin{enumerate} [label=(\alph*), leftmargin=2\parindent]
			\item
			\(x + y = x + z\)이면 \(y = z\).
			\item
			\(y \in F\)가 [\(x + y = y + x = x\) for \(x \in F\)]를 만족하면 \(y = 0\).\footnote{사실 이것이 없으면 (F3)은 non-sense.}
			\item
			\(x + y = y + x = 0\)이면 \(y = -x\).
			\item
			For \(x \in F\), \(-(-x) = x\).
		\end{enumerate}
	\end{prop}
	\begin{proof}
		\quad\\
		(a): \(y = 0 + y = (-x + x) + y = -x + (x + y) = -x + (x + z) = (-x + x) + z = 0 + z = z\).\\
		(b): \(x + y = x = x + 0\)이므로, (a)에 의해 \(y=0\).\\
		(c): \(x + y = 0 = x + (-x)\)이므로, (a)에 의해 \(y = -x\).\\
		(d): \(-x + x = x + (-x) = 0\)이므로, (c)에 의해 \(x = -(-x)\).
	\end{proof}

	\begin{prop} \label{field mult}
		Field \(F\)에 대하여 다음이 성립한다(\(x, y, z \in F\)).
		\begin{enumerate} [label=(\alph*), leftmargin=2\parindent]
			\item
			\(xy = xz\)이면 \(y = z\). (\(x, y, z \in F, x \neq 0\))
			\item
			\(y \in F\)가 [\(xy = yx = x\) for \(x \in F\)]를 만족하면 \(y = 1\).\footnote{사실 이것이 없으면 (F7)은 non-sense.}
			\item
			\(xy = yx = 1\)이면 \(y = x^{-1}\). (\(x, y \in F, x \neq 0\))
			\item
			For \(x \in F \backslash \{0\}\), \((x^{-1})^{-1} = x\).
		\end{enumerate}
	\end{prop}
	\begin{proof}
		Proposition \ref{field addition}의 증명과 거의 같다.
	\end{proof}

	\begin{prop}
		Field \(F\)에 대하여 다음이 성립한다(\(x, y \in F\)).
		\begin{enumerate} [label=(\alph*), leftmargin=2\parindent]
			\item
			\(0x = 0\).
			\item
			\(xy = 0\)이면 \(x = 0\) or \(y = 0\).
			\item
			\((-1)x = -x\).
			\item
			\((-x)y = x(-y) = -(xy)\)이고 \((-x)(-y)=xy\).
		\end{enumerate}
	\end{prop}
	\begin{proof}
		\quad\\
		(a): \(0x + 0 = 0x = (0+0)x = 0x + 0x\)이므로 Proposition \ref{field addition} (a)에 의해.\\
		(b): \(xy = 0\)이면서 \(x \neq 0, y \neq 0\)이면 \(x^{-1}, y^{-1}\)이 존재하므로
		\begin{gather*}
			0 = 0(y^{-1}x^{-1}) = (xy)(y^{-1}x^{-1}) = xx^{-1} = 1
		\end{gather*}
		이므로 모순.\\
		(c): \(x + (-x) = 0 = 0x = (1 + (-1))x = 1x + (-1)x = x + (-1)x\)이므로 Proposition \ref{field addition} (a)에 의해.\\
		(d): (c)에 의해 당연.
	\end{proof}

	\begin{ex}
		\(\QQ, \RR, \CC\) 등은 모두 field입니다. 주로 \(\RR\)을 다룰 것입니다.
	\end{ex}

\subsection{Order}

어떤 수가 다른 수보다 `크다'고 하는 것은 어떤 의미일까요? 순서의 개념을 일반화합니다.

	\begin{defn}
		집합 \(S\)에 주어진 관계(relation) \(>\)가 다음 조건을 만족하면 \(>\)를 \textbf{strict partial order}라고 한다.
		\begin{enumerate} [label=(\alph*), leftmargin=2\parindent]
			\item
			\(x > x\)인 \(x \in S\)는 존재하지 않는다.
			\item
			\(x > y\)이고 \(y > z\)이면 \(x > y > z\). (\(x, y, z \in S\))
		\end{enumerate}
		Strict partial order \(>\)가 다음 조건까지 만족하면 \(>\)를 \textbf{strict total order}라고 한다.
		\begin{enumerate}[label=(\alph*), leftmargin=2\parindent]
			\setcounter{enumi}{2}
			\item
			\(x, y \in S\)에 대해 \(x > y, x = y, y > x\) 중 오직 하나만 참.
		\end{enumerate}
	\end{defn}

	\begin{ex}
		\(\RR, \QQ, \ZZ, \NN\) 등에 주어진 일상적인 의미의 \(>\)가 strict total order임은 쉽게 확인할 수 있습니다.
	\end{ex}

	\begin{rem}
		편의를 위해 \(<, \ge, \le\)의 기호도 우리가 생각하는 의미 그대로 사용하기로 합니다.
	\end{rem}

	\begin{defn}
		Strict total order \(>\)가 주어진 field \(F\)가 다음 조건을 만족할 때 \(F\)를 \textbf{ordered field(순서체)}라고 한다.
		\begin{enumerate} [label=(\alph*), leftmargin=2\parindent]
			\item
			\(x < y\)이면 \(x + z < y + z\). (\(x, y, z \in F\))
			\item
			\(x > 0, y > 0\)이면 \(xy > 0\). (\(x, y \in F\))
		\end{enumerate}
	\end{defn}

	\begin{ex}
		\(\RR\)이나 \(\QQ\)에 일상적인 의미의 \(>\)를 주면 ordered field가 됩니다.
	\end{ex}

	\begin{defn} \label{bdd in R}
		Ordered field \(F\)의 부분집합 \(S \subseteq F\)가 \textbf{bounded above(위로 유계)}라는 것은 \(\beta \in F\)가 존재하여 [\(x \le \beta\) for all \(x \in S\)]라는 것이다. 이때 \(\beta\)를 \textbf{upper bound(상계) of \(S\)}라고 한다. 부등호 방향을 반대로 하여 \textbf{bounded below(하계)}와 \textbf{lower bound(하계)}도 정의한다. \(S\)가 bounded above and bounde below이면 \(S\)는 \textbf{bounded(유계)}라고 한다.
	\end{defn}
	
	\begin{defn}
		Ordered field \(F\)의 bounded above subset \(S\)에 대하여 \(\alpha \in F\)가 다음 조건을 만족할 때 \(\alpha\)를 \textbf{least upper bound=supremum(최소상계=상한) of \(S\)}라고 하고, \(\alpha = \sup S\)로 쓴다.
		\begin{enumerate} [label=(\alph*), leftmargin=2\parindent]
			\item
			\(\alpha\) is an upper bound of \(S\).
			\item
			If \(\gamma \in F\) is an upper bound of \(S\), then \(\alpha \le \gamma\).
		\end{enumerate}
	\end{defn}

\subsection{Complete Ordered Field}

Ordered field \(F\)의 bounded above subset \(S \subseteq F\)의 supremum은 항상 존재할까요? 이 질문에 긍정적인 답을 하게 할 수 있게 하는 \(F\)가 우리의 관심사입니다.

	\begin{defn} \label{comp axiom}
		다음 두 성질을 만족하는 ordered field \(F\)는 \textbf{least-upper-bound property} 또는 \textbf{completeness axiom(완비성공리)}을 만족시킨다고 하며 \(F\)를 \textbf{complete ordered field(완비순서체)}라고 한다.
		\begin{enumerate} [label=(\alph*), leftmargin=2\parindent]
			\item
			Nonempty bounded above subset \(S \subseteq F\) has a supremum in \(F\).
			\item
			Nonempty bounded below subset \(S \subseteq F\) has a infimum in \(F\).
		\end{enumerate}
	\end{defn}

사실 (a)와 (b)는 동치입니다.

	\begin{prop}
		Definition \ref{comp axiom}의 (a)와 (b)는 동치이다.
	\end{prop}
	\begin{proof}
		(a)$\Rightarrow$(b)만 보이면 역은 마찬가지 방법으로 보일 수 있다. (a)를 가정하고, nonempty bounded above subset \(S \subseteq F\)와 \(S\)의 모든 lower bound들의 집합 \(T\)를 생각하자. \(S\)가 bounded below이므로 \(T\)는 공집합이 아니고, \(S\)의 아무 원소를 가져오면 그 원소는 \(T\)의 upper bound이므로 \(T\)는 nonempty bounded above이다. 가정 (a)에 의해 \(T\)는 supremum \(\alpha \in F\)를 가진다. 우리의 주장: \(\alpha\)는 \(S\)의 infimum이다.\\
		임의의 \(x \in S\)에 대해 \(x\)는 \(T\)의 upper bound이므로 \(\alpha \le x\)이다. 따라서 \(x\)는 \(S\)의 lower bound이다. 한편 \(\gamma \in F\)가 \(S\)의 lower bound라고 하면 \(T\)의 정의에 의해 \(\gamma \in T\)인데, \(\alpha = \sup T\)이므로 \(\gamma \le \alpha\)이다. 따라서 \(\alpha = \inf S\)이다.
	\end{proof}

모든 ordered field가 complete인 것은 아닙니다. 대표적으로 \(\QQ\)는 complete가 아닌데, \(\{x \in \QQ: x^2 < 2\}\)는 nonempty bounded above subset of \(\QQ\)이지만 \(\QQ\)에서 supremum을 가지지 않음을 쉽게 확인할 수 있습니다.

그렇다면 실제로 complete ordered field가 존재하는지가 궁금한데, 그 답은 긍정적입니다.

	\begin{thm}
		Complete ordered field \(F\)가 존재하고, 그 \(F\)를 실수체 \(\RR\)이라고 부른다.
	\end{thm}

수학에서 어떤 것의 존재성을 보일 때는 다른 정리의 도움을 받거나, 아니면 실제로 그 대상을 구성(construct)해야 합니다.

우리가 수 체계를 배웠던 과정을 생각해보면, 자연수에서 정수, 유리수까지는 나름 자연스러운 확장이었고, 실수에서 복소수도 자연스러운 확장인데 – 유리수와 실수 사이에 어떤 gap이 있었습니다. 적어도 저한테는 그랬습니다. 제곱해서 2가 되는 유리수가 없다는 것으로 유리수체가 어떤 의미로 ‘불완전함’을 설명하곤 하지만, 이는 유리수체에서 실수체로의 확장이라기보다는 유리수체에서 field of algebraic number(algebraic number, 대수적 수: 유리계수 다항식의 근)로의 확장에 가깝습니다. 이제서야 우리는 실수가 어떻게 만들어졌는지 알았습니다. \textbf{유리수체 \(\QQ\)의 빈틈을 채워 넣어서 completeness axiom을 만족하도록 실수체 \(\RR\)을 직접 ‘구성’한 것입니다.} 그래서 실수체가 complete인 것은 증명의 대상이 아닌데, 왜냐하면 그것을 만족하도록 실수체를 만들었기 때문입니다.

\(\QQ\)로부터 \(\RR\)를 구성하여 정의하는(실수의 구성적 정의) 대표적인 방법에는 두 가지가 있는데, 하나는 Cantor의 방법이고 하나는 Dedekind의 방법입니다. 다른 글에서 이를 다룰 기회가 있을 것입니다.

여기서 드는 한 가지 의문은, 두 가지 방법으로 만든 실수체가 동일한가 하는 것입니다. 좀 더 일반적으로, complete ordered field는 유일한가라는 질문을 던질 수 있는데, 실제로 그렇습니다. 이 증명 역시 다른 글에서 다룰 기회가 있을 것입니다.

	\begin{thm}
		The complete ordered field is unique up to isomorphism. i.e., for any complete ordered fields \(F_1\) and \(F_2\), there exists a bijection \(f: f_1 \rightarrow F_2\) satisfying:
		\begin{enumerate} [label=(\alph*), leftmargin=2\parindent]
			\item
			\(x < y \Longrightarrow f(x) < f(y) \quad (x, y \in F_1)\)
			\item
			\(f(x + y) = f(x)+f(y), f(xy)=f(x)f(y) \quad (x, y \in F_1)\)
		\end{enumerate}
		따라서 \(\RR\)은 유일한 complete ordered field이다.
	\end{thm}

\newpage

\section{Sequence} \label{sec sequence}

\subsection{Metric Space} \label{sec metric space}

수열의 수렴을 이야기하기 위해 거리의 개념을 도입하려고 합니다.

	\begin{defn} \label{metric}
		집합 \(X\)와 함수 \(d: X \times X \rightarrow \RR\)가 주어졌을 때 임의의 \(x, y, z \in X\)에 대해 다음이 성립하면 \((X, d)\) 또는 간단하게 \(X\)가 \textbf{metric space(거리공간)}라고 하고 \(d\)를 \textbf{metric on \(X\)}라고 한다.
		\begin{enumerate} [label=(M\arabic*), leftmargin=2\parindent]
			\item
			\(d(x, y) = 0\) if and only if \(x = y\).
			\item
			\(d(x, y) = d(y, x)\).
			\item
			\(d(x, z) \le d(x, y) + d(y, z)\)
		\end{enumerate}
	\end{defn}

	\begin{prop}
		Metric space \(X\)에서 \(d(x, y) \ge 0\) for all \(x, y \in X\).
	\end{prop}
	\begin{proof}
		\(0 = d(x, x) \le d(x, y) + d(y, x) = 2d(x, y)\).
	\end{proof}

	\begin{ex}
		\quad
		\begin{enumerate} [label=(\alph*), leftmargin=2\parindent]
			\item
			\(\RR\) 또는 \(\CC\)에서 \(d(x, y) = \abs{x - y}\)로 정의하면 \(d\)는 (M1)-(M3)을 만족하므로 \((\RR, d), (\CC, d)\)는 metric space.
			\item
			평면이나 공간에서 우리가 일상적으로 생각하는 `거리'는 (M1)-(M3)을 만족하므로\footnote{Definition \ref{def euc space} 참조.} 평면과 공간은 metric space.
			\item
			아무 집합 \(X\)에서 \(x, y \in X\)에 대해 \(d(x, y)\)를 \(x = y\)일 때 0, \(x \neq y\)일 때 1로 정의하면 (M1)-(M3)을 만족하므로 \((X, d)\)는 metric space.
		\end{enumerate}
	\end{ex}

\subsection{Convergence of a Sequence}

	\begin{defn} \label{converge}
		집합 \(X\)에 대하여 \(\NN\)(또는 \(\ZZ\))에서 \(X\)로 가는 함수를 \textbf{sequence(수열) in \(X\)}라고 하고 \((x_n)\)과 같이 쓴다.\\
		\(X\)가 metric space일 때 \(X\) 안의 sequence \((x_n)\)이 \(x \in X\)로 \textbf{converge(수렴)}한다는 것은 임의의 양수 \(\eps > 0\)에 대해 다음을 만족시키는 \(N \in \NN\)이 존재하는 것이다.
		\begin{gather*}
			n \ge N \Longrightarrow d(x_n, x) < \eps
		\end{gather*}
		이때 \(\lim_{n \rightarrow \infty}x_n = x, \lim_n x_n = x, x_n \rightarrow x\)와 같이 쓴다.
	\end{defn}

	\begin{thm} \label{thm metric limit unq}
		Metric space \(X\) 안의 sequence \((x_n)\)이 \(x \in X\)로 수렴하고 \(x' \in X\)로 수렴하면 \(x = x'\).
	\end{thm}
	\begin{proof}
		\(x \neq x'\)라고 가정하면 metric의 정의에 의해 \(d(x, x') > 0\)이다. \(r = d(x, x') / 2\)라고 하자. 극한의 정의에 의해 다음을 만족하는 \(N_1, N_2\)가 존재한다.
		\begin{gather*}
			n \ge N_1 \Longrightarrow d(x_n, x) < r\\
			n \ge N_2 \Longrightarrow d(x_n, x') < r
		\end{gather*}
		\(N = \max\{N_1, N_2\}\)라고 하면 \(d(x, x') \le d(x, x_N) + d(x, x_N') < 2r = d(x, x')\)이므로 모순.
	\end{proof}

	\begin{defn} \label{bdd}
		Metric space \(X\)의 부분집합 \(S \subseteq X\)가 \textbf{bounded(유계)}라는 것은 어떤 양수 \(M > 0\)이 존재하여 [\(d(x, y) < M\) for all \(x, y \in S\)]라는 것이다.\\
		Sequence \((x_n)\) in \(X\)가 \textbf{bounded}라는 것은 \(X\)의 부분집합 \(\{x_n\}_{n \in \NN}\)이 bounded라는 것이다.
	\end{defn}
	
	\begin{rem}
		\(X = \RR\)일 때 Definition \ref{bdd in R}에서의 bounded의 정의와 Definition \ref{bdd}의 정의는 동치입니다(증명 생략).
	\end{rem}

	\begin{prop} \label{conv bdd}
		Metric space \(X\) 안의 sequence \((x_n)\)이 수렴하면 \((x_n)\)은 유계이다.
	\end{prop}
	\begin{proof}
		\(x_n \rightarrow x\)라고 하면 극한의 정의에 의해 \(n \ge N \Rightarrow d(x_n, x) < 1\) for some \(N\)이다. 이때 \(\{x_n\}_{n < N}\)은 유한집합이므로 유계이고, \(\{x_n\}_{n \ge N}\)의 아무 점 2개를 잡아도 두 점 사이의 거리는 2보다 작으므로 유계이다. 유계 부분집합 2개의 합집합은 유계이므로 \(\{x_n\}_{n \in \NN}\)도 유계이다.
	\end{proof}

이제부터 Section \ref{sec sequence}\가 끝날 때까지 metric space \(X\)를 \(d(x, y) = \abs{x - y}\)가 주어진 \(\RR\)로 한정하고 이 안의 실수열만 생각합니다.

다음 성질을 증명하는 것은 연습문제로 남깁니다.

	\begin{prop}
		실수열 \((x_n\)과 \((y_n)\)이 각각 \(x, y \in \RR\)로 수렴할 때 다음이 성립한다.
		\begin{enumerate} [label=(\alph*), leftmargin=2\parindent]
			\item
			\(x_n + y_n \rightarrow x + y\).
			\item
			\(cx_n \rightarrow cx\) for \(c \in \RR\).
			\item
			\(x_ny_n \rightarrow xy\).
			\item
			\(x_n/y_n \rightarrow x/y\), if \(y_n \neq 0\) for all \(n\) and \(y \neq 0\).
		\end{enumerate}
	\end{prop}

\subsection{Monotonic Sequence}

Definition \ref{converge}\를 보면, 어떤 수열 \((x_n)\)의 수렴을 보이기 위해서 \textbf{수렴할 것으로 예상되는 점 \(x\)를 먼저 찾아야} 합니다. 그런데 그 \(x\)를 찾는 것이 쉽지 않을 때가 많습니다. 그래서 우리는 어디로 수렴하는지 모르는 상태에서 수열의 수렴을 판정할 수 있는 많은 방법들을 개발해야 하는데, 단조수열은 그 중 첫 번째입니다.

	\begin{defn}
		실수열 \((x_n)\)이 \textbf{monotonically increasing=nondecreasing sequence(단조증가수열)}이라는 것은 \(x_n \le x_{n+1}\) for all \(n\)이라는 것이다. 마찬가지 방법으로 \textbf{monotonically decreasing=nonincreasing sequence(단조감소수열)}도 정의한다. Monotonically increasing or monotonically decreasing sequence를 \textbf{monotonic sequence(단조수열)}이라고 한다.
	\end{defn}

다음은 단조수열이 \(\RR\)에서 수렴할 필요충분조건입니다.

	\begin{thm} [Monotone Convergence Theorem, 단조수렴정리]
		실수열 \((x_n)\)이 단조수열일 때, \((x_n)\) is convergent if and only if \((x_n)\) is bounded.
	\end{thm}
	\begin{proof}
		($\Rightarrow$) 방향은 Proposition \ref{conv bdd}에 의해 성립하므로 그 역만 보이면 된다. 일반성을 잃지 않고 \((x_n)\)이 유계 단조증가수열이라고 가정하자. 그러면 \(S = \{x_n\}_{n \in \NN}\)은 nonempty bounded above subset of \(\RR\)이므로, \textbf{\(\RR\)의 완비성에 의해} \(S\)의 supremum \(x \in \RR\)가 존재한다. 주장: \(x_n \rightarrow x\).\\
		임의의 \(\eps > 0\)에 대해 \(x - \eps\)은 \(S\)의 upper bound가 아니므로, \(x_N > x - \eps\)인 \(N\)이 존재한다. \((x_n)\)은 단조증가수열이므로 \(n \ge N\)에 대해 \(x - \eps < x_N \le x_n \le x\). 따라서 \((x_n)\)은 \(x\)로 수렴한다.
	\end{proof}
	\begin{rem}
		\(\RR\)의 완비성의 핵심적인 결과들 중 하나입니다. 이 정리는 complete가 아닌 field, 예를 들면 \(\QQ\)에서 성립하지 않습니다.
	\end{rem}

\subsection{Upper Limit and Lower Limit} \label{sec limsup}

	\begin{ex}
		실수열 \((x_n)\)을 다음과 같이 정의합니다.
		\begin{gather*}
			x_n = 
			\begin{cases}
				\frac{1}{n}, & \text{if } n \text{ is odd}\\
				-n, & \text{if } n \text{ is even}
			\end{cases}
		\end{gather*}
		\((x_n)\)이 수렴하지 않는 것은 명백합니다. 홀수항만 취해서 보면 이 수열의 `일부분'은 0으로 수렴하는 단조감소수열입니다. 이 0에 어떤 의미를 부여하려고 하는 것이 Section \ref{sec limsup}의 목표입니다.
	\end{ex}

먼저 표기법 하나.

	\begin{notn}
		실수열 \((x_n)\)에 대하여, 임의의 \(M > 0\)에 대해 적당한 \(N \in \NN\)이 존재하여 \(n \ge N \Rightarrow x_n > M\)을 만족시키면 \(\lim_n x_n = \infty\)로 쓴다. 마찬가지 방법으로 \(\lim_n x_n = -\infty\)로 쓴다.\\
		Nonempty \(S \subseteq \RR\)가 위로 유계가 아닐 때 \(\sup S = \infty\), 아래로 유계가 아닐 때 \(\inf S = -\infty\)로 쓴다.\\
		\(x_n \rightarrow \alpha\) for \(\alpha \in \RR \cup \{\pm \infty\}\)라는 것은 \((x_n)\)이 \(\alpha \in \RR\)로 수렴하거나, \(\lim_n x_n = \infty\) 또는 \(\lim_n x_n = -\infty\)라는 것이다. (즉, 진동하지 않는다는 의미.)
	\end{notn}

	\begin{obs} \label{obs sup}
		Nonempty subset \(A, B \subseteq \RR\)에 대하여 \(A \subseteq B\)이면 \(\sup A \le \sup B\)이고 \(\inf A \ge \inf B\). \(\sup\)이나 \(\inf\)의 값이 \(\pm\infty\)인 경우에도 성립한다.
	\end{obs}
	\begin{proof}
		당연하다.
	\end{proof}

	\begin{defn} \label{def limsup}
		실수열 \((x_n)\)의 \textbf{upper limit(상극한)}을 다음과 같이 정의하고, 그 값을 \(\limsup_n x_n \in \RR \cup \{\pm \infty\}\)으로 쓴다.
		\begin{enumerate} [label=(\alph*), leftmargin=2\parindent]
			\item
			\((x_n)\)이 위로 유계가 아니면 \(\limsup_n x_n = \infty\)로 정의.
			\item
			\((x_n)\)이 위로 유계이면 각 자연수 \(n\)에 대하여 집합 \(S_n = \{x_k\}_{k \ge n}\)을 생각한다. 이때 각 \(S_n\)은 위로 유계이므로 supremum이 유한한 실수로 존재한다. 이 값을 \(y_n\)으로 정의하자. 즉 \(y_n = \sup S_n\). Observation \ref{obs sup}에 의해 \((y_n)\)은 단조감소수열이다. 이때 \(\limsup_n x_n = \lim_n y_n \in \RR \cup \{-\infty\}\)로 정의한다. (\((y_n)\)이 아래로 유계이면 \(\lim_n y_n\)은 유한한 실수이고, 그렇지 않으면 \(-\infty\)이다.)
		\end{enumerate}
		마찬가지 방법으로 \textbf{lower limit(하극한)}을 정의하고 그 값을 \(\liminf_n x_n \in \RR \cup \{\pm \infty\}\)으로 쓴다.
	\end{defn}

도대체 이게 무슨 소리인가? 필자도 이 개념을 이해하는 데 상당히 많은 시간이 걸렸습니다. 다음 정리는 이 정의를 좀 더 구체적으로 설명해 줍니다.

	\begin{thm} \label{thm limsup}
		실수열 \((x_n)\)과 \(\alpha \in \RR\)에 대하여 \(\limsup_n x_n = \alpha\)일 필요충분조건은 다음 (a), (b)가 성립하는 것이다.
		\begin{enumerate} [label=(\alph*), leftmargin=2\parindent]
			\item
			임의의 \(\eps > 0\)에 대해 적당한 \(N \in \NN\)이 존재하여 \(n \ge N \Rightarrow x_n < \alpha + \eps\).
			\item
			임의의 \(\eps > 0\)에 대해 \(x_n > \alpha - \eps\)를 만족하는 \(n\)이 무수히 많이 존재한다.
		\end{enumerate}
	\end{thm}
	\begin{proof}
		($\Leftarrow$): 먼저 \(\limsup_n x_n = \alpha\)라고 가정하고 \(\eps > 0\)이 주어졌다고 하자. Definition \ref{def limsup}에서와 같이 단조감소수열 \((y_n)\)을 정의하면 \(\lim_n y_n = \alpha\)이므로 \(y_N < \alpha + \eps\)인 \(N\)이 존재한다. 이때 \(n \ge N\)이면 \(x_n \le y_N < \alpha + \eps\)이므로 (a)가 성립한다. 이제 (b)를 부정하여 \(x_n > \alpha + \eps\)을 만족하는 \(n\)이 유한 개만 존재한다고 가정하자. 그러면 충분히 큰 자연수 \(K\)가 존재하여 \(n \ge K \Rightarrow x_n \le \alpha - \eps\), 즉 \(y_K \le \alpha - \eps\)이다. 그런데 \((y_n)\)은 단조감소수열이므로 \(\alpha \le y_K = \alpha - \eps\)이고 따라서 모순이다.\\
		($\Rightarrow$): 이제 (a)와 (b)를 가정하고 \(\eps > 0\)이 주어졌다고 하자. (a)에 의해 \((x_n)\)은 위로 유계이므로 Definition \ref{def limsup}에서와 같이 \((y_n)\)이 잘 정의되고, \(\lim_n y_n \le \alpha + \eps\)이다. 한편 (b)에 의해, 모든 \(n\)에 대해 [\(x_n > \alpha - \eps\) for some \(m > n\)]이므로 \(y_n \ge \alpha - \eps\)이다. 따라서 \((y_n)\)은 유계 단조수열이므로 수렴하고 \(\alpha- \eps \le \lim_n y_n \le \alpha + \eps\)이다. 그런데 \(\eps > 0\)은 임의의 양수이므로 \(\limsup_n x_n = \lim_n y_n = \alpha\)이다.
	\end{proof}

하극한에 대해 마찬가지의 정리를 state하고 증명하는 것은 연습문제로 남깁니다.

상극한과 하극한이 유용한 한 가지 이유는 다음 따름정리 때문입니다.

	\begin{cor}
		실수열 \((x_n)\)과 \(\alpha \in \RR\)에 대하여 \(\lim_n x_n = \alpha\) if and only if \(\limsup_n x_n = \liminf_n x_n = \alpha\).
	\end{cor}
	\begin{proof}
		Theorem \ref{thm limsup}\과 그 하극한 version에 의해 자명.
	\end{proof}

	\begin{cor}
		\(n = 0, 1, \ldots\)에서 정의된 실수열 \((s_n)\)에 대하여
		\begin{gather*}
			\sigma_n = \frac{1}{n}\sum_{k=0}^{n-1}s_k
		\end{gather*}
		로 정의하자. \((s_n)\)이 \(s \in \RR\)로 수렴하면 \((\sigma_n)\)도 \(s\)로 수렴한다.
	\end{cor}
	\begin{proof}
		일반성을 잃지 않고 \(s = 0\)이라고 가정할 수 있다. 양수 \(\eps > 0\)을 고정하고, 다음을 만족시키는 \(N\)을 찾는다.
		\begin{gather*}
			n \ge N \Longrightarrow \abs{s_n} < \eps
		\end{gather*}
		그러면 \(n > N\)에 대해
		\begin{align*}
			\abs{\sigma_n} = \frac{1}{n}\abs{\sum_{k=0}^{n-1}s_k} &\le \frac{1}{n}\abs{\sum_{k=0}^{N}s_k} + \frac{1}{n}\sum_{k=N+1}^{n}\abs{s_k}\\
			&< \frac{1}{n}\abs{\sum_{k=0}^{N}s_k} + \frac{(n-N)\eps}{n}\\
			&<  \frac{1}{n}\abs{\sum_{k=0}^{N}s_k} + \eps
		\end{align*}
		이다. 이제 양변에 \textbf{상극한}을 취한다. (바로 극한을 못 취하는 이유는 좌변이 수렴하는지를 아직 알지 못하기 때문이다. 상극한은 항상 \(\RR \cup \{\pm \infty\}\) 안에서 존재한다.) \(\abs{\sum_{k=0}^{N}s_k}\)은 \(n\)과 무관한 상수이므로,
		\begin{gather*}
			\limsup_n \abs{\sigma_n} \le \limsup_n \left ( \frac{1}{n}\abs{\sum_{k=0}^{N}s_k} + \eps \right )= \lim_n \left ( \frac{1}{n}\abs{\sum_{k=0}^{N}s_k} + \eps \right )= \eps
		\end{gather*}
		이다. 따라서 \(0 \le \liminf_n \abs{\sigma_n} \le \limsup_n \abs{\sigma_n} \le \eps\)인데, \(\eps > 0\)은 임의의 양수였으므로 \(\liminf_n \abs{\sigma_n} = \limsup_n \abs{\sigma_n} = 0\)이고 \(\lim_n \abs{\sigma_n} = \lim_n \sigma_n = 0\)이다.
	\end{proof}

\newpage

\section{Euclidean Space} \label{sec euc}
\subsection{Vector Space}

	\begin{defn}
		Field \(F\)에 대하여 집합 \(V\)에 덧셈과 \(F\)에 의한 스칼라곱이 정의되어 있을 때, 다음 (V1)-(V8)을 만족하면 \(V\)를 \textbf{\(F\)-vector space}라고 하고 V의 원소를 \textbf{vector}라고 한다.
		\begin{enumerate}[label=(V\arabic*), leftmargin=2\parindent]
			\item
			\((u+v)+w=u+(v+w)\) for \(u, v, w \in V\).
			\item
			\(0 \in V\)가 존재하여 [\(v+0=0+v=v\) for \(v \in V\)].
			\item
			\(v \in V\)에 대하여 \(-v \in V\)가 존재하여 \(v+(-v)=(-v)+v=0\).
			\item
			\(v+w=w+v\) for \(v, w \in V\).
			\item
			\(1v = v\) for \(v \in V\).
			\item
			\((ab)v = a(bv)\) for \(a, b \in F, v \in V\).
			\item
			\((a+b)v=av+bv\) for \(a, b \in F, v \in V\).
			\item
			\(a(v+w) = av + aw\) for \(a \in F, v, w \in V\).
		\end{enumerate}
	\end{defn}

다음 성질들을 증명하는 것은 연습문제로 남깁니다.

	\begin{prop}
		\(F\)-vector space \(V\)에 대하여 다음이 성립한다\(u, v, w \in V\).
		\begin{enumerate}[label=(\alph*), leftmargin=2\parindent]
			\item
			\(u + v = u + w\)이면 \(v = w\).
			\item
			\(w \in V\)가 [\(v + w = w + v = v\) for \(v \in V\)]를 만족하면 \(w = 0\).\footnote{사실 이것이 없으면 (V3)은 non-sense.}
			\item
			\(v + w = w + v = v\)이면 \(w = -v\).
			\item
			For \(v \in V\), \(-(-v) = v\).
		\end{enumerate}
	\end{prop}

	\begin{prop}
		\(F\)-vector space \(V\)에 대하여 다음이 성립한다\(a \in F, v, w \in V\).
		\begin{enumerate}[label=(\alph*), leftmargin=2\parindent]
			\item
			\(0v = v\).
			\item
			\(av = 0\)이면 \(a = 0\) or \(v = 0\).
			\item
			\((-1)v = -v\).
			\item
			\((-a)v = a(-v) = -(av)\)이고 \((-a)(-v) = av\).
		\end{enumerate}
	\end{prop}

당분간 다룰 vector space는 다음의 형태입니다.

	\begin{prop}
		Field \(F\)에 대하여 집합
		\begin{gather*}
			F^n = \{(x_1, \ldots, x_n): x_i \in F, i = 1, \ldots, n\}
		\end{gather*}
		에 덧셈과 스칼라곱을 다음과 같이 정의하면 \((c \in F, (x_1, \ldots, x_n), (y_1, \ldots, y_n) \in F^n)\)
		\begin{gather*}
			(x_1, \ldots, x_n) + (y_1, \ldots, y_n) = (x_1 + y_1, \ldots, x_n + y_n), \quad c(x_1, \ldots, x_n) = (cx_1, \ldots, cx_n)
		\end{gather*}
		\(F^n\)은 \(F\)-vector space이다.
	\end{prop}
	\begin{proof}
		시간이 남으면 한번 해 보자.
	\end{proof}


특히 \(F=\RR\)일 때, \(\RR^2\)와 \(\RR^3\)을 각각 \textbf{평면}, \textbf{공간}이라고 불렀습니다. \(\RR^n\) 공간은 우리의 `마음의 고향'이고, 해석개론의 주된 관심사 중 하나입니다. 우리의 목적은, metric space의 구조를 \(\mathbb{R}^n\)에 주는 것입니다.

\subsection{Inner Product Space}
고등학교 때 배웠던 \(\mathbb{R}^2\)와 \(\mathbb{R}^3\)에서의 내적을 임의의 \(\RR\)- 또는 \(\CC\)- vector space로 일반화합니다. Section \ref{sec euc}\이 끝날 때까지 아무 말이 없으면 \(V\)는 \(F\)-vector space이고 \(F\)는 \(\RR\) 또는 \(\CC\).

	\begin{defn}
		함수 \(\inner{\cdot}{\cdot}: V \times V \rightarrow F\)가 모든 \(u, v, w \in V, c \in F\)에 대해 (IP1)-(IP3)를 만족할 때, \(\inner{\cdot}{\cdot}\)를 \textbf{inner product on \(V\)}라고 하고, \((V, \inner{\cdot}{\cdot})\) 또는 간단하게 \(V\)를 \textbf{inner product space}라고 한다.
		\begin{enumerate}[label=(IP\arabic*), leftmargin=2\parindent]
			\item
			\(\inner{\cdot}{\cdot}\)는 첫 번째 성분에 대해 linear.\\
			즉, \(\inner{u+v}{w} = \inner{u}{w}+ \inner{v}{w}\)이고 \(\inner{cv}{w} = c\inner{v}{w}\).
			\item
			\(\inner{v}{w} = \overline{\inner{w}{v}}\)
			\item
			\(v \neq 0\)이면 \(\inner{v}{v} \in \mathbb{R}\)이고 \(\inner{v}{v} > 0\).
		\end{enumerate}
	\end{defn}

\(F=\RR\)이면, \(\inner{v}{w} = \inner{w}{v}\)이고(symmetric) \(\inner{\cdot}{\cdot}\)는 첫 번째와 두 번째 성분에 대해 각각 linear입니다(bilinear).\footnote {fancy하게 말해서 \(\inner{\cdot}{\cdot}\)는 \textbf{symmetric bilinear form}.}

	\begin{prop}
		\(x = (x_1, \ldots, x_n), y = (y_1, \ldots, y_n) \in \RR^n\)에 대하여
		\begin{gather*}
			\inner{x}{y} = x \cdot y = \sum_{i=1}^n x_i y_i
		\end{gather*}
		로 정의하면 \(\inner{\cdot}{\cdot}\)은 inner product이다. 이러한 inner product를 \textbf{dot product} on \(\RR^n\)이라고 한다.
	\end{prop}
	\begin{proof}
		시간이 남으면 한번 해 보자.
	\end{proof}

	\begin{ex}
		\(V\)의 inner product는 여러 가지 방법으로 줄 수 있습니다. 예를 들어 \(\RR^2\)에서
		\begin{gather*}
			\inner{(x_1, x_2)}{(y_1, y_2)} = 3x_1 y_1 + 2x_2 y_2
		\end{gather*}
		로 정의하면 \(\inner{\cdot}{\cdot}\)은 inner product on \(\RR^2\)가 됩니다.
	\end{ex}

고등학교 때 dot product에 대해 배운 다음 정리는 일반적인 inner product에 대해 성립합니다. 증명은 orthogonality의 개념을 공부한 이후로 미루겠습니다.

	\begin{thm}[Cauchy-Schwarz inequality]
		Inner product space \(V\)의 원소 \(v, w\)에 대하여 다음 부등식이 성립한다.
		\begin{gather*}
			\abs{\inner{v}{w}}^2 \le \inner{v}{v}\inner{w}{w}
		\end{gather*}
	\end{thm}

\subsection{Normed Vector Space}
이제 우리의 목표에 거의 근접했습니다. 다음은 vector space의 원소의 `크기'에 대한 정의입니다.

	\begin{defn}
		함수 \(\norm{\cdot}: V \rightarrow \mathbb{R}\)가 모든 \(v, w \in V, c \in F\)에 대해 (N1)-(N3)을 만족할 때, \(\norm{\cdot}\)를 \textbf{norm on \(V\)}라고 하고, \((V, \norm{\cdot})\) 또는 \(V\)를 \textbf{normed vector space}라고 한다.
		\begin{enumerate}[label=(N\arabic*), leftmargin=2\parindent]
			\item
			\(\norm{v+w} \le \norm{v}{w}\)
			\item
			\(\norm{cv} = \abs{c}\norm{v}\)
			\item
			\(\norm{v}=0\)이면 \(v=0\).
		\end{enumerate}
	\end{defn}
	
	\begin{prop}
		Normed vector space \(V\)에서 \(\norm{v} \ge 0\) for all \(v \in V\).
	\end{prop}
	\begin{proof}
		\(0 = \norm{v + (-v)} \le \norm{v} + \norm{-v} = 2\norm{v}\).
	\end{proof}


Norm이 주어지면 자동으로 metric이 주어집니다. 왜냐하면, 두 vector 사이의 거리는 두 vector의 차의 norm으로 정의할 수 있기 때문입니다.

	\begin{prop}
		Normed vector space \(V\)에 대해, \(d:V\times V \rightarrow \mathbb{R}\)을 \(d(v, w)=\norm{v-w}\)로 정의하면 \(d\)는 metric. 즉, normed vector space에 자연스러운 방식으로 metric을 정의할 수 있다.
	\end{prop}
	\begin{proof}
		(M1)-(M3) 모두 norm의 정의로부터 당연하다.
	\end{proof}

한편 inner product가 주어지면 자동으로 norm이 주어지는데, 이는 우리가 dot product로부터 vector의 길이를 정의하던 방법을 일반화한 것입니다.

	\begin{prop}
		Inner product space \(V\)에 대해, \(\norm{\cdot}: V \rightarrow \mathbb{R}\)을 \(\norm{v} = \sqrt{\inner{v}{v}}\)로 정의하면 \(\norm{\cdot}\)는 norm. 즉, inner product space에 자연스러운 방식으로 norm을 정의할 수 있다.
	\end{prop}

	\begin{proof}
		(N2)와 (N3)은 당연하다. (N1)은 복소수의 성질과 Cauchy-Schwarz inequality를 이용.
		\begin{align*}
			\norm{v+w}^2 = \inner{v+w}{v+w} &= \inner{v}{v} + \inner{v}{w} + \inner{w}{v} + \inner{w}{w}\\
			&= \norm{v}^2 + \norm{w}^2 + \inner{v}{w} + \overline{\inner{v}{w}}\\
			&= \norm{v}^2 + \norm{w}^2 + 2\mathrm{Re}(\inner{v}{w})\\
			&\le \norm{v}^2 + \norm{w}^2 + 2\abs{\inner{v}{w}}\\
			&\le \norm{v}^2 + \norm{w}^2 + 2\norm{v}\norm{w}\\
			&= (\norm{v} + \norm{w})^2
		\end{align*}
		(\(\mathrm{Re}(z)\)는 \(z \in \CC\)의 real part. Imaginary part는 \(\mathrm{Im}(z)\).)
	\end{proof}

	\begin{ex}
		Inner product으로부터 유도되지 않는 norm을 정의할 수 있습니다. 예를 들어 \(\RR^n\)에서
		\begin{align*}
			&\norm{(x_1, \ldots, x_n)}_p = \left(\abs{x_1}^p + \ldots + \abs{x_n}^p\right)^{1/p} \quad (p > 1, p \neq 2)\\
			&\norm{(x_1, \ldots, x_n)}_\infty = \max \{\abs{x_1}, \ldots, \abs{x_n}\}
		\end{align*}
		으로 정의하면 \(\norm{\cdot}_p, \norm{\cdot}_\infty\)는 각각 norm on \(\RR^n\)이지만 \(\norm{v}_p = \sqrt{\inner{v}{v}}\) for \(v \in \RR^n\)이거나 \(\norm{v}_\infty = \sqrt{\inner{v}{v}}\) for \(v \in \RR^n\)인 inner product \(\inner{\cdot}{\cdot}\)이 존재하지 않습니다.
	\end{ex}


	\begin{prop}[Parallelogram law]
		Normed vector space \(V\)에 대해 다음은 동치.
		\begin{enumerate}[label=(\alph*), leftmargin=2\parindent]
			\item
			Inner product \(\inner{\cdot}{\cdot}\) on V가 존재하여 \(\norm{v} = \sqrt{\inner{v}{v}}\) for \(v \in V\).
			\item
			\(\norm{v+w}^2 + \norm{v-w}^2 = 2\norm{v}^2 + 2\norm{w}^2 \quad (v, w \in V)\).
		\end{enumerate}
	\end{prop}

	\begin{proof}
		(a)\(\Rightarrow\)(b)는 당연하다. (b)\(\Rightarrow\)(a)를 보이기 위해,
		\begin{gather*}
			\inner{v}{w} = \frac{1}{4}(\norm{v+w}^2 - \norm{v-w}^2)
		\end{gather*}
		로 정의하고, 이것이 우리가 원하는 inner product임을 보이면 된다.
	\end{proof}

따라서 다음과 같이 생각할 수 있겠습니다. (각 화살표의 역은 성립하지 않습니다.)

	\begin{center}
		Inner Product Space \(\Rightarrow\) Normed Vector Space \(\Rightarrow\) Metric Space
	\end{center}

이제 마지막으로, 앞으로 우리가 여러 가지 재밌는 일들을 벌일 Euclidean space를 소개하는 것으로 이 Section을 마무리하겠습니다.

	\begin{defn} \label{def euc space}
		Dot product가 주어진 inner product space \(\RR^n\)을 \textbf{Euclidean space}라고 한다. 따라서 Euclidean space는 normed vector space이고 metric space이며, Euclidean space의 dot product에 의해 유도되는 자연스러운 norm을 \textbf{Euclidean norm}이라고 한다.
	\end{defn}


\section{Open Set and Closed Set} 

\subsection{Metric Space Case} \label{sec metric open}


이제 \(\mathbb{R}\)에서 열린구간과 닫힌구간의 개념을 metric space로 확장하려 합니다. 

	\begin{defn} \label{def metric top}
		Metric space \(X\)와 \(p \in X, E \subseteq X\)에 대하여,
		\begin{enumerate}[label=(\alph*), leftmargin=2\parindent]
			\item
			양수 \(r\)에 대하여 집합 \(N_r (p) = \{q \in X : d(p, q) < r\}\)를 \textbf{\(r\)-neighborhood(근방) of \(p\)}라고 한다. \(r\)은 \textbf{radius of \(N_r (p)\)}.
			\item
			\(E\)가 \textbf{open in \(X\)(\(X\)의 열린집합)}라는 것은 각 \(p \in E\)에 대해 적당한 양수 \(r_p > 0\)이 존재하여 다음을 만족하는 것이다.
			\begin{gather*}
				E = \bigcup_{p \in E} N_{r_p}(p)
			\end{gather*}
			\item
			\(p\)를 포함하는 open set of \(X\)를 \textbf{(open) neighborhood of \(p\)}라고 한다.
			\item
			\(E\)가 \textbf{closed in \(X\)(\(X\)의 닫힌집합)}라는 것은 \(X \backslash E\)가 open in \(X\)라는 것이다.
		\end{enumerate}
		
	\end{defn}

Section \ref{sec metric open}\이 끝날 때까지 \(X\)는 metric space.

다음은 open set의 (더 유용한) 새로운 정의입니다.

	\begin{defn}
		\(p \in X, E \subseteq X\)에 대하여, \(p\)가 \textbf{interior point(내점) of \(E\)}라는 것은 적당한 양수 \(r > 0\)이 존재하여 \(N_r (p) \subseteq E\)인 것이다. \(E\)의 모든 interior point들의 집합을 \textbf{interior(내부) of \(E\)}라고 하고 \(\text{int}E\)로 쓴다.
	\end{defn}

	\begin{thm}
		\(U \subseteq X\) is open if and only if \(\text{int}U = U\).
	\end{thm}
	\begin{proof}
		\(U \subseteq X\)가 open이라고 가정하면 각 \(p \in U\)에 대해 \(r_p > 0\)가 존재하여 \(U = \bigcup_{p \in U} N_{r_p}(p)\)이다. 따라서 \(N_{r_p}(p) \in U\)이므로 \(p \in \text{int}U\)이고 \(\text{int}U \subseteq U\). 이 증명을 반대로 읽으면 역이 증명된다.
	\end{proof}

	\begin{prop}
		\(N_r (p)\)는 open in \(X\).
	\end{prop}
	\begin{proof}
		\(q \in N_r (p)\)를 택하고, \(r'=d(p, q) < r\)이라고 하면 \(N_{r - r'} (q) \subseteq N_r (p)\)이므로(왜 그런가?) \(N_r (p)\)는 open.
	\end{proof}

	\begin{prop} \label{prop metric haus}
		\(X\)의 서로 다른 두 원소 \(p, q\)에 대하여, \(p, q\) 각각의 적당한 neighborhood \(U, V\)가 존재하여 \(U \cap V = \emptyset\)이다.
	\end{prop}
	
	\begin{proof}
		Metric의 정의에 의해 \(d(p, q) > 0\)이다. \(r = d(p, q) / 2\)로 놓고, \(U = N_r (p), V = N_r(q)\)로 놓으면 triangular inequality에 의해 \(U \cap V = \emptyset\).
	\end{proof}

	\begin{prop} \label{prop metric open ax}
		\(X\)에서 다음이 성립한다.
		\begin{enumerate}[label=(\alph*), leftmargin=2\parindent]
			\item
			\(\emptyset\)과 \(X\)는 open in \(X\).
			\item
			\(X\)의 open set들의 union(무한 개 포함)은 open in \(X\).
			\item
			\(X\)의 open set들의 \textbf{finite} intersection은 open in \(X\).
			\item
			\(\emptyset\)과 \(X\)는 closed in \(X\).
			\item
			\(X\)의 closed set들의 intersection(무한 개 포함)은 closed in \(X\).
			\item
			\(X\)의 closed set들의 \textbf{finite} union은 closed in \(X\).
		\end{enumerate}
	\end{prop}

	\begin{proof}
		(a)는 당연. (b)를 보이기 위해, \(\{U_i\}_{i \in I}\)가 \(X\)의 open set들의 모임이라고 하고 \(p \in \cup_{i\in I} U_i\)를 택한다. 그러면 어떤 \(i \in I\)에 대하여 \(p \in U_i\)이고 open set의 정의에 의해 \(\exists\) neighborhood \(N\) of \(p \in X\) s.t. \(N \subseteq U_i\). 그러면 \(N \subseteq \cup_{i\in I} U_i\)이므로 \(\cup_{i\in I} U_i\)는 open. (c)를 보이기 위해, \(U_1, ..., U_n\)이 \(X\)의 open set이라고 하고 \(p \in \cap_{i=1}^n U_i\)라고 하자. 그러면 각 \(i=1, ..., n\)에 대해 양수 \(r_i\)가 존재하여 \(N_{r_i} (p) \subseteq U_i\)이다. \(r = \mathrm{max}\{r_1, ..., r_n\}>0\)으로 놓으면 \(N_r (p) \subseteq \cap_{i=1}^n U_i\)이므로 \(\cap_{i=1}^n U_i\)는 open. (d)-(f)는 closed set의 정의와 (a)-(c)에 의해 성립.
	\end{proof}

	\begin{rem}
		(c)와 (f)에서 finite 조건은 필수적입니다. 예를 들어 각 \(n=1, 2, ...\)에 대해 \(U_n = (-1, 1/n)\)으로 정의하면 \(\cap_{n \in \NN} U_n = (-1, 0]\)으로 open in \(\mathbb{R}\)이 아닙니다. 또 \(F_n = [-1, 1-1/n]\)으로 정의하면 \(\cup_{n \in \NN} F_n = [-1, 1)\)로 closed in \(\mathbb{R}\)이 아닙니다.
	\end{rem}


어떤 집합이 closed인지 정의에 의해 확인하려면 그 집합의 여집합이 open인지를 확인해야 하는데, 이는 불편할 때가 많습니다. 그래서 \(X\)의 부분집합의 closedness를 결정하는 특징을 알아보려고 합니다. \(\RR\)의 닫힌구간 \(I\) 안의 수렴하는 수열은 그 극한이 \(I\) 안에 존재한다는 점에 주목합시다.

	\begin{defn}
		\(E \subseteq X, p \in X\)에 대하여,
		\begin{enumerate}[label=(\alph*), leftmargin=2\parindent]
			\item
			\(p\)가 \textbf{limit point(극한점) of \(E\)}라는 것은, 임의의 양수 \(\eps > 0\)에 대하여 \((N_\eps (p) \backslash \{p\}) \cap E \neq \emptyset\)이라는 것이다. \(E\)의 모든 limit point들의 집합을 \textbf{derive set of \(E\)}라고 하고 \(E'\)로 쓴다.
			\item
			\(p\)가 \textbf{isolated point(고립점) of \(E\)}라는 것은 \(p \in E \backslash E'\)라는 것이다.
			\item
			집합 \(\overline{E} = E \cup E'\)를 \textbf{closure(폐포) of \(E\)}라고 한다.
			\item
			\(E\)가 \textbf{dense(조밀) in \(X\)}하다는 것은 \(\overline{E} = X\)라는 것이다.
		\end{enumerate}
	\end{defn}

	\begin{prop}
		\(E \subseteq X, p \in X\)에 대하여,
		\begin{enumerate}[label=(\alph*), leftmargin=2\parindent]
			\item
			\(p\)가 isolated point of \(E\)일 필요충분조건은 다음을 만족하는 \(\eps > 0\)이 존재하는 것이다.
			\begin{gather*}
				N_\eps(p) \cap E = \{p\}
			\end{gather*}
			\item
			\(p \in \overline{E}\)일 필요충분조건은 임의의 양수 \(\eps > 0\)에 대해 다음이 성립하는 것이다.
			\begin{gather*}
				N_\eps (p) \cap E \neq \emptyset
			\end{gather*}
		\end{enumerate}
	\end{prop}
	\begin{proof}
		정의에 의해 당연.
	\end{proof}

	\begin{prop}
		\(p \in X\)가 \(E \subseteq X\)의 limit point이면 임의의 \(\eps > 0\)에 대해 \(N_\eps (p) \cap E\)는 infinite.
	\end{prop}
	\begin{proof}
		\(N_\eps (p) \cap E\)가 finite이라고 가정하고 모순을 보이면 된다.
	\end{proof}

	\begin{ex}
		대표적인 예시들입니다.
		\begin{enumerate}[label=(\alph*), leftmargin=2\parindent]
			\item
			\(X\)의 유한 부분집합의 limit point는 존재하지 않습니다. 즉 유한 부분집합은 모든 원소가 isolated point.
			\item
			\begin{gather*}
				S = \{\frac{1}{n} \in \RR\}_{n \in \NN}
			\end{gather*}
			로 정의하면, \(S'=\{0\}\).
			\item
			\(\QQ' = \RR\). 따라서 \(\overline{\QQ} = \RR\)이므로 \(\QQ\)는 dense in \(\RR\).
		\end{enumerate}
	\end{ex}

Limit point와 관련된 개념들과 sequence는 밀접한 관련이 있습니다.

	\begin{prop} \label{prop limpt seq}
		\(E \subseteq X, p \in X\)에 대하여 다음이 성립한다.
		\begin{enumerate}[label=(\alph*), leftmargin=2\parindent]
			\item
			\(p \in E'\) if and only if there exists a sequence in \(E \backslash \{p\}\) converging to p.
			\item
			\(p \in \overline{E}\) if and only if there exists a sequence in \(E\) converging to p.
		\end{enumerate}
	\end{prop}

	\begin{proof}
		(a): 먼저 임의의 \(p \in E'\)에 대해, 각 \(n = 1, 2, ...\)에 대해 \((N_{1/n} (p) \backslash {p}) \cap E \neq \emptyset\)이므로 \(d(x_n, p) < 1/n\)인 \(x_n \in E \backslash \{p\}\)가 존재한다. 이제 \((x_n)\)이 \(p\)로 수렴함은 쉽게 알 수 있다. 역으로 \(p\)로 수렴하는 \(E \backslash \{p\}\) 안의 sequence \((x_n)\)을 생각하면, 임의의 양수 \(\eps > 0\)에 대해 \(x_N \in (N_\eps (p) \backslash {p}) \cap E\) for some \(N\)이므로 \(p \in E'\).\\
		(b): \(p \in \overline{E}\)이면 \(p \in E\)이거나 \(p \in E'\)이다. 첫 번째 경우는 \(E\) 안에서 모든 항이 \(p\)인 sequence를 잡으면 되고, 두 번째 경우는 (a)에 의해서 성립. 역으로 \(p\)로 수렴하는 \(E\) 안의 sequence \((x_n)\)을 생각하면, 임의의 양수 \(\eps > 0\)에 대해 \(x_N \in N_\eps (p) \cap E\) for some \(N\)이므로 \(p \in \overline{E}\).
	\end{proof}

이제 closed set의 필요충분조건을 소개할 준비가 되었습니다.

	\begin{thm}
		\(F \subseteq X\)에 대하여 다음은 동치.
		\begin{enumerate}[label=(\alph*), leftmargin=2\parindent]
			\item
			\(F\) is closed in \(X\).
			\item
			\(F' \subseteq F\).
			\item
			\(\overline{F} = F\).
			\item
			If a sequence \((x_n)\) in \(F\) converges to some \(p \in X\), then \(p \in F\).
		\end{enumerate}
	\end{thm}

	\begin{proof}
		(a)\(\Leftrightarrow\)(b):
		\begin{align*}
			F \text{ is closed in } X & \Leftrightarrow X \backslash F \text{ is open in } X\\
			& \Leftrightarrow \forall x \in X \backslash F, \: \exists \epsilon > 0 \text{ s.t. } N_\epsilon (x) \subseteq X \backslash F\\
			& \Leftrightarrow \forall x \in X \backslash F, \: x \notin F'\\
			& \Leftrightarrow F' \subseteq F
		\end{align*}
		(b)\(\Leftrightarrow\)(c): \(\overline{F}\)의 정의에 의해.\\
		(c)\(\Leftrightarrow\)(d): Proposition \ref{prop limpt seq}.
	\end{proof}

	\begin{thm}
		\(E \subseteq X\)에 대하여 \(E'\)와 \(\overline{E}\)는 closed in \(X\). 또 \(E \subseteq F \subseteq X\)이고 \(F\)가 closed in \(X\)이면 \(\overline{E} \subseteq F\). 즉, \(\overline{E}\)는 \(E\)를 포함하는 가장 작은 closed set.
	\end{thm}

	\begin{proof}
		\(E'\)가 closed in \(X\)임을 보이기 위해 \((E')' \subseteq E'\)임을 보이면 된다. \(E' = \emptyset\)이면 당연하므로 \(E' \neq \emptyset\)라 가정하자. 임의의 \(x \in (E')', \eps > 0\)에 대해 \(d(x, y) < \eps\)인 \(y \in E'\)가 존재한다. \(\eps' = \min\{d(x, y), \eps - d(x, y)\} > 0\)으로 두면, \(0 < d(y, z) < \eps'\)을 만족하는 \(z \in E\)가 존재한다. 이제
		\begin{gather*}
			0 < \abs{d(x, y) - d(y, z)} \le  d(x, z) \le d(x, y) + d(y, z) \le d(x, y) + (\eps - d(x, y)) = \eps
		\end{gather*}
		이므로 \(x \in E'\).\\
		한편 \(x \in X \backslash \overline{E}\)일 필요충분조건은 다음을 만족하는 \(\eps > 0\)이 존재하는 것이다.
		\begin{gather*}
			N_\eps (x) \cap E = \emptyset
		\end{gather*}
		따라서 \(N_\eps (x)\)는 \(E\)의 점을 포함하지 않는다. 또 \(N_\eps (x)\)가 \(E'\)의 점도 포함하지 않음을 알 수 있다(왜 그런가?). 따라서 \( N_\eps (x) \in X \backslash \overline{E}\) 이므로 \(X \backslash \overline{E}\)는 open.\\
		Derive set의 정의에 의해 \(E \subseteq F\)이면 \(E' \subseteq F'\)이다. 그런데 \(F' \subseteq F\)이므로 \(E' \subseteq F\)이고 따라서 \(\overline{E} \subseteq F\).
	\end{proof}

전체 집합을 무엇으로 생각하는지에 따라 open과 closed 여부가 달라질 수 있습니다. 특히 \(X\)의 부분집합 안에서 open과 closed를 생각하는 것이 편리할 때가 많습니다.

	\begin{defn}
		\(E \subseteq Y \subseteq X\)일 때, \(E\)가 \textbf{open in \(Y\)}라는 것은 어떤 open set \(U\) of \(X\)가 존재하여 \(E = U \cap Y\)인 것이다. \textbf{closed in \(Y\)}도 마찬가지로 정의한다.
	\end{defn}

\subsection{Topological Space Case} \label{sec top}

이 부분을 쓸지 말지 굉장히 고민했는데, 계속해서 관련된 내용이 나와서 모든 것을 밝히고 가는 것이 좋겠다고 생각했습니다. 이 글 전체에서 이 이상의 추상화는 없을 것입니다.

	\begin{defn}
		집합 \(X\)의 부분집합들의 collection \(\mathcal{T}\)가 다음 조건을 만족하면 \(\mathcal{T}\)는 \textbf{topology(위상) of \(X\)}라고 하고, \((X, \mathcal{T})\) 또는 간단하게 \(X\)를 \textbf{topological space(위상공간)}라고 한다.
		\begin{enumerate}[label=(TP\arabic*), leftmargin=2\parindent]
			\item
			\(\emptyset \in \mathcal{T}, X \in \mathcal{T}\)
			\item
			\(\mathcal{T}\)의 원소들의 union(무한 개 포함)은 \(\mathcal{T}\)의 원소이다.
			\item
			\(\mathcal{T}\)의 원소들의 finite intersection은 \(\mathcal{T}\)의 원소이다.
		\end{enumerate}
		이때 각 \(U \in \mathcal{T}\)에 대해 \(U\) \textbf{open set of \(X\)}이라고 하고, \(p\)를 포함하는 open set of \(X\)를 \textbf{(open) neighborhood of \(X\)}라고 한다. \(U \in \mathcal{T}\)에 대해 \(X \backslash U\)를 \textbf{closed set of \(X\)}라고 한다.
	\end{defn}

	\begin{thm}
		Metric space \(X\)에서 Definition \ref{def metric top}\과 같이 정의된 모든 open set들의 collection을 \(\mathcal{T}\)라고 하면 \(\mathcal{T}\) is a topology of \(X\). 즉 metric space에 자연스럽게 topological space의 구조를 줄 수 있다.
	\end{thm}
	\begin{proof}
		Proposition \ref{prop metric open ax}의 실체는 바로 이 정리였다.
	\end{proof}

Section \ref{sec top}\이 끝날 때까지 \(X\)는 아무 말이 없으면 topological space.

Topological space에서 limit point와 convergence의 개념을 정의할 수 있습니다.

	\begin{defn}
		\(E \subseteq X, p \in X\)에 대하여,
		\begin{enumerate}[label=(\alph*), leftmargin=2\parindent]
			\item
			\(p\)가 \textbf{limit point(극한점) of \(E\)}라는 것은, 임의의 open neighborhood \(U\) of \(p\)에 대하여 \((U \backslash \{p\}) \cap E \neq \emptyset\)이라는 것이다. \(E\)의 모든 limit point들의 집합을 \textbf{derive set of \(E\)}라고 하고 \(E'\)로 쓴다.
			\item
			\(p\)가 \textbf{isolated point(고립점) of \(E\)}라는 것은 \(p \in E \backslash E'\)라는 것이다.
			\item
			집합 \(\overline{E} = E \cup E'\)를 \textbf{closure(폐포) of \(E\)}라고 한다.
			\item
			\(E\)가 \textbf{dense(조밀) in \(X\)}하다는 것은 \(\overline{E} = X\)라는 것이다.
		\end{enumerate}
	\end{defn}

	\begin{defn}
		Sequence \((x_n)\)이 \(x \in X\)로 수렴한다는 것은 임의의 open neighborhood \(U\) of \(x\)에 대해 \(N \in \NN\)이 존재하여 \(n \ge N \Longrightarrow x_n \in U\)라는 것이다.
	\end{defn}

	\begin{prop}
		Metric space에서 정의된 limit point, isolated point, closure, denseness, convergence의 정의는 위 정의의 특수한 경우이다.
	\end{prop}
	\begin{proof}
		연습문제로 남김.
	\end{proof}

우리는 지금 수준에서는 다음 성질을 만족하는 topological space에 관심이 있습니다.

	\begin{defn}
		\(X\)가 \textbf{Hausdorff}라는 것은, \(X\)의 서로 다른 두 원소 \(p, q\)에 대하여 \(p, q\) 각각의 적당한 neighborhood \(U, V\)가 존재하여 \(U \cap V = \emptyset\)를 만족하는 것이다.
	\end{defn}

	\begin{thm}
		Metric space is Hausdorff.
	\end{thm}
	\begin{proof}
		Proposition \ref{prop metric haus}에 의해.
	\end{proof}
	
	\begin{thm}
		Hausdorff space \(X\) 안의 sequence \((x_n)\)이 \(x \in X\)로 수렴하고 \(x' \in X\)로 수렴하면 \(x = x'\).
	\end{thm}
	\begin{proof}
		\(x, x'\)의 disjoint한 각각의 open neighborhood \(U, V\)를 찾고\ldots 그 이후는 Theorem \ref{thm metric limit unq}의 증명과 거의 같다. 
	\end{proof}

Hausdorff space는 Section \ref{sec cpt}의 compactness에서 한번 더 언급될 것입니다.

\newpage

\section{Completeness of Euclidean Spaces}
\subsection{Bolzano-Weierstrass Theorem}

앞 Section에서 limit point의 개념을 정의하였는데, \(X\)의 유한 부분집합은 limit point를 가지지 않음을 알고 있습니다.이제 \(X=\RR^n\)인 경우로 한정하여, 그렇다면 \(\RR^n\)의 무한집합 중 어떤 집합이 limit point를 가질지에 대해 논의하려고 하는데, 결론은 `\(\RR^n\)의 bounded infinite subset은 limit point를 가진다'라는 것입니다.

\(\RR^n\)에 대해 이야기하기 전에 먼저 \(\RR\)의 중요한 성질 중 하나를 언급하겠습니다. 이는 \(\RR\)의 단조수렴정리에 의해 obvious합니다.

	\begin{thm}[Nested Intervals Theorem: 축소구간정리]
		각 \(n \in \NN\)에 대하여 \(I_n = [a_n, b_n] \subset \RR\)이 유계닫힌구간이고 \(I_{n+1} \subseteq I_n\)이라고 하자. 즉, \(a_n \le a_{n+1}, b_n \ge b_{n+1}\). (이러한 \((I_n)\)을 \textbf{nested interval}이라고 한다.) 그러면 \(\cap_n I_n\)은 nonempty.
	\end{thm}
	\begin{proof}
		실수열 \((a_n)\)을 생각하자. 각 \(n \in \mathbb{N}\)에 대해 \(a_n \le b_1\)이므로 \((a_n)\)은 유계이고, 가정에 의해 \(a_n\)은 단조증가수열이다. 따라서 단조수렴정리에 의해 \((a_n)\)은 어떤 \(\alpha \in \RR\)로 수렴한다. 주장: \(\alpha \in \cap_n I_n\)\\
		이를 보이기 위하여 for all \(n \in \NN\), \(\alpha \le b_n\)임을 보이면 된다. 어떤 \(N\)에 대하여 \(\alpha > b_N\)이라고 가정하자. 그런데 for all \(n \in \NN\), \(a_n \le b_N\)이므로 \(\alpha \le b_N\)이다. 따라서 모순이고, \(\alpha \le b_n\) for all \(n \in \NN\)이다.
	\end{proof}

	\begin{rem}
		\(I_n\)이 닫힌구간인 것이 중요합니다. \(I_n = (0, \frac{1}{n})\)이면 \(\cap_n I_n = \emptyset \).
	\end{rem}

\(\RR\)에서의 축소구간정리를 \(\RR^N\)\footnote{\(n\)이 아니라 \(N\)을 쓰는 이유는 sequence에 \(n\)을 쓰기 위해서.}으로 자연스럽게 확장할 수 있습니다. \(k = 1, ..., N\)에 대하여 \(I^k \subset \RR\)가 유계닫힌구간일 때, \(B = I^1 \times ... \times I^N \subset \RR^N\)을 \textbf{\(N\)-cell}이라고 합니다. 평면에서는 직사각형, 공간에서는 직육면체의 개념을 일반화한 것입니다.

	\begin{cor} \label{cor Ncell}
		\(k = 1, ..., N, n \in \NN\)에 대하여 \(I_n^k \subset \RR\)가 유계닫힌구간, \(B_n = I_n^1 \times ... \times I_n^N \subset \RR^N\)이라고 하자. 각 \(n \in \NN\)에 대하여 \(B_{n+1} \subseteq B_n\)이면 \(\cap_n B_n\)은 nonempty.
	\end{cor}
	\begin{proof}
		\(B_{n+1} \subseteq B_n\)이므로 각 \(k = 1, ..., N\)에 대하여 \(I_{n+1}^k \subseteq I_n^k\)이다. 따라서 축소구간정리에 의해 \(\exists \alpha^k \in \cap_n I_n^k\). 이제 \((\alpha^1, ..., \alpha^N) \in \cap_n B_n\)임은 당연하다.
	\end{proof}

이제 Bolzano-Weierstrass Theorem을 증명할 준비가 되었습니다.

	\begin{thm}[Bolzano-Weierstrass Theorem]
		\(A \subset \RR^N\)가 bounded and infinite이면 \(A\)는 limit point를 가진다.
	\end{thm}
	\begin{proof}
		\(A\)가 유계이므로 \(A\)를 포함하는 \(N\)-cell \(B_1\)을 잡을 수 있다. 이제 각 \(n \in \NN\)에 대하여 \(N\)-cell \(B_n\)을 다음 성질을 만족하도록 귀납적으로 정의하려고 한다.
		\begin{enumerate}[label=(\alph*), leftmargin=2\parindent]
			\item
			\(B_{n+1} \subseteq B_n\)
			\item
			\(B_n \cap A\) is infinite.
		\end{enumerate}
		위 조건을 만족하는 \(B_1, ..., B_m\)이 주어졌다고 하자. \(B_m = I_m^1 \times ... \times I_m^N\)로 쓰고, 각 \(I_m^k\)를 이등분한다. 그러면 \(2^N\)개의 \(B_m\)의 sub-\(N\)-cell을 얻는데, 이중 적어도 1개는 \(A\)의 점을 무수히 많이 포함하고 있다(왜 그런가?). 따라서 이 \(N\)-cell을 \(B_{m+1}\)으로 정의하면 된다.\\
		이제 Corollary \ref{cor Ncell}에 의해 \(\exists x \in \cap_n B_n\)이다. 주장: \(x\) is a limit point of \(A\).
		\(B_n\)의 \textbf{diameter}를 \(\mathrm{diam}B_n = \sup \{\norm{x-y}: x, y \in B_n\}\)으로 정의하면, \(\lim_n \mathrm{diam}B_n = 0\)인 것은 쉽게 알 수 있다. 따라서 임의의 양수 \(\eps > 0\)에 대해 \(\mathrm{diam}B_n < \eps / 2\)인 \(n\)이 존재한다. 이제 for all \(y \in B_n\), \(\norm{x-y} \le \eps / 2 < \eps\)이므로 \(B_n \subseteq N_\eps (x)\)이다. 그런데 \(B_n\)이 \(A\)의 점을 무한히 많이 포함하고 있으므로, \(N_\eps (x)\)도 마찬가지이다. 따라서 \(x\)는 limit point of \(A\).
	\end{proof}

Bolzano-Weierstrass Theorem의 direct result를 설명하기 위해 다음 정의가 필요합니다.

	\begin{defn}
		함수 \(n: \NN \rightarrow \NN\)이 각 \(k \in \NN\)에 대해 \(n(k) < n(k+1)\)을 만족한다고 하자. Sequence \(n \mapsto x_n\)에 대하여 \(k \mapsto x_{n(k)}\)를 \textbf{subsequence(부분수열) of \((x_n)\)}이라고 한다.
	\end{defn}

예를 들어 \(x_n = 2n\)으로 정의된 실수열 \((x_n)\)와 \(n(k) = 2k\)에 대하여 \(x_{n(k)} = 4k\)는 \((x_n)\)의 subsequence입니다.

Limit point와 convergent sequence는 밀접한 관계가 있음을 확인한 바 있는데, 마찬가지로 Bolzano-Weierstrass Theorem에 의해서 다음 따름정리를 얻게 됩니다.

	\begin{cor} \label{cor bdd conv}
		\(\RR^N\)의 bounded sequence는 convergent subsequence를 가진다.
	\end{cor}
	\begin{proof}
		\((x_n)\)이 \(\RR^N\)의 bounded sequence라고 하자. \(A = \{x_n: n \in \NN \}\)이 finite이면, 어떤 \(x \in A\)에 대하여 \(x_n = x\)인 \(n\)이 무수히 많이 존재한다. 따라서 이 \(n\)들을 모아서 subsequence를 만들면 이 수열은 상수수열이므로 당연히 수렴한다. 이제 \(A\)가 infinite이라고 가정하자. Bolzano-Weierstrass Theorem에 의해 \(A\)의 limit point \(x\)가 존재한다. 이제 \(x\)로 수렴하는 subsequence \((x_{n(k)})\)를 \(\norm{x_{n(k)} - x} < 1/k\)를 만족하도록 귀납적으로 정의하려고 한다.\\
		먼저 \(N_1 (x) \cap A\)는 nonempty이므로 여기에 속하는 \(x_{n(1)}\)을 잡을 수 있다. 다음으로 위 성질을 만족하는 \(x_{n(1)}, ..., x_{n(m)} \: (n(1) < ... < n(m))\)이 정해졌다고 가정하자. \(N_{1/(m+1)} (x) \cap A\)은 infinite이므로 \(n(m) < n(m+1)\)이면서 \(x_{n(m+1)} \in N_{1/(m+1)} (x) \cap A\)인 \(n(m+1)\)을 잡을 수 있다. 이제 이 subsequence가 \(x\)로 수렴하는 것은 쉽게 알 수 있다.
	\end{proof}

\subsection{Cauchy Sequence}

해석학에서 정말 중요한 개념인 Cauchy sequence를 정의하려고 합니다. 앞에서 단조수렴정리를 소개할 때, 어떤 sequence가 수렴할 것으로 예상되는 점조차 알지 못하는 경우에 수렴성을 보일 수 있는 방법을 계속 찾게 될 것이라고 말씀드렸습니다. Cauchy sequence가 그 대표적인 방법입니다.

	\begin{defn}
		Metric space \(X\)의 sequence \(x_n\)이 Cauchy sequence라는 것은, 임의의 양수 \(\eps > 0\)에 대하여 적당한 \(N \in \NN\)이 존재하여 다음을 만족하는 것이다.
		\begin{gather*}
			n, m \ge N \Longrightarrow d(x_n, x_m) < \eps
		\end{gather*} 인 것이다.
	\end{defn}

	\begin{ex}
		\(x_n = 1/n\)은 Cauchy sequence. 양수 \(\eps > 0\)이 주어졌다고 하고 \(1/N < \eps\)인 \(N\)을 잡는다. 이때 \(n, m \ge N\)이면 \(|1/n - 1/m| \le 1/N < \eps\)이므로, \((x_n)\)은 Cauchy sequence.
	\end{ex}

수렴하는 수열과 밀접한 관계가 있는데, 실제로 모든 수렴하는 수열은 Cauchy입니다.

	\begin{prop}
		A convergent sequence in a metric space \(X\) is Cauchy.
	\end{prop}
	\begin{proof}
		\((x_n)\)이 \(x \in X\)로 수렴한다고 하자. 양수 \(\eps > 0\)에 대하여,
		\begin{gather*}
			n \ge N \Longrightarrow d(x_n, x) < \frac{\eps}{2}
		\end{gather*}
		\(N\)을 잡을 수 있다. 이제
		\begin{gather*}
			n, m \ge N \Longrightarrow d(x_n, x_m) \le d(x_n, x) + d(x_m, x) < \eps.
		\end{gather*}
	\end{proof}

Cauchy sequence는 다음 성질들을 가집니다.

	\begin{prop} \label{cauchy bdd}
		A Cauchy sequence in a metric space \(X\) is bounded.
	\end{prop}
	\begin{proof}
		\((x_n)\)이 Cauchy sequence in \(X\)라고 하고, \(n, m \ge N \Rightarrow d(x_n, x_m) < 1\)인 \(N\)을 잡을 수 있다. 따라서 \(\{x_n: n \ge N\}\)은 유계이고, \(\{x_n: n < N\}\)은 finite이므로 유계. 따라서 \((x_n)\)은 bounded.
	\end{proof}

	\begin{prop} \label{cauchy subs}
		Metric space \(X\)의 Cauchy sequence \((x_n)\)이 \(x \in X\)로 수렴하는 subsequence를 가지면 \((x_n)\)도 \(x\)로 수렴.
	\end{prop}
	\begin{proof}
		\(\eps > 0\)이 주어졌다고 하자. \((x_n)\)은 Cauchy이므로
		\begin{gather*}
			n, m \ge N \Longrightarrow d(x_n, x_m) < \frac{\eps}{2}
		\end{gather*}
		인 \(N\)이 존재한다. 또 \((x_{n(k)})\)가 \(x \in X\)로 수렴하는, \((x_n)\)의 convergent subsequence라고 하면
		\begin{gather*}
			k \ge K \Longrightarrow d(x_{n(k)}, x) < \frac{\eps}{2}, \quad n(K) \ge N
		\end{gather*}
		인 \(K\)를 잡을 수 있다. 따라서
		\begin{gather*}
			n \ge n(K) \Longrightarrow d(x_n, x) \le d(x_n, x_{n(K)}) + d(x_{n(K)}, x) < \frac{\eps}{2} + \frac{\eps}{2} = \eps.
		\end{gather*}
	\end{proof}

위의 두 Proposition에 의해, \(X = \RR^N\)인 경우에 Cauchy sequence와 convergent sequence가 동치임을 알 수 있습니다. 모든 Cauchy sequence가 수렴하는 metric space를 \textbf{complete metric space}라고 합니다. 즉 Euclidean space는 complete metric space.

	\begin{thm}[Completeness of a Euclidean Space]
		\(\RR^N\)의 sequence \((x_n)\)에 대하여, \((x_n)\) is convergent if and only if \((x_n)\) is Cauchy.
	\end{thm}
	\begin{proof}
		\((\Rightarrow)\) 방향은 이미 보였다(아무 metric space에서 성립). 이제 \((x_n)\)이 Cauchy라고 가정하면, Proposition \ref{cauchy bdd}에 의해 \((x_n)\)은 bounded이다. 따라서 Corollary \ref{cor bdd conv}에 의해 \((x_n)\)은 convergent subsequence를 가진다. 따라서 Proposition \ref{cauchy subs}에 의해 \((x_n)\)은 convergent.
	\end{proof}

위 증명에서 Corollary \ref{cor bdd conv}를 이용하였음에 유의하기 바랍니다. 이것이 다른 metric space와 Euclidean space를 구분짓는 한 가지 성질입니다.

\newpage

\section{Series} \label{sec series}

Section \ref{sec series}에서는 해석학의 또 다른 중요한 주제인 series(급수)의 수렴판정을 공부하려고 합니다. 이 글이 끝날 때까지 모든 sequence는 아무 말이 없으면 (usual한 metric이 주어진) \(\CC\) 안에 있는 것으로 생각합니다. \(\CC = \RR^2\)로 `자연스럽게' identify할 수 있기 때문에, \(\CC\)는 complete metric space.

\subsection{Cauchy Criterion}

	\begin{defn}
		Sequence \((x_n)\)에 대하여 \((s_n)\)을 \(s_n = \sum_{k=1}^n x_k\)으로 정의하고 \(\sum_{n=1}^\infty x_n\) 또는 \(\sum_n x_n\)을 \textbf{series(급수)}라고 하고 \((s_n)\)을 \textbf{partial sum(부분합) of series}라고 한다. 만약 \((s_n)\)이 \(s\)로 converge하면 \textbf{series \(\sum_n x_n\)이 converge(수렴)}한다고 하고 \(\sum_n x_n = s\)로 쓴다. 만약 \((s_n)\)이 diverge하면 \textbf{series \(\sum_n x_n\)이 diverge(발산)}한다고 한다.
	\end{defn}

	\begin{prop}
		급수 \(\sum_n x_n\)과 \(\sum_n y_n\)이 각각 \(x, y \in \CC\)로 수렴할 때 다음이 성립한다.
		\begin{enumerate} [label=(\alph*), leftmargin=2\parindent]
			\item
			\(\sum_n (x_n + y_n) = x + y\).
			\item
			\(c \in \CC\)에 대하여 \(\sum_n cx_n = cx\).
		\end{enumerate}
	\end{prop}
	\begin{proof}
		각 급수의 partial sum을 생각하면 끝.
	\end{proof}

\(\CC\)에서 Cauchy sequence와 convergent sequence는 동치이므로, 다음 판정법을 얻습니다.

	\begin{thm}[Cauchy criterion]
		\(\sum_n x_n\) converges if and only if for every \(\eps > 0\), there exists \(N \in \NN\) s.t.
		\begin{align*}
			m \ge n \ge N \Longrightarrow \left | \sum_{k=n}^{m}x_n \right | < \eps.
		\end{align*}
	\end{thm}
	\begin{proof}
		\
		\(\sum_n x_n\) converges if and only if \((s_n)\) converges if and only if \((s_n)\) is Cauchy.
	\end{proof}

특히 위 정리에서 \(m=n\)으로 잡으면, \( \left | x_n \right | < \epsilon\)이므로 다음 따름정리를 얻습니다.

	\begin{cor}
		If \(\sum _n x_n\) converges, then \(\lim _n x_n = 0\).
	\end{cor}

역은 성립하지 않습니다(반례: \(x_n = 1/n\)).

\subsection{Alternating Series Test}

항들이 특정한 형태를 가지는 series는 쉽게 그 수렴을 알 수 있습니다. Real sequence의 각 항의 부호가 +와 -가 반복되고(alternating) 그 절댓값이 0으로 수렴할 때 series는 수렴합니다.

	\begin{lem} \label{alt}
		Real-valued nonnegative sequence \((x_n)\)이 단조감소하면, 임의의 \(m > n\)에 대하여
		\begin{align*}
			\abs{\sum_{k=n}^{m} (-1)^{k-1} x_k} \le x_n.
		\end{align*}
	\end{lem}
	\begin{proof}
		\begin{align*}
			\abs{\sum_{k=n}^{m} (-1)^{k-1} x_k} = x_n - x_{n+1} + ... + (-1)^{m-n} x_m
		\end{align*}
		이다. \(m-n\)이 짝수이면, 각 \(k\)에 대해 \(x_k - x_{k+1} \ge 0\)이므로
		\begin{align*}
			x_n - x_{n+1} + ... + x_m &= x_n -((x_{n+1} - x_{n+2}) + ... + (x_{m-1} - x_m))\\
			& \le x_n.
		\end{align*}
		\(m-n\)이 홀수이면 \(m-n+1\)은 짝수이므로
		\begin{align*}
			x_n - x_{n+1} + ... - x_m &\le x_n - x_{n+1} + ... - x_m + x_{m+1}\\
			& \le x_n.
		\end{align*}
	\end{proof}

	\begin{thm}[Alternating Series Test: 교대급수판정법]
		Real nonnegative sequence \((x_n)\)이 단조감소할 때, \(\lim_n x_n = 0\)이면 \(\sum_n (-1)^{n-1} x_n\)은 수렴한다.
	\end{thm}
	\begin{proof}
		주어진 series의 partial sum \((s_n)\)이 Cauchy sequence인 것을 보이는 것과 동치이다. 임의의 양수 \(\eps > 0\)에 대해 \(n \ge N \Longrightarrow x_n < \eps\)인 \(N\)이 존재한다. 이제 \(m > n \ge N\)이면 Lemma \ref{alt}에 의해
		\begin{align*}
			\abs{s_m - s_n} &= \abs{\sum_{k=n+1}^m (-1)^{k-1} x_k}\\
			& \le x_{n+1} < \eps.
		\end{align*}
	\end{proof}

	\begin{ex}
		\(\sum_n (-1)^{n-1}/n\), \(\sum_{n=2}^\infty (-1)^{n-1}/(\log n)\) 등은 모두 수렴합니다.
	\end{ex}


\subsection{Comparison Test}

급수의 comparison test(비교판정법)는 대단히 직관적입니다.


	\begin{thm}[Comparison Test: 비교판정법]
		\((c_n)\)은 real nonnegative sequence. 각 \(n \in \NN\)에 대하여 \(\abs{a_n} \le c_n\)이고 \(\sum _n c_n\)이 수렴하면 \(\sum _n a_n\)도 수렴.
	\end{thm}
	\begin{proof}
		Cauchy criterion에 의해, 임의의 양수 \(\eps > 0\)에 대하여 \(N \in \NN\)이 존재하여
		\begin{align*}
			m \ge n \ge N \Rightarrow \abs{\sum_{k=n}^m c_k} < \eps
		\end{align*}
		이다. 따라서
		\begin{align*}
			\abs{\sum_{k=n}^m a_k} \le  \sum_{k=n}^m \abs{a_k} \le \sum_{k=n}^m c_k < \eps
		\end{align*}
		이므로, Cauchy criterion에 의해 \(\sum _n a_n\)도 수렴.
	\end{proof}

급수나 수열의 수렴 판정을 할 때 유한 개의 항들은 전혀 영향을 미치지 않습니다. 따라서 위 정리의 `각 \(n \in \NN\)에 대하여'를 `\(n \ge N\)에 대하여'로 바꾸어도 상관이 없습니다.

위 정리의 역은 성립하지 않습니다. 예를 들어 \(x_n = (-1)^n / n\)에 대하여 \(\sum_n x_n\)은 alternating series test에 의하여 수렴하지만 \(\sum_n \abs{x_n} = \sum_n 1/n\)은 발산합니다. 따라서 다음 정의를 생각하는 것이 자연스럽습니다.

	\begin{defn}
		Series \(\sum_n x_n\)이 \textbf{absolutely converge(절대수렴)}한다는 것은 \(\sum_n \abs{x_n}\)이 수렴한다는 것이다. \(\sum_n x_n\)이 \textbf{conditionally converge(조건수렴)}한다는 것은 \(\sum_n x_n\)은 수렴하지만 \(\sum_n \abs{x_n}\)은 발산한다는 것이다.
	\end{defn}

즉 convergent series는 absolutely convergent이거나 conditionally convergent입니다. 특히 real nonnegative series의 경우에 absolutely convergent와 convergent는 동치입니다.

\subsection{Rearrangement}

유한 개의 실수의 덧셈의 순서는 바꾸어도 그 결과가 같은데, 무한 개의 덧셈의 순서를 바꾸어도 결과가 같겠는가, 즉 항들의 순서를 바꾼 series의 성질에 대해 논의할 것입니다.

	\begin{defn}
		\(r: \NN \rightarrow \NN\)이 bijection이라고 하자. Sequence \((x_n)\)에 대하여 \(\sum_n x_{r(n)}\)을 \textbf{rearrangement(재배열급수) of \(\sum_n x_n\)}이라고 한다.
	\end{defn}

이 subsection에서는 \(\sum_n x_n = \sum_n x_{r(n)}\)일 충분조건 2개를 소개합니다. 첫 번째 충분조건은 \((x_n)\)이 real nonnegative sequence인 것입니다.

	\begin{thm}
		\((x_n)\)이 real nonnegative sequence라고 하자. \(\sum_n x_n = s\) for some \(s \in \RR\)이면 임의의 rearrangement에 대하여 \(\sum_n x_{r(n)} = s\)이다. 또 \(\sum_n x_n\)이 발산하면 \(\sum_n x_{r(n)}\)도 발산한다.
	\end{thm}
	\begin{proof}
		각 \(n\)에 대하여 \(m(n) = \max \{r(1), ..., r(n)\}\)으로 정의하면, \(\sum_{k=1}^n x_{r(k)} \le \sum_{k=1}^{m(n)} x_k \le s\)이다(여기서 각 항이 nonnegative임을 사용했다). 따라서 각 항이 nonnegative인 series \(\sum_n x_{r(n)}\)의 partial sum이 bounded이므로, \(\sum_n x_{r(n)}\)은 수렴하고 그 극한 \(t = \sum_n x_{r(n)} \le s\)이다. 그런데 \(r^{-1}:\NN \rightarrow \NN\)도 bijection이므로 같은 논리에 의해 \(s \le t\)이다. 따라서 \(s = t\). \(\sum_n x_n\)이 발산할 때 \(\sum_n x_{r(n)}\)도 발산하는 것은 이제 쉽게 알 수 있다.
	\end{proof}

또 하나의 충분조건은 \(\sum_n x_n\)이 absolutely convergent인 것입니다. index를 조작하는 방법이 조금 까다로우니 유의해서 보시기 바랍니다.

	\begin{thm}
		\(\sum_n x_n\)이 absolutely convergent이고 \(\sum_n x_n = s\)라고 하자. 그러면 임의의 rearrangement에 대해 \(\sum_n x_{r(n)} = s\)이다. 
	\end{thm}
	\begin{proof}
		\(\sum_n x_n\)과 \(\sum_n x_{r(n)}\)의 partial sum을 각각 \((s_n), \: (s'_n)\)이라고 하자. 양수 \(\eps > 0\)이 주어졌다고 하고, Cauchy criterion에 의해
		\begin{align*}
			m \ge n \ge N \Rightarrow \sum_{k=n}^{m} \abs{x_k} < \frac{\eps}{2}
		\end{align*}
		인 \(N\)을 잡는다. 또 \(K \in \NN\)을
		\begin{align*}
			\{1, 2, ..., N\} \subseteq \{r(1), r(2), ..., r(K)\}
		\end{align*}
		을 만족하도록 잡을 수 있고, \(M = \max\{r(1), ..., r(K)\}\)라고 하자. 이때 \(M \ge K \ge N\)임은 쉽게 알 수 있다. 이제 \(n > K\)에 대하여,
		\begin{align*}
			\abs{s'_n - s_n} &= \abs{\sum_{k=1}^{n} x_{r(k)} - \sum_{k=1}^n x_k}\\
			&= \abs{( \sum_{i=1}^N x_i + \sum_{\substack{1 \le k \le n \\ r(k) > N}} x_{r(k)}) - (\sum_{i=1}^N x_i + \sum_{i=N+1}^n x_i)}\\
			&= \abs{\sum_{\substack{1 \le k \le n \\ r(k) > N}} x_{r(k)} - \sum_{i=N+1}^n x_i}\\
			& \le \sum_{\substack{1 \le k \le n \\ r(k) > N}} \abs{x_{r(k)}} + \sum_{i=N+1}^n \abs{x_i}\\
			& \le \sum_{i=N+1}^M \abs{x_i} + \sum_{i=N+1}^n \abs{x_i} < \frac{\eps}{2} + \frac{\eps}{2} = \eps
		\end{align*}
		이므로
		\begin{align*}
			\limsup_n \abs{s'_n - s_n} \le \eps
		\end{align*}
		이다. \(\eps > 0\)은 임의의 양수였으므로 \(\lim_n \abs{s'_n - s_n} = 0\)이고, 따라서
		\begin{align*}
			\lim_n s'_n = \lim_n s_n = s.
		\end{align*}
	\end{proof}

그렇다면 \(\sum_n x_n\)이 conditinally convergent일 때는 어떻게 될까요? 아쉽게도 이때는 rearrangement series가 수렴하는 것조차 보장할 수 없으며, 수렴한다 해도 그 극한이 같다고 할 수 없습니다. 대신 real series의 경우에 상당히 흥미로운 결과가 도출되는데, `양/음의 무한대로 발산하거나 임의의 실수로 수렴하는 rearrangement를 항상 잡을 수 있다'는 것입니다.

이를 위해서는 상극한과 하극한의 새로운 정의를 소개하는 것이 도움이 됩니다. 이는 앞에서 배운 정의와 동치입니다. 약간 흐름에서 벗어난 내용이기 때문에 정리의 statement만 봐도 될 것 같습니다.

	\begin{lem}
		Metric space \(X\)의 sequence \((x_n)\)에 대하여
		\begin{center}
			\(x \in X\)로 수렴하는 \((x_n)\)의 subsequence가 존재한다
		\end{center}
		를 만족하는 모든 \(x \in X\)의 집합을 \(E\)라고 하면, \(E\)는 closed subset of \(X\).
	\end{lem}
	\begin{proof}
		Derive set \(E' = \emptyset\)이면 증명 끝. \(E'\)가 nonempty라고 가정하고, \(E\)의 임의의 limit point \(p\)가 \(E\)의 원소임을 보이면 된다.\\
		먼저 \(x_{n(1)} \neq p\)인 \(n(1)\)을 택한다. (만약 이러한 \(n(1)\)이 없으면 \(E = \{p\}\)이므로 증명이 끝난다.) \(\delta = d(p, x_{n(1)})\)라고 하고, \(d(p, x_{n(k)}) \le 2^{k-1} \delta\)를 만족하도록 \(n(k)\)를 귀납적으로 정하려고 한다. \\
		\(n(1) < ... < n(k)\)가 위의 성질을 만족하도록 정해졌다고 가정하자. \(p\)는 \(E\)의 limit point이므로, \(d(p, q) < 2^{-k-1} \delta\)인 \(q \in E\)가 존재한다. \(E\)의 정의에 의해, \(n(k+1) > n(k)\)이고 \(d(q, x_{n(k+1)}) < 2^{-k-1}\delta\)인 \(n(k+1)\)을 찾을 수 있다. 이제 \(d(p, x_{n(k+1)}) \le d(p, q) + d(q, x_{n(k+1)}) < 2^{-k} \delta\)이므로 우리가 원하는 \(n(k+1)\)을 찾았다.\\
		이제 subsequence \((x_{n(k)})\)가 \(p\)로 수렴하는 것은 쉽게 알 수 있다. 따라서 \(p \in E\).
	\end{proof}

양의 무한대나 음의 무한대로 발산하는 경우에 \(x_n \rightarrow \infty\), \(x_n \rightarrow -\infty\) 같은 표기를 자연스럽게 쓰기로 하겠습니다.

	\begin{thm}[New Definition of Upper and Lower Limits] \label{limsup}
		Real sequence \((x_n)\)에 대하여
		\begin{center}
			\(x_{n(k)} \rightarrow x\)인 subsequence \((x_{n(k)})\)가 존재한다
		\end{center}
		를 만족하는 모든 \(x \in \RR \cup \{\pm \infty\}\)의 집합을 \(E\)라고 하면,
		\begin{align*}
			\limsup_n x_n = \sup E, \quad \liminf_n x_n = \inf E
		\end{align*}
		이다. 특히 \(x_n\)이 bounded이면 Lemma 2.11.에 의해 \(E\)는 closed bounded subset of \(\RR\)이므로 maximum과 minimum이 존재한다. 따라서
		\begin{align*}
			\limsup_n x_n = \max E, \quad \liminf_n x_n = \min E.
		\end{align*}
	\end{thm}
	\begin{proof}
		(언제나 그랬듯이) \(\limsup\)에 대해서만 증명하면 된다.\\
		먼저 \(\limsup_n x_n = \infty\)이면 \((x_n)\)은 not bounded above이므로, 각 \(k\)에 대해 \(x_{n(k)} > k\)인 \(n(1) < n(2) < ...\)을 잡을 수 있다. 이때 \(\lim_n x_{n(k)} = \infty\)이므로 \(\sup E = \infty\). 또 \(\limsup_n x_n = -\infty\)이면 \(\liminf_n x_n = -\infty\)이므로 \(\lim_n x_n = -\infty\)이다. 따라서 \(E = \{-\infty\}\)이므로(왜 그런가?) \(\sup E = -\infty\).\\
		이제 \(\limsup_n x_n = x\)가 finite real number라고 하자. Theorem \ref{thm limsup}\을 생각해 보면, 각 \(k\)에 대하여 \(\abs{x_n - x} < \frac{1}{k}\)인 \(n\)이 무한히 많이 존재한다. 따라서 \(\abs{x_{n(k)} - x} < \frac{1}{k}\)인 \(n(1) < n(2) < ...\)를 잡을 수 있다. 이제 \(x_{n(k)} \rightarrow x\)이므로 \(x \in E\)이다. 따라서 \(\sup E \ge x\).\\
		\(y > x\)인 \(y \in E\)가 존재한다고 가정하자. 그러면 \(x_{n'(k)} \rightarrow y\)인 subsequence \((x_{n'(k)})\)가 존재한다. 즉 \(\abs{x_n - x} > \frac{y-x}{2}\)인 \(n\)이 무한히 많이 존재하는데, 이는 상극한의 조건에 모순이다. 따라서 \(x = \sup E\)이다.
	\end{proof}

이제 우리의 결론을 state할 준비가 되었습니다.

	\begin{thm}
		\(\sum_n x_n\)이 conditionally convergent real series라고 하자.
		\begin{align*}
			-\infty \le \alpha \le \beta \le \infty
		\end{align*}
		인 \(\alpha, \beta\)에 대하여(즉 \(\alpha = -\infty\)나 \(\beta = \infty\) 같은 경우도 포함), 적당한 rearrangement \(\sum_n x_{r(n)}\)이 존재하여 그 partial sum \((s'_n)\)이
		\begin{align*}
			\liminf_n s'_n = \alpha, \quad \limsup_n s'_n = \beta
		\end{align*}
		를 만족한다.
	\end{thm}
	\begin{proof}
		\begin{align*}
			p_n = \frac{\abs{x_n} + x_n}{2}, \quad p_n = \frac{\abs{x_n} - x_n}{2}
		\end{align*}
		으로 정의하자. 즉 \((p_n)\)은 \((x_n)\)의 음수 항을 모두 0으로 만든 것이고, \((q_n)\)은 \((x_n)\)의 양수 항을 모두 0으로 만든 뒤 음수 항의 부호를 반대로 한 것이다. 이때 \(p_n + q_n = \abs{x_n},\: p_n - q_n = x_n, \: p_n, q_n \ge 0\)이다. \(\sum_n x_n\)이 conditionally converge하기 위해서는 \(\sum_n p_n\)과 \(\sum_n q_n\) 모두 발산해야 한다.\\
		\((x_n)\)의 0 이상의 항들을 순서대로 나열한 것을 \(P_1, P_2, P_3, ...\)이라고 하고, \((x_n)\)의 음수 항들을 순서대로 나열한 후 부호를 반대로 한 것을 \(Q_1, Q_2, Q_3, ...\)라고 하자(0 이상의 항과 음수 항은 각각 무수히 많이 존재해야 한다. 왜 그런가?). 그러면 \(\sum_n P_n, \sum_n Q_n\)과 \(\sum_n p_n, \sum_n q_n\)은 zero term밖에 차이가 없으므로, \(\sum_n P_n, \sum_n Q_n\)은 모두 발산해야 한다. 그리고 \(\lim_n x_n = 0\)이므로 \(\lim_n P_n = \lim_n Q_n = 0\)이다.\\
		증가수열 \(m, l: \NN \rightarrow \NN\)에 대하여,
		\begin{equation}
			P_1 + ... + P_{m(1)} - Q_1 - ... - Q_{l(1)} + P_{m(1)+1} + ... + P_{m(2)} - Q_{l(1)+1} - ... - Q_{l(2)} + ... \notag
		\end{equation}
		은 rearrangement series이다. 이 series가 원하는 조건을 만족하도록 \(m, l\)을 잡으려고 한다.\\
		먼저
		\begin{align*}
			\alpha_n \rightarrow \alpha, \beta_n \rightarrow \beta, \alpha_n < \beta_n, \beta_1 > 0
		\end{align*}
		을 만족하는 real sequence \((\alpha_n), (\beta_n)\)을 잡을 수 있다. \(m(1), l(1)\)을
		\begin{gather*}
			P_1 + ... + P_{m(1)} > \beta_1\\
			P_1 + ... + P_{m(1)} - Q_1 - ... - Q_{l(1)} < \alpha_1
		\end{gather*}
		을 만족하는 최소의 자연수로 정의한다. 다음으로 \(m(2), l(2)\)를
		\begin{gather*}
			P_1 + ... + P_{m(1)} - Q_1 - ... - Q_{l(1)} + P_{m(1) + 1} + ... + P_{m(2)} > \beta_2\\
			P_1 + ... + P_{m(1)} - Q_1 - ... - Q_{l(1)} + P_{m(1) + 1} + ... + P_{m(2)} - Q_{l(1)+1} - ... - Q_{l(2)} < \alpha_2
		\end{gather*}
		을 만족하는 최소의 자연수로 정의한다. \(\sum_n P_n, \sum_n Q_n\)이 발산하기 때문에, 이 과정을 반복하여 \(m(k), l(k)\)를 계속 정의할 수 있다.\\
		이렇게 만들어진 rearrangement series의 parial sum \((s'_n)\)의 두 subsequence
		\begin{gather*}
			x_k = P_1 + ... + P_{m(1)} - ... + P_{m(k-1)+1} + ... + P_{m(k)}\\
			y_k = P_1 + ... + P_{m(1)} - ... - Q_{l(k-1)+1} - ... - Q_{l(k)}
		\end{gather*}
		를 생각하자. 만약 두 sequence가 수렴한다면, upper limit과 lower limit의 새로운 정의(Theorem \ref{limsup})에 의해
		\begin{align*}
			\lim_k x_k = \limsup_n s'_n, \quad \lim_k y_k = \liminf_n s'_n
		\end{align*}
		인 것은 쉽게 알 수 있다. 따라서 \(\lim_n x_n = \beta, \lim_n y_n = \alpha\)인 것을 보이면 된다. \(m(k)\)와 \(l(k)\)의 정의에 의해,
		\begin{gather*}
			x_k - P_{m(k)} \le \beta_k < x_{n(k)}\\
			y_k + Q_{l(k)} \ge \alpha_k > y_{n(k)}
		\end{gather*}
		이다. 즉
		\begin{gather*}
			\abs{x_k - \beta_k} \le P_{m(k)}, \quad \abs{y_k - \alpha_k} \le -Q_{l(k)}
		\end{gather*}
		이다. 그런데 \(\lim_k P_{m(k)} = \lim_k Q_{l(k)} = 0\)이므로, \(x_k \rightarrow \beta, y_k \rightarrow \alpha\)이다.
		
	\end{proof}

위 정리에서 \(\alpha = \beta\)인 경우에는 바로 다음 따름정리를 얻을 수 있습니다.

	\begin{cor}
		\(\sum_n x_n\)이 conditionally convergent real series라고 하자. \(-\infty \le \alpha \le \infty\)에 대하여(즉 \(\alpha = \pm \infty\)인 경우도 포함), 적당한 rearrangement \(\sum_n x_{r(n)}\)이 존재하여 \(\sum_n x_{r(n)} = \alpha\)이다.
	\end{cor}
	\begin{proof}
		\(\limsup_n s'_n = \liminf_n s'_n = \alpha\)이면 \(\lim_n s'_n = \alpha\).
	\end{proof}

	\begin{ex}
		\(\sum_n (-1)^{n-1}\frac{1}{n}\)은 conditionally convergent series입니다. 우선 이 series의 partial sum을 \((s_n)\)이라 하고, \(\lim_n s_n = s\)라고 하겠습니다(나중에 \(s = \log 2\)임을 알게 됩니다). 그리고 다음 rearrangement의 partial sum을 \((s'_n)\)이라 하겠습니다.
		\begin{align*}
			1 + \frac{1}{3} - \frac{1}{2} + \frac{1}{5} + \frac{1}{7} - \frac{1}{4} + \frac{1}{9} +\frac{1}{11} - \frac{1}{6} + ...
		\end{align*}
		먼저 \(\lim_n s'_{3n}\)이 수렴한다는 것을 보이려고 합니다.
		\begin{align*}
			s'_{3n} &= \sum_{k=1}^n (\frac{1}{4k-3} + \frac{1}{4k-1} - \frac{1}{2k})\\
			&= \sum_{k=1}^n (\frac{1}{4k-3} - \frac{1}{4k-2} + \frac{1}{4k-1} - \frac{1}{4k}) + \sum_{k=1}^{n} (\frac{1}{4k-2} - \frac{1}{4k})\\
			&= s_{4n} + \frac{1}{2} s_{2n}
		\end{align*}
		따라서 \(\lim_n s'_{3n} = \frac{3}{2}s\)로 수렴합니다. 그리고
		\begin{align*}
			\abs{s'_{3n+1} - s'_{3n}} < \abs{s'_{3n+2} - s'_{3n}} = \frac{1}{4n+1} + \frac{1}{4n+3}
		\end{align*}
		이므로, \(\lim_n s'_{3n+1} = \lim_n s'_{3n+2} = \lim_n s'_{3n} = \frac{3}{2} s\)입니다. 이제 \(\lim_n s'_n = \frac{3}{2}s\)인 것은 극한의 정의에 의해 쉽게 알 수 있습니다.
	\end{ex}

\subsection{Abel's Test}
교대급수판정법 일반화하려고 합니다. 먼저 다음이 성립하는 것을 관찰합시다(부분적분과 식의 형태가 비슷함에 주목하기 바랍니다).

\begin{prop}[Summation by Parts]
	Sequence \((a_n), (b_n)\)에 대하여,
	\begin{align*}
		A_n = \sum_{k=1}^{n} a_k, A_0 = 0
	\end{align*}
	로 정의하자. 그러면 임의의 \(1 \le m < n\)에 대해
	\begin{align*}
		\sum_{k=m+1}^{n} a_k b_k = A_n b_{n+1} - A_m b_{m+1} - \sum_{k=m+1}^n A_k (b_{k+1} - b_k)
	\end{align*}
	이다.
\end{prop}
\begin{proof}
	\begin{align*}
		\sum_{k=1}^n a_k b_k &= \sum_{k=1}^n (A_k - A_{k-1})b_k\\
		&= \sum_{k=1}^n A_k b_k - \sum_{k=1}^n A_{k-1} b_k\\
		&= \sum_{k=1}^n A_k b_k - (\sum_{k=1}^n A_{k} b_{k+1} - A_{n} b_{n+1})\\
		&= A_n b_{n+1} - \sum_{k=1}^n A_k (b_{k+1} - b_k)
	\end{align*}
	이므로,
	\begin{align*}
		\sum_{k=1}^n a_k b_k - \sum_{k=1}^m a_k b_k
	\end{align*}
	를 계산하면 원하는 등식을 얻는다.
\end{proof}

이를 이용하면 \(\sum_n a_n b_n\)꼴의 series의 수렴을 판정할 때 도움이 됩니다.

\begin{thm}[Abel's Test]
	Sequence \((a_n), (b_n)\)이 다음을 만족한다고 하자.
	\begin{enumerate}[label=(\alph*), leftmargin=2\parindent]
		\item
		\(\sum_n a_n\)의 partial sum은 bounded.
		\item
		\((b_n)\)은 real-valued이고 단조감소수열.
		\item
		\(\lim_n b_n = 0\)
	\end{enumerate}
	그러면 series \(\sum_n a_n b_n\)은 수렴한다.
\end{thm}
\begin{proof}
	Cauchy criterion에 의해, 임의의 양수 \(\eps > 0\)에 대해 
	\begin{align*}
		n > m \ge N \Rightarrow \abs{\sum_{k=m+1}^{n} a_k b_k} < \eps
	\end{align*}
	인 \(N\)이 존재하는 것을 보이면 된다.\\
	\(\sum_n a_n\)의 partial sum을 \((A_n)\)이라 하면, 모든 \(n\)에 대해 \(\abs{A_n} \le M\)인 \(M > 0\)이 존재한다. 그리고 \(b_N < \frac{\eps}{3M}\)인 \(N\)을 잡는다. 이제 \(n > m \ge N\)이면,
	\begin{align*}
		\abs{\sum_{k=m+1}^{n} a_k b_k} &= \abs{A_n b_{n+1} - A_m b_{m+1} - \sum_{k=m+1}^n A_k (b_{k+1} - b_k)} \\
		& \le \abs{A_n} b_{n+1} + \abs{A_m} b_{m+1} + \sum_{k=m+1}^n \abs{A_k}\abs{b_{k+1}-b_k}\\
		& \le  Mb_{n+1} + M b_{m+1} + M \sum_{k=m+1}^n (b_k - b_{k+1})\\
		& < M \times \frac{\eps}{3M} + M \times \frac{\eps}{3M} + M \times b_{m+1}\\
		& < \frac{\eps}{M} + \frac{\eps}{M} + \frac{\eps}{M} = \eps
	\end{align*}
\end{proof}

\(a_n = (-1)^{n-1}\)로 두면, alternating series test와 동일합니다.

(b)는 `\(\sum_n \abs{b_{n+1} - b_n}\)이 수렴한다'로 약화시킬 수 있습니다.

\subsection{Cauchy Product}

이제 두 급수의 곱을 정의하려고 합니다. (엄밀한 표기는 아니지만) 다음을 관찰합시다. (Index의 편리함을 위해, 0부터 시작하는 급수를 생각합니다.)

\begin{align*}
	&(a_0 + a_1 + a_2 + \ldots + a_n + \ldots)(b_0 + b_1 + b_2 + \ldots + b_n + \ldots)\\
	= &\: a_0 b_0 + (a_0 b_1 + a_1 b_0) + (a_0 b_2 + a_1 b_1 + a_2 b_0) + \ldots + \sum_{k=0}^n a_k b_{n-k} + \ldots
\end{align*}

따라서 급수의 곱을 다음과 같이 정의하는 것은 자연스럽습니다.

\begin{defn}
	두 급수 \(\sum_n a_n, \sum_n b_n\)에 대하여
	\begin{gather*}
		c_n = \sum_{k=0}^{n} a_k b_{n-k}
	\end{gather*}
	로 정의하자. 이때 급수 \(\sum_n c_n\)을 두 급수의 \textbf{Cauchy product(코시곱)}라고 한다.
\end{defn}

이제 우리의 관심사는 Cauchy product가 진정한 급수의 곱인가, 즉 \(\sum_n a_n = A, \sum_n b_n = B\)로 수렴하면
\begin{enumerate}[label=(\alph*), leftmargin=2\parindent]
	\item
	\(\sum_n c_n\)이 수렴하는가?
	\item
	\(\sum_n c_n = AB\)인가?
\end{enumerate}
에 대한 답이 긍정적인지입니다. 안타깝게도 항상 그렇지는 않습니다.

\begin{ex}
	\begin{gather*}
		a_n = b_n = \frac{(-1)^n}{\sqrt{n+1}}
	\end{gather*}
	으로 정의하면 alternating series test에 의해 \(\sum_n a_n, \sum_n b_n\)은 수렴합니다. 그 Cauchy product \(\sum_n c_n\)은
	\begin{gather*}
		c_n = (-1)^n \sum_{k=0}^{n} \frac{1}{\sqrt{(k+1)(n-k+1)}}
	\end{gather*}
	로 정의되는데, 산술-기하평균 부등식에 의해
	\begin{align*}
		\abs{c_n} &= \sum_{k=0}^{n} \frac{1}{\sqrt{(k+1)(n-k+1)}}\\
		& \ge \sum_{k=0}^{n} \frac{1}{(n+2)/2} = \sum_{k=0}^{n} \frac{2}{n+2} = \frac{2(n+1)}{n+2}
	\end{align*}
	이므로 \(c_n \rightarrow 0\)이 아닙니다. 따라서 \(\sum_n c_n\)은 발산.
\end{ex}

이제 (a)와 (b)의 답이 긍정적이도록 하는 충분조건 하나를 소개하겠습니다. 그것은 \(\sum_n a_n\)과 \(\sum_n b_n\) 중 적어도 하나가 절대수렴하는 것입니다.

\begin{thm}
	\(\sum_n a_n=A, \sum_n b_n=B\)로 수렴하고, \(\sum_n a_n\)이 절대수렴한다고 하자. 그러면 그 Cauchy product \(\sum_n c_n\)은 수렴하고, \(\sum_n c_n = AB\)이다.
\end{thm}
\begin{proof}
	\(\sum_n a_n, \sum_n b_n, \sum_n c_n\)의 partial sum들을 각각 \((A_n), (B_n), (C_n)\)으로 두자. \(A_n B \rightarrow AB\)임은 명백하므로, \(C_n - A_n B \rightarrow 0\)임을 보이면 된다. [왜 이런 발상이 가능한가? 절대수렴하는 급수가 좀 더 `빠르게' 극한에 가까워진다고 생각할 수 있기 때문이다. 그래서 \(C_n - A B_n\)이 아니라, \(C_n - A_n B\)의 수렴을 보인다(굉장히 수학적이지 않은 문장이다\ldots).]\\
	\(\beta_n = B_n - B\)로 두고, 다음을 관찰하자.
	\begin{align*}
		C_n &= a_0 b_0 + (a_0 b_1 + a_1 b_0) + \ldots (a_0 b_n + a_n b_{n-1} + a_n b_0)\\
		&= a_0 B_n + a_1 B_{n-1} + \ldots + a_n B_0\\
		&= a_0 (B + \beta_n) + a_1 (B + \beta_{n-1}) + \ldots + a_n (B + \beta_0)\\
		&= A_n B + a_0 \beta_n + a_1 \beta_{n-1} + \ldots + a_n \beta_0
	\end{align*}
	이제
	\begin{gather*}
		\gamma_n = a_0 \beta_n + a_1 \beta_{n-1} + \ldots + a_n \beta_0
	\end{gather*}
	로 정의하고, (a)를 이용하기 위해
	\begin{gather*}
		M = \sum_n \abs{a_n} < \infty
	\end{gather*}
	로 정의하자. 이제 양수 \(\eps > 0\)이 주어졌다고 하고 \(n \ge N \Rightarrow \abs{\beta_n} < \eps\)인 \(N\)을 택한다. \(n > N\)에 대하여
	\begin{align*}
		\abs{\gamma_n} &= \abs{\beta_0 a_n + \ldots + \beta_N a_{n-N}} + \abs{\beta_{N+1} a_{n-N-1} + \ldots + \beta_n a_0}\\
		& \le \abs{\beta_0 a_n + \ldots + \beta_N a_{n-N}} + \eps(\abs{a_{n-N-1}} + \ldots + \abs{a_0})\\
		& \le \abs{\beta_0 a_n + \ldots + \beta_N a_{n-N}} + M\eps
	\end{align*}
	이다. \(N\)을 고정하고 양변에 \(\limsup_n\)을 취하면,
	\begin{gather*}
		\limsup_n \abs{\gamma_n} \le M\eps
	\end{gather*}
	이다. \(\eps > 0\)이 임의의 양수였으므로, \(\lim_n \gamma_n = 0\).
\end{proof}
\begin{rem}
	\(\sum_n \abs{a_n} < \infty\)인 성질이 없으면 위 증명이 어디서 깨지는지 생각해보기 바랍니다.
\end{rem}

또 하나의 궁금증: (a)는 성립하는데 (b)가 성립하지 않는 경우, 즉 \(\sum_n c_n\)이 수렴하는데 그 극한이 \(AB\)는 아닌 경우가 가능할까요? 이러한 경우는 불가능하다는 것이 알려져 있습니다. 즉 (a)가 성립하면 (b)는 항상 성립합니다. 이 경우에 \(\sum_n a_n\)이나 \(\sum_n b_n\) 중 하나의 절대수렴 조건은 요구되지 않습니다.

\begin{thm}[Abel]
	\(\sum_n a_n = A, \sum_n b_n = B\)로 수렴하고 그 Cauchy product도 \(\sum_n c_n = C\)로 수렴하면 \(AB = C\)이다.
\end{thm}

한참 후에 power series에 대해 공부하고 나면, 이 정리는 당연해집니다. 증명은 그때로 미루겠습니다.

\subsection{Cauchy Condensation Test}

다음 판정법은 좀 재미있습니다.

\begin{thm}[Cauchy Condensation Test]
	Real sequence \((a_n)\)이 nonnegative이고 단조감소할 때, 다음은 동치.
	\begin{enumerate}[label=(\alph*), leftmargin=2\parindent]
		\item
		\(\sum_{n=1}^\infty a_n\)이 수렴.
		\item
		\(\sum_{k=0}^\infty 2^k a_{2^k} = a_1 + 2a_2 + 4a_4 + 8a_8 + \ldots\)가 수렴.
	\end{enumerate}
\end{thm}
\begin{proof}
	두 급수의 partial sum을 각각
	\begin{gather*}
		s_n = a_1 + a_2 + \ldots + a_n\\
		t_k = a_1 + 2a_2 + \ldots + 2^k a_{2^k}
	\end{gather*}
	로 두자. 먼저 \(n \le 2^N - 1\)인 \(N\)을 택하면,
	\begin{align*}
		s_n \le s_{2^N - 1} &= a_1 + (a_2 + a_3) + (a_4 + \ldots + a_7) + \ldots  (a_{2^{N-1}} + \ldots + a_{2^N - 1})\\
		& \le a_1 + 2a_2 + 4a_4 + \ldots + 2^{N-1} a_{2^{N-1}} = t_{N-1}
	\end{align*}
	이므로 \((t_k)\)가 bounded이면 \((s_n)\)도 bounded이다. 따라서 (b)\(\Rightarrow\)(a). 한편
	\begin{align*}
		t_k &= a_1 + 2a_2 + 4a_4 + \ldots + 2^k a_{2^k}\\
		&\le 2(a_1 + a_2 + 2a_4 + \ldots 2^{k-1} a_{2^k})\\
		&\le 2(a_1 + a_2 + (a_3 + a_4) + \ldots + (a_{2^{k-1}+1} + \ldots a_{2^k}))\\
		&= 2s_{2^k}
	\end{align*}
	이므로 \((s_n)\)이 bounded이면 \((t_k)\)도 bounded이다. 따라서 (a)\(\Rightarrow\)(b).
\end{proof}

의외로 다음 판정법을 가장 쉽게 유도하는 방법이 위의 Cauchy condensation test입니다(이거 증명하려고 적분을 이용하는 건 좀 쥐 잡는 데 소 잡는 칼 쓰는 느낌\ldots 게다가 우리는 아직 적분을 정의하지도 않았습니다).

\begin{cor}[\(p\)-series Test]
	\(\sum_n 1 / n^p\) converges if and only if \(p > 1\).
\end{cor}
\begin{proof}
	Cauchy condensation test에 의해, \(\sum_n 1 / n^p\) converges if and only if \(\sum_k 2^k / (2^k)^p = \sum_k 1 / (2^k)^{p-1}\) converges if and only if \(p - 1 > 0\).
\end{proof}

\(\log\)가 들어간 급수도 Cauchy condensation test를 이용하면 편리합니다.

\begin{ex}
	\(\sum_{n=2}^\infty 1/(n(\log n)^p)\) converges if and only if \(p > 1\).
\end{ex}

\subsection{Root Test and Ratio Test}

이제 마지막 수렴판정법 2개입니다. 사실상 실전에서 가장 많이 쓰게 되는 것들이 아닐까 생각합니다.

\begin{thm}[Root Test: 근 판정법]
	\(\sum_n a_n\)에 대하여 \(\alpha = \limsup_n \sqrt[n]{\abs{a_n}}\)으로 정의하자. 이때
	\begin{enumerate}[label=(\alph*), leftmargin=2\parindent]
		\item
		\(\alpha < 1\)이면 \(\sum_n a_n\)은 절대수렴한다.
		\item
		\(\alpha > 1\)이면 \(\sum_n a_n\)은 발산한다.
		\item
		\(\alpha = 1\)이면 \(\sum_n a_n\)의 수렴 여부를 알 수 없다.
	\end{enumerate}
\end{thm}
\begin{proof}
	\(\alpha < 1\)이면 \(\alpha < \beta < 1\)인 \(\beta\)를 택할 수 있다. 상극한의 정의에 의해, \(n \ge N \Rightarrow \sqrt[n]{\abs{a_n}} < \beta\)인 \(N\)이 존재한다. 따라서 \(n \ge N\)이면 \(\abs{a_n} < \beta^n\)이므로
	\begin{gather*}
		\sum_{n=N}^{\infty} \abs{a_n} \le \sum_{n=N}^{\infty} \beta^n < \infty.
	\end{gather*}
	\(\alpha > 1\)이면 \(\lim_n \abs{a_n} = 0\)일 수 없으므로 \(\sum_n a_n\)이 발산한다.\\
	마지막으로 \(a_n = 1/n, b_n = 1/n^2\)을 생각하면 \(\limsup_n \sqrt[n]{\abs{a_n}} = \limsup_n \sqrt[n]{\abs{b_n}} = 1\)이지만(\(\lim_n n^{1/n}=1\)임을 이용하였다) \(\sum_n a_n\)은 발산, \(\sum_n b_n\)은 수렴한다.
\end{proof}

\begin{thm}[Ratio Test: 비 판정법]
	\(\sum_n a_n\)에 대하여 다음이 성립한다(단, \(a_n \neq 0\)).
	\begin{enumerate}[label=(\alph*), leftmargin=2\parindent]
		\item
		\(\limsup_n \abs{a_{n+1}/a_n} < 1\)이면 \(\sum_n a_n\)은 절대수렴한다.
		\item
		\(n \ge n_0 \Rightarrow \abs{a_{n+1}/a_n} \ge 1\) for some \(n_0\)이면 \(\sum_n a_n\)은 발산한다.
	\end{enumerate}
\end{thm}
\begin{proof}
	\(\alpha = \limsup_n \abs{a_{n+1}/a_n} < 1\)이면 \(\alpha < \beta < 1\)인 \(\beta\)를 택할 수 있다. 상극한의 정의에 의해 \(n \ge N \Rightarrow \abs{a_{n+1}/a_n} < \beta\)인 \(N\)이 존재한다. 따라서 \(n \ge N\)이면 \(\abs{a_n} < \beta^{n-N}\abs{a_N}\)이므로
	\begin{gather*}
		\sum_{n=N}^\infty \abs{a_n} \le \sum_{n=N}^\infty \beta^{n-N}\abs{a_N} < \infty.
	\end{gather*}
	\(n \ge n_0 \Rightarrow \abs{a_{n+1}/a_n} \ge 1\) for some \(n_0\)이면 \(\lim_n \abs{a_n} = 0\)일 수 없으므로 \(\sum_n a_n\)이 발산한다.
\end{proof}
\begin{rem}
	(b)의 충분조건은 \(\liminf_n \abs{a_{n+1}/a_n} > 1\)인 것입니다. 따라서 \(\limsup_n \abs{a_{n+1}/a_n} < 1\)이거나 \(\liminf_n \abs{a_{n+1}/a_n} > 1\)인 경우에는 상극한과 하극한의 계산만으로 ratio test를 적용할 수 있고, 그렇지 않은 경우에는 \(\abs{a_{n+1}/a_n}\) 자체에 대한 관찰이 필요합니다.
\end{rem}

다음 정리는 root test와 ratio test의 판정 범위를 비교해 줍니다.

\begin{thm}
	\((a_n)\)이 sequence of real positive numbers라고 하자. 이때 다음 부등식이 성립한다.
	\begin{gather*}
		\liminf_n \frac{a_{n+1}}{a_n} \le \liminf_n \sqrt[n]{a_n} \le \limsup_n \sqrt[n]{a_n} \le \limsup_n \frac{a_{n+1}}{a_n}
	\end{gather*}
\end{thm}
\begin{proof}
	두 번째 부등호는 당연하다. 세 번째 부등호가 성립하는 것만 보이면, 첫 번째 부등호가 성립하는 것은 비슷하게 보일 수 있다.\\
	\(\alpha = \limsup_n a_{n+1}/a_n\)으로 정의하자. \(\alpha = \infty\)이면 증명할 것이 없으므로 \(\alpha < \infty\)라고 가정하면, \(\alpha < \beta < \infty\)인 \(\beta\)를 택할 수 있다. 상극한의 정의에 의해 \(n \ge N \Rightarrow a_{n+1}/a_n < \beta\)인 \(N\)이 존재한다. 이제 \(n \ge N\)에 대해
	\begin{align*}
		a_n &< \beta^{n-N} a_N\\
		\sqrt[n]{a_n} &< \beta^{1-\frac{N}{n}} \sqrt[n]{a_N}
	\end{align*}
	이므로, 양변에 \(\limsup_n\)을 취하면
	\begin{align*}
		\limsup_n \sqrt[n]{a_n} \le \beta
	\end{align*}
	이다. 그런데 \(\beta\)는 \(\alpha\)보다 큰 임의의 실수이므로, \(\limsup_n \sqrt[n]{a_n} \le \alpha\).
\end{proof}

위 정리의 의미를 생각해봅시다. Ratio test에 의해 \(\sum_n a_n\)의 수렴을 보일 수 있으면, 즉 \(\limsup_n a_{n+1}/a_n < 1\)이면 \(\limsup_n \sqrt[n]{a_n} \le \limsup_n a_{n+1}/a_n < 1\)이므로 root test에 의해서도 \(\sum_n a_n\)의 수렴을 보일 수 있습니다. 또 root test로 \(\sum_n a_n\)의 수렴 여부를 결정할 수 없으면, 즉 \(\limsup_n \sqrt[n]{a_n} = 1\)이면 \(\limsup_n a_{n+1}/a_n \ge 1\)이고 \(\liminf_n a_{n+1}/a_n \le 1\)이므로 ratio test에 의해서도 \(\sum_n a_n\)의 수렴 여부를 결정할 수 없습니다.
\par 다음 예시는 root test에 의해서 수렴 판정이 가능하지만 ratio test에 의해서는 수렴 판정이 불가능한 급수의 존재를 보여줍니다. 즉 root test가 ratio test보다 `강력한' 판정법입니다. 그런데 \(\limsup_n \sqrt[n]{a_n}\)의 값을 구하기가 어려울 때가 많기 때문에 실전에서는 ratio test도 자주 쓰이는 것입니다.

\begin{ex}
	다음 급수 \(\sum_n a_n\)을 생각합시다.
	\begin{gather*}
		\frac{1}{2^1} + \frac{1}{2^0} + \frac{1}{2^3} + \frac{1}{2^2} + \frac{1}{2^5} + \frac{1}{2^4} + \ldots
	\end{gather*}
	다음은 쉽게 계산할 수 있습니다.
	\begin{gather*}
		\liminf_n \sqrt[n]{a_n} = \limsup_n \sqrt[n]{a_n} = \frac{1}{2}\\
		\liminf_n \frac{a_{n+1}}{a_n} = \frac{1}{8}, \quad \limsup_n \frac{a_{n+1}}{a_n} = 2
	\end{gather*}
	따라서 root test에 의해서는 \(\sum_n a_n\)이 수렴함을 알 수 있지만 ratio test는 정보를 주지 못합니다.
\end{ex}

\subsection{Application: \(e\)}

Ratio test에 의해 \(\sum_{n=0}^\infty 1/n!\)이 수렴합니다.

\begin{defn}
	\(e = \sum_{n=0}^\infty 1/n!\).
\end{defn}

\(n! \ge 2^{n-1}\)임을 이용하면, \(2 < e \le 1 + \sum_{n=1}^\infty 1/2^{n-1} = 3\)임을 알 수 있습니다.

고등학교 미적분에서는 다음 정리를 주어진 사실로 받아들이고 이를 \(e\)의 정의로 사용하였습니다. 이제 우리는 이를 증명할 수 있습니다.

\begin{thm} \label{thm e}
	\(\lim_n (1 + 1/n)^n = e\).
\end{thm}
\begin{proof}
	\begin{gather*}
		s_n = \sum_{k=0}^{n} \frac{1}{k!}, \quad t_n = \left(1 + \frac{1}{n}\right)^n
	\end{gather*}
	으로 두자. 먼저
	\begin{align*}
		t_n &= \sum_{k=0}^{n} \frac{n(n-1)\ldots(n-k+1)}{k!}\left (\frac{1}{n}\right )^k\\
		&= \sum_{k=0}^{n}\frac{1}{k!}\left(1 - \frac{1}{n}\right)\ldots\left(1 - \frac{k-1}{n}\right) \le \sum_{k=0}^{n}\frac{1}{k!} = s_n
	\end{align*}
	이므로 양변에 \(\limsup_n\)을 취하면
	\begin{align*}
		\limsup_n t_n \le \limsup_n s_n = \lim_n s_n = e.
	\end{align*}
	또 \(n \ge m\)이면,
	\begin{align*}
		t_n &= \sum_{k=0}^{n}\frac{1}{k!}\left(1 - \frac{1}{n}\right)\ldots\left(1 - \frac{k-1}{n}\right)\\
		&\ge \sum_{k=0}^{m}\frac{1}{k!}\left(1 - \frac{1}{n}\right)\ldots\left(1 - \frac{k-1}{n}\right)
	\end{align*}
	이다. \(m\)을 고정하고 양변에 \(\liminf_n\)을 취하면,
	\begin{align*}
		\liminf_n t_n &\ge \liminf_n \sum_{k=0}^{m}\frac{1}{k!}\left(1 - \frac{1}{n}\right)\ldots\left(1 - \frac{k-1}{n}\right)\\
		&= \lim_n \sum_{k=0}^{m}\frac{1}{k!}\left(1 - \frac{1}{n}\right)\ldots\left(1 - \frac{k-1}{n}\right) = \sum_{k=0}^{m}\frac{1}{k!}
	\end{align*}
	이므로
	\begin{align*}
		\liminf_n t_n \ge t_m
	\end{align*}
	이다. 그런데 이는 임의의 \(m\)에 대해 성립하므로 양변에 \(\lim_m\)을 취하면
	\begin{align*}
		\liminf_n t_n \ge e
	\end{align*}
	이다. 따라서 \(\liminf_n t_n \ge e \ge \limsup_n t_n\)이므로, \(\lim_n t_n = e\).
\end{proof}

다음 사실도 이제서야 증명할 수 있습니다.

\begin{thm}
	\(e\)는 무리수이다.
\end{thm}
\begin{proof}
	\(s_n\)을 Theorem \ref{thm e}의 증명에서와 같이 정의한다.\\
	\(e\)가 유리수라고 가정하면 어떤 자연수 \(p, q\)에 대해 \(e = p/q\)로 쓸 수 있다. \(n > m\)일 때 \(n!/m! > m^{n-m}\)인 것을 이용하면,
	\begin{align*}
		0 < e - s_q &= \sum_{k=1}^\infty \frac{1}{(q+k)!}\\
		&< \frac{1}{(q+1)!}\sum_{k=1}^\infty \frac{1}{(q+1)^{k-1}} = \frac{1}{(q+1)!} \frac{1}{q} < \frac{1}{q!q}
	\end{align*}
	이다. 따라서
	\begin{align*}
		0 < q!e - q!s_q < \frac{1}{q}
	\end{align*}
	이다. 그런데
	\begin{align*}
		q!e - q!s_q = (q-1)!p - \sum_{k=0}^q \frac{q!}{k!}
	\end{align*}
	는 정수이므로 모순이다. 따라서 \(e\)는 무리수이다.
\end{proof}

\newpage

\section{Compactness and Connectedness} \label{sec cpt}

\section{Continuity}

\section{Differentiation}

\section{Riemann-Stieltjes Integral}

\section{Sequence of Functions}

\section{Function Space}

\section{Functions Defined by Integral}

\section{Fourier Series}

\section{Multivariable Calculus (미정)}


\begin{thebibliography}{99}
	
	\bibitem{Cd94}\label{ref1} 김성기, 김도한, 계승혁. \emph{해석개론}. 서울대학교출판문화원. 2011.
	\bibitem{Cd94}\label{ref2} W. Rudin. \emph{Principles of Mathematical Analysis}. McGraw-Hill. 1976.
	\bibitem{Cd94} 이인석. \emph{선형대수와 군}. 서울대학교출판문화원. 2015.
	\bibitem{Cd94} J. R. Munkres, \emph{Topology}. Prentice Hall. 2000.
		
\end{thebibliography}
\end{document}