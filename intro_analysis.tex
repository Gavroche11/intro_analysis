\documentclass[12pt]{article}

\usepackage[a4paper,left=20mm,right=20mm,top=40mm,bottom=40mm]{geometry}
\usepackage[utf8]{inputenc}
\usepackage{amsmath, amsthm, amsfonts, kotex, enumitem, setspace, hyperref, lipsum}
\setstretch{1.4}


\theoremstyle{definition}
\newtheorem{thm}{Theorem}[section]
\newtheorem{cor}[thm]{Corollary}
\newtheorem{lem}[thm]{Lemma}
\newtheorem{prop}[thm]{Proposition}
\newtheorem{defn}[thm]{Definition}
\newtheorem{notn}[thm]{Notation}
\newtheorem{obs}[thm]{Observation}
\newtheorem*{rem}{Remark}
\newtheorem*{ex}{Example}

\def\NN{\mathbb{N}}
\def\ZZ{\mathbb{Z}}
\def\QQ{\mathbb{Q}}
\def\RR{\mathbb{R}}
\def\CC{\mathbb{C}}
\def\eps{\epsilon}
\def\calR{\mathcal{R}}

\newcommand{\abs}[1]{\left\vert#1\right\vert}
\newcommand{\norm}[1]{\left\vert\left\vert#1\right\vert\right\vert}
\newcommand{\inner}[2]{\langle #1, #2\rangle}

\title{해석개론}
\author{\texttt{@Gavroche11}}

\begin{document}
\maketitle

\tableofcontents
\newpage


1년 동안 공부한 해석개론의 내용을 [\ref{ref1}]\과 [\ref{ref2}]의 내용을 중심으로 복습하는 김에, {\LaTeX} 연습도 좀 하고, (정말 낮은 확률로) 여기에 관심 있는 독자가 존재한다면 재미있는 내용을 전달하기 위해서 쓰는 글입니다. 저도 일개 학부생이니 오류나 오타가 있을 수 있으며, 같은 대상을 지칭하는 영어와 한국어 용어도 혼용하고 있고, 한번에 쓰는 글이 아니므로 서술 방식과 문체도 글 내부에서 상당히 다를 수 있습니다.

\section{Completeness of \(\RR\)}

해석개론의 출발점은 실수 집합 \(\RR\)의 성질을 규명하는 것입니다. 이를 위해서 몇 가지 정의가 필요합니다.

\subsection{Field}

	\begin{defn}
		집합 \(F\)에 이항 연산 \(+\)와 \(\cdot\)이 정의되어\footnote{집합 \(X\)에 이항 연산 \(*\)이 정의되었다는 것은 함수 \(*: X \times X \rightarrow X\)가 잘 정의되었다는 뜻. 다시 말해서, 임의의 \(x, y \in F\)에 대해 \(*(x, y) = x * y\)가 \(X\) 안의 유일한 원소로 결정된다는 뜻이다.} 있을 때, 다음 (F1)-(F9)를 만족하면 \((F, +, \cdot)\) 또는 간단하게 \(F\)를 \textbf{field(체)}라고 한다. \(+\)와 \(\cdot\)을 각각 덧셈, 곱셈이라고 한다.
		\begin{enumerate} [label=(F\arabic*), leftmargin=2\parindent]
			\item
			\(x + (y+z) = (x+y)+z\) for \(x, y, z \in F\).
			\item
			\(0 \in F\)가 존재하여 [\(x+0=0+x=x\) for \(x \in F\)].
			\item
			\(x \in F\)에 대하여 \(-x \in F\)가 존재하여 \mbox{\(x+(-x)=(-x)+x=0\).}
			\item
			\(x+y=y+x\) for \(x, y \in F\).
			\item
			\(x(yz)=(xy)z\) for \(x, y, z \in F\).
			\item
			\(1 \in F \backslash \{0\}\)이 존재하여 [\(x\cdot 1=1 \cdot x=x\) for \(x \in F\)].
			\item
			\(x \in F \backslash \{0\}\)에 대하여 \(x^{-1} \in F\)가 존재하여 \(xx^{-1}=x^{-1}x=1\).
			\item
			\(xy=yx\) for \(x, y \in F\).
			\item
			\(x(y+z)=xy+xz, (x+y)z = xz+yz\) for \(x, y, z \in F\).
		\end{enumerate}
	\end{defn}
	
다음이 성립하는 것을 관찰할 수 있습니다.

	\begin{prop} \label{field addition}
		Field \(F\)에 대하여 다음이 성립한다(\(x, y, z \in F\)).
		\begin{enumerate} [label=(\alph*), leftmargin=2\parindent]
			\item
			\(x + y = x + z\)이면 \(y = z\).
			\item
			\(y \in F\)가 [\(x + y = y + x = x\) for \(x \in F\)]를 만족하면 \(y = 0\).\footnote{사실 이것이 없으면 (F3)은 non-sense.}
			\item
			\(x + y = y + x = 0\)이면 \(y = -x\).
			\item
			For \(x \in F\), \(-(-x) = x\).
		\end{enumerate}
	\end{prop}
	\begin{proof}
		\quad\\
		(a): \(y = 0 + y = (-x + x) + y = -x + (x + y) = -x + (x + z) = (-x + x) + z = 0 + z = z\).\\
		(b): \(x + y = x = x + 0\)이므로, (a)에 의해 \(y=0\).\\
		(c): \(x + y = 0 = x + (-x)\)이므로, (a)에 의해 \(y = -x\).\\
		(d): \(-x + x = x + (-x) = 0\)이므로, (c)에 의해 \(x = -(-x)\).
	\end{proof}

	\begin{prop} \label{field mult}
		Field \(F\)에 대하여 다음이 성립한다(\(x, y, z \in F\)).
		\begin{enumerate} [label=(\alph*), leftmargin=2\parindent]
			\item
			\(xy = xz\)이면 \(y = z\). (\(x, y, z \in F, x \neq 0\))
			\item
			\(y \in F\)가 [\(xy = yx = x\) for \(x \in F\)]를 만족하면 \(y = 1\).\footnote{사실 이것이 없으면 (F7)은 non-sense.}
			\item
			\(xy = yx = 1\)이면 \(y = x^{-1}\). (\(x, y \in F, x \neq 0\))
			\item
			For \(x \in F \backslash \{0\}\), \((x^{-1})^{-1} = x\).
		\end{enumerate}
	\end{prop}
	\begin{proof}
		Proposition \ref{field addition}의 증명과 거의 같다.
	\end{proof}

	\begin{prop}
		Field \(F\)에 대하여 다음이 성립한다(\(x, y \in F\)).
		\begin{enumerate} [label=(\alph*), leftmargin=2\parindent]
			\item
			\(0x = 0\).
			\item
			\(xy = 0\)이면 \(x = 0\) or \(y = 0\).
			\item
			\((-1)x = -x\).
			\item
			\((-x)y = x(-y) = -(xy)\)이고 \((-x)(-y)=xy\).
		\end{enumerate}
	\end{prop}
	\begin{proof}
		\quad\\
		(a): \(0x + 0 = 0x = (0+0)x = 0x + 0x\)이므로 Proposition \ref{field addition} (a)에 의해.\\
		(b): \(xy = 0\)이면서 \(x \neq 0, y \neq 0\)이면 \(x^{-1}, y^{-1}\)이 존재하므로
		\begin{gather*}
			0 = 0(y^{-1}x^{-1}) = (xy)(y^{-1}x^{-1}) = xx^{-1} = 1
		\end{gather*}
		이므로 모순.\\
		(c): \(x + (-x) = 0 = 0x = (1 + (-1))x = 1x + (-1)x = x + (-1)x\)이므로 Proposition \ref{field addition} (a)에 의해.\\
		(d): (c)에 의해 당연.
	\end{proof}

	\begin{ex}
		\(\QQ, \RR, \CC\) 등은 모두 field입니다. 주로 \(\RR\)을 다룰 것입니다.
	\end{ex}

\subsection{Order}

어떤 수가 다른 수보다 `크다'고 하는 것은 어떤 의미일까요? 순서의 개념을 일반화합니다.

	\begin{defn}
		집합 \(S\)에 주어진 관계(relation) \(>\)가 다음 조건을 만족하면 \(>\)를 \textbf{strict partial order}라고 한다.
		\begin{enumerate} [label=(\alph*), leftmargin=2\parindent]
			\item
			\(x > x\)인 \(x \in S\)는 존재하지 않는다.
			\item
			\(x > y\)이고 \(y > z\)이면 \(x > y > z\). (\(x, y, z \in S\))
		\end{enumerate}
		Strict partial order \(>\)가 다음 조건까지 만족하면 \(>\)를 \textbf{strict total order}라고 한다.
		\begin{enumerate}[label=(\alph*), leftmargin=2\parindent]
			\setcounter{enumi}{2}
			\item
			\(x, y \in S\)에 대해 \(x > y, x = y, y > x\) 중 오직 하나만 참.
		\end{enumerate}
	\end{defn}

	\begin{ex}
		\(\RR, \QQ, \ZZ, \NN\) 등에 주어진 일상적인 의미의 \(>\)가 strict total order임은 쉽게 확인할 수 있습니다.
	\end{ex}

	\begin{rem}
		편의를 위해 \(<, \ge, \le\)의 기호도 우리가 생각하는 의미 그대로 사용하기로 합니다.
	\end{rem}

	\begin{defn}
		Strict total order \(>\)가 주어진 field \(F\)가 다음 조건을 만족할 때 \(F\)를 \textbf{ordered field(순서체)}라고 한다.
		\begin{enumerate} [label=(\alph*), leftmargin=2\parindent]
			\item
			\(x < y\)이면 \(x + z < y + z\). (\(x, y, z \in F\))
			\item
			\(x > 0, y > 0\)이면 \(xy > 0\). (\(x, y \in F\))
		\end{enumerate}
	\end{defn}

	\begin{ex}
		\(\RR\)이나 \(\QQ\)에 일상적인 의미의 \(>\)를 주면 ordered field가 됩니다.
	\end{ex}

	\begin{defn} \label{bdd in R}
		Ordered field \(F\)의 부분집합 \(S \subseteq F\)가 \textbf{bounded above(위로 유계)}라는 것은 \(\beta \in F\)가 존재하여 [\(x \le \beta\) for all \(x \in S\)]라는 것이다. 이때 \(\beta\)를 \textbf{upper bound(상계) of \(S\)}라고 한다. 부등호 방향을 반대로 하여 \textbf{bounded below(하계)}와 \textbf{lower bound(하계)}도 정의한다. \(S\)가 bounded above and bounde below이면 \(S\)는 \textbf{bounded(유계)}라고 한다.
	\end{defn}
	
	\begin{defn}
		Ordered field \(F\)의 bounded above subset \(S\)에 대하여 \(\alpha \in F\)가 다음 조건을 만족할 때 \(\alpha\)를 \textbf{least upper bound=supremum(최소상계=상한) of \(S\)}라고 하고, \(\alpha = \sup S\)로 쓴다.
		\begin{enumerate} [label=(\alph*), leftmargin=2\parindent]
			\item
			\(\alpha\) is an upper bound of \(S\).
			\item
			If \(\gamma \in F\) is an upper bound of \(S\), then \(\alpha \le \gamma\).
		\end{enumerate}
	\end{defn}

\subsection{Complete Ordered Field}

Ordered field \(F\)의 bounded above subset \(S \subseteq F\)의 supremum은 항상 존재할까요? 이 질문에 긍정적인 답을 하게 할 수 있게 하는 \(F\)가 우리의 관심사입니다.

	\begin{defn} \label{comp axiom}
		다음 두 성질을 만족하는 ordered field \(F\)는 \textbf{least-upper-bound property} 또는 \textbf{completeness axiom(완비성공리)}을 만족시킨다고 하며 \(F\)를 \textbf{complete ordered field(완비순서체)}라고 한다.
		\begin{enumerate} [label=(\alph*), leftmargin=2\parindent]
			\item
			Nonempty bounded above subset \(S \subseteq F\) has a supremum in \(F\).
			\item
			Nonempty bounded below subset \(S \subseteq F\) has a infimum in \(F\).
		\end{enumerate}
	\end{defn}

사실 (a)와 (b)는 동치입니다.

	\begin{prop}
		Definition \ref{comp axiom}의 (a)와 (b)는 동치이다.
	\end{prop}
	\begin{proof}
		(a)$\Rightarrow$(b)만 보이면 역은 마찬가지 방법으로 보일 수 있다. (a)를 가정하고, nonempty bounded above subset \(S \subseteq F\)와 \(S\)의 모든 lower bound들의 집합 \(T\)를 생각하자. \(S\)가 bounded below이므로 \(T\)는 공집합이 아니고, \(S\)의 아무 원소를 가져오면 그 원소는 \(T\)의 upper bound이므로 \(T\)는 nonempty bounded above이다. 가정 (a)에 의해 \(T\)는 supremum \(\alpha \in F\)를 가진다. 우리의 주장: \(\alpha\)는 \(S\)의 infimum이다.\\
		임의의 \(x \in S\)에 대해 \(x\)는 \(T\)의 upper bound이므로 \(\alpha \le x\)이다. 따라서 \(x\)는 \(S\)의 lower bound이다. 한편 \(\gamma \in F\)가 \(S\)의 lower bound라고 하면 \(T\)의 정의에 의해 \(\gamma \in T\)인데, \(\alpha = \sup T\)이므로 \(\gamma \le \alpha\)이다. 따라서 \(\alpha = \inf S\)이다.
	\end{proof}

모든 ordered field가 complete인 것은 아닙니다. 대표적으로 \(\QQ\)는 complete가 아닌데, \(\{x \in \QQ: x^2 < 2\}\)는 nonempty bounded above subset of \(\QQ\)이지만 \(\QQ\)에서 supremum을 가지지 않음을 쉽게 확인할 수 있습니다.

그렇다면 실제로 complete ordered field가 존재하는지가 궁금한데, 그 답은 긍정적입니다.

	\begin{thm}
		Complete ordered field \(F\)가 존재하고, 그 \(F\)를 실수체 \(\RR\)이라고 부른다.
	\end{thm}

수학에서 어떤 것의 존재성을 보일 때는 다른 정리의 도움을 받거나, 아니면 실제로 그 대상을 구성(construct)해야 합니다.

우리가 수 체계를 배웠던 과정을 생각해보면, 자연수에서 정수, 유리수까지는 나름 자연스러운 확장이었고, 실수에서 복소수도 자연스러운 확장인데 – 유리수와 실수 사이에 어떤 gap이 있었습니다. 적어도 저한테는 그랬습니다. 제곱해서 2가 되는 유리수가 없다는 것으로 유리수체가 어떤 의미로 ‘불완전함’을 설명하곤 하지만, 이는 유리수체에서 실수체로의 확장이라기보다는 유리수체에서 field of algebraic number(algebraic number, 대수적 수: 유리계수 다항식의 근)로의 확장에 가깝습니다. 이제서야 우리는 실수가 어떻게 만들어졌는지 알았습니다. \textbf{유리수체 \(\QQ\)의 빈틈을 채워 넣어서 completeness axiom을 만족하도록 실수체 \(\RR\)을 직접 ‘구성’한 것입니다.} 그래서 실수체가 complete인 것은 증명의 대상이 아닌데, 왜냐하면 그것을 만족하도록 실수체를 만들었기 때문입니다.

\(\QQ\)로부터 \(\RR\)를 구성하여 정의하는(실수의 구성적 정의) 대표적인 방법에는 두 가지가 있는데, 하나는 Cantor의 방법이고 하나는 Dedekind의 방법입니다. 다른 글에서 이를 다룰 기회가 있을 것입니다.

여기서 드는 한 가지 의문은, 두 가지 방법으로 만든 실수체가 동일한가 하는 것입니다. 좀 더 일반적으로, complete ordered field는 유일한가라는 질문을 던질 수 있는데, 실제로 그렇습니다. 이 증명 역시 다른 글에서 다룰 기회가 있을 것입니다.

	\begin{thm}
		The complete ordered field is unique up to isomorphism. i.e., for any complete ordered fields \(F_1\) and \(F_2\), there exists a bijection \(f: f_1 \rightarrow F_2\) satisfying:
		\begin{enumerate} [label=(\alph*), leftmargin=2\parindent]
			\item
			\(x < y \Longrightarrow f(x) < f(y) \quad (x, y \in F_1)\)
			\item
			\(f(x + y) = f(x)+f(y), f(xy)=f(x)f(y) \quad (x, y \in F_1)\)
		\end{enumerate}
		따라서 \(\RR\)은 유일한 complete ordered field이다.
	\end{thm}

\newpage

\section{Sequence} \label{sec sequence}

\subsection{Metric Space} \label{sec metric space}

수열의 수렴을 이야기하기 위해 거리의 개념을 도입하려고 합니다.

	\begin{defn} \label{metric}
		집합 \(X\)와 함수 \(d: X \times X \rightarrow \RR\)가 주어졌을 때 임의의 \(x, y, z \in X\)에 대해 다음이 성립하면 \((X, d)\) 또는 간단하게 \(X\)가 \textbf{metric space(거리공간)}라고 하고 \(d\)를 \textbf{metric on \(X\)}라고 한다.
		\begin{enumerate} [label=(M\arabic*), leftmargin=2\parindent]
			\item
			\(d(x, y) = 0\) if and only if \(x = y\).
			\item
			\(d(x, y) = d(y, x)\).
			\item
			\(d(x, z) \le d(x, y) + d(y, z)\)
		\end{enumerate}
	\end{defn}

	\begin{prop}
		Metric space \(X\)에서 \(d(x, y) \ge 0\) for all \(x, y \in X\).
	\end{prop}
	\begin{proof}
		\(0 = d(x, x) \le d(x, y) + d(y, x) = 2d(x, y)\).
	\end{proof}

	\begin{ex}
		\quad
		\begin{enumerate} [label=(\alph*), leftmargin=2\parindent]
			\item
			\(\RR\) 또는 \(\CC\)에서 \(d(x, y) = \abs{x - y}\)로 정의하면 \(d\)는 (M1)-(M3)을 만족하므로 \((\RR, d), (\CC, d)\)는 metric space.
			\item
			평면이나 공간에서 우리가 일상적으로 생각하는 `거리'는 (M1)-(M3)을 만족하므로\footnote{Definition \ref{def euc space} 참조.} 평면과 공간은 metric space.
			\item
			아무 집합 \(X\)에서 \(x, y \in X\)에 대해 \(d(x, y)\)를 \(x = y\)일 때 0, \(x \neq y\)일 때 1로 정의하면 (M1)-(M3)을 만족하므로 \((X, d)\)는 metric space.
		\end{enumerate}
	\end{ex}

\subsection{Convergence of a Sequence}

	\begin{defn} \label{converge}
		집합 \(X\)에 대하여 \(\NN\)(또는 \(\ZZ\))에서 \(X\)로 가는 함수를 \textbf{sequence(수열) in \(X\)}라고 하고 \((x_n)\)과 같이 쓴다.\\
		\(X\)가 metric space일 때 \(X\) 안의 sequence \((x_n)\)이 \(x \in X\)로 \textbf{converge(수렴)}한다는 것은 임의의 양수 \(\eps > 0\)에 대해 다음을 만족시키는 \(N \in \NN\)이 존재하는 것이다.
		\begin{gather*}
			n \ge N \Longrightarrow d(x_n, x) < \eps
		\end{gather*}
		이때 \(\lim_{n \rightarrow \infty}x_n = x, \lim_n x_n = x, x_n \rightarrow x\)와 같이 쓴다.
	\end{defn}

	\begin{thm} \label{thm metric limit unq}
		Metric space \(X\) 안의 sequence \((x_n)\)이 \(x \in X\)로 수렴하고 \(x' \in X\)로 수렴하면 \(x = x'\).
	\end{thm}
	\begin{proof}
		\(x \neq x'\)라고 가정하면 metric의 정의에 의해 \(d(x, x') > 0\)이다. \(r = d(x, x') / 2\)라고 하자. 극한의 정의에 의해 다음을 만족하는 \(N_1, N_2\)가 존재한다.
		\begin{gather*}
			n \ge N_1 \Longrightarrow d(x_n, x) < r\\
			n \ge N_2 \Longrightarrow d(x_n, x') < r
		\end{gather*}
		\(N = \max\{N_1, N_2\}\)라고 하면 \(d(x, x') \le d(x, x_N) + d(x, x_N') < 2r = d(x, x')\)이므로 모순.
	\end{proof}

	\begin{defn} \label{bdd}
		Metric space \(X\)의 부분집합 \(S \subseteq X\)가 \textbf{bounded(유계)}라는 것은 어떤 양수 \(M > 0\)이 존재하여 [\(d(x, y) < M\) for all \(x, y \in S\)]라는 것이다.\\
		Sequence \((x_n)\) in \(X\)가 \textbf{bounded}라는 것은 \(X\)의 부분집합 \(\{x_n\}_{n \in \NN}\)이 bounded라는 것이다.
	\end{defn}
	
	\begin{rem}
		\(X = \RR\)일 때 Definition \ref{bdd in R}에서의 bounded의 정의와 Definition \ref{bdd}의 정의는 동치입니다(증명 생략).
	\end{rem}

	\begin{prop} \label{conv bdd}
		Metric space \(X\) 안의 sequence \((x_n)\)이 수렴하면 \((x_n)\)은 유계이다.
	\end{prop}
	\begin{proof}
		\(x_n \rightarrow x\)라고 하면 극한의 정의에 의해 \(n \ge N \Rightarrow d(x_n, x) < 1\) for some \(N\)이다. 이때 \(\{x_n\}_{n < N}\)은 유한집합이므로 유계이고, \(\{x_n\}_{n \ge N}\)의 아무 점 2개를 잡아도 두 점 사이의 거리는 2보다 작으므로 유계이다. 유계 부분집합 2개의 합집합은 유계이므로 \(\{x_n\}_{n \in \NN}\)도 유계이다.
	\end{proof}

이제부터 Section \ref{sec sequence}\가 끝날 때까지 metric space \(X\)를 \(d(x, y) = \abs{x - y}\)가 주어진 \(\RR\)로 한정하고 이 안의 실수열만 생각합니다.

다음 성질을 증명하는 것은 연습문제로 남깁니다.

	\begin{prop}
		실수열 \((x_n)\)과 \((y_n)\)이 각각 \(x, y \in \RR\)로 수렴할 때 다음이 성립한다.
		\begin{enumerate} [label=(\alph*), leftmargin=2\parindent]
			\item
			\(x_n + y_n \rightarrow x + y\).
			\item
			\(cx_n \rightarrow cx\) for \(c \in \RR\).
			\item
			\(x_ny_n \rightarrow xy\).
			\item
			\(x_n/y_n \rightarrow x/y\), if \(y_n \neq 0\) for all \(n\) and \(y \neq 0\).
		\end{enumerate}
	\end{prop}

\subsection{Monotonic Sequence}

Definition \ref{converge}\를 보면, 어떤 수열 \((x_n)\)의 수렴을 보이기 위해서 \textbf{수렴할 것으로 예상되는 점 \(x\)를 먼저 찾아야} 합니다. 그런데 그 \(x\)를 찾는 것이 쉽지 않을 때가 많습니다. 그래서 우리는 어디로 수렴하는지 모르는 상태에서 수열의 수렴을 판정할 수 있는 많은 방법들을 개발해야 하는데, 단조수열은 그 중 첫 번째입니다.

	\begin{defn}
		실수열 \((x_n)\)이 \textbf{monotonically increasing=nondecreasing sequence(단조증가수열)}이라는 것은 \(x_n \le x_{n+1}\) for all \(n\)이라는 것이다. 마찬가지 방법으로 \textbf{monotonically decreasing=nonincreasing sequence(단조감소수열)}도 정의한다. Monotonically increasing or monotonically decreasing sequence를 \textbf{monotonic sequence(단조수열)}이라고 한다.
	\end{defn}

다음은 단조수열이 \(\RR\)에서 수렴할 필요충분조건입니다.

	\begin{thm} [Monotone Convergence Theorem, 단조수렴정리]
		실수열 \((x_n)\)이 단조수열일 때, \((x_n)\) is convergent if and only if \((x_n)\) is bounded.
	\end{thm}
	\begin{proof}
		($\Rightarrow$) 방향은 Proposition \ref{conv bdd}에 의해 성립하므로 그 역만 보이면 된다. 일반성을 잃지 않고 \((x_n)\)이 유계 단조증가수열이라고 가정하자. 그러면 \(S = \{x_n\}_{n \in \NN}\)은 nonempty bounded above subset of \(\RR\)이므로, \textbf{\(\RR\)의 완비성에 의해} \(S\)의 supremum \(x \in \RR\)가 존재한다. 주장: \(x_n \rightarrow x\).\\
		임의의 \(\eps > 0\)에 대해 \(x - \eps\)은 \(S\)의 upper bound가 아니므로, \(x_N > x - \eps\)인 \(N\)이 존재한다. \((x_n)\)은 단조증가수열이므로 \(n \ge N\)에 대해 \(x - \eps < x_N \le x_n \le x\). 따라서 \((x_n)\)은 \(x\)로 수렴한다.
	\end{proof}
	\begin{rem}
		\(\RR\)의 완비성의 핵심적인 결과들 중 하나입니다. 이 정리는 complete가 아닌 field, 예를 들면 \(\QQ\)에서 성립하지 않습니다.
	\end{rem}

\subsection{Upper Limit and Lower Limit} \label{sec limsup}

	\begin{ex}
		실수열 \((x_n)\)을 다음과 같이 정의합니다.
		\begin{gather*}
			x_n = 
			\begin{cases}
				\frac{1}{n}, & \text{if } n \text{ is odd}\\
				-n, & \text{if } n \text{ is even}
			\end{cases}
		\end{gather*}
		\((x_n)\)이 수렴하지 않는 것은 명백합니다. 홀수항만 취해서 보면 이 수열의 `일부분'은 0으로 수렴하는 단조감소수열입니다. 이 0에 어떤 의미를 부여하려고 하는 것이 Section \ref{sec limsup}의 목표입니다.
	\end{ex}

먼저 표기법 하나.

	\begin{notn}
		실수열 \((x_n)\)에 대하여, 임의의 \(M > 0\)에 대해 적당한 \(N \in \NN\)이 존재하여 \(n \ge N \Rightarrow x_n > M\)을 만족시키면 \(\lim_n x_n = \infty\)로 쓴다. 마찬가지 방법으로 \(\lim_n x_n = -\infty\)로 쓴다.\\
		Nonempty \(S \subseteq \RR\)가 위로 유계가 아닐 때 \(\sup S = \infty\), 아래로 유계가 아닐 때 \(\inf S = -\infty\)로 쓴다.\\
		\(x_n \rightarrow \alpha\) for \(\alpha \in \RR \cup \{\pm \infty\}\)라는 것은 \((x_n)\)이 \(\alpha \in \RR\)로 수렴하거나, \(\lim_n x_n = \infty\) 또는 \(\lim_n x_n = -\infty\)라는 것이다. (즉, 진동하지 않는다는 의미.)
	\end{notn}

	\begin{obs} \label{obs sup}
		Nonempty subset \(A, B \subseteq \RR\)에 대하여 \(A \subseteq B\)이면 \(\sup A \le \sup B\)이고 \(\inf A \ge \inf B\). \(\sup\)이나 \(\inf\)의 값이 \(\pm\infty\)인 경우에도 성립한다.
	\end{obs}
	\begin{proof}
		당연하다.
	\end{proof}

	\begin{defn} \label{def limsup}
		실수열 \((x_n)\)의 \textbf{upper limit(상극한)}을 다음과 같이 정의하고, 그 값을 \(\limsup_n x_n \in \RR \cup \{\pm \infty\}\)으로 쓴다.
		\begin{enumerate} [label=(\alph*), leftmargin=2\parindent]
			\item
			\((x_n)\)이 위로 유계가 아니면 \(\limsup_n x_n = \infty\)로 정의.
			\item
			\((x_n)\)이 위로 유계이면 각 자연수 \(n\)에 대하여 집합 \(S_n = \{x_k\}_{k \ge n}\)을 생각한다. 이때 각 \(S_n\)은 위로 유계이므로 supremum이 유한한 실수로 존재한다. 이 값을 \(y_n\)으로 정의하자. 즉 \(y_n = \sup S_n\). Observation \ref{obs sup}에 의해 \((y_n)\)은 단조감소수열이다. 이때 \(\limsup_n x_n = \lim_n y_n \in \RR \cup \{-\infty\}\)로 정의한다. (\((y_n)\)이 아래로 유계이면 \(\lim_n y_n\)은 유한한 실수이고, 그렇지 않으면 \(-\infty\)이다.)
		\end{enumerate}
		마찬가지 방법으로 \textbf{lower limit(하극한)}을 정의하고 그 값을 \(\liminf_n x_n \in \RR \cup \{\pm \infty\}\)으로 쓴다.
	\end{defn}

도대체 이게 무슨 소리인가? 필자도 이 개념을 이해하는 데 상당히 많은 시간이 걸렸습니다. 다음 정리는 이 정의를 좀 더 구체적으로 설명해 줍니다.

	\begin{thm} \label{thm limsup}
		실수열 \((x_n)\)과 \(\alpha \in \RR\)에 대하여 \(\limsup_n x_n = \alpha\)일 필요충분조건은 다음 (a), (b)가 성립하는 것이다.
		\begin{enumerate} [label=(\alph*), leftmargin=2\parindent]
			\item
			임의의 \(\eps > 0\)에 대해 적당한 \(N \in \NN\)이 존재하여 \(n \ge N \Rightarrow x_n < \alpha + \eps\).
			\item
			임의의 \(\eps > 0\)에 대해 \(x_n > \alpha - \eps\)를 만족하는 \(n\)이 무수히 많이 존재한다.
		\end{enumerate}
	\end{thm}
	\begin{proof}
		($\Leftarrow$): 먼저 \(\limsup_n x_n = \alpha\)라고 가정하고 \(\eps > 0\)이 주어졌다고 하자. Definition \ref{def limsup}에서와 같이 단조감소수열 \((y_n)\)을 정의하면 \(\lim_n y_n = \alpha\)이므로 \(y_N < \alpha + \eps\)인 \(N\)이 존재한다. 이때 \(n \ge N\)이면 \(x_n \le y_N < \alpha + \eps\)이므로 (a)가 성립한다. 이제 (b)를 부정하여 \(x_n > \alpha + \eps\)을 만족하는 \(n\)이 유한 개만 존재한다고 가정하자. 그러면 충분히 큰 자연수 \(K\)가 존재하여 \(n \ge K \Rightarrow x_n \le \alpha - \eps\), 즉 \(y_K \le \alpha - \eps\)이다. 그런데 \((y_n)\)은 단조감소수열이므로 \(\alpha \le y_K = \alpha - \eps\)이고 따라서 모순이다.\\
		($\Rightarrow$): 이제 (a)와 (b)를 가정하고 \(\eps > 0\)이 주어졌다고 하자. (a)에 의해 \((x_n)\)은 위로 유계이므로 Definition \ref{def limsup}에서와 같이 \((y_n)\)이 잘 정의되고, \(\lim_n y_n \le \alpha + \eps\)이다. 한편 (b)에 의해, 모든 \(n\)에 대해 [\(x_n > \alpha - \eps\) for some \(m > n\)]이므로 \(y_n \ge \alpha - \eps\)이다. 따라서 \((y_n)\)은 유계 단조수열이므로 수렴하고 \(\alpha- \eps \le \lim_n y_n \le \alpha + \eps\)이다. 그런데 \(\eps > 0\)은 임의의 양수이므로 \(\limsup_n x_n = \lim_n y_n = \alpha\)이다.
	\end{proof}

하극한에 대해 마찬가지의 정리를 state하고 증명하는 것은 연습문제로 남깁니다.

상극한과 하극한이 유용한 한 가지 이유는 다음 따름정리 때문입니다.

	\begin{cor}
		실수열 \((x_n)\)과 \(\alpha \in \RR\)에 대하여 \(\lim_n x_n = \alpha\) if and only if \(\limsup_n x_n = \liminf_n x_n = \alpha\).
	\end{cor}
	\begin{proof}
		Theorem \ref{thm limsup}\과 그 하극한 version에 의해 자명.
	\end{proof}

	\begin{cor}
		\(n = 0, 1, \ldots\)에서 정의된 실수열 \((s_n)\)에 대하여
		\begin{gather*}
			\sigma_n = \frac{1}{n}\sum_{k=0}^{n-1}s_k
		\end{gather*}
		로 정의하자. \((s_n)\)이 \(s \in \RR\)로 수렴하면 \((\sigma_n)\)도 \(s\)로 수렴한다.
	\end{cor}
	\begin{proof}
		일반성을 잃지 않고 \(s = 0\)이라고 가정할 수 있다. 양수 \(\eps > 0\)을 고정하고, 다음을 만족시키는 \(N\)을 찾는다.
		\begin{gather*}
			n \ge N \Longrightarrow \abs{s_n} < \eps
		\end{gather*}
		그러면 \(n > N\)에 대해
		\begin{align*}
			\abs{\sigma_n} = \frac{1}{n}\abs{\sum_{k=0}^{n-1}s_k} &\le \frac{1}{n}\abs{\sum_{k=0}^{N}s_k} + \frac{1}{n}\sum_{k=N+1}^{n}\abs{s_k}\\
			&< \frac{1}{n}\abs{\sum_{k=0}^{N}s_k} + \frac{(n-N)\eps}{n}\\
			&<  \frac{1}{n}\abs{\sum_{k=0}^{N}s_k} + \eps
		\end{align*}
		이다. 이제 양변에 \textbf{상극한}을 취한다. (바로 극한을 못 취하는 이유는 좌변이 수렴하는지를 아직 알지 못하기 때문이다. 상극한은 항상 \(\RR \cup \{\pm \infty\}\) 안에서 존재한다.) \(\abs{\sum_{k=0}^{N}s_k}\)은 \(n\)과 무관한 상수이므로,
		\begin{gather*}
			\limsup_n \abs{\sigma_n} \le \limsup_n \left ( \frac{1}{n}\abs{\sum_{k=0}^{N}s_k} + \eps \right )= \lim_n \left ( \frac{1}{n}\abs{\sum_{k=0}^{N}s_k} + \eps \right )= \eps
		\end{gather*}
		이다. 따라서 \(0 \le \liminf_n \abs{\sigma_n} \le \limsup_n \abs{\sigma_n} \le \eps\)인데, \(\eps > 0\)은 임의의 양수였으므로 \(\liminf_n \abs{\sigma_n} = \limsup_n \abs{\sigma_n} = 0\)이고 \(\lim_n \abs{\sigma_n} = \lim_n \sigma_n = 0\)이다.
	\end{proof}

\newpage

\section{Euclidean Space} \label{sec euc}
\subsection{Vector Space}

	\begin{defn}
		Field \(F\)에 대하여 집합 \(V\)에 덧셈과 \(F\)에 의한 스칼라곱이 정의되어 있을 때, 다음 (V1)-(V8)을 만족하면 \(V\)를 \textbf{\(F\)-vector space}라고 하고 V의 원소를 \textbf{vector}라고 한다.
		\begin{enumerate}[label=(V\arabic*), leftmargin=2\parindent]
			\item
			\((u+v)+w=u+(v+w)\) for \(u, v, w \in V\).
			\item
			\(0 \in V\)가 존재하여 [\(v+0=0+v=v\) for \(v \in V\)].
			\item
			\(v \in V\)에 대하여 \(-v \in V\)가 존재하여 \(v+(-v)=(-v)+v=0\).
			\item
			\(v+w=w+v\) for \(v, w \in V\).
			\item
			\(1v = v\) for \(v \in V\).
			\item
			\((ab)v = a(bv)\) for \(a, b \in F, v \in V\).
			\item
			\((a+b)v=av+bv\) for \(a, b \in F, v \in V\).
			\item
			\(a(v+w) = av + aw\) for \(a \in F, v, w \in V\).
		\end{enumerate}
	\end{defn}

다음 성질들을 증명하는 것은 연습문제로 남깁니다.

	\begin{prop}
		\(F\)-vector space \(V\)에 대하여 다음이 성립한다\(u, v, w \in V\).
		\begin{enumerate}[label=(\alph*), leftmargin=2\parindent]
			\item
			\(u + v = u + w\)이면 \(v = w\).
			\item
			\(w \in V\)가 [\(v + w = w + v = v\) for \(v \in V\)]를 만족하면 \(w = 0\).\footnote{사실 이것이 없으면 (V3)은 non-sense.}
			\item
			\(v + w = w + v = v\)이면 \(w = -v\).
			\item
			For \(v \in V\), \(-(-v) = v\).
		\end{enumerate}
	\end{prop}

	\begin{prop}
		\(F\)-vector space \(V\)에 대하여 다음이 성립한다\(a \in F, v, w \in V\).
		\begin{enumerate}[label=(\alph*), leftmargin=2\parindent]
			\item
			\(0v = v\).
			\item
			\(av = 0\)이면 \(a = 0\) or \(v = 0\).
			\item
			\((-1)v = -v\).
			\item
			\((-a)v = a(-v) = -(av)\)이고 \((-a)(-v) = av\).
		\end{enumerate}
	\end{prop}

당분간 다룰 vector space는 다음의 형태입니다.

	\begin{prop}
		Field \(F\)에 대하여 집합
		\begin{gather*}
			F^n = \{(x_1, \ldots, x_n): x_i \in F, i = 1, \ldots, n\}
		\end{gather*}
		에 덧셈과 스칼라곱을 다음과 같이 정의하면 \((c \in F, (x_1, \ldots, x_n), (y_1, \ldots, y_n) \in F^n)\)
		\begin{gather*}
			(x_1, \ldots, x_n) + (y_1, \ldots, y_n) = (x_1 + y_1, \ldots, x_n + y_n), \quad c(x_1, \ldots, x_n) = (cx_1, \ldots, cx_n)
		\end{gather*}
		\(F^n\)은 \(F\)-vector space이다.
	\end{prop}
	\begin{proof}
		시간이 남으면 한번 해 보자.
	\end{proof}


특히 \(F=\RR\)일 때, \(\RR^2\)와 \(\RR^3\)을 각각 \textbf{평면}, \textbf{공간}이라고 불렀습니다. \(\RR^n\) 공간은 우리의 `마음의 고향'이고, 해석개론의 주된 관심사 중 하나입니다. 우리의 목적은, metric space의 구조를 \(\mathbb{R}^n\)에 주는 것입니다.

\subsection{Inner Product Space}
고등학교 때 배웠던 \(\mathbb{R}^2\)와 \(\mathbb{R}^3\)에서의 내적을 임의의 \(\RR\)- 또는 \(\CC\)- vector space로 일반화합니다. Section \ref{sec euc}\이 끝날 때까지 아무 말이 없으면 \(V\)는 \(F\)-vector space이고 \(F\)는 \(\RR\) 또는 \(\CC\).

	\begin{defn}
		함수 \(\inner{\cdot}{\cdot}: V \times V \rightarrow F\)가 모든 \(u, v, w \in V, c \in F\)에 대해 (IP1)-(IP3)를 만족할 때, \(\inner{\cdot}{\cdot}\)를 \textbf{inner product on \(V\)}라고 하고, \((V, \inner{\cdot}{\cdot})\) 또는 간단하게 \(V\)를 \textbf{inner product space}라고 한다.
		\begin{enumerate}[label=(IP\arabic*), leftmargin=2\parindent]
			\item
			\(\inner{\cdot}{\cdot}\)는 첫 번째 성분에 대해 linear.\\
			즉, \(\inner{u+v}{w} = \inner{u}{w}+ \inner{v}{w}\)이고 \(\inner{cv}{w} = c\inner{v}{w}\).
			\item
			\(\inner{v}{w} = \overline{\inner{w}{v}}\)
			\item
			\(v \neq 0\)이면 \(\inner{v}{v} \in \mathbb{R}\)이고 \(\inner{v}{v} > 0\).
		\end{enumerate}
	\end{defn}

\(F=\RR\)이면, \(\inner{v}{w} = \inner{w}{v}\)이고(symmetric) \(\inner{\cdot}{\cdot}\)는 첫 번째와 두 번째 성분에 대해 각각 linear입니다(bilinear).\footnote {fancy하게 말해서 \(\inner{\cdot}{\cdot}\)는 \textbf{symmetric bilinear form}.}

	\begin{prop}
		\(x = (x_1, \ldots, x_n), y = (y_1, \ldots, y_n) \in \RR^n\)에 대하여
		\begin{gather*}
			\inner{x}{y} = x \cdot y = \sum_{i=1}^n x_i y_i
		\end{gather*}
		로 정의하면 \(\inner{\cdot}{\cdot}\)은 inner product이다. 이러한 inner product를 \textbf{dot product} on \(\RR^n\)이라고 한다.
	\end{prop}
	\begin{proof}
		시간이 남으면 한번 해 보자.
	\end{proof}

	\begin{ex}
		\(V\)의 inner product는 여러 가지 방법으로 줄 수 있습니다. 예를 들어 \(\RR^2\)에서
		\begin{gather*}
			\inner{(x_1, x_2)}{(y_1, y_2)} = 3x_1 y_1 + 2x_2 y_2
		\end{gather*}
		로 정의하면 \(\inner{\cdot}{\cdot}\)은 inner product on \(\RR^2\)가 됩니다.
	\end{ex}

고등학교 때 dot product에 대해 배운 다음 정리는 일반적인 inner product에 대해 성립합니다. 증명은 orthogonality의 개념을 공부한 이후로 미루겠습니다.

	\begin{thm}[Cauchy-Schwarz inequality]
		Inner product space \(V\)의 원소 \(v, w\)에 대하여 다음 부등식이 성립한다.
		\begin{gather*}
			\abs{\inner{v}{w}}^2 \le \inner{v}{v}\inner{w}{w}
		\end{gather*}
	\end{thm}

\subsection{Normed Vector Space}
이제 우리의 목표에 거의 근접했습니다. 다음은 vector space의 원소의 `크기'에 대한 정의입니다.

	\begin{defn}
		함수 \(\norm{\cdot}: V \rightarrow \mathbb{R}\)가 모든 \(v, w \in V, c \in F\)에 대해 (N1)-(N3)을 만족할 때, \(\norm{\cdot}\)를 \textbf{norm on \(V\)}라고 하고, \((V, \norm{\cdot})\) 또는 \(V\)를 \textbf{normed vector space}라고 한다.
		\begin{enumerate}[label=(N\arabic*), leftmargin=2\parindent]
			\item
			\(\norm{v+w} \le \norm{v}{w}\)
			\item
			\(\norm{cv} = \abs{c}\norm{v}\)
			\item
			\(\norm{v}=0\)이면 \(v=0\).
		\end{enumerate}
	\end{defn}
	
	\begin{prop}
		Normed vector space \(V\)에서 \(\norm{v} \ge 0\) for all \(v \in V\).
	\end{prop}
	\begin{proof}
		\(0 = \norm{v + (-v)} \le \norm{v} + \norm{-v} = 2\norm{v}\).
	\end{proof}


Norm이 주어지면 자동으로 metric이 주어집니다. 왜냐하면, 두 vector 사이의 거리는 두 vector의 차의 norm으로 정의할 수 있기 때문입니다.

	\begin{prop}
		Normed vector space \(V\)에 대해, \(d:V\times V \rightarrow \mathbb{R}\)을 \(d(v, w)=\norm{v-w}\)로 정의하면 \(d\)는 metric. 즉, normed vector space에 자연스러운 방식으로 metric을 정의할 수 있다.
	\end{prop}
	\begin{proof}
		(M1)-(M3) 모두 norm의 정의로부터 당연하다.
	\end{proof}

한편 inner product가 주어지면 자동으로 norm이 주어지는데, 이는 우리가 dot product로부터 vector의 길이를 정의하던 방법을 일반화한 것입니다.

	\begin{prop}
		Inner product space \(V\)에 대해, \(\norm{\cdot}: V \rightarrow \mathbb{R}\)을 \(\norm{v} = \sqrt{\inner{v}{v}}\)로 정의하면 \(\norm{\cdot}\)는 norm. 즉, inner product space에 자연스러운 방식으로 norm을 정의할 수 있다.
	\end{prop}

	\begin{proof}
		(N2)와 (N3)은 당연하다. (N1)은 복소수의 성질과 Cauchy-Schwarz inequality를 이용.
		\begin{align*}
			\norm{v+w}^2 = \inner{v+w}{v+w} &= \inner{v}{v} + \inner{v}{w} + \inner{w}{v} + \inner{w}{w}\\
			&= \norm{v}^2 + \norm{w}^2 + \inner{v}{w} + \overline{\inner{v}{w}}\\
			&= \norm{v}^2 + \norm{w}^2 + 2\mathrm{Re}(\inner{v}{w})\\
			&\le \norm{v}^2 + \norm{w}^2 + 2\abs{\inner{v}{w}}\\
			&\le \norm{v}^2 + \norm{w}^2 + 2\norm{v}\norm{w}\\
			&= (\norm{v} + \norm{w})^2
		\end{align*}
		(\(\mathrm{Re}(z)\)는 \(z \in \CC\)의 real part. Imaginary part는 \(\mathrm{Im}(z)\).)
	\end{proof}

	\begin{ex}
		Inner product으로부터 유도되지 않는 norm을 정의할 수 있습니다. 예를 들어 \(\RR^n\)에서
		\begin{align*}
			&\norm{(x_1, \ldots, x_n)}_p = \left(\abs{x_1}^p + \ldots + \abs{x_n}^p\right)^{1/p} \quad (p > 1, p \neq 2)\\
			&\norm{(x_1, \ldots, x_n)}_\infty = \max \{\abs{x_1}, \ldots, \abs{x_n}\}
		\end{align*}
		으로 정의하면 \(\norm{\cdot}_p, \norm{\cdot}_\infty\)는 각각 norm on \(\RR^n\)이지만 \(\norm{v}_p = \sqrt{\inner{v}{v}}\) for \(v \in \RR^n\)이거나 \(\norm{v}_\infty = \sqrt{\inner{v}{v}}\) for \(v \in \RR^n\)인 inner product \(\inner{\cdot}{\cdot}\)이 존재하지 않습니다.
	\end{ex}


	\begin{prop}[Parallelogram law]
		Normed vector space \(V\)에 대해 다음은 동치.
		\begin{enumerate}[label=(\alph*), leftmargin=2\parindent]
			\item
			Inner product \(\inner{\cdot}{\cdot}\) on V가 존재하여 \(\norm{v} = \sqrt{\inner{v}{v}}\) for \(v \in V\).
			\item
			\(\norm{v+w}^2 + \norm{v-w}^2 = 2\norm{v}^2 + 2\norm{w}^2 \quad (v, w \in V)\).
		\end{enumerate}
	\end{prop}

	\begin{proof}
		(a)\(\Rightarrow\)(b)는 당연하다. (b)\(\Rightarrow\)(a)를 보이기 위해,
		\begin{gather*}
			\inner{v}{w} = \frac{1}{4}(\norm{v+w}^2 - \norm{v-w}^2)
		\end{gather*}
		로 정의하고, 이것이 우리가 원하는 inner product임을 보이면 된다.
	\end{proof}

따라서 다음과 같이 생각할 수 있겠습니다. (각 화살표의 역은 성립하지 않습니다.)

	\begin{center}
		Inner Product Space \(\Rightarrow\) Normed Vector Space \(\Rightarrow\) Metric Space
	\end{center}

이제 마지막으로, 앞으로 우리가 여러 가지 재밌는 일들을 벌일 Euclidean space를 소개하는 것으로 이 Section을 마무리하겠습니다.

	\begin{defn} \label{def euc space}
		Dot product가 주어진 inner product space \(\RR^n\)을 \textbf{Euclidean space}라고 한다. 따라서 Euclidean space는 normed vector space이고 metric space이며, Euclidean space의 dot product에 의해 유도되는 자연스러운 norm을 \textbf{Euclidean norm}이라고 한다.
	\end{defn}


\section{Open Set and Closed Set} \label{sec open}


이제 \(\mathbb{R}\)에서 열린구간과 닫힌구간의 개념을 metric space로 확장하려 합니다. 

	\begin{defn} \label{def metric top}
		Metric space \(X\)와 \(p \in X, E \subseteq X\)에 대하여,
		\begin{enumerate}[label=(\alph*), leftmargin=2\parindent]
			\item
			양수 \(r\)에 대하여 집합 \(N_r (p) = \{q \in X : d(p, q) < r\}\)를 \textbf{\(r\)-neighborhood(근방) of \(p\)}라고 한다. \(r\)은 \textbf{radius of \(N_r (p)\)}.
			\item
			\(E\)가 \textbf{open in \(X\)(\(X\)의 열린집합)}라는 것은 각 \(p \in E\)에 대해 적당한 양수 \(r_p > 0\)이 존재하여 다음을 만족하는 것이다.
			\begin{gather*}
				E = \bigcup_{p \in E} N_{r_p}(p)
			\end{gather*}
			\item
			\(p\)를 포함하는 open set of \(X\)를 \textbf{(open) neighborhood of \(p\)}라고 한다.
			\item
			\(E\)가 \textbf{closed in \(X\)(\(X\)의 닫힌집합)}라는 것은 \(X \backslash E\)가 open in \(X\)라는 것이다.
		\end{enumerate}
		
	\end{defn}

Section \ref{sec open}\이 끝날 때까지 \(X\)는 metric space.

다음은 open set의 (더 유용한) 새로운 정의입니다.

	\begin{defn}
		\(p \in X, E \subseteq X\)에 대하여, \(p\)가 \textbf{interior point(내점) of \(E\)}라는 것은 적당한 양수 \(r > 0\)이 존재하여 \(N_r (p) \subseteq E\)인 것이다. \(E\)의 모든 interior point들의 집합을 \textbf{interior(내부) of \(E\)}라고 하고 \(\text{int}E\)로 쓴다.
	\end{defn}

	\begin{thm}
		\(U \subseteq X\) is open if and only if \(\text{int}U = U\).
	\end{thm}
	\begin{proof}
		\(U \subseteq X\)가 open이라고 가정하면 각 \(p \in U\)에 대해 \(r_p > 0\)가 존재하여 \(U = \bigcup_{p \in U} N_{r_p}(p)\)이다. 따라서 \(N_{r_p}(p) \in U\)이므로 \(p \in \text{int}U\)이고 \(\text{int}U \subseteq U\). 이 증명을 반대로 읽으면 역이 증명된다.
	\end{proof}

	\begin{prop}
		\(N_r (p)\)는 open in \(X\).
	\end{prop}
	\begin{proof}
		\(q \in N_r (p)\)를 택하고, \(r'=d(p, q) < r\)이라고 하면 \(N_{r - r'} (q) \subseteq N_r (p)\)이므로(왜 그런가?) \(N_r (p)\)는 open.
	\end{proof}

	\begin{prop} \label{prop metric haus}
		\(X\)의 서로 다른 두 원소 \(p, q\)에 대하여, \(p, q\) 각각의 적당한 neighborhood \(U, V\)가 존재하여 \(U \cap V = \emptyset\)이다.
	\end{prop}
	
	\begin{proof}
		Metric의 정의에 의해 \(d(p, q) > 0\)이다. \(r = d(p, q) / 2\)로 놓고, \(U = N_r (p), V = N_r(q)\)로 놓으면 triangular inequality에 의해 \(U \cap V = \emptyset\).
	\end{proof}

	\begin{prop} \label{prop metric open ax}
		\(X\)에서 다음이 성립한다.
		\begin{enumerate}[label=(\alph*), leftmargin=2\parindent]
			\item
			\(\emptyset\)과 \(X\)는 open in \(X\).
			\item
			\(X\)의 open set들의 union(무한 개 포함)은 open in \(X\).
			\item
			\(X\)의 open set들의 \textbf{finite} intersection은 open in \(X\).
			\item
			\(\emptyset\)과 \(X\)는 closed in \(X\).
			\item
			\(X\)의 closed set들의 intersection(무한 개 포함)은 closed in \(X\).
			\item
			\(X\)의 closed set들의 \textbf{finite} union은 closed in \(X\).
		\end{enumerate}
	\end{prop}

	\begin{proof}
		(a)는 당연. (b)를 보이기 위해, \(\{U_i\}_{i \in I}\)가 \(X\)의 open set들의 모임이라고 하고 \(p \in \cup_{i\in I} U_i\)를 택한다. 그러면 어떤 \(i \in I\)에 대하여 \(p \in U_i\)이고 open set의 정의에 의해 \(\exists\) neighborhood \(N\) of \(p \in X\) s.t. \(N \subseteq U_i\). 그러면 \(N \subseteq \cup_{i\in I} U_i\)이므로 \(\cup_{i\in I} U_i\)는 open. (c)를 보이기 위해, \(U_1, ..., U_n\)이 \(X\)의 open set이라고 하고 \(p \in \cap_{i=1}^n U_i\)라고 하자. 그러면 각 \(i=1, ..., n\)에 대해 양수 \(r_i\)가 존재하여 \(N_{r_i} (p) \subseteq U_i\)이다. \(r = \mathrm{max}\{r_1, ..., r_n\}>0\)으로 놓으면 \(N_r (p) \subseteq \cap_{i=1}^n U_i\)이므로 \(\cap_{i=1}^n U_i\)는 open. (d)-(f)는 closed set의 정의와 (a)-(c)에 의해 성립.
	\end{proof}

	\begin{rem}
		(c)와 (f)에서 finite 조건은 필수적입니다. 예를 들어 각 \(n=1, 2, ...\)에 대해 \(U_n = (-1, 1/n)\)으로 정의하면 \(\cap_{n \in \NN} U_n = (-1, 0]\)으로 open in \(\mathbb{R}\)이 아닙니다. 또 \(F_n = [-1, 1-1/n]\)으로 정의하면 \(\cup_{n \in \NN} F_n = [-1, 1)\)로 closed in \(\mathbb{R}\)이 아닙니다.
	\end{rem}


어떤 집합이 closed인지 정의에 의해 확인하려면 그 집합의 여집합이 open인지를 확인해야 하는데, 이는 불편할 때가 많습니다. 그래서 \(X\)의 부분집합의 closedness를 결정하는 특징을 알아보려고 합니다. \(\RR\)의 닫힌구간 \(I\) 안의 수렴하는 수열은 그 극한이 \(I\) 안에 존재한다는 점에 주목합시다.

	\begin{defn}
		\(E \subseteq X, p \in X\)에 대하여,
		\begin{enumerate}[label=(\alph*), leftmargin=2\parindent]
			\item
			\(p\)가 \textbf{limit point(극한점) of \(E\)}라는 것은, 임의의 양수 \(\eps > 0\)에 대하여 \((N_\eps (p) \backslash \{p\}) \cap E \neq \emptyset\)이라는 것이다. \(E\)의 모든 limit point들의 집합을 \textbf{derive set of \(E\)}라고 하고 \(E'\)로 쓴다.
			\item
			\(p\)가 \textbf{isolated point(고립점) of \(E\)}라는 것은 \(p \in E \backslash E'\)라는 것이다.
			\item
			집합 \(\overline{E} = E \cup E'\)를 \textbf{closure(폐포) of \(E\)}라고 한다.
			\item
			\(E\)가 \textbf{dense(조밀) in \(X\)}하다는 것은 \(\overline{E} = X\)라는 것이다.
		\end{enumerate}
	\end{defn}

	\begin{prop}
		\(E \subseteq X, p \in X\)에 대하여,
		\begin{enumerate}[label=(\alph*), leftmargin=2\parindent]
			\item
			\(p\)가 isolated point of \(E\)일 필요충분조건은 다음을 만족하는 \(\eps > 0\)이 존재하는 것이다.
			\begin{gather*}
				N_\eps(p) \cap E = \{p\}
			\end{gather*}
			\item
			\(p \in \overline{E}\)일 필요충분조건은 임의의 양수 \(\eps > 0\)에 대해 다음이 성립하는 것이다.
			\begin{gather*}
				N_\eps (p) \cap E \neq \emptyset
			\end{gather*}
		\end{enumerate}
	\end{prop}
	\begin{proof}
		정의에 의해 당연.
	\end{proof}

	\begin{prop}
		\(p \in X\)가 \(E \subseteq X\)의 limit point이면 임의의 \(\eps > 0\)에 대해 \(N_\eps (p) \cap E\)는 infinite.
	\end{prop}
	\begin{proof}
		\(N_\eps (p) \cap E\)가 finite이라고 가정하고 모순을 보이면 된다.
	\end{proof}

	\begin{ex}
		대표적인 예시들입니다.
		\begin{enumerate}[label=(\alph*), leftmargin=2\parindent]
			\item
			\(X\)의 유한 부분집합의 limit point는 존재하지 않습니다. 즉 유한 부분집합은 모든 원소가 isolated point.
			\item
			\begin{gather*}
				S = \{\frac{1}{n} \in \RR\}_{n \in \NN}
			\end{gather*}
			로 정의하면, \(S'=\{0\}\).
			\item
			\(\QQ' = \RR\). 따라서 \(\overline{\QQ} = \RR\)이므로 \(\QQ\)는 dense in \(\RR\).
		\end{enumerate}
	\end{ex}

Limit point와 관련된 개념들과 sequence는 밀접한 관련이 있습니다.

	\begin{prop} \label{prop limpt seq}
		\(E \subseteq X, p \in X\)에 대하여 다음이 성립한다.
		\begin{enumerate}[label=(\alph*), leftmargin=2\parindent]
			\item
			\(p \in E'\) if and only if there exists a sequence in \(E \backslash \{p\}\) converging to \(p\).
			\item
			\(p \in \overline{E}\) if and only if there exists a sequence in \(E\) converging to \(p\).
		\end{enumerate}
	\end{prop}

	\begin{proof}
		(a): 먼저 임의의 \(p \in E'\)에 대해, 각 \(n = 1, 2, ...\)에 대해 \((N_{1/n} (p) \backslash {p}) \cap E \neq \emptyset\)이므로 \(d(x_n, p) < 1/n\)인 \(x_n \in E \backslash \{p\}\)가 존재한다. 이제 \((x_n)\)이 \(p\)로 수렴함은 쉽게 알 수 있다. 역으로 \(p\)로 수렴하는 \(E \backslash \{p\}\) 안의 sequence \((x_n)\)을 생각하면, 임의의 양수 \(\eps > 0\)에 대해 \(x_N \in (N_\eps (p) \backslash {p}) \cap E\) for some \(N\)이므로 \(p \in E'\).\\
		(b): \(p \in \overline{E}\)이면 \(p \in E\)이거나 \(p \in E'\)이다. 첫 번째 경우는 \(E\) 안에서 모든 항이 \(p\)인 sequence를 잡으면 되고, 두 번째 경우는 (a)에 의해서 성립. 역으로 \(p\)로 수렴하는 \(E\) 안의 sequence \((x_n)\)을 생각하면, 임의의 양수 \(\eps > 0\)에 대해 \(x_N \in N_\eps (p) \cap E\) for some \(N\)이므로 \(p \in \overline{E}\).
	\end{proof}

이제 closed set의 필요충분조건을 소개할 준비가 되었습니다.

	\begin{thm} \label{thm closed}
		\(F \subseteq X\)에 대하여 다음은 동치.
		\begin{enumerate}[label=(\alph*), leftmargin=2\parindent]
			\item
			\(F\) is closed in \(X\).
			\item
			\(F' \subseteq F\).
			\item
			\(\overline{F} = F\).
			\item
			If a sequence \((x_n)\) in \(F\) converges to some \(p \in X\), then \(p \in F\).
		\end{enumerate}
	\end{thm}

	\begin{proof}
		(a)\(\Leftrightarrow\)(b):
		\begin{align*}
			F \text{ is closed in } X & \Leftrightarrow X \backslash F \text{ is open in } X\\
			& \iff \forall x \in X \backslash F, \: \exists \epsilon > 0 \text{ s.t. } N_\epsilon (x) \subseteq X \backslash F\\
			& \iff \forall x \in X \backslash F, \: x \notin F'\\
			& \iff F' \subseteq F
		\end{align*}
		(b)\(\Leftrightarrow\)(c): \(\overline{F}\)의 정의에 의해.\\
		(c)\(\Leftrightarrow\)(d): Proposition \ref{prop limpt seq}.
	\end{proof}

	\begin{thm} \label{thm cls}
		\(E \subseteq X\)에 대하여 \(E'\)와 \(\overline{E}\)는 closed in \(X\). 또 \(E \subseteq F \subseteq X\)이고 \(F\)가 closed in \(X\)이면 \(\overline{E} \subseteq F\). 즉, \(\overline{E}\)는 \(E\)를 포함하는 가장 작은 closed set.
	\end{thm}

	\begin{proof}
		\(E'\)가 closed in \(X\)임을 보이기 위해 \((E')' \subseteq E'\)임을 보이면 된다. \(E' = \emptyset\)이면 당연하므로 \(E' \neq \emptyset\)라 가정하자. 임의의 \(x \in (E')', \eps > 0\)에 대해 \(d(x, y) < \eps\)인 \(y \in E'\)가 존재한다. \(\eps' = \min\{d(x, y), \eps - d(x, y)\} > 0\)으로 두면, \(0 < d(y, z) < \eps'\)을 만족하는 \(z \in E\)가 존재한다. 이제
		\begin{gather*}
			0 < \abs{d(x, y) - d(y, z)} \le  d(x, z) \le d(x, y) + d(y, z) \le d(x, y) + (\eps - d(x, y)) = \eps
		\end{gather*}
		이므로 \(x \in E'\).\\
		한편 \(x \in X \backslash \overline{E}\)일 필요충분조건은 다음을 만족하는 \(\eps > 0\)이 존재하는 것이다.
		\begin{gather*}
			N_\eps (x) \cap E = \emptyset
		\end{gather*}
		따라서 \(N_\eps (x)\)는 \(E\)의 점을 포함하지 않는다. 또 \(N_\eps (x)\)가 \(E'\)의 점도 포함하지 않음을 알 수 있다(왜 그런가?). 따라서 \( N_\eps (x) \in X \backslash \overline{E}\) 이므로 \(X \backslash \overline{E}\)는 open.\\
		Derive set의 정의에 의해 \(E \subseteq F\)이면 \(E' \subseteq F'\)이다. 그런데 \(F' \subseteq F\)이므로 \(E' \subseteq F\)이고 따라서 \(\overline{E} \subseteq F\).
	\end{proof}

전체 집합을 무엇으로 생각하는지에 따라 open과 closed 여부가 달라질 수 있습니다. 특히 \(X\)의 부분집합 안에서 open과 closed를 생각하는 것이 편리할 때가 많습니다.

	\begin{defn}
		\(E \subseteq Y \subseteq X\)일 때, \(E\)가 \textbf{open in \(Y\)}라는 것은 어떤 open set \(U\) of \(X\)가 존재하여 \(E = U \cap Y\)인 것이다. \textbf{closed in \(Y\)}도 마찬가지로 정의한다.
	\end{defn}


\newpage

\section{Completeness of Euclidean Spaces}
\subsection{Bolzano-Weierstrass Theorem}

앞 Section에서 limit point의 개념을 정의하였는데, \(X\)의 유한 부분집합은 limit point를 가지지 않음을 알고 있습니다.이제 \(X=\RR^n\)인 경우로 한정하여, 그렇다면 \(\RR^n\)의 무한집합 중 어떤 집합이 limit point를 가질지에 대해 논의하려고 하는데, 결론은 `\(\RR^n\)의 bounded infinite subset은 limit point를 가진다'라는 것입니다.

\(\RR^n\)에 대해 이야기하기 전에 먼저 \(\RR\)의 중요한 성질 중 하나를 언급하겠습니다. 이는 \(\RR\)의 단조수렴정리에 의해 obvious합니다.

	\begin{thm}[Nested Intervals Theorem: 축소구간정리]
		각 \(n \in \NN\)에 대하여 \(I_n = [a_n, b_n] \subset \RR\)이 유계닫힌구간이고 \(I_{n+1} \subseteq I_n\)이라고 하자. 즉, \(a_n \le a_{n+1}, b_n \ge b_{n+1}\). (이러한 \((I_n)\)을 \textbf{nested interval}이라고 한다.) 그러면 \(\cap_n I_n\)은 nonempty.
	\end{thm}
	\begin{proof}
		실수열 \((a_n)\)을 생각하자. 각 \(n \in \mathbb{N}\)에 대해 \(a_n \le b_1\)이므로 \((a_n)\)은 유계이고, 가정에 의해 \(a_n\)은 단조증가수열이다. 따라서 단조수렴정리에 의해 \((a_n)\)은 어떤 \(\alpha \in \RR\)로 수렴한다. 주장: \(\alpha \in \cap_n I_n\)\\
		이를 보이기 위하여 for all \(n \in \NN\), \(\alpha \le b_n\)임을 보이면 된다. 어떤 \(N\)에 대하여 \(\alpha > b_N\)이라고 가정하자. 그런데 for all \(n \in \NN\), \(a_n \le b_N\)이므로 \(\alpha \le b_N\)이다. 따라서 모순이고, \(\alpha \le b_n\) for all \(n \in \NN\)이다.
	\end{proof}

	\begin{rem}
		\(I_n\)이 닫힌구간인 것이 중요합니다. \(I_n = (0, \frac{1}{n})\)이면 \(\cap_n I_n = \emptyset \).
	\end{rem}

\(\RR\)에서의 축소구간정리를 \(\RR^N\)\footnote{\(n\)이 아니라 \(N\)을 쓰는 이유는 sequence에 \(n\)을 쓰기 위해서.}으로 자연스럽게 확장할 수 있습니다. \(k = 1, ..., N\)에 대하여 \(I^k \subset \RR\)가 유계닫힌구간일 때, \(B = I^1 \times ... \times I^N \subset \RR^N\)을 \textbf{\(N\)-cell}이라고 합니다. 평면에서는 직사각형, 공간에서는 직육면체의 개념을 일반화한 것입니다.

	\begin{cor} \label{cor Ncell}
		\(k = 1, ..., N, n \in \NN\)에 대하여 \(I_n^k \subset \RR\)가 유계닫힌구간, \(B_n = I_n^1 \times ... \times I_n^N \subset \RR^N\)이라고 하자. 각 \(n \in \NN\)에 대하여 \(B_{n+1} \subseteq B_n\)이면 \(\cap_n B_n\)은 nonempty.
	\end{cor}
	\begin{proof}
		\(B_{n+1} \subseteq B_n\)이므로 각 \(k = 1, ..., N\)에 대하여 \(I_{n+1}^k \subseteq I_n^k\)이다. 따라서 축소구간정리에 의해 \(\exists \alpha^k \in \cap_n I_n^k\). 이제 \((\alpha^1, ..., \alpha^N) \in \cap_n B_n\)임은 당연하다.
	\end{proof}

이제 Bolzano-Weierstrass Theorem을 증명할 준비가 되었습니다.

	\begin{thm}[Bolzano-Weierstrass Theorem]
		\(A \subset \RR^N\)가 bounded and infinite이면 \(A\)는 limit point를 가진다.
	\end{thm}
	\begin{proof}
		\(A\)가 유계이므로 \(A\)를 포함하는 \(N\)-cell \(B_1\)을 잡을 수 있다. 이제 각 \(n \in \NN\)에 대하여 \(N\)-cell \(B_n\)을 다음 성질을 만족하도록 귀납적으로 정의하려고 한다.
		\begin{enumerate}[label=(\alph*), leftmargin=2\parindent]
			\item
			\(B_{n+1} \subseteq B_n\)
			\item
			\(B_n \cap A\) is infinite.
		\end{enumerate}
		위 조건을 만족하는 \(B_1, ..., B_m\)이 주어졌다고 하자. \(B_m = I_m^1 \times ... \times I_m^N\)로 쓰고, 각 \(I_m^k\)를 이등분한다. 그러면 \(2^N\)개의 \(B_m\)의 sub-\(N\)-cell을 얻는데, 이중 적어도 1개는 \(A\)의 점을 무수히 많이 포함하고 있다(왜 그런가?). 따라서 이 \(N\)-cell을 \(B_{m+1}\)으로 정의하면 된다.\\
		이제 Corollary \ref{cor Ncell}에 의해 \(\exists x \in \cap_n B_n\)이다. 주장: \(x\) is a limit point of \(A\).
		\(B_n\)의 \textbf{diameter}를 \(\mathrm{diam}B_n = \sup \{\norm{x-y}: x, y \in B_n\}\)으로 정의하면, \(\lim_n \mathrm{diam}B_n = 0\)인 것은 쉽게 알 수 있다. 따라서 임의의 양수 \(\eps > 0\)에 대해 \(\mathrm{diam}B_n < \eps / 2\)인 \(n\)이 존재한다. 이제 for all \(y \in B_n\), \(\norm{x-y} \le \eps / 2 < \eps\)이므로 \(B_n \subseteq N_\eps (x)\)이다. 그런데 \(B_n\)이 \(A\)의 점을 무한히 많이 포함하고 있으므로, \(N_\eps (x)\)도 마찬가지이다. 따라서 \(x\)는 limit point of \(A\).
	\end{proof}

Bolzano-Weierstrass Theorem의 direct result를 설명하기 위해 다음 정의가 필요합니다.

	\begin{defn}
		함수 \(n: \NN \rightarrow \NN\)이 각 \(k \in \NN\)에 대해 \(n(k) < n(k+1)\)을 만족한다고 하자. Sequence \(n \mapsto x_n\)에 대하여 \(k \mapsto x_{n(k)}\)를 \textbf{subsequence(부분수열) of \((x_n)\)}이라고 한다.
	\end{defn}

예를 들어 \(x_n = 2n\)으로 정의된 실수열 \((x_n)\)와 \(n(k) = 2k\)에 대하여 \(x_{n(k)} = 4k\)는 \((x_n)\)의 subsequence입니다.

Limit point와 convergent sequence는 밀접한 관계가 있음을 확인한 바 있는데, 마찬가지로 Bolzano-Weierstrass Theorem에 의해서 다음 따름정리를 얻게 됩니다.

	\begin{cor} \label{cor bdd conv}
		\(\RR^N\)의 bounded sequence는 convergent subsequence를 가진다.
	\end{cor}
	\begin{proof}
		\((x_n)\)이 \(\RR^N\)의 bounded sequence라고 하자. \(A = \{x_n: n \in \NN \}\)이 finite이면, 어떤 \(x \in A\)에 대하여 \(x_n = x\)인 \(n\)이 무수히 많이 존재한다. 따라서 이 \(n\)들을 모아서 subsequence를 만들면 이 수열은 상수수열이므로 당연히 수렴한다. 이제 \(A\)가 infinite이라고 가정하자. Bolzano-Weierstrass Theorem에 의해 \(A\)의 limit point \(x\)가 존재한다. 이제 \(x\)로 수렴하는 subsequence \((x_{n(k)})\)를 \(\norm{x_{n(k)} - x} < 1/k\)를 만족하도록 귀납적으로 정의하려고 한다.\\
		먼저 \(N_1 (x) \cap A\)는 nonempty이므로 여기에 속하는 \(x_{n(1)}\)을 잡을 수 있다. 다음으로 위 성질을 만족하는 \(x_{n(1)}, ..., x_{n(m)} \: (n(1) < ... < n(m))\)이 정해졌다고 가정하자. \(N_{1/(m+1)} (x) \cap A\)은 infinite이므로 \(n(m) < n(m+1)\)이면서 \(x_{n(m+1)} \in N_{1/(m+1)} (x) \cap A\)인 \(n(m+1)\)을 잡을 수 있다. 이제 이 subsequence가 \(x\)로 수렴하는 것은 쉽게 알 수 있다.
	\end{proof}

\subsection{Cauchy Sequence}

해석학에서 정말 중요한 개념인 Cauchy sequence를 정의하려고 합니다. 앞에서 단조수렴정리를 소개할 때, 어떤 sequence가 수렴할 것으로 예상되는 점조차 알지 못하는 경우에 수렴성을 보일 수 있는 방법을 계속 찾게 될 것이라고 말씀드렸습니다. Cauchy sequence가 그 대표적인 방법입니다.

	\begin{defn}
		Metric space \(X\)의 sequence \(x_n\)이 Cauchy sequence라는 것은, 임의의 양수 \(\eps > 0\)에 대하여 적당한 \(N \in \NN\)이 존재하여 다음을 만족하는 것이다.
		\begin{gather*}
			n, m \ge N \Longrightarrow d(x_n, x_m) < \eps
		\end{gather*} 인 것이다.
	\end{defn}

	\begin{ex}
		\(x_n = 1/n\)은 Cauchy sequence. 양수 \(\eps > 0\)이 주어졌다고 하고 \(1/N < \eps\)인 \(N\)을 잡는다. 이때 \(n, m \ge N\)이면 \(|1/n - 1/m| \le 1/N < \eps\)이므로, \((x_n)\)은 Cauchy sequence.
	\end{ex}

수렴하는 수열과 밀접한 관계가 있는데, 실제로 모든 수렴하는 수열은 Cauchy입니다.

	\begin{prop}
		A convergent sequence in a metric space \(X\) is Cauchy.
	\end{prop}
	\begin{proof}
		\((x_n)\)이 \(x \in X\)로 수렴한다고 하자. 양수 \(\eps > 0\)에 대하여,
		\begin{gather*}
			n \ge N \Longrightarrow d(x_n, x) < \frac{\eps}{2}
		\end{gather*}
		\(N\)을 잡을 수 있다. 이제
		\begin{gather*}
			n, m \ge N \Longrightarrow d(x_n, x_m) \le d(x_n, x) + d(x_m, x) < \eps.
		\end{gather*}
	\end{proof}

Cauchy sequence는 다음 성질들을 가집니다.

	\begin{prop} \label{cauchy bdd}
		A Cauchy sequence in a metric space \(X\) is bounded.
	\end{prop}
	\begin{proof}
		\((x_n)\)이 Cauchy sequence in \(X\)라고 하고, \(n, m \ge N \Rightarrow d(x_n, x_m) < 1\)인 \(N\)을 잡을 수 있다. 따라서 \(\{x_n: n \ge N\}\)은 유계이고, \(\{x_n: n < N\}\)은 finite이므로 유계. 따라서 \((x_n)\)은 bounded.
	\end{proof}

	\begin{prop} \label{cauchy subs}
		Metric space \(X\)의 Cauchy sequence \((x_n)\)이 \(x \in X\)로 수렴하는 subsequence를 가지면 \((x_n)\)도 \(x\)로 수렴.
	\end{prop}
	\begin{proof}
		\(\eps > 0\)이 주어졌다고 하자. \((x_n)\)은 Cauchy이므로
		\begin{gather*}
			n, m \ge N \Longrightarrow d(x_n, x_m) < \frac{\eps}{2}
		\end{gather*}
		인 \(N\)이 존재한다. 또 \((x_{n(k)})\)가 \(x \in X\)로 수렴하는, \((x_n)\)의 convergent subsequence라고 하면
		\begin{gather*}
			k \ge K \Longrightarrow d(x_{n(k)}, x) < \frac{\eps}{2}, \quad n(K) \ge N
		\end{gather*}
		인 \(K\)를 잡을 수 있다. 따라서
		\begin{gather*}
			n \ge n(K) \Longrightarrow d(x_n, x) \le d(x_n, x_{n(K)}) + d(x_{n(K)}, x) < \frac{\eps}{2} + \frac{\eps}{2} = \eps.
		\end{gather*}
	\end{proof}

위의 두 Proposition에 의해, \(X = \RR^N\)인 경우에 Cauchy sequence와 convergent sequence가 동치임을 알 수 있습니다. 모든 Cauchy sequence가 수렴하는 metric space를 \textbf{complete metric space}라고 합니다. 즉 Euclidean space는 complete metric space.

	\begin{thm}[Completeness of a Euclidean Space]
		\(\RR^N\)의 sequence \((x_n)\)에 대하여, \((x_n)\) is convergent if and only if \((x_n)\) is Cauchy.
	\end{thm}
	\begin{proof}
		\((\Rightarrow)\) 방향은 이미 보였다(아무 metric space에서 성립). 이제 \((x_n)\)이 Cauchy라고 가정하면, Proposition \ref{cauchy bdd}에 의해 \((x_n)\)은 bounded이다. 따라서 Corollary \ref{cor bdd conv}에 의해 \((x_n)\)은 convergent subsequence를 가진다. 따라서 Proposition \ref{cauchy subs}에 의해 \((x_n)\)은 convergent.
	\end{proof}

위 증명에서 Corollary \ref{cor bdd conv}를 이용하였음에 유의하기 바랍니다. 이것이 다른 metric space와 Euclidean space를 구분짓는 한 가지 성질입니다.

\newpage

\section{Series} \label{sec series}


Section \ref{sec series}에서는 해석학의 또 다른 중요한 주제인 series(급수)의 수렴판정을 공부하려고 합니다. 이 글이 끝날 때까지 모든 sequence는 아무 말이 없으면 (usual한 metric이 주어진) \(\CC\) 안에 있는 것으로 생각합니다. \(\CC = \RR^2\)로 `자연스럽게' identify할 수 있기 때문에, \(\CC\)는 complete metric space.

\subsection{Cauchy Criterion}

	\begin{defn}
		Sequence \((x_n)\)에 대하여 \((s_n)\)을 \(s_n = \sum_{k=1}^n x_k\)으로 정의하고 \(\sum_{n=1}^\infty x_n\) 또는 \(\sum_n x_n\)을 \textbf{series(급수)}라고 하고 \((s_n)\)을 \textbf{partial sum(부분합) of series}라고 한다. 만약 \((s_n)\)이 \(s\)로 converge하면 \textbf{series \(\sum_n x_n\)이 converge(수렴)}한다고 하고 \(\sum_n x_n = s\)로 쓴다. 만약 \((s_n)\)이 diverge하면 \textbf{series \(\sum_n x_n\)이 diverge(발산)}한다고 한다.
	\end{defn}

	\begin{prop}
		급수 \(\sum_n x_n\)과 \(\sum_n y_n\)이 각각 \(x, y \in \CC\)로 수렴할 때 다음이 성립한다.
		\begin{enumerate} [label=(\alph*), leftmargin=2\parindent]
			\item
			\(\sum_n (x_n + y_n) = x + y\).
			\item
			\(c \in \CC\)에 대하여 \(\sum_n cx_n = cx\).
		\end{enumerate}
	\end{prop}
	\begin{proof}
		각 급수의 partial sum을 생각하면 끝.
	\end{proof}

\(\CC\)에서 Cauchy sequence와 convergent sequence는 동치이므로, 다음 판정법을 얻습니다.

	\begin{thm}[Cauchy criterion]
		\(\sum_n x_n\) converges if and only if for every \(\eps > 0\), there exists \(N \in \NN\) s.t.
		\begin{align*}
			m \ge n \ge N \Longrightarrow \left | \sum_{k=n}^{m}x_n \right | < \eps.
		\end{align*}
	\end{thm}
	\begin{proof}
		\
		\(\sum_n x_n\) converges if and only if \((s_n)\) converges if and only if \((s_n)\) is Cauchy.
	\end{proof}

특히 위 정리에서 \(m=n\)으로 잡으면, \( \left | x_n \right | < \epsilon\)이므로 다음 따름정리를 얻습니다.

	\begin{cor}
		If \(\sum _n x_n\) converges, then \(\lim _n x_n = 0\).
	\end{cor}

역은 성립하지 않습니다(반례: \(x_n = 1/n\)).

\subsection{Alternating Series Test}

항들이 특정한 형태를 가지는 series는 쉽게 그 수렴을 알 수 있습니다. Real sequence의 각 항의 부호가 +와 -가 반복되고(alternating) 그 절댓값이 0으로 수렴할 때 series는 수렴합니다.

	\begin{lem} \label{alt}
		Real-valued nonnegative sequence \((x_n)\)이 단조감소하면, 임의의 \(m > n\)에 대하여
		\begin{align*}
			\abs{\sum_{k=n}^{m} (-1)^{k-1} x_k} \le x_n.
		\end{align*}
	\end{lem}
	\begin{proof}
		\begin{align*}
			\abs{\sum_{k=n}^{m} (-1)^{k-1} x_k} = x_n - x_{n+1} + ... + (-1)^{m-n} x_m
		\end{align*}
		이다. \(m-n\)이 짝수이면, 각 \(k\)에 대해 \(x_k - x_{k+1} \ge 0\)이므로
		\begin{align*}
			x_n - x_{n+1} + ... + x_m &= x_n -((x_{n+1} - x_{n+2}) + ... + (x_{m-1} - x_m))\\
			& \le x_n.
		\end{align*}
		\(m-n\)이 홀수이면 \(m-n+1\)은 짝수이므로
		\begin{align*}
			x_n - x_{n+1} + ... - x_m &\le x_n - x_{n+1} + ... - x_m + x_{m+1}\\
			& \le x_n.
		\end{align*}
	\end{proof}

	\begin{thm}[Alternating Series Test: 교대급수판정법]
		Real nonnegative sequence \((x_n)\)이 단조감소할 때, \(\lim_n x_n = 0\)이면 \(\sum_n (-1)^{n-1} x_n\)은 수렴한다.
	\end{thm}
	\begin{proof}
		주어진 series의 partial sum \((s_n)\)이 Cauchy sequence인 것을 보이는 것과 동치이다. 임의의 양수 \(\eps > 0\)에 대해 \(n \ge N \Longrightarrow x_n < \eps\)인 \(N\)이 존재한다. 이제 \(m > n \ge N\)이면 Lemma \ref{alt}에 의해
		\begin{align*}
			\abs{s_m - s_n} &= \abs{\sum_{k=n+1}^m (-1)^{k-1} x_k}\\
			& \le x_{n+1} < \eps.
		\end{align*}
	\end{proof}

	\begin{ex}
		\(\sum_n (-1)^{n-1}/n\), \(\sum_{n=2}^\infty (-1)^{n-1}/(\log n)\) 등은 모두 수렴합니다.
	\end{ex}


\subsection{Comparison Test}

급수의 comparison test(비교판정법)는 대단히 직관적입니다.


	\begin{thm}[Comparison Test: 비교판정법]
		\((c_n)\)은 real nonnegative sequence. 각 \(n \in \NN\)에 대하여 \(\abs{a_n} \le c_n\)이고 \(\sum _n c_n\)이 수렴하면 \(\sum _n a_n\)도 수렴.
	\end{thm}
	\begin{proof}
		Cauchy criterion에 의해, 임의의 양수 \(\eps > 0\)에 대하여 \(N \in \NN\)이 존재하여
		\begin{align*}
			m \ge n \ge N \Rightarrow \abs{\sum_{k=n}^m c_k} < \eps
		\end{align*}
		이다. 따라서
		\begin{align*}
			\abs{\sum_{k=n}^m a_k} \le  \sum_{k=n}^m \abs{a_k} \le \sum_{k=n}^m c_k < \eps
		\end{align*}
		이므로, Cauchy criterion에 의해 \(\sum _n a_n\)도 수렴.
	\end{proof}

급수나 수열의 수렴 판정을 할 때 유한 개의 항들은 전혀 영향을 미치지 않습니다. 따라서 위 정리의 `각 \(n \in \NN\)에 대하여'를 `\(n \ge N\)에 대하여'로 바꾸어도 상관이 없습니다.

위 정리의 역은 성립하지 않습니다. 예를 들어 \(x_n = (-1)^n / n\)에 대하여 \(\sum_n x_n\)은 alternating series test에 의하여 수렴하지만 \(\sum_n \abs{x_n} = \sum_n 1/n\)은 발산합니다. 따라서 다음 정의를 생각하는 것이 자연스럽습니다.

	\begin{defn}
		Series \(\sum_n x_n\)이 \textbf{absolutely converge(절대수렴)}한다는 것은 \(\sum_n \abs{x_n}\)이 수렴한다는 것이다. \(\sum_n x_n\)이 \textbf{conditionally converge(조건수렴)}한다는 것은 \(\sum_n x_n\)은 수렴하지만 \(\sum_n \abs{x_n}\)은 발산한다는 것이다.
	\end{defn}

즉 convergent series는 absolutely convergent이거나 conditionally convergent입니다. 특히 real nonnegative series의 경우에 absolutely convergent와 convergent는 동치입니다.

\subsection{Rearrangement}

유한 개의 실수의 덧셈의 순서는 바꾸어도 그 결과가 같은데, 무한 개의 덧셈의 순서를 바꾸어도 결과가 같겠는가, 즉 항들의 순서를 바꾼 series의 성질에 대해 논의할 것입니다.

	\begin{defn}
		\(r: \NN \rightarrow \NN\)이 bijection이라고 하자. Sequence \((x_n)\)에 대하여 \(\sum_n x_{r(n)}\)을 \textbf{rearrangement(재배열급수) of \(\sum_n x_n\)}이라고 한다.
	\end{defn}

이 subsection에서는 \(\sum_n x_n = \sum_n x_{r(n)}\)일 충분조건 2개를 소개합니다. 첫 번째 충분조건은 \((x_n)\)이 real nonnegative sequence인 것입니다.

	\begin{thm}
		\((x_n)\)이 real nonnegative sequence라고 하자. \(\sum_n x_n = s\) for some \(s \in \RR\)이면 임의의 rearrangement에 대하여 \(\sum_n x_{r(n)} = s\)이다. 또 \(\sum_n x_n\)이 발산하면 \(\sum_n x_{r(n)}\)도 발산한다.
	\end{thm}
	\begin{proof}
		각 \(n\)에 대하여 \(m(n) = \max \{r(1), ..., r(n)\}\)으로 정의하면, \(\sum_{k=1}^n x_{r(k)} \le \sum_{k=1}^{m(n)} x_k \le s\)이다(여기서 각 항이 nonnegative임을 사용했다). 따라서 각 항이 nonnegative인 series \(\sum_n x_{r(n)}\)의 partial sum이 bounded이므로, \(\sum_n x_{r(n)}\)은 수렴하고 그 극한 \(t = \sum_n x_{r(n)} \le s\)이다. 그런데 \(r^{-1}:\NN \rightarrow \NN\)도 bijection이므로 같은 논리에 의해 \(s \le t\)이다. 따라서 \(s = t\). \(\sum_n x_n\)이 발산할 때 \(\sum_n x_{r(n)}\)도 발산하는 것은 이제 쉽게 알 수 있다.
	\end{proof}

또 하나의 충분조건은 \(\sum_n x_n\)이 absolutely convergent인 것입니다. index를 조작하는 방법이 조금 까다로우니 유의해서 보시기 바랍니다.

	\begin{thm}
		\(\sum_n x_n\)이 absolutely convergent이고 \(\sum_n x_n = s\)라고 하자. 그러면 임의의 rearrangement에 대해 \(\sum_n x_{r(n)} = s\)이다. 
	\end{thm}
	\begin{proof}
		\(\sum_n x_n\)과 \(\sum_n x_{r(n)}\)의 partial sum을 각각 \((s_n), \: (s'_n)\)이라고 하자. 양수 \(\eps > 0\)이 주어졌다고 하고, Cauchy criterion에 의해
		\begin{align*}
			m \ge n \ge N \Rightarrow \sum_{k=n}^{m} \abs{x_k} < \frac{\eps}{2}
		\end{align*}
		인 \(N\)을 잡는다. 또 \(K \in \NN\)을
		\begin{align*}
			\{1, 2, ..., N\} \subseteq \{r(1), r(2), ..., r(K)\}
		\end{align*}
		을 만족하도록 잡을 수 있고, \(M = \max\{r(1), ..., r(K)\}\)라고 하자. 이때 \(M \ge K \ge N\)임은 쉽게 알 수 있다. 이제 \(n > K\)에 대하여,
		\begin{align*}
			\abs{s'_n - s_n} &= \abs{\sum_{k=1}^{n} x_{r(k)} - \sum_{k=1}^n x_k}\\
			&= \abs{( \sum_{i=1}^N x_i + \sum_{\substack{1 \le k \le n \\ r(k) > N}} x_{r(k)}) - (\sum_{i=1}^N x_i + \sum_{i=N+1}^n x_i)}\\
			&= \abs{\sum_{\substack{1 \le k \le n \\ r(k) > N}} x_{r(k)} - \sum_{i=N+1}^n x_i}\\
			& \le \sum_{\substack{1 \le k \le n \\ r(k) > N}} \abs{x_{r(k)}} + \sum_{i=N+1}^n \abs{x_i}\\
			& \le \sum_{i=N+1}^M \abs{x_i} + \sum_{i=N+1}^n \abs{x_i} < \frac{\eps}{2} + \frac{\eps}{2} = \eps
		\end{align*}
		이므로
		\begin{align*}
			\limsup_n \abs{s'_n - s_n} \le \eps
		\end{align*}
		이다. \(\eps > 0\)은 임의의 양수였으므로 \(\lim_n \abs{s'_n - s_n} = 0\)이고, 따라서
		\begin{align*}
			\lim_n s'_n = \lim_n s_n = s.
		\end{align*}
	\end{proof}

그렇다면 \(\sum_n x_n\)이 conditinally convergent일 때는 어떻게 될까요? 아쉽게도 이때는 rearrangement series가 수렴하는 것조차 보장할 수 없으며, 수렴한다 해도 그 극한이 같다고 할 수 없습니다. 대신 real series의 경우에 상당히 흥미로운 결과가 도출되는데, `양/음의 무한대로 발산하거나 임의의 실수로 수렴하는 rearrangement를 항상 잡을 수 있다'는 것입니다.

이를 위해서는 상극한과 하극한의 새로운 정의를 소개하는 것이 도움이 됩니다. 이는 앞에서 배운 정의와 동치입니다. 약간 흐름에서 벗어난 내용이기 때문에 정리의 statement만 봐도 될 것 같습니다.

	\begin{lem}
		Metric space \(X\)의 sequence \((x_n)\)에 대하여
		\begin{center}
			\(x \in X\)로 수렴하는 \((x_n)\)의 subsequence가 존재한다
		\end{center}
		를 만족하는 모든 \(x \in X\)의 집합을 \(E\)라고 하면, \(E\)는 closed subset of \(X\).
	\end{lem}
	\begin{proof}
		Derive set \(E' = \emptyset\)이면 증명 끝. \(E'\)가 nonempty라고 가정하고, \(E\)의 임의의 limit point \(p\)가 \(E\)의 원소임을 보이면 된다.\\
		먼저 \(x_{n(1)} \neq p\)인 \(n(1)\)을 택한다. (만약 이러한 \(n(1)\)이 없으면 \(E = \{p\}\)이므로 증명이 끝난다.) \(\delta = d(p, x_{n(1)})\)라고 하고, \(d(p, x_{n(k)}) \le 2^{k-1} \delta\)를 만족하도록 \(n(k)\)를 귀납적으로 정하려고 한다. \\
		\(n(1) < ... < n(k)\)가 위의 성질을 만족하도록 정해졌다고 가정하자. \(p\)는 \(E\)의 limit point이므로, \(d(p, q) < 2^{-k-1} \delta\)인 \(q \in E\)가 존재한다. \(E\)의 정의에 의해, \(n(k+1) > n(k)\)이고 \(d(q, x_{n(k+1)}) < 2^{-k-1}\delta\)인 \(n(k+1)\)을 찾을 수 있다. 이제 \(d(p, x_{n(k+1)}) \le d(p, q) + d(q, x_{n(k+1)}) < 2^{-k} \delta\)이므로 우리가 원하는 \(n(k+1)\)을 찾았다.\\
		이제 subsequence \((x_{n(k)})\)가 \(p\)로 수렴하는 것은 쉽게 알 수 있다. 따라서 \(p \in E\).
	\end{proof}

양의 무한대나 음의 무한대로 발산하는 경우에 \(x_n \rightarrow \infty\), \(x_n \rightarrow -\infty\) 같은 표기를 자연스럽게 쓰기로 하겠습니다.

	\begin{thm}[New Definition of Upper and Lower Limits] \label{limsup}
		Real sequence \((x_n)\)에 대하여
		\begin{center}
			\(x_{n(k)} \rightarrow x\)인 subsequence \((x_{n(k)})\)가 존재한다
		\end{center}
		를 만족하는 모든 \(x \in \RR \cup \{\pm \infty\}\)의 집합을 \(E\)라고 하면,
		\begin{align*}
			\limsup_n x_n = \sup E, \quad \liminf_n x_n = \inf E
		\end{align*}
		이다. 특히 \(x_n\)이 bounded이면 Lemma 2.11.에 의해 \(E\)는 closed bounded subset of \(\RR\)이므로 maximum과 minimum이 존재한다. 따라서
		\begin{align*}
			\limsup_n x_n = \max E, \quad \liminf_n x_n = \min E.
		\end{align*}
	\end{thm}
	\begin{proof}
		(언제나 그랬듯이) \(\limsup\)에 대해서만 증명하면 된다.\\
		먼저 \(\limsup_n x_n = \infty\)이면 \((x_n)\)은 not bounded above이므로, 각 \(k\)에 대해 \(x_{n(k)} > k\)인 \(n(1) < n(2) < ...\)을 잡을 수 있다. 이때 \(\lim_n x_{n(k)} = \infty\)이므로 \(\sup E = \infty\). 또 \(\limsup_n x_n = -\infty\)이면 \(\liminf_n x_n = -\infty\)이므로 \(\lim_n x_n = -\infty\)이다. 따라서 \(E = \{-\infty\}\)이므로(왜 그런가?) \(\sup E = -\infty\).\\
		이제 \(\limsup_n x_n = x\)가 finite real number라고 하자. Theorem \ref{thm limsup}\을 생각해 보면, 각 \(k\)에 대하여 \(\abs{x_n - x} < \frac{1}{k}\)인 \(n\)이 무한히 많이 존재한다. 따라서 \(\abs{x_{n(k)} - x} < \frac{1}{k}\)인 \(n(1) < n(2) < ...\)를 잡을 수 있다. 이제 \(x_{n(k)} \rightarrow x\)이므로 \(x \in E\)이다. 따라서 \(\sup E \ge x\).\\
		\(y > x\)인 \(y \in E\)가 존재한다고 가정하자. 그러면 \(x_{n'(k)} \rightarrow y\)인 subsequence \((x_{n'(k)})\)가 존재한다. 즉 \(\abs{x_n - x} > \frac{y-x}{2}\)인 \(n\)이 무한히 많이 존재하는데, 이는 상극한의 조건에 모순이다. 따라서 \(x = \sup E\)이다.
	\end{proof}

이제 우리의 결론을 state할 준비가 되었습니다.

	\begin{thm}
		\(\sum_n x_n\)이 conditionally convergent real series라고 하자.
		\begin{align*}
			-\infty \le \alpha \le \beta \le \infty
		\end{align*}
		인 \(\alpha, \beta\)에 대하여(즉 \(\alpha = -\infty\)나 \(\beta = \infty\) 같은 경우도 포함), 적당한 rearrangement \(\sum_n x_{r(n)}\)이 존재하여 그 partial sum \((s'_n)\)이
		\begin{align*}
			\liminf_n s'_n = \alpha, \quad \limsup_n s'_n = \beta
		\end{align*}
		를 만족한다.
	\end{thm}
	\begin{proof}
		\begin{align*}
			p_n = \frac{\abs{x_n} + x_n}{2}, \quad p_n = \frac{\abs{x_n} - x_n}{2}
		\end{align*}
		으로 정의하자. 즉 \((p_n)\)은 \((x_n)\)의 음수 항을 모두 0으로 만든 것이고, \((q_n)\)은 \((x_n)\)의 양수 항을 모두 0으로 만든 뒤 음수 항의 부호를 반대로 한 것이다. 이때 \(p_n + q_n = \abs{x_n},\: p_n - q_n = x_n, \: p_n, q_n \ge 0\)이다. \(\sum_n x_n\)이 conditionally converge하기 위해서는 \(\sum_n p_n\)과 \(\sum_n q_n\) 모두 발산해야 한다.\\
		\((x_n)\)의 0 이상의 항들을 순서대로 나열한 것을 \(P_1, P_2, P_3, ...\)이라고 하고, \((x_n)\)의 음수 항들을 순서대로 나열한 후 부호를 반대로 한 것을 \(Q_1, Q_2, Q_3, ...\)라고 하자(0 이상의 항과 음수 항은 각각 무수히 많이 존재해야 한다. 왜 그런가?). 그러면 \(\sum_n P_n, \sum_n Q_n\)과 \(\sum_n p_n, \sum_n q_n\)은 zero term밖에 차이가 없으므로, \(\sum_n P_n, \sum_n Q_n\)은 모두 발산해야 한다. 그리고 \(\lim_n x_n = 0\)이므로 \(\lim_n P_n = \lim_n Q_n = 0\)이다.\\
		증가수열 \(m, l: \NN \rightarrow \NN\)에 대하여,
		\begin{equation}
			P_1 + ... + P_{m(1)} - Q_1 - ... - Q_{l(1)} + P_{m(1)+1} + ... + P_{m(2)} - Q_{l(1)+1} - ... - Q_{l(2)} + ... \notag
		\end{equation}
		은 rearrangement series이다. 이 series가 원하는 조건을 만족하도록 \(m, l\)을 잡으려고 한다.\\
		먼저
		\begin{align*}
			\alpha_n \rightarrow \alpha, \beta_n \rightarrow \beta, \alpha_n < \beta_n, \beta_1 > 0
		\end{align*}
		을 만족하는 real sequence \((\alpha_n), (\beta_n)\)을 잡을 수 있다. \(m(1), l(1)\)을
		\begin{gather*}
			P_1 + ... + P_{m(1)} > \beta_1\\
			P_1 + ... + P_{m(1)} - Q_1 - ... - Q_{l(1)} < \alpha_1
		\end{gather*}
		을 만족하는 최소의 자연수로 정의한다. 다음으로 \(m(2), l(2)\)를
		\begin{gather*}
			P_1 + ... + P_{m(1)} - Q_1 - ... - Q_{l(1)} + P_{m(1) + 1} + ... + P_{m(2)} > \beta_2\\
			P_1 + ... + P_{m(1)} - Q_1 - ... - Q_{l(1)} + P_{m(1) + 1} + ... + P_{m(2)} - Q_{l(1)+1} - ... - Q_{l(2)} < \alpha_2
		\end{gather*}
		을 만족하는 최소의 자연수로 정의한다. \(\sum_n P_n, \sum_n Q_n\)이 발산하기 때문에, 이 과정을 반복하여 \(m(k), l(k)\)를 계속 정의할 수 있다.\\
		이렇게 만들어진 rearrangement series의 parial sum \((s'_n)\)의 두 subsequence
		\begin{gather*}
			x_k = P_1 + ... + P_{m(1)} - ... + P_{m(k-1)+1} + ... + P_{m(k)}\\
			y_k = P_1 + ... + P_{m(1)} - ... - Q_{l(k-1)+1} - ... - Q_{l(k)}
		\end{gather*}
		를 생각하자. 만약 두 sequence가 수렴한다면, upper limit과 lower limit의 새로운 정의(Theorem \ref{limsup})에 의해
		\begin{align*}
			\lim_k x_k = \limsup_n s'_n, \quad \lim_k y_k = \liminf_n s'_n
		\end{align*}
		인 것은 쉽게 알 수 있다. 따라서 \(\lim_n x_n = \beta, \lim_n y_n = \alpha\)인 것을 보이면 된다. \(m(k)\)와 \(l(k)\)의 정의에 의해,
		\begin{gather*}
			x_k - P_{m(k)} \le \beta_k < x_{n(k)}\\
			y_k + Q_{l(k)} \ge \alpha_k > y_{n(k)}
		\end{gather*}
		이다. 즉
		\begin{gather*}
			\abs{x_k - \beta_k} \le P_{m(k)}, \quad \abs{y_k - \alpha_k} \le -Q_{l(k)}
		\end{gather*}
		이다. 그런데 \(\lim_k P_{m(k)} = \lim_k Q_{l(k)} = 0\)이므로, \(x_k \rightarrow \beta, y_k \rightarrow \alpha\)이다.
		
	\end{proof}

위 정리에서 \(\alpha = \beta\)인 경우에는 바로 다음 따름정리를 얻을 수 있습니다.

	\begin{cor}
		\(\sum_n x_n\)이 conditionally convergent real series라고 하자. \(-\infty \le \alpha \le \infty\)에 대하여(즉 \(\alpha = \pm \infty\)인 경우도 포함), 적당한 rearrangement \(\sum_n x_{r(n)}\)이 존재하여 \(\sum_n x_{r(n)} = \alpha\)이다.
	\end{cor}
	\begin{proof}
		\(\limsup_n s'_n = \liminf_n s'_n = \alpha\)이면 \(\lim_n s'_n = \alpha\).
	\end{proof}

	\begin{ex}
		\(\sum_n (-1)^{n-1}\frac{1}{n}\)은 conditionally convergent series입니다. 우선 이 series의 partial sum을 \((s_n)\)이라 하고, \(\lim_n s_n = s\)라고 하겠습니다(나중에 \(s = \log 2\)임을 알게 됩니다). 그리고 다음 rearrangement의 partial sum을 \((s'_n)\)이라 하겠습니다.
		\begin{align*}
			1 + \frac{1}{3} - \frac{1}{2} + \frac{1}{5} + \frac{1}{7} - \frac{1}{4} + \frac{1}{9} +\frac{1}{11} - \frac{1}{6} + ...
		\end{align*}
		먼저 \(\lim_n s'_{3n}\)이 수렴한다는 것을 보이려고 합니다.
		\begin{align*}
			s'_{3n} &= \sum_{k=1}^n (\frac{1}{4k-3} + \frac{1}{4k-1} - \frac{1}{2k})\\
			&= \sum_{k=1}^n (\frac{1}{4k-3} - \frac{1}{4k-2} + \frac{1}{4k-1} - \frac{1}{4k}) + \sum_{k=1}^{n} (\frac{1}{4k-2} - \frac{1}{4k})\\
			&= s_{4n} + \frac{1}{2} s_{2n}
		\end{align*}
		따라서 \(\lim_n s'_{3n} = \frac{3}{2}s\)로 수렴합니다. 그리고
		\begin{align*}
			\abs{s'_{3n+1} - s'_{3n}} < \abs{s'_{3n+2} - s'_{3n}} = \frac{1}{4n+1} + \frac{1}{4n+3}
		\end{align*}
		이므로, \(\lim_n s'_{3n+1} = \lim_n s'_{3n+2} = \lim_n s'_{3n} = \frac{3}{2} s\)입니다. 이제 \(\lim_n s'_n = \frac{3}{2}s\)인 것은 극한의 정의에 의해 쉽게 알 수 있습니다.
	\end{ex}

\subsection{Abel's Test}
교대급수판정법 일반화하려고 합니다. 먼저 다음이 성립하는 것을 관찰합시다(부분적분과 식의 형태가 비슷함에 주목하기 바랍니다).

\begin{prop}[Summation by Parts]
	Sequence \((a_n), (b_n)\)에 대하여,
	\begin{align*}
		A_n = \sum_{k=1}^{n} a_k, A_0 = 0
	\end{align*}
	로 정의하자. 그러면 임의의 \(1 \le m < n\)에 대해
	\begin{align*}
		\sum_{k=m+1}^{n} a_k b_k = A_n b_{n+1} - A_m b_{m+1} - \sum_{k=m+1}^n A_k (b_{k+1} - b_k)
	\end{align*}
	이다.
\end{prop}
\begin{proof}
	\begin{align*}
		\sum_{k=1}^n a_k b_k &= \sum_{k=1}^n (A_k - A_{k-1})b_k\\
		&= \sum_{k=1}^n A_k b_k - \sum_{k=1}^n A_{k-1} b_k\\
		&= \sum_{k=1}^n A_k b_k - (\sum_{k=1}^n A_{k} b_{k+1} - A_{n} b_{n+1})\\
		&= A_n b_{n+1} - \sum_{k=1}^n A_k (b_{k+1} - b_k)
	\end{align*}
	이므로,
	\begin{align*}
		\sum_{k=1}^n a_k b_k - \sum_{k=1}^m a_k b_k
	\end{align*}
	를 계산하면 원하는 등식을 얻는다.
\end{proof}

이를 이용하면 \(\sum_n a_n b_n\)꼴의 series의 수렴을 판정할 때 도움이 됩니다.

\begin{thm}[Abel's Test]
	Sequence \((a_n), (b_n)\)이 다음을 만족한다고 하자.
	\begin{enumerate}[label=(\alph*), leftmargin=2\parindent]
		\item
		\(\sum_n a_n\)의 partial sum은 bounded.
		\item
		\((b_n)\)은 real-valued이고 단조감소수열.
		\item
		\(\lim_n b_n = 0\)
	\end{enumerate}
	그러면 series \(\sum_n a_n b_n\)은 수렴한다.
\end{thm}
\begin{proof}
	Cauchy criterion에 의해, 임의의 양수 \(\eps > 0\)에 대해 
	\begin{align*}
		n > m \ge N \Rightarrow \abs{\sum_{k=m+1}^{n} a_k b_k} < \eps
	\end{align*}
	인 \(N\)이 존재하는 것을 보이면 된다.\\
	\(\sum_n a_n\)의 partial sum을 \((A_n)\)이라 하면, 모든 \(n\)에 대해 \(\abs{A_n} \le M\)인 \(M > 0\)이 존재한다. 그리고 \(b_N < \frac{\eps}{3M}\)인 \(N\)을 잡는다. 이제 \(n > m \ge N\)이면,
	\begin{align*}
		\abs{\sum_{k=m+1}^{n} a_k b_k} &= \abs{A_n b_{n+1} - A_m b_{m+1} - \sum_{k=m+1}^n A_k (b_{k+1} - b_k)} \\
		& \le \abs{A_n} b_{n+1} + \abs{A_m} b_{m+1} + \sum_{k=m+1}^n \abs{A_k}\abs{b_{k+1}-b_k}\\
		& \le  Mb_{n+1} + M b_{m+1} + M \sum_{k=m+1}^n (b_k - b_{k+1})\\
		& < M \times \frac{\eps}{3M} + M \times \frac{\eps}{3M} + M \times b_{m+1}\\
		& < \frac{\eps}{M} + \frac{\eps}{M} + \frac{\eps}{M} = \eps
	\end{align*}
\end{proof}

\(a_n = (-1)^{n-1}\)로 두면, alternating series test와 동일합니다.

(b)는 `\(\sum_n \abs{b_{n+1} - b_n}\)이 수렴한다'로 약화시킬 수 있습니다.

\subsection{Cauchy Product}

이제 두 급수의 곱을 정의하려고 합니다. (엄밀한 표기는 아니지만) 다음을 관찰합시다. (Index의 편리함을 위해, 0부터 시작하는 급수를 생각합니다.)

\begin{align*}
	&(a_0 + a_1 + a_2 + \ldots + a_n + \ldots)(b_0 + b_1 + b_2 + \ldots + b_n + \ldots)\\
	= &\: a_0 b_0 + (a_0 b_1 + a_1 b_0) + (a_0 b_2 + a_1 b_1 + a_2 b_0) + \ldots + \sum_{k=0}^n a_k b_{n-k} + \ldots
\end{align*}

따라서 급수의 곱을 다음과 같이 정의하는 것은 자연스럽습니다.

\begin{defn}
	두 급수 \(\sum_n a_n, \sum_n b_n\)에 대하여
	\begin{gather*}
		c_n = \sum_{k=0}^{n} a_k b_{n-k}
	\end{gather*}
	로 정의하자. 이때 급수 \(\sum_n c_n\)을 두 급수의 \textbf{Cauchy product(코시곱)}라고 한다.
\end{defn}

이제 우리의 관심사는 Cauchy product가 진정한 급수의 곱인가, 즉 \(\sum_n a_n = A, \sum_n b_n = B\)로 수렴하면
\begin{enumerate}[label=(\alph*), leftmargin=2\parindent]
	\item
	\(\sum_n c_n\)이 수렴하는가?
	\item
	\(\sum_n c_n = AB\)인가?
\end{enumerate}
에 대한 답이 긍정적인지입니다. 안타깝게도 항상 그렇지는 않습니다.

\begin{ex}
	\begin{gather*}
		a_n = b_n = \frac{(-1)^n}{\sqrt{n+1}}
	\end{gather*}
	으로 정의하면 alternating series test에 의해 \(\sum_n a_n, \sum_n b_n\)은 수렴합니다. 그 Cauchy product \(\sum_n c_n\)은
	\begin{gather*}
		c_n = (-1)^n \sum_{k=0}^{n} \frac{1}{\sqrt{(k+1)(n-k+1)}}
	\end{gather*}
	로 정의되는데, 산술-기하평균 부등식에 의해
	\begin{align*}
		\abs{c_n} &= \sum_{k=0}^{n} \frac{1}{\sqrt{(k+1)(n-k+1)}}\\
		& \ge \sum_{k=0}^{n} \frac{1}{(n+2)/2} = \sum_{k=0}^{n} \frac{2}{n+2} = \frac{2(n+1)}{n+2}
	\end{align*}
	이므로 \(c_n \rightarrow 0\)이 아닙니다. 따라서 \(\sum_n c_n\)은 발산.
\end{ex}

이제 (a)와 (b)의 답이 긍정적이도록 하는 충분조건 하나를 소개하겠습니다. 그것은 \(\sum_n a_n\)과 \(\sum_n b_n\) 중 적어도 하나가 절대수렴하는 것입니다.

\begin{thm}
	\(\sum_n a_n=A, \sum_n b_n=B\)로 수렴하고, \(\sum_n a_n\)이 절대수렴한다고 하자. 그러면 그 Cauchy product \(\sum_n c_n\)은 수렴하고, \(\sum_n c_n = AB\)이다.
\end{thm}
\begin{proof}
	\(\sum_n a_n, \sum_n b_n, \sum_n c_n\)의 partial sum들을 각각 \((A_n), (B_n), (C_n)\)으로 두자. \(A_n B \rightarrow AB\)임은 명백하므로, \(C_n - A_n B \rightarrow 0\)임을 보이면 된다. [왜 이런 발상이 가능한가? 절대수렴하는 급수가 좀 더 `빠르게' 극한에 가까워진다고 생각할 수 있기 때문이다. 그래서 \(C_n - A B_n\)이 아니라, \(C_n - A_n B\)의 수렴을 보인다(굉장히 수학적이지 않은 문장이다\ldots).]\\
	\(\beta_n = B_n - B\)로 두고, 다음을 관찰하자.
	\begin{align*}
		C_n &= a_0 b_0 + (a_0 b_1 + a_1 b_0) + \ldots (a_0 b_n + a_n b_{n-1} + a_n b_0)\\
		&= a_0 B_n + a_1 B_{n-1} + \ldots + a_n B_0\\
		&= a_0 (B + \beta_n) + a_1 (B + \beta_{n-1}) + \ldots + a_n (B + \beta_0)\\
		&= A_n B + a_0 \beta_n + a_1 \beta_{n-1} + \ldots + a_n \beta_0
	\end{align*}
	이제
	\begin{gather*}
		\gamma_n = a_0 \beta_n + a_1 \beta_{n-1} + \ldots + a_n \beta_0
	\end{gather*}
	로 정의하고, (a)를 이용하기 위해
	\begin{gather*}
		M = \sum_n \abs{a_n} < \infty
	\end{gather*}
	로 정의하자. 이제 양수 \(\eps > 0\)이 주어졌다고 하고 \(n \ge N \Rightarrow \abs{\beta_n} < \eps\)인 \(N\)을 택한다. \(n > N\)에 대하여
	\begin{align*}
		\abs{\gamma_n} &= \abs{\beta_0 a_n + \ldots + \beta_N a_{n-N}} + \abs{\beta_{N+1} a_{n-N-1} + \ldots + \beta_n a_0}\\
		& \le \abs{\beta_0 a_n + \ldots + \beta_N a_{n-N}} + \eps(\abs{a_{n-N-1}} + \ldots + \abs{a_0})\\
		& \le \abs{\beta_0 a_n + \ldots + \beta_N a_{n-N}} + M\eps
	\end{align*}
	이다. \(N\)을 고정하고 양변에 \(\limsup_n\)을 취하면,
	\begin{gather*}
		\limsup_n \abs{\gamma_n} \le M\eps
	\end{gather*}
	이다. \(\eps > 0\)이 임의의 양수였으므로, \(\lim_n \gamma_n = 0\).
\end{proof}
\begin{rem}
	\(\sum_n \abs{a_n} < \infty\)인 성질이 없으면 위 증명이 어디서 깨지는지 생각해보기 바랍니다.
\end{rem}

또 하나의 궁금증: (a)는 성립하는데 (b)가 성립하지 않는 경우, 즉 \(\sum_n c_n\)이 수렴하는데 그 극한이 \(AB\)는 아닌 경우가 가능할까요? 이러한 경우는 불가능하다는 것이 알려져 있습니다. 즉 (a)가 성립하면 (b)는 항상 성립합니다. 이 경우에 \(\sum_n a_n\)이나 \(\sum_n b_n\) 중 하나의 절대수렴 조건은 요구되지 않습니다.

\begin{thm}[Abel]
	\(\sum_n a_n = A, \sum_n b_n = B\)로 수렴하고 그 Cauchy product도 \(\sum_n c_n = C\)로 수렴하면 \(AB = C\)이다.
\end{thm}

한참 후에 power series에 대해 공부하고 나면, 이 정리는 당연해집니다. 증명은 그때로 미루겠습니다.

\subsection{Cauchy Condensation Test}

다음 판정법은 좀 재미있습니다.

\begin{thm}[Cauchy Condensation Test]
	Real sequence \((a_n)\)이 nonnegative이고 단조감소할 때, 다음은 동치.
	\begin{enumerate}[label=(\alph*), leftmargin=2\parindent]
		\item
		\(\sum_{n=1}^\infty a_n\)이 수렴.
		\item
		\(\sum_{k=0}^\infty 2^k a_{2^k} = a_1 + 2a_2 + 4a_4 + 8a_8 + \ldots\)가 수렴.
	\end{enumerate}
\end{thm}
\begin{proof}
	두 급수의 partial sum을 각각
	\begin{gather*}
		s_n = a_1 + a_2 + \ldots + a_n\\
		t_k = a_1 + 2a_2 + \ldots + 2^k a_{2^k}
	\end{gather*}
	로 두자. 먼저 \(n \le 2^N - 1\)인 \(N\)을 택하면,
	\begin{align*}
		s_n \le s_{2^N - 1} &= a_1 + (a_2 + a_3) + (a_4 + \ldots + a_7) + \ldots  (a_{2^{N-1}} + \ldots + a_{2^N - 1})\\
		& \le a_1 + 2a_2 + 4a_4 + \ldots + 2^{N-1} a_{2^{N-1}} = t_{N-1}
	\end{align*}
	이므로 \((t_k)\)가 bounded이면 \((s_n)\)도 bounded이다. 따라서 (b)\(\Rightarrow\)(a). 한편
	\begin{align*}
		t_k &= a_1 + 2a_2 + 4a_4 + \ldots + 2^k a_{2^k}\\
		&\le 2(a_1 + a_2 + 2a_4 + \ldots 2^{k-1} a_{2^k})\\
		&\le 2(a_1 + a_2 + (a_3 + a_4) + \ldots + (a_{2^{k-1}+1} + \ldots a_{2^k}))\\
		&= 2s_{2^k}
	\end{align*}
	이므로 \((s_n)\)이 bounded이면 \((t_k)\)도 bounded이다. 따라서 (a)\(\Rightarrow\)(b).
\end{proof}

의외로 다음 판정법을 가장 쉽게 유도하는 방법이 위의 Cauchy condensation test입니다(이거 증명하려고 적분을 이용하는 건 좀 쥐 잡는 데 소 잡는 칼 쓰는 느낌\ldots 게다가 우리는 아직 적분을 정의하지도 않았습니다).

\begin{cor}[\(p\)-series Test]
	\(\sum_n 1 / n^p\) converges if and only if \(p > 1\).
\end{cor}
\begin{proof}
	Cauchy condensation test에 의해, \(\sum_n 1 / n^p\) converges if and only if \(\sum_k 2^k / (2^k)^p = \sum_k 1 / (2^k)^{p-1}\) converges if and only if \(p - 1 > 0\).
\end{proof}

\(\log\)가 들어간 급수도 Cauchy condensation test를 이용하면 편리합니다.

\begin{ex}
	\(\sum_{n=2}^\infty 1/(n(\log n)^p)\) converges if and only if \(p > 1\).
\end{ex}

\subsection{Root Test and Ratio Test}

이제 마지막 수렴판정법 2개입니다. 사실상 실전에서 가장 많이 쓰게 되는 것들이 아닐까 생각합니다.

\begin{thm}[Root Test: 근 판정법]
	\(\sum_n a_n\)에 대하여 \(\alpha = \limsup_n \sqrt[n]{\abs{a_n}}\)으로 정의하자. 이때
	\begin{enumerate}[label=(\alph*), leftmargin=2\parindent]
		\item
		\(\alpha < 1\)이면 \(\sum_n a_n\)은 절대수렴한다.
		\item
		\(\alpha > 1\)이면 \(\sum_n a_n\)은 발산한다.
		\item
		\(\alpha = 1\)이면 \(\sum_n a_n\)의 수렴 여부를 알 수 없다.
	\end{enumerate}
\end{thm}
\begin{proof}
	\(\alpha < 1\)이면 \(\alpha < \beta < 1\)인 \(\beta\)를 택할 수 있다. 상극한의 정의에 의해, \(n \ge N \Rightarrow \sqrt[n]{\abs{a_n}} < \beta\)인 \(N\)이 존재한다. 따라서 \(n \ge N\)이면 \(\abs{a_n} < \beta^n\)이므로
	\begin{gather*}
		\sum_{n=N}^{\infty} \abs{a_n} \le \sum_{n=N}^{\infty} \beta^n < \infty.
	\end{gather*}
	\(\alpha > 1\)이면 \(\lim_n \abs{a_n} = 0\)일 수 없으므로 \(\sum_n a_n\)이 발산한다.\\
	마지막으로 \(a_n = 1/n, b_n = 1/n^2\)을 생각하면 \(\limsup_n \sqrt[n]{\abs{a_n}} = \limsup_n \sqrt[n]{\abs{b_n}} = 1\)이지만(\(\lim_n n^{1/n}=1\)임을 이용하였다) \(\sum_n a_n\)은 발산, \(\sum_n b_n\)은 수렴한다.
\end{proof}

\begin{thm}[Ratio Test: 비 판정법]
	\(\sum_n a_n\)에 대하여 다음이 성립한다(단, \(a_n \neq 0\)).
	\begin{enumerate}[label=(\alph*), leftmargin=2\parindent]
		\item
		\(\limsup_n \abs{a_{n+1}/a_n} < 1\)이면 \(\sum_n a_n\)은 절대수렴한다.
		\item
		\(n \ge n_0 \Rightarrow \abs{a_{n+1}/a_n} \ge 1\) for some \(n_0\)이면 \(\sum_n a_n\)은 발산한다.
	\end{enumerate}
\end{thm}
\begin{proof}
	\(\alpha = \limsup_n \abs{a_{n+1}/a_n} < 1\)이면 \(\alpha < \beta < 1\)인 \(\beta\)를 택할 수 있다. 상극한의 정의에 의해 \(n \ge N \Rightarrow \abs{a_{n+1}/a_n} < \beta\)인 \(N\)이 존재한다. 따라서 \(n \ge N\)이면 \(\abs{a_n} < \beta^{n-N}\abs{a_N}\)이므로
	\begin{gather*}
		\sum_{n=N}^\infty \abs{a_n} \le \sum_{n=N}^\infty \beta^{n-N}\abs{a_N} < \infty.
	\end{gather*}
	\(n \ge n_0 \Rightarrow \abs{a_{n+1}/a_n} \ge 1\) for some \(n_0\)이면 \(\lim_n \abs{a_n} = 0\)일 수 없으므로 \(\sum_n a_n\)이 발산한다.
\end{proof}
\begin{rem}
	(b)의 충분조건은 \(\liminf_n \abs{a_{n+1}/a_n} > 1\)인 것입니다. 따라서 \(\limsup_n \abs{a_{n+1}/a_n} < 1\)이거나 \(\liminf_n \abs{a_{n+1}/a_n} > 1\)인 경우에는 상극한과 하극한의 계산만으로 ratio test를 적용할 수 있고, 그렇지 않은 경우에는 \(\abs{a_{n+1}/a_n}\) 자체에 대한 관찰이 필요합니다.
\end{rem}

다음 정리는 root test와 ratio test의 판정 범위를 비교해 줍니다.

\begin{thm}
	\((a_n)\)이 sequence of real positive numbers라고 하자. 이때 다음 부등식이 성립한다.
	\begin{gather*}
		\liminf_n \frac{a_{n+1}}{a_n} \le \liminf_n \sqrt[n]{a_n} \le \limsup_n \sqrt[n]{a_n} \le \limsup_n \frac{a_{n+1}}{a_n}
	\end{gather*}
\end{thm}
\begin{proof}
	두 번째 부등호는 당연하다. 세 번째 부등호가 성립하는 것만 보이면, 첫 번째 부등호가 성립하는 것은 비슷하게 보일 수 있다.\\
	\(\alpha = \limsup_n a_{n+1}/a_n\)으로 정의하자. \(\alpha = \infty\)이면 증명할 것이 없으므로 \(\alpha < \infty\)라고 가정하면, \(\alpha < \beta < \infty\)인 \(\beta\)를 택할 수 있다. 상극한의 정의에 의해 \(n \ge N \Rightarrow a_{n+1}/a_n < \beta\)인 \(N\)이 존재한다. 이제 \(n \ge N\)에 대해
	\begin{align*}
		a_n &< \beta^{n-N} a_N\\
		\sqrt[n]{a_n} &< \beta^{1-\frac{N}{n}} \sqrt[n]{a_N}
	\end{align*}
	이므로, 양변에 \(\limsup_n\)을 취하면
	\begin{align*}
		\limsup_n \sqrt[n]{a_n} \le \beta
	\end{align*}
	이다. 그런데 \(\beta\)는 \(\alpha\)보다 큰 임의의 실수이므로, \(\limsup_n \sqrt[n]{a_n} \le \alpha\).
\end{proof}

위 정리의 의미를 생각해봅시다. Ratio test에 의해 \(\sum_n a_n\)의 수렴을 보일 수 있으면, 즉 \(\limsup_n a_{n+1}/a_n < 1\)이면 \(\limsup_n \sqrt[n]{a_n} \le \limsup_n a_{n+1}/a_n < 1\)이므로 root test에 의해서도 \(\sum_n a_n\)의 수렴을 보일 수 있습니다. 또 root test로 \(\sum_n a_n\)의 수렴 여부를 결정할 수 없으면, 즉 \(\limsup_n \sqrt[n]{a_n} = 1\)이면 \(\limsup_n a_{n+1}/a_n \ge 1\)이고 \(\liminf_n a_{n+1}/a_n \le 1\)이므로 ratio test에 의해서도 \(\sum_n a_n\)의 수렴 여부를 결정할 수 없습니다.
\par 다음 예시는 root test에 의해서 수렴 판정이 가능하지만 ratio test에 의해서는 수렴 판정이 불가능한 급수의 존재를 보여줍니다. 즉 root test가 ratio test보다 `강력한' 판정법입니다. 그런데 \(\limsup_n \sqrt[n]{a_n}\)의 값을 구하기가 어려울 때가 많기 때문에 실전에서는 ratio test도 자주 쓰이는 것입니다.

\begin{ex}
	다음 급수 \(\sum_n a_n\)을 생각합시다.
	\begin{gather*}
		\frac{1}{2^1} + \frac{1}{2^0} + \frac{1}{2^3} + \frac{1}{2^2} + \frac{1}{2^5} + \frac{1}{2^4} + \ldots
	\end{gather*}
	다음은 쉽게 계산할 수 있습니다.
	\begin{gather*}
		\liminf_n \sqrt[n]{a_n} = \limsup_n \sqrt[n]{a_n} = \frac{1}{2}\\
		\liminf_n \frac{a_{n+1}}{a_n} = \frac{1}{8}, \quad \limsup_n \frac{a_{n+1}}{a_n} = 2
	\end{gather*}
	따라서 root test에 의해서는 \(\sum_n a_n\)이 수렴함을 알 수 있지만 ratio test는 정보를 주지 못합니다.
\end{ex}

\subsection{Application: \(e\)}

Ratio test에 의해 \(\sum_{n=0}^\infty 1/n!\)이 수렴합니다.

\begin{defn}
	\(e = \sum_{n=0}^\infty 1/n!\).
\end{defn}

\(n! \ge 2^{n-1}\)임을 이용하면, \(2 < e \le 1 + \sum_{n=1}^\infty 1/2^{n-1} = 3\)임을 알 수 있습니다.

고등학교 미적분에서는 다음 정리를 주어진 사실로 받아들이고 이를 \(e\)의 정의로 사용하였습니다. 이제 우리는 이를 증명할 수 있습니다.

\begin{thm} \label{thm e}
	\(\lim_n (1 + 1/n)^n = e\).
\end{thm}
\begin{proof}
	\begin{gather*}
		s_n = \sum_{k=0}^{n} \frac{1}{k!}, \quad t_n = \left(1 + \frac{1}{n}\right)^n
	\end{gather*}
	으로 두자. 먼저
	\begin{align*}
		t_n &= \sum_{k=0}^{n} \frac{n(n-1)\ldots(n-k+1)}{k!}\left (\frac{1}{n}\right )^k\\
		&= \sum_{k=0}^{n}\frac{1}{k!}\left(1 - \frac{1}{n}\right)\ldots\left(1 - \frac{k-1}{n}\right) \le \sum_{k=0}^{n}\frac{1}{k!} = s_n
	\end{align*}
	이므로 양변에 \(\limsup_n\)을 취하면
	\begin{align*}
		\limsup_n t_n \le \limsup_n s_n = \lim_n s_n = e.
	\end{align*}
	또 \(n \ge m\)이면,
	\begin{align*}
		t_n &= \sum_{k=0}^{n}\frac{1}{k!}\left(1 - \frac{1}{n}\right)\ldots\left(1 - \frac{k-1}{n}\right)\\
		&\ge \sum_{k=0}^{m}\frac{1}{k!}\left(1 - \frac{1}{n}\right)\ldots\left(1 - \frac{k-1}{n}\right)
	\end{align*}
	이다. \(m\)을 고정하고 양변에 \(\liminf_n\)을 취하면,
	\begin{align*}
		\liminf_n t_n &\ge \liminf_n \sum_{k=0}^{m}\frac{1}{k!}\left(1 - \frac{1}{n}\right)\ldots\left(1 - \frac{k-1}{n}\right)\\
		&= \lim_n \sum_{k=0}^{m}\frac{1}{k!}\left(1 - \frac{1}{n}\right)\ldots\left(1 - \frac{k-1}{n}\right) = \sum_{k=0}^{m}\frac{1}{k!}
	\end{align*}
	이므로
	\begin{align*}
		\liminf_n t_n \ge t_m
	\end{align*}
	이다. 그런데 이는 임의의 \(m\)에 대해 성립하므로 양변에 \(\lim_m\)을 취하면
	\begin{align*}
		\liminf_n t_n \ge e
	\end{align*}
	이다. 따라서 \(\liminf_n t_n \ge e \ge \limsup_n t_n\)이므로, \(\lim_n t_n = e\).
\end{proof}

다음 사실도 이제서야 증명할 수 있습니다.

\begin{thm}
	\(e\)는 무리수이다.
\end{thm}
\begin{proof}
	\(s_n\)을 Theorem \ref{thm e}의 증명에서와 같이 정의한다.\\
	\(e\)가 유리수라고 가정하면 어떤 자연수 \(p, q\)에 대해 \(e = p/q\)로 쓸 수 있다. \(n > m\)일 때 \(n!/m! > m^{n-m}\)인 것을 이용하면,
	\begin{align*}
		0 < e - s_q &= \sum_{k=1}^\infty \frac{1}{(q+k)!}\\
		&< \frac{1}{(q+1)!}\sum_{k=1}^\infty \frac{1}{(q+1)^{k-1}} = \frac{1}{(q+1)!} \frac{1}{q} < \frac{1}{q!q}
	\end{align*}
	이다. 따라서
	\begin{align*}
		0 < q!e - q!s_q < \frac{1}{q}
	\end{align*}
	이다. 그런데
	\begin{align*}
		q!e - q!s_q = (q-1)!p - \sum_{k=0}^q \frac{q!}{k!}
	\end{align*}
	는 정수이므로 모순이다. 따라서 \(e\)는 무리수이다.
\end{proof}


\newpage


\section{Compactness} \label{sec cpt}

Section \ref{sec cpt}\이 끝날 때까지 아무 말이 없으면 \(X\)는 metric space. 우리의 이야기는 지금부터 잠시 일반적인 metric space \(X\)로 돌아갔다가, 다시 우리의 고향 Euclidean space로 되돌아올 예정입니다.

\subsection{Compact Sets in a Metric Space}

	\begin{defn}
		\quad
		\begin{enumerate}[label=(\alph*), leftmargin=2\parindent]
			\item
			\(E \subseteq X\)에 대하여 \(X\)의 subset들의 collection \(\{U_i\}_{i \in I}\)가 \textbf{cover(덮개) of \(E\)}라는 것은 \(E \subseteq \bigcup_{i \in I} U_i\)라는 것이다. 각 \(U_i\)가 open set in \(X\)이면 \(\{U_i\}_{i \in I}\)를 \textbf{open cover(열린덮개) of \(E\)}라고 한다.
			\item
			\(\{U_i\}_{i \in I}\)가 \(E \subseteq X\)의 cover일 때, index set \(I\)의 subset \(J\)에 대하여 \(\{U_i\}_{i \in J}\)도 \(E\)의 open cover이면 \(\{U_i\}_{i \in J}\)를 \textbf{subcover of \(\{U_i\}_{i \in I}\)}라고 한다.
			\item
			\(K \subseteq X\)가 \textbf{compact set(옹골집합\footnote{Compact set의 번역어인데, 아무도 이 말을 쓰지 않습니다.})}라는 것은 \(K\)의 임의의 open cover가 finite subcover를 가진다는 것이다. 즉, \(\{U_i\}_{i \in I}\)가 \(K\)의 open cover이면 유한한 개수의 \(i_1, \ldots, i_n \in I\)가 존재하여 \(K \subseteq \bigcup_{k=1}^n U_{i_k}\)라는 것이다.
		\end{enumerate}	
	\end{defn}


	\begin{ex}
		\quad
		\begin{enumerate}[label=(\alph*), leftmargin=2\parindent]
			\item
			\(X\)의 finite subset \(E=\{x_1, \ldots, x_n\}\)은 compact입니다. \(\{U_i\}_{i \in I}\)가 open cover of \(E\)라고 하면, 각 \(k\)에 대해 \(x_k \in U_{i_k}\)인 \(i_k \in I\)가 존재합니다. 따라서 finite subcover \(\{U_i\}_{i=i_1, \ldots, i_n}\). 앞으로 계속 보게 되겠지만, compact set은 \textbf{finite set의 일반화}입니다.
			\item
			\(X\)는 usual한 metric이 주어진 \(\RR\)이고, \(E = \RR\). 그러면 \(\{(-i, i)\}_{i \in \NN}\)은 open cover of \(E\)이다. 이 open cover가 finite subcover \(\{(-i, i)\}_{i = i_1, \ldots i_n}\)를 가진다고 가정합시다. 그런데 \(M = \max \{i_1, \ldots i_n\}\)으로 정의하면 \(M+1 \notin \bigcup_{k=1}^n (-i_k, i_k)\)이므로 모순. 따라서 \(E\) is not compact.
			\item
			\(X\)는 (b)에서와 같고, \(E = (0, 1)\). 그러면 \(\{(1/i, 1)\}_{i \in \NN}\)은 open cover of \(E\)입니다. 이 open cover가 finite subcover \(\{(1/i, 1)\}_{i = i_1, \ldots i_n}\)를 가진다고 가정합시다. 그런데 \(M = \max \{i_1, \ldots i_n\}\)으로 정의하면 \(1/(M+1) \notin \bigcup_{k=1}^n (1/i, 1)\)이므로 모순. 따라서 \(E\) is not compact.
		\end{enumerate}	
	\end{ex}

Compact set의 좋은 점은, open / closed와 달리 어느 공간을 기준으로 생각하는지에 따라 compact 여부가 달라지지 않는다는 것입니다. 따라서 ``compact space \(X\)''라는 말이 가능합니다(open space나 closed space라는 말은 없습니다).

	\begin{prop}
		\(K \subseteq Y \subseteq X\)에 대하여, \(K\) is compact relative to  \(X\) if and only if \(K\) is compact relative to \(Y\).
	\end{prop}
	\begin{proof}
		\(K\) is compact relative to \(X\)라고 가정하고, \(\{V_i\}_{i \in I}\)가 open cover of \(K\) in \(Y\)라고 하자. 그러면 \(V_i = U_i \cap Y\) for some open set \(U_i\) relative to \(X\)이고, \(\{U_i\}_{i \in I}\)가 open cover of \(K\) in \(X\)인 것은 당연하다. 이제 가정에 의해 finite subcover \(\{U_i\}_{i=i_1, \ldots, i_n}\)을 찾을 수 있는데, \(K \subseteq Y\)이므로
		\begin{align*}
			K = K \cap Y &\subseteq \bigcup_{k=1}^n U_{i_k} \cap Y = \bigcup_{k=1}^n (U_{i_k} \cap Y) = \bigcup_{k=1}^n V_{i_k}
		\end{align*}
		이다. 따라서 \(\{V_i\}_{i \in I}\)의 finite subcover \(\{V_i\}_{i=i_1, \ldots, i_n}\)를 찾을 수 있으므로, \(\Rightarrow\) 방향이 증명되었다. 이 증명을 밑에서부터 거꾸로 올라가면 역이 증명된다.
	\end{proof}

다음은 metric space에서 compact set의 필요조건입니다.

	\begin{thm}
		Compact subsets of \(X\) are closed.
	\end{thm}
	\begin{proof}
		\(K \subseteq X\)가 compact라고 하고, \(p \in X \backslash K\)를 고정하자. 각 \(q \in K\)에 대하여, \(U_q \cap V_q = \emptyset\)을 만족하는 \(p, q\)의 neighborhood \(U_q, V_q\) in \(X\)가 존재한다. 그러면 \(\{V_q\}_{q \in K}\)는 open cover of \(K\). 가정에 의해 finite subcover \(\{V_q\}_{q=q_1, \ldots, q_n}\)이 존재한다. \(U=\bigcap_{k=1}^n U_{q_k}\)로 정의하면 \(U\)는 \(p\)를 포함하는 open set of \(X\)이고,
		\begin{align*}
			U \cap K \subseteq \bigcap_{k=1}^n U_{q_k} \cap \bigcup_{k=1}^n V_{q_k} \subseteq \bigcup_{k=1}^n (U_{q_k} \cap V_{q_k}) = \emptyset
		\end{align*}
		이므로 \(U \subseteq X \backslash K\)이다. 따라서 \(K\) is closed.
	\end{proof}
	
	\begin{thm}
		Compact subsets of a metric space \(X\) are bounded.
	\end{thm}
	\begin{proof}
		\(K \subseteq X\)가 compact라고 하고, \(p \in X\)를 고정하자. \(\{N_r (p)\}_{r \in \NN}\)은 open cover of \(K\)이므로 finite subcover \(\{N_r (p)\}_{r=r_1, \ldots, r_n}\)이 존재한다. \(r = \max \{r_1, \ldots, r_n\}\)으로 정의하면 \(K \subseteq N_r (p)\)이므로, \(K\) is bounded.
	\end{proof}

주어진 compact set으로부터 다른 compact set을 만들어낼 때 다음이 유용합니다.

	\begin{thm} \label{thm cpt subset}
		Closed subsets of compact sets are compact.
	\end{thm}
	\begin{proof}
		\(K \subseteq X\)가 compact이고 \(F \subseteq K\)가 closed relative to \(K\)라고 하자. 임의의 open cover \(\{U_i\}_{i \in I}\) of \(F\) relative to \(K\)에 대하여, \(\{U_i\}_{i \in I} \cup (K \backslash F)\)는 open cover of \(K\) relative to \(K\)이다. \(K\)는 compact이므로 finite subcover \(\{U_i\}_{i=i_1, \ldots, i_n} \cup (K \backslash F)\)가 존재한다(일반성을 잃지 않고 \(K \backslash F\)가 이 안에 들어 있다고 말해도 된다). 그런데 \(F \cap (K \backslash F) = \emptyset \)이므로 \(F \subseteq \bigcup_{k=1}^n U_{i_k}\)이어야 한다. 따라서 \(\{U_i\}_{i=i_1, \ldots, i_n}\)는 finite subcover of \(\{U_i\}_{i \in I}\)이므로 \(F\) is compact.
	\end{proof}

Nested intervals theorem과 같은 이야기를 compact set에 대해서도 할 수 있는데, 이는 compactness의 정의에 의해 당연합니다.

	\begin{defn}
		집합 \(X\)의 subset들의 collection \(\mathcal{C}\)가 \textbf{finite intersection property}를 가진다는 것은, 임의의 finite subcollection \(\{C_1, \ldots, C_n\}\) of \(\mathcal{C}\)에 대하여 \(\bigcap_{k=1}^n C_k\)가 nonempty라는 것이다.
	\end{defn}

	\begin{ex}
		\(\{(0, \frac{1}{n})\}_{n \in \NN}\)은 finite intersection property를 가집니다. 반면 \(\{(n, n+1)\}_{n \in \NN}\)은 finite intersection property를 가지지 않습니다.
	\end{ex}

	\begin{thm} \label{thm fip}
		다음은 동치.
		\begin{enumerate}[label=(\alph*), leftmargin=2\parindent]
			\item
			\(X\) is compact.
			\item
			Collection \(\mathcal{C}\) of closed subsets of \(X\)에 대하여, \(\mathcal{C}\)가 finite intersection property를 가지면 \(\bigcap_{C \in \mathcal{C}} C\) is nonempty.
		\end{enumerate}	
	\end{thm}

	\begin{proof}
		\(X\)가 compact라고 하고 \(\mathcal{C}\)가 finite intersection property를 가지는 collection of closed subsets of \(X\)라고 하자. 또 \(\mathcal{A} = \{X \backslash C\}_{C \in \mathcal{C}}\)로 정의하면, \(\mathcal{A}\)는 \(X\)의 open set들의 collection. 이제 \(\bigcap_{C \in \mathcal{C}} C = \emptyset\)이면 \(\mathcal{A}\)는 \(X\)의 open cover이므로 finite subcover \(\{X \backslash C_i\}_{i = 1, \ldots n}\)을 가진다. 그러면 \(\bigcap_{i=1}^n C_i = \emptyset\)이므로 \(\mathcal{C}\)가 finite intersection property를 가진다는 것에 모순이다. 따라서 (a)$\Rightarrow$(b)가 증명되었고, 비슷한 방식으로 역도 증명할 수 있다.
	\end{proof}

\(\mathcal{C} = \{C_1, C_2, \ldots\}\)이고 \(C_1 \supseteq C_2 \supseteq \ldots, C_i \neq \emptyset\)이면 finite intersection property를 만족합니다. 뒤에서 \(\RR\)의 유계닫힌구간이 compact set임을 증명하게 되는데, 따라서 Theorem \ref{thm fip}의 특수한 경우가 \(\RR\)에서 nested intervals theorem입니다. 

\subsection{Limit Point Compactness and Sequential Compactness} \label{sec lpc}


	\begin{defn}
		\(X\)가 \textbf{limit point compact}라는 것은 \(X\)의 임의의 infinite subset이 \(X\) 안에서 limit point를 가진다는 것이다.
	\end{defn}

Bolzano-Weierstrass theorem이 떠오릅니다. 실제로 limit point compactness를 Bolzano-Weierstrass property라고 부르기도 합니다.

	\begin{thm}
		Compact이면 limit point compact.
	\end{thm}
	\begin{proof}
		\(X\)가 compact라고 하자. \(X\)가 limit point compact임을 보이는 것은 다음을 보이는 것과 동치이다.
		\begin{center}
			\(A \subseteq X\)가 limit point를 가지지 않으면 \(A\)는 finite이다.
		\end{center}
		\(A \subseteq X\)가 limit point를 가지지 않는다고 가정하면 \(A\)는 closed in \(X\)이고, Theorem \ref{thm cpt subset}에 의해 \(A\)는 compact. 한편 각 \(a \in A\)는 isolated point이므로, 적당한 open neighborhood \(U_a\) of \(a\) in \(A\)가 존재하여 \(U_a \cap  A= \{a\}\)이다. 이제 \(\{U_a\}_{a \in A}\)는 open cover of \(A\)이고, \(A\)가 compact이므로 finite subcover \(\{U_{a_1}, \ldots, U_{a_n}\}\)이 존재한다. 그런데 각 \(U_{a_k}\)에 포함되는 \(A\)의 원소는 \(a_k\)뿐이므로, \(A = \{a_1, \ldots, a_n\}\)이다. 즉 \(A\)는 finite.
	\end{proof}

역은 metric space에서 성립한다는 것을 잠시 후에 증명할 것입니다.

정의 하나 더.

	\begin{defn}
		\(X\)가 \textbf{sequentially compact}라는 것은 \(X\) 안의 임의의 sequence가 \(X\) 안에서 수렴하는 subsequence를 가진다는 것이다.
	\end{defn}

Section \ref{sec open}에서 공부했던 개념들과 비슷한 것들이 재등장하고 있습니다.

이제 Section \ref{sec lpc}의 highlight입니다.

	\begin{thm} \label{1.11}
		Metric space \(X\)에 대하여 다음은 동치.
		\begin{enumerate}[label=(\alph*), leftmargin=2\parindent]
			\item
			\(X\) is compact.
			\item
			\(X\) is limit point compact.
			\item
			\(X\) is sequentially compact.
		\end{enumerate}	
	\end{thm}
	\begin{proof}
		(a)$\Rightarrow$(b)는 이미 보였다.\\
		(b)$\Rightarrow$(c): \(X\)가 LPC라고 하고, \((x_n)\)이 sequence in \(X\)라고 하자. \(A = \{x_n\}_{n \in \NN}\)이 finite이면, 어떤 \(x \in A\)에 대해 \(x_n = x\)인 \(n\)이 무수히 많이 존재하므로 \(x\)로 수렴하는 subsequence를 찾을 수 있다. 이제 \(A\)가 infinite이라고 가정하자. \(X\)가 LPC이므로 \(A\)는 limit point \(x \in X\)를 가진다. 임의의 \(k\)에 대해 \(N_{1/k} (x)\)는 \(A\)의 점을 무수히 많이 포함하고 있으므로, \(x_{n(k)} \in N_{1/k} (x)\)인 \(n(1) < n(2) < \ldots\)를 찾을 수 있다. 이제 \((x_{n(k)})\)는 \(x \in X\)로 수렴하는 subsequence.\footnote{이런 증명은 전에 한번 한 적이 있습니다. 어디서\ldots?}\\
		(c)$\Rightarrow$(a): \(X\)가 SC라고 하고 \(\{U_i\}_{i \in I}\)가 open cover of \(X\)라고 하자.
		\begin{enumerate}[label={Step \arabic*}, leftmargin=3\parindent]
			\item
			먼저 다음 성질을 만족하는 \(\delta > 0\)이 존재함을 보이려고 한다.\footnote{이러한 \(\delta > 0\)을 \textbf{Lesbesgue number}라고 합니다.}
			\begin{center}
				\(A \subseteq X\)의 diameter \(\sup \{d(x, y) : x, y \in A\} < \delta\)이면 \(A \subseteq U_i\) for some \(U_i\)
			\end{center}
			모순을 보이기 위해 다음을 가정하자.
			\begin{center}
				각 \(n \in \NN\)에 대해, subset \(C_n \subseteq X\)이 존재하여\\
				\(\mathrm{diam} C_n < 1/n\) but there exists no \(U_i\) s.t. \(C_n \subseteq U_i\)
			\end{center}
			각 \(C_n\)은 nonempty이므로 \(x_n \in C_n\)을 택할 수 있다. 이제 \(X\)가 SC이므로 sequence \((x_n)\)은 어떤 \(a \in X\)로 수렴하는 subsequence \((x_{n(k)})\)를 가진다. 이제 \(a \in U\)인 \(U \in \{U_i\}_{i \in I}\)가 존재한다. \(U\)는 open이므로, 적당한 양수 \(\eps > 0\)에 대하여 \(N_\eps (a) \subseteq A\)이다.\\
			\(n(k_1) > 2/\eps\)인 \(k_1\)을 잡고, \(k \ge k_2 \Rightarrow d(x_{n(k)}, a) < \eps\)인 \(k_2\)를 잡을 수 있다. \(K = \max\{k_1, k_2\}\)라고 하면 \(\mathrm{diam}C_K < \eps / 2\)이므로 각 \(p \in C_K\)에 대해 \(d(x_{n(K)}, p) \le \eps / 2\). 따라서 \(d(a, p) \le d(a, x_{n(K)}) + d(x_{n(K)}, p) < \eps\)이므로, \(C_K \subseteq N_\eps (a) \subseteq A\)이다. 따라서 모순. 
			\item
			다음으로 임의의 \(\eps > 0\)에 대해 유한 개의 \(x_1, \ldots, x_n \in X\)가 존재하여 \(X = \bigcup_{k=1}^n N_\eps (x_k)\)임을 보인다. 만약 그러한 유한 개의 점들이 존재하지 않는다고 가정하자. 그러면 다음과 같이 \((x_n)\)을 구성할 수 있다.\\
			\(x_1\)은 \(X\)의 아무 점으로 하자. \(N_\eps (x_1)\)은 \(X\) 전체가 될 수 없으므로 \(x_2 \in X \backslash N_\eps (x_1)\)이 존재한다. 계속해서 \(x_1, \ldots, x_k\)가 주어지면, \(x_{k+1} \in X \backslash \bigcup_{i=1}^k N_\eps (x_i)\)를 택한다. 그러면 정의에 의해 \(d(x_{k+1}, x_i) \ge \eps\) for \(i = 1, \ldots, k\)인데, 따라서 \((x_n)\)의 어느 subsequence도 Cauchy sequence가 아니므로 수렴하지 않는다. 이는 \(X\)가 SC인 것에 모순이다.
			\item
			마지막으로 \(\{U_i\}_{i \in I}\)가 finite subcover를 가짐을 보인다. Step 1에서 구한 \(\delta > 0\)에 대하여, \(\eps = \delta / 3\)으로 놓는다. 그러면 Step 2에 의하여, 유한 개의 \(x_1, \ldots, x_n\)이 존재하여 \(X = \bigcup_{k=1}^n N_\eps (x_k)\)이다. 각 \(k\)에 대해 \(\mathrm{diam}N_\eps(x_k) \le 2 \delta / 3\)이므로, \(\delta\)의 정의에 의해 \(N_\eps(x_k) \subseteq U_k\) for some \(U_k\). 이제 \(\{U_k\}_{k = 1, \ldots, n}\)은 finite subcover이다. 따라서 \(X\)는 compact.
		\end{enumerate}
	\end{proof}

\subsection{Compact Sets in a Euclidean Space} \label{sec hb}

드디어 우리의 고향으로 돌아왔습니다. Section \ref{sec hb}의 목표는 Euclidean space \(\RR^N\)에서 compact set의 필요충분조건을 찾는 것입니다.

임의의 metric space에서 compact이면 closed and bounded임을 알고 있습니다. 그런데 놀랍게도, Euclidean space에서는 그 역이 성립합니다.

	\begin{thm} [Heine-Borel Theorem]
		\(K \subseteq \RR^N\) is compact if and only if \(K\) is closed and bounded.
	\end{thm}
	\begin{proof}
		($\Leftarrow$) 방향만 보이면 된다. \(K \subseteq \RR^N\)가 closed and bounded라고 하자. \(K\)의 임의의 infinite subset \(A\)에 대하여, \(A\)도 bounded이므로 Bolzano-Weierstrass theorem에 의해 \(A\)는 limit point \(x\)를 가진다. 그런데 가정에 의해 \(K\)는 closed이므로, \(x \in A' \subseteq K' \subseteq K\)이다. 즉 \(K\)는 LPC이고 따라서 \(K\)는 compact이다.
	\end{proof}

Bolzano-Weierstrass theorem의 강력함을 다시 한 번 음미하기 바랍니다.

지금까지의 내용을 정리하면 다음과 같습니다.

	\begin{cor}
		\(K \subseteq \RR^N\)에 대하여 다음은 동치.
		\begin{enumerate}[label=(\alph*), leftmargin=2\parindent]
			\item
			\(K\) is closed and bounded.
			\item
			\(K\) is compact.
			\item
			Any infinite subset of \(K\) has a limit point in \(K\). i.e., \(K\) is LPC.
			\item
			Any sequence in \(K\) has a subseqnece converging to some point in \(K\). i.e., \(K\) is SC.
		\end{enumerate}	
	\end{cor}

\newpage

\section{Connectedness} \label{sec conn}

Section \ref{sec conn}\이 끝날 때까지 아무 말이 없으면 \(X\)는 metric space.

Compact set은 어떤 의미에서 finite set의 위상적인 일반화였고, 그것이 \(\RR\)에서는 유계닫힌집합과 동치였습니다. Connected set은 \(\RR\)의 구간을 일반화한 것입니다. `connected'라는 것이 무엇인가를 생각해 보면, 다음 정의에 도달합니다.

	\begin{defn}
		\(U, V \subset X\)가 nonempty open subsets of \(X\)이고 \(U \cap V = \emptyset, U \cup V = X\)일 때 \((U, V)\)를 \textbf{separation of \(X\)}라고 한다. \(X\)의 separation이 존재하지 않을 때 \(X\)를 \textbf{connected set(연결집합)}이라고 한다.
	\end{defn}

	\begin{ex}
		\(\RR \backslash \{0\} \subset \RR\)은 \((-\infty, 0)\cup(0, \infty)\)의 separation을 가지므로 connected set이 아닙니다.
	\end{ex}

Separation의 또 다른 정의는 다음 정리와 같은데, 이는 위의 정의와 동치입니다.

	\begin{thm} \label{conn def}
		\((U, V)\) is a separation of \(Y \subseteq X\) if and only if \(U, V\) are both nonempty, \(U \cup V = Y\) and \(U \cap \overline{V} = \emptyset, \overline{U} \cap V = \emptyset\). (여기서 \(\overline{U}\)는 closure of \(U\) in \(X\))
	\end{thm}
	\begin{proof}
		($\Rightarrow$): \((U, V)\)가 \(Y\)의 separation이면, \(U\)는 open subset of \(Y\)이면서 동시에 closed subset of \(Y\)이다(왜 그런가?) Closure of \(U\) in \(Y\)는 \(\overline{U} \cap Y\)인데 \(U\)가 closed in \(Y\)이므로 \(U = \overline{U} \cap Y\)이다. 즉 \(\overline{U} \cap V = \emptyset\). \(U \cap \overline{V} = \emptyset\)도 마찬가지로 보일 수 있다.\\
		($\Leftarrow$): 가정에 의해 \(U \cap V = \emptyset, \overline{U} \cap Y = U, \overline{V} \cap Y = V\)이다. 즉 \(U, V\)는 closed in \(Y\). 한편 \(U = Y \backslash V\)이므로, \(U, V\)는 open in \(Y\). 따라서 \((U, V)\)는 \(Y\)의 separation이다.
	\end{proof}

\(\RR\)의 구간과 connected set은 우리가 예상하는 대로 동치입니다.

	\begin{thm}
		\(I \subseteq \RR\)에 대하여 다음은 동치.
		\begin{enumerate}[label=(\alph*), leftmargin=2\parindent]
			\item
			\(I\)는 구간이다.
			\item
			\(a, b \in I, a < c < b \Rightarrow c \in I\)
			\item
			\(I\) is connected.
		\end{enumerate}	
	\end{thm}
	\begin{proof}
		(a)$\Leftrightarrow$(b)는 쉽게 알 수 있다.\\
		(b)$\Rightarrow$(c): (b)를 가정하고, 모순을 보이기 위해 \(I\)의 separation \(U, V\)가 존재한다고 하자. \(U, V\)는 nonempty이므로, 일반성을 잃지 않고, \(a \in U, b \in V, a < b\)를 택할 수 있다. \(c = \sup (U \cap [a, b])\)라고 정의하면 \(a \le c \le b\)이다. (b)에 의해 \(c \in I\)이므로 \(c \in U\) or \(c \in V\)이다.\\
		만약 \(c \in U\)이면, \(a \le c < b\)이다. \(U \cap [a, b)\)는 \([a, b)\)의 open set이므로, 양수 \(\delta_1 > 0\)이 존재하여 \(N_{\delta_1}(c) \cap [a, b) \subseteq U \cap [a, b)\)이다. 따라서 \(d_1 \in U \cap [a, b)\)인 \(d_1 \in (c, b)\)가 존재하는데, 이는 \(c = \sup (U \cap [a, b])\)에 모순이다.\\
		한편 \(c \in V\)이면, \(a < c \le b\)이다. 앞에서와 마찬가지로 양수 \(\delta_2 > 0\)가 존재하여 \(N_{\delta_2}(c) \cap (a, b] \subseteq V \cap (a, b]\)이다. \(\sup\)의 정의에 의해 \(d_2 \in U \cap [a, b]\)이면서 \(N_{\delta_2}(c) \cap (a, b]\)인 \(d_2\)가 존재한다. 그러면 \(d_2 \in U \cap V\)이므로, \(U \cap V\)인 것에 모순이다. 따라서 (b)가 참이면 \(I\)는 separation을 가지지 않는다.
		(c)$\Rightarrow$(b): \(a, b \in I, c \notin I\)인 \(a < c < b\)가 존재한다고 하면 \(U = I \cap (-\infty, c), V = I \cap (c, \infty)\)는 \(I\)의 separation이므로(왜 그런가?) \(I\) is not connected.
	\end{proof}
	
	다음은 connectedness의 성질 몇 가지.
	
	\begin{lem} \label{conn lem}
		\(U, V \subset X\)가 \(X\)의 separation이고, \(Y \subset X\)가 connected subset of \(X\)이면 \(Y \subseteq U\)이거나 \(Y \subseteq V\)이다.
	\end{lem}
	\begin{proof}
		\(U \cap Y, V \cap Y\)는 모두 \(Y\)의 open set이고, \((U \cap Y) \cap (V \cap Y) = Y\)이다. 따라서 \(Y\)가 connected이기 위해서는 \(U \cap Y = \emptyset\)이거나 \(V \cap Y = \emptyset\)이어야 한다.
	\end{proof}
	
	\begin{thm}
		\(\mathcal{C}\)가 collection of connected subsets of \(X\)라고 하자. \(\bigcap_{C \in \mathcal{C}} C\)가 nonempty이면 \(\bigcup_{C \in \mathcal{C}} C\)도 connected이다.
	\end{thm}
	\begin{proof}
		\(Y = \bigcup_{C \in \mathcal{C}} C\)의 separation \(U, V\)가 존재한다고 가정하자. 가정에 의해 \(p \in \bigcap_{C \in \mathcal{C}} C\)를 택할 수 있는데, \(p \in U\)이거나 \(p \in V\)이다. 일반성을 잃지 않고 \(p \in U\)라고 하자. \(\bigcap_{C \in \mathcal{C}} C\)는 connected이므로 Lemma \ref{conn lem}에 의해 \(\bigcap_{C \in \mathcal{C}} C \subseteq U\)이거나 \(\bigcap_{C \in \mathcal{C}} C \subseteq V\)인데, \(p \in U\)이므로 \(\bigcap_{C \in \mathcal{C}} C \subseteq U\)이다. 이는 \(V\)가 nonempty인 것에 모순이므로, \(Y\) is connected.
	\end{proof}
	
	\begin{thm}
		\(A \subseteq X\)가 connected이고 \(A \subseteq B \subseteq \overline{A}\)이면 \(B\)도 connected.
	\end{thm}
	\begin{proof}
		\(B\)의 separation \(U, V\)가 존재한다고 가정하자. Lemma \ref{conn lem}에 의해 \(A \subseteq U\)이거나 \(A \subseteq V\)이다. 일반성을 잃지 않고 \(A \subseteq U\)라고 하자. 그러면 \(\overline{A} \subseteq \overline{U} \subseteq \overline{B} \subseteq \overline{A}\)이므로 \(B \subseteq \overline{U} = \overline{A}\). 그런데 Theorem \ref{conn def}에 의해 \(\overline{U} \cap V = \emptyset\)이다. 이는 \(V\)가 nonempty인 것에 모순이다.
	\end{proof}

Connectedness는 이후 연속함수의 성질에서 한번 더 언급됩니다.

\newpage

\section{Continuity} \label{sec cont}

\subsection{Limit of Functions and Continuous Functions}

이제 드디어 연속함수를 시작합니다. 친숙한 개념이니까 좀 더 재밌지 않을까요?

역시 아무 말이 없으면 Section \ref{sec cont}에서 \(X, Y\)는 metric space.

\begin{defn}
	함수 \(f: E \subseteq X \rightarrow Y\)와 점 \(p \in E', q \in Y\)에 대하여
	\begin{gather*}
		\lim_{x \rightarrow p} f(x) = q
	\end{gather*}
	라는 것은 임의의 \(\eps > 0\)에 대하여 \(\delta > 0\)이 존재하여
	\begin{gather*}
		0 < d_X (x, p) < \delta \Longrightarrow d_Y (f(x), q) < \eps \quad (x \in E)
	\end{gather*}
	를 만족하는 것이다. 이때 \(q\)를 \textbf{limit of \(f\) as \(x \rightarrow p\)}라고 한다.
\end{defn}
\begin{rem}
	\(p\)가 꼭 \(E\) 안에 있지 않아도 됩니다.
\end{rem}

\begin{ex}
	\(\lim_{x \rightarrow 2} x^2 = 4\)임을 증명하려 합니다. 먼저 양수 \(\delta > 0\)에 대해 다음이 성립하는 것을 알 수 있습니다.
	\begin{gather*}
		0 < \abs{x-2} < \delta \Longrightarrow \abs{x^2 - 4} = \abs{x + 2}\abs{x - 2} < (4 + \delta)\delta
	\end{gather*}
	이제 양수 \(\eps > 0\)이 주어졌을 때, \((4 + \delta)\delta < \eps\)이도록 하는 \(\delta > 0\)을 찾으면 됩니다. 여러 가지 방법이 있는데, 생각나는 가장 쉬운 방법은 \(\delta = \min\{1, \frac{\eps}{5}\}\)로 잡는 것입니다. 그러면
	\begin{gather*}
		0 < \abs{x-2} < \delta \Longrightarrow \abs{x^2 - 4} < (4 + \delta)\delta \le 5 \times \eps / 5 = \eps
	\end{gather*}
	이므로 증명이 끝났습니다.
\end{ex}

함수의 극한을 sequence의 수렴으로도 설명할 수 있습니다.

\begin{thm} \label{thm lim seq}
	함수 \(f: E \subseteq X \rightarrow Y\)와 점 \(p \in E', q \in Y\)에 대하여 다음은 동치.
	\begin{enumerate}[label=(\alph*), leftmargin=2\parindent]
		\item
		\(\lim_{x \rightarrow p} f(x) = q\).
		\item
		Sequence \((p_n)\) in \(E\)가 \(p_n \neq p, \lim_n p_n = p\)를 만족하면 \(\lim_n f(p_n) = q\).
	\end{enumerate}
\end{thm}
\begin{proof}
	($\Rightarrow$): \(p_n \neq p, \lim_n p_n = p\)인 sequence \((p_n)\) in \(E\)를 잡는다. 임의의 \(\eps > 0\)에 대하여 \(\delta > 0\) s.t. [\(d_Y (f(x), q) < \eps\) if \(x \in E\) and \(0 < d_X (x, p) < \delta\)]인 \(\delta\)를 찾는다. 그리고 \(n \ge N\)이면 \(0 < d_X (p_n, p) < \delta\)인 \(N\)을 찾으면, \(n \ge N\)에 대하여 \(d_Y (f(p_n), q) < \eps\)이다. 따라서 \(\lim_n f(p_n) = q\).\\
	($\Leftarrow$): 대우를 보이기 위해 결론을 부정하자. 그러면 for some \(\eps > 0\), for any \(n \in \NN\) there exists some \(p_n \in E\) s.t. \(0 < d_X (p_n, p) < 1/n\) but \(d_Y (f(p_n), q) \ge \eps\). 그러면 \(p_n \neq p, p_n \rightarrow p\)인데 \((f(p_n))\)은 \(q\)로 수렴하지 않는다.
\end{proof}

\begin{cor}
	\(f\)의 \(p\)에서의 limit이 존재하면 유일하다.
\end{cor}
\begin{proof}
	Metric space에서 limit의 유일성에 의해\ldots
\end{proof}

다음을 증명하는 것은 연습문제로 남깁니다.

\begin{prop}
	\(f, g : E \subseteq X \rightarrow \RR\)과 \(p \in E'\)에 대하여 \(\lim_{x \rightarrow p} f(x) = \alpha, \lim_{x \rightarrow p} g(x) = \beta\)일 때 다음이 성립한다.
	\begin{enumerate}[label=(\alph*), leftmargin=2\parindent]
		\item
		\(\lim_{x \rightarrow p} f(x)+g(x) = \alpha + \beta\).
		\item
		\(\lim_{x \rightarrow p} cf(x) = c\alpha\) for \(c \in \RR\).
		\item
		\(\lim_{x \rightarrow p} f(x)g(x) = \alpha\beta\).
		\item
		\(\lim_{x \rightarrow p} f(x)/g(x) = \alpha/\beta\), if \(g(x) \neq 0\) for all \(x \in E\) and \(\beta \neq 0\).
	\end{enumerate}
\end{prop}

\begin{defn}
	함수 \(f: E \subseteq X \rightarrow Y\) \textbf{continuous at \(p \in E\)}라는 것은 임의의 \(\eps > 0\)에 대하여 \(\delta > 0\)이 존재하여
	\begin{gather*}
		d_X (x, p) < \delta \Longrightarrow d_U (f(x), f(p)) < \eps \quad (x \in E)
	\end{gather*}
	라는 것이다. \(f\)가 \(E\)의 모든 점에서 연속이면 \(f\)는 \textbf{continuous on \(E\)}이다.
\end{defn}

Limit의 정의와 비교해 보면 다음을 얻습니다(이것이 고등학교에서 연속의 정의입니다).

\begin{prop}
	\(p \in E \cap E'\)에 대하여 \(f\) is continuous at \(p\) if and only if \(\lim_{x \rightarrow p} f(x) = f(p)\).
\end{prop}

만약 \(p \in E \backslash E'\)이면, 즉 \(p\)가 isolated point of \(E\)이면 \(p\)에서의 극한은 정의되지 않지만, 정의에 의해 \(f\)는 \(p\)에서 항상 연속입니다. 이 사실을 염두에 두면, Theorem \ref{thm lim seq}로부터 continuity와 sequence의 수렴의 관계를 얻습니다.

\begin{thm} \label{conti seq}
	함수 \(f: E \subseteq X \rightarrow Y\)와 점 \(p \in E\)에 대하여 다음은 동치.
	\begin{enumerate}[label=(\alph*), leftmargin=2\parindent]
		\item
		\(f\) is continuous at \(p\).
		\item
		Sequence \((p_n)\) in \(E\)가 \(\lim_n p_n = p\)를 만족하면 \(\lim_n f(p_n) = f(p)\).
	\end{enumerate}
\end{thm}

\begin{prop}
	\(f, g : E \subseteq X \rightarrow \RR\)가 \(p \in E\)에서 연속이면 \(f+g, cf, fg\)도 \(p\)에서 연속이고(\(c \in \RR\)), \(g(x) \neq 0\) for all \(x \in E\)이면 \(f/g\)도 \(p\)에서 연속.
\end{prop}

연속함수의 개념을 metric을 사용하지 않고 open set만을 이용하여 더 일반적인 상황에서 정의할 수 있는데, metric space에서의 정의와 동치입니다.

\begin{defn}
	\(f: X \rightarrow Y\) (여기서 \(X, Y\)는 그냥 집합)에 대하여, \(f(A) = \{f(x) \in Y: x \in A\}\)를 \textbf{image(상) of \(A \subseteq X\) under \(f\)}라고 하고, \(f^{-1}(B) = \{x \in X: f(x) \in B\}\)를 \textbf{preimage(역상) of \(B \subseteq Y\) under \(f\)}라고 한다.
\end{defn}

\begin{thm} \label{conti}
	\(f: E \subseteq X \rightarrow Y\)에 대하여 다음은 동치.
	\begin{enumerate}[label=(\alph*), leftmargin=2\parindent]
		\item
		\(f\) is continuous on \(E\).
		\item
		\(B \subseteq Y\)가 open in \(Y\)이면 \(f^{-1}(B)\)도 open in \(E\).
		\item
		\(B \subseteq Y\)가 closed in \(Y\)이면 \(f^{-1}(B)\)도 closed in \(E\).
	\end{enumerate}
	즉 연속함수는 `open set의 preimage가 open set인 함수'이다.
\end{thm}
\begin{proof}
	(a)$\Rightarrow$(b): \(B \subseteq Y\)가 open이라고 하자. 임의의 \(x \in f^{-1}(B)\)에 대해, \(\eps > 0\)이 존재하여 \(N_\eps (f(x)) \subseteq B\)이다. \(f\)가 continuous at \(x\)이므로 \(\delta > 0\)가 존재하여 \(f(N_\delta (x_0) \cap E) \subseteq N_\eps (f(x))\)이다. \(A \subseteq f^{-1}(f(A))\)인 사실을 이용하면(왜 그런가?),
	\begin{gather*}
		N_\delta (x_0) \cap E \subseteq f^{-1}(f(N_\delta (x_0) \cap E)) \cap f^{-1}(N_\eps (f(x))) \subseteq f^{-1}(B)
	\end{gather*}
	이다. 따라서 \(f^{-1}(B)\) is open in \(E\).\\
	(b)$\Rightarrow$(a): \(x \in E\)라고 하고, 임의의 \(\eps > 0\)에 대해 \(N_\eps (f(x))\)는 open in \(Y\)이므로 \(f^{-1}(N_\eps (f(x)))\)도 open in \(E\)이다. 따라서 \(\delta > 0\)가 존재하여 \(N_\delta (x) \cap E \subseteq f^{-1}(N_\eps (f(x)))\)이고, \(f(N_\delta (x) \cap E) \subseteq f(f^{-1}(N_\eps (f(x)))) \subseteq N_\eps (f(x))\)이다. 따라서 \(f\) is continous at \(x\). \(x\)는 임의의 점이었으므로 \(f\) is continuous on \(E\).\\
	(b)$\Leftrightarrow$(c): 연습문제로 남김.
\end{proof}

\begin{cor}
	\(f: X \rightarrow Y, g: Y \rightarrow Z\)가 연속이면 \(g \circ f: X \rightarrow Z\)도 연속이다.
\end{cor}
\begin{proof}
	Open set \(U \subseteq Z\)에 대하여, \(g\)가 연속이므로 \(g^{-1}(U) \subseteq Y\)도 open. \(f\)가 연속이므로 \((g \circ f)^{-1}(U) = f^{-1}(g^{-1}(U)) \subseteq X\)도 open.
\end{proof}

\subsection{Continuity and Compactness} \label{sec cont cpt}

Section \ref{sec cont cpt}의 주요한 결과는 \textbf{최대최소정리}입니다.

\begin{thm} \label{conticpt}
	연속함수 \(f: X \rightarrow Y\)에 대하여 \(X\)가 compact이면 \(f(X)\)도 compact.
\end{thm}
\begin{proof}
	\(\{U_i\}_{i \in I}\)가 \(f(X)\)의 open cover라고 하자. Theorem \ref{conti}에 의해 각 \(f^{-1}(U_i)\)는 open in \(X\)이고, 따라서 \(\{f^{-1}(U_i)\}_{i \in I}\)는 open cover of \(X\). \(X\)가 compact이므로 finite subcover \(\{f^{-1}(U_i)\}_{i = i_1, \ldots, i_n}\)이 존재한다. 따라서
	\begin{gather*}
		f(X) = f(\bigcup_{k = 1}^n f^{-1}(U_{i_k})) = \bigcup_{k = 1}^n f(f^{-1}(U_{i_k})) \subseteq \bigcup_{k = 1}^n U_{i_k}
	\end{gather*}
	이므로, \(\{U_i\}_{i \in I}\)의 finite subcover를 찾았다.
\end{proof}

이제 최대최소정리는 당연한 따름정리입니다.

\begin{cor} [Maximum-Minimum Theorem: 최대최소정리]
	\(f: [a, b] \rightarrow \RR\)가 연속이면 \(f\)는 \([a, b]\)에서 최댓값과 최솟값을 가진다.
\end{cor}
\begin{proof}
	\([a, b] \subseteq \RR\)은 compact이므로, Theorem \ref{conticpt}에 의해 \(f([a, b]) \subseteq \RR\)도 compact이고 따라서 bounded and closed이다. \(\RR\)의 bounded and closed subset은 최댓값과 최솟값을 가지므로 증명 끝.
\end{proof}

뭔가 허무한 느낌입니다. 고등학교 때부터 증명을 미뤄온 최대최소정리의 증명 자체는 단 몇 줄로 끝났습니다. 하지만 여기에 이르기까지 무엇을 공부했는지\textemdash 실수의 완비성, 수열, open set, limit point, compactness, Heine-Borel \ldots \textemdash 를 생각해보면 실로 대단한 정리인 것입니다. 같은 느낌을 잠시 후에 사잇값정리를 증명할 때 받게 될 것입니다.

연속함수의 역함수는 연속함수 일까요? `Yes'라고 대답하고 싶은 충동이 들겠지만\ldots

\begin{ex}
	\(\{0\} \cup (1, 2]\)에서 \(f\)를 다음과 같이 정의합니다.
	\begin{gather*}
		f(x) = 
		\begin{cases}
			x - 1, & \text{if } 1 < x \le 2\\
			0, & \text{if } x = 0
		\end{cases}
	\end{gather*}
	그러면 \(f\)는 연속이고(\(0\)은 정의역의 isolated point이므로 항상 연속), \(f\)는 injective\footnote{injection = 단사함수 = 일대일함수. surjection = 전사함수 = 치역과 공역이 같은 함수. bijection = injection and surjection = 전단사함수 = 일대일대응.}이므로 그 치역에서 역함수를 정의할 수 있습니다. 그런데
	\begin{gather*}
		f^{-1}(x) = 
		\begin{cases}
			x + 1, & \text{if } 0 < x \le 1\\
			0, & \text{if } x = 0
		\end{cases}
	\end{gather*}
	이므로 \(f^{-1}\)은 \([0, 1]\)에서 연속이 아닙니다.
\end{ex}


정의역이 compact인 전단사 연속함수는 그 역함수도 연속함수입니다.

\begin{thm}
	Bijection \(f: X \rightarrow Y\)가 \(X\)에서 연속이고 \(X\)이 compact이면 \(f^{-1}: Y \rightarrow X\)도 연속이다.
\end{thm}
\begin{proof}
	Theorem \ref{conti}에 의하여, open set \(U \subseteq X\)에 대하여 \(f(U) \subseteq Y\)가 open인 것을 보이면 된다.\\
	\begin{align*}
		U \text{ is open in } X. &\iff X \backslash U \text{ is closed in } X.\\
		&\iff X \backslash U \text{ is compact.}\\
		&\iff f(X \backslash U)=Y \backslash f(U) \text{ is closed in } Y.\\
		&\iff f(U) \text{ is open in } Y.			
	\end{align*}
\end{proof}

\subsection{Continuity and Connectedness} \label{sec cont conn}

Section \ref{sec cont conn}의 주요한 결과는 \textbf{사잇값정리}입니다.

\begin{lem} \label{lem cont conn}
	\(E \subseteq X\)에 대하여 다음은 동치.
	\begin{enumerate}[label=(\alph*), leftmargin=2\parindent]
		\item
		\(S\) is not connected.
		\item
		There exists a surjective continuous function \(f: S \rightarrow \{0, 1\} \subset \RR\).
	\end{enumerate}
\end{lem}
\begin{proof}
	\(U, V\)가 \(S\)의 separation이라고 하자.
	\begin{gather*}
		f(x)=
		\begin{cases}
			0, &\text{if } x \in U\\
			1, &\text{if } x \in V
		\end{cases}
	\end{gather*}
	로 정의하면, \(f\)는 전사 연속함수. 역으로 그러한 \(f\)가 존재한다고 하면,
	\begin{gather*}
		U = f^{-1}(\{0\}), \quad V = f^{-1}(\{1\})
	\end{gather*}
	로 정의하면 \(U, V\)는 \(S\)의 separation이다.
\end{proof}

\begin{thm} \label{conti conn}
	연속함수 \(f: X \rightarrow Y\)에 대하여 \(X\)가 connected이면 \(f(X)\)도 connected.
\end{thm}
\begin{proof}
	\(f(X)\)가 connected가 아니라고 가정하면, Lemma \ref{lem cont conn}에 의해 전사 연속함수 \(g: f(X) \rightarrow \{0, 1\} \subset \RR\)이 존재한다. 따라서 \(g \circ f: X \rightarrow \{0, 1\}\)도 전사 연속함수이므로 \(X\)는 not connected.
\end{proof}

\begin{cor} [Intermediate Value Theorem: 사잇값정리]
	\(f: [a, b] \rightarrow \RR\)가 연속이고 \(f(a) \neq f(b)\)이면 \(f(a)\)와 \(f(b)\) 사이의 임의의 실수 \(k\)에 대하여 \(f(c) = k\)인 \(c \in (a, b)\)가 존재한다.
\end{cor}
\begin{proof}
	\(f([a, b]) \subseteq \RR\)이 connected이고, \(\RR\)의 conencted set은 구간이므로\ldots
\end{proof}

사잇값정리를 (한 줄로!) 증명했습니다.

Connectedness의 정의는 `separation이 존재하지 않는다'이므로, 정의만을 이용하여 어떤 집합이 connected임을 보이는 것은 어려울 때가 많습니다. Theorem \ref{conti conn}을 이용하면 어떤 집합 \(S\)에 대하여 다른 connected set(예: \(\RR\)의 구간에서 \(S\)로 가는 전사 연속함수만 찾으면 됩니다.

\begin{ex}
	\(\RR^N\)의 두 점 \(x, y\)에 대하여, \textbf{line segment(선분) \([x, y]\)}를
	\begin{gather*}
		[x, y] = \{(1-t)x + ty \in \RR^N: t \in [0, 1]\}
	\end{gather*}
	로 정의하면, \([0, 1]\)에서 정의된 전사 연속함수
	\begin{align*}
		f: [0, 1] &\rightarrow [x, y]\\
		t &\mapsto (1-t)x + ty
	\end{align*}
	가 존재하므로 \([x, y]\)는 connected.
\end{ex}

Theorem \ref{conti conn}의 중요한 결과 하나를 언급하고 Section 1.3을 마무리하겠습니다. 함수 \(f: X \rightarrow X\)에 대하여 \(f(x) = x\)인 \(x \in X\)를 \textbf{fixed point(고정점) of \(f\)}라고 합니다. Fixed point의 존재성은 수학의 각 분야에서 자주 언급되는데, \(X=[a, b] \subset \RR\)이고 \(f\)가 연속일 때 fixed point의 존재성을 주장할 수 있습니다.

\begin{cor}
	\(f:[a, b] \subset \RR \rightarrow [a, b] \subset \RR\)이 연속일 때, \(f\)는 fixed point를 가진다.
\end{cor}
\begin{proof}
	\(g(x) = x - f(x)\)로 정의하면,
	\begin{gather*}
		g(a) = a - f(a) \le 0 \le b - f(b) = g(b)
	\end{gather*}
	이므로 사잇값 정리에 의해 \(g(c) = c - f(c) = 0\)인 \(c \in [a, b]\)가 존재한다.
\end{proof}

\subsection{Uniform Continuity}

이제 고등학교나 미적분학에서는 전혀 볼 수 없었던 새로운 개념인 uniform continuity를 정의하려고 합니다.

\begin{defn}
	\textbf{Metric space}\footnote{그냥 연속과 달리 거리의 개념이 꼭 필요합니다.} \(X, Y\)에 대하여 함수 \(f: X \rightarrow Y\)가 \textbf{uniformly continuous(고른연속) on \(X\)}라는 것은, 임의의 \(\eps > 0\)에 대하여 양수 \(\delta > 0\)이 존재하여\\
	\begin{gather} \label{uniconti}
		d_X(x, y) < \delta \Longrightarrow d_Y (f(x), f(y)) < \eps \quad (x, y \in X)
	\end{gather}
	를 만족하는 것이다.
\end{defn}

\begin{rem}
	`어떤 함수가 continuous on \(X\)'의 정의는 \(f\)가 \textbf{각 \(x \in X\)에서 continuous}인 것입니다. 즉, continuity는 기본적으로 각 점의 근방에서의 local한 성질입니다. 그런데 uniform continuity는 정의역 전체를 control할 수 있는 \(\delta > 0\)을 요구하는, global한 성질입니다.\footnote{수학에서 uniform이라는 단어는 보통 어떤 집합 전체를 control할 수 있는 어떤 대상의 존재를 의미합니다.} 그래서 `\(f\)가 \(x \in X\)에서 uniform continuous하다'라는 말은 그 자체로 non-sense.
\end{rem}

\begin{prop}
	\(f: X \rightarrow Y\)가 uniformly continuous on \(X\)이면 continuous on \(X\).
\end{prop}
\begin{proof}
	\(\eps > 0\)에 대하여 (\ref{uniconti})을 만족하는 \(\delta > 0\)를 찾는다. 각 \(x \in X\)에 대하여,
	\begin{gather*}
		y \in X, d_X (x, y) < \delta \rightarrow d_Y (f(x), f(Y)) < \eps
	\end{gather*}
	이므로 \(f\)는 continuous at \(x\). \(x \in X\)는 임의의 점이므로, \(f\)는 continuous on \(X\).
\end{proof}

\begin{ex}
	\quad
	\begin{enumerate} [label=(\alph*), leftmargin=2\parindent]
		\item
		\(f:\RR \rightarrow \RR\)을 \(f(x) = x^2\)로 정의하면, \(f\) is not uniformly continuous on \(\RR\).\\
		\(f\)가 정의역에서 uniformly continuous라고 가정하면,
		\begin{gather*}
			x, y \in \RR, \abs{x - y} < \delta \Longrightarrow \abs{x^2 - y^2} < 1
		\end{gather*}
		인 \(\delta > 0\)이 존재합니다. 그런데
		\begin{gather*}
			x = \frac{1}{\delta}, \quad y = x + \frac{\delta}{2}
		\end{gather*}
		로 잡으면,
		\begin{gather*}
			\abs{x^2 - y^2} = \abs{x + y}\abs{x - y} \ge \frac{2}{\delta}\times \frac{\delta}{2} = 1
		\end{gather*}
		이므로 모순.
		\item
		그런데 (a)의 \(f\)의 정의역을 \([0, 1]\)로 제한시키면 \(f\) is uniformly continuous on \([0, 1]\).\\
		\(\eps > 0\)에 대하여 \(\delta = \eps / 2\)\footnote{분모의 2의 정체는 Corollary \ref{diff} 참고.}로 정의하면
		\begin{gather*}
			x, y \in \RR, \abs{x - y} < \delta \Longrightarrow \abs{x^2 - y^2} = \abs{x + y}\abs{x - y} \le 2\delta = \eps.
		\end{gather*}
		즉 uniform continuity는 정의역에 의존한다.
	\end{enumerate}
\end{ex}

\(f: X \rightarrow Y\)가 uniformly continuous on \(X\)일 충분조건 몇 개를 소개합니다.

\begin{prop} \label{Lip}
	\(f: X \rightarrow Y\)에 대하여
	\begin{gather*}
		d_Y (f(x), f(y)) \le M d_X (x, y) \quad (x, y \in X)
	\end{gather*}
	를 만족하는 \(M > 0\)이 존재하면 \(f\) is uniformly continuous on \(X\).
\end{prop}
\begin{proof}
	\(\eps > 0\)에 대하여 \(\delta = \eps / M\)으로 정의하면 (\ref{uniconti})을 만족한다.
\end{proof}

\begin{cor} \label{diff}
	\(f: \RR \rightarrow \RR\)이 미분가능하고,\footnote{미분을 아직 정의하지는 않았지만\ldots} \(\abs{f'} < M\)인 \(M > 0\)이 존재하면 \(f\) is uniformly continuous on \(\RR\). 
\end{cor}
\begin{proof}
	평균값 정리에 의해
	\begin{gather*}
		\abs{f(x) - f(y)} \le M \abs{x - y} \quad (x, y \in X)
	\end{gather*}
	이므로, Proposition \ref{Lip}에 의해.
\end{proof}

다음 정리는 compact set이 어떤 의미에서 finite set의 일반화라고 하는 것인지를 알 수 있게 해 줍니다.

\begin{thm}
	\(f: X \rightarrow Y\)가 연속함수이고 \(X\)가 compact이면 \(f\) is uniformly continuous on \(X\).
\end{thm}
\begin{proof}
	\(\eps > 0\)이 주어졌다고 하고, 각 \(x \in X\)에 대하여
	\begin{gather*} \label{pf2}
		d_X (x, y) < \delta (x) \Longrightarrow d_Y (f(x), f(y)) < \eps / 2, \quad (y \in X)
	\end{gather*}
	인 \(\delta(x) > 0\)을 찾는다(\(\delta(x)\)는 \(x\)에 의존한다는 뜻). 이제 \(\{N_{\delta(x) / 2} (x)\}_{x \in X}\)는 open cover of \(X\)이므로, finite subcover \(\{N_{\delta(x) / 2} (x)\}_{x = x_1, \ldots, x_n}\)이 존재한다. 이제 \(\delta = \min\{\delta(x_1) / 2, \ldots, \delta(x_n) / 2\}\)로 정의하고 (\ref{uniconti})이 성립함을 보이자.\\
	\(x, y \in X\)에 대하여 \(d_X (x, y) < \delta\)라고 가정하자. 먼저 \(\delta = \min\{\delta(x_1) / 2, \ldots, \delta(x_n) / 2\}\)가 open cover이므로 \(x \in N_{\delta(x_i) / 2} (x_i)\)인 \(i = 1, \ldots, n\)이 존재한다. 따라서
	\begin{gather*}
		d_X (y, x_i) \le d_X (y, x) + d_X (x, x_i) < \delta + \frac{\delta(x_i)}{2} \le \delta(x_i)
	\end{gather*}
	이므로 (\ref{pf2})에 의해
	\begin{gather*}
		d_Y (f(x), f(y)) \le d_Y (f(x), f(x_i)) + d_Y (f(y), f(x_i)) < \frac{\eps}{2} + \frac{\eps}{2} = \eps.
	\end{gather*}
\end{proof}

Uniform continuity의 좋은 점은 Cauchy sequence를 보존해준다는 것입니다. 일반적인 연속함수는 Cauchy sequence를 보존해준다고 보장할 수 없습니다.

\begin{ex}
	\(f: (0, 1] \rightarrow \RR\)을 \(f(x) = 1/x\)로 정의하고 \(p_n = 1/n\)이라 하면 \(p_n \rightarrow 0\)이므로 \((p_n)\)은 Cauchy이지만 \((f(p_n))\)은 발산하므로 not Cauchy.
\end{ex}

\begin{thm} \label{Cauchy}
	\(f: X \rightarrow Y\)가 uniformly continuous on \(X\)이고 \((p_n)\)이 Cauchy sequence in \(X\)이면 \((f(p_n))\)도 Cauchy sequence.
\end{thm}
\begin{proof}
	\(\eps > 0\)에 대하여 (\ref{uniconti})을 만족하는 \(\delta > 0\)을 잡고,
	\begin{gather*}
		m, n \ge N \Longrightarrow d_X (p_m, p_n) < \delta
	\end{gather*}
	인 \(N\)을 잡는다. 이제
	\begin{gather*}
		m, n \ge N \Longrightarrow d_X (f(p_m), f(p_n)) < \eps
	\end{gather*}
	이므로 \((f(p_n))\)은 Cauchy.
\end{proof}

연속함수 \(f: E \subseteq X \rightarrow Y\)에 대하여 \(g: X \rightarrow Y\)가 연속이고 \(g \vert_E = f\)이면 \(g\)는 \textbf{continuous extension of \(f\)}라고 합니다. 모든 연속함수가 continuous extension을 가지는 것은 아닙니다.

\begin{ex}
	\quad
	\begin{enumerate} [label=(\alph*), leftmargin=2\parindent]
		\item
		\((0, 1]\)에서 정의된 함수 \(x \mapsto 1/x\)은 \([0, 1]\)에서의 continuous extension이 존재하지 않습니다(\(0\)에서의 값을 어떤 것으로 해도 불연속이므로).
		\item
		\((0, 1]\)에서 정의된 함수 \(x \mapsto \sin(1/x)\)은 \([0, 1]\)에서의 continuous extension이 존재하지 않습니다(\(0\)에서의 값을 어떤 것으로 해도 불연속이므로).
	\end{enumerate}
\end{ex}

\begin{cor}
	Complete metric space \(Y\)에 대하여 \(f: E \subseteq X \rightarrow Y\)이 uniformly continuous on \(E\)이면 continuous extension \(g: \overline{E} \subseteq X \rightarrow Y\)가 존재한다.
\end{cor}
\begin{proof}
	\(E\)의 limit point에서의 \(g\)의 값을 지정해주면 된다. \(x_0 \in X\)가 \(E\)의 limit point이라고 하자. 그러면 \(x_0\)으로 수렴하는 sequence \((x_n)\) in \(E \backslash \{x_0\}\)이 존재한다. 그러면 \((x_n)\)은 Cauchy이고, \ref{Cauchy}에 의해 \((f(x_n))\)도 Cauchy이다. \(Y\)가 complete이므로 \(f(x_n) \rightarrow \alpha\) for some \(\alpha \in Y\)이다. 이제 \(g(x_0) = \alpha\)로 정의한다.\\
	이 정의의 well-definedness를 확인하기 위해\footnote{여러 가지 가능성 중에서 대표 하나를 이용해서 새로운 것을 정의했을 경우, 대표를 다른 것으로 뽑았을 때도 정의가 같은지를 확인해야 합니다.}, \((x'_n)\)이 \(x_0\)로 수렴하는 sequence in \(E \ \{x_0\}\)라고 하자. 마찬가지 이유로 \(f(x'_n) \rightarrow \beta\) for some \(\beta \in Y\)이다. 이제 \(\alpha = \beta\)인 것을 보이면 된다.
	\begin{gather*}
		\limsup_n d_X (x_n, x'_n) \le \limsup_n (d_X (x_n, x_0) + d_X (x'_n, x_0)) = 0
	\end{gather*}
	이므로 \(d_X (x_n, x'_n) \rightarrow 0\)이다. 이제 \(\eps > 0\)에 대하여 (\ref{uniconti})을 만족하는 \(\delta > 0\)을 찾고,
	\begin{gather*}
		n \ge N_1 \Longrightarrow d_X(x_n, x'_n) < \delta \Longrightarrow d_Y (f(x_n), f(x'_n)) < \eps
	\end{gather*}
	인 \(N_1\)을 찾는다. \(N_2, N_3\)을
	\begin{gather*}
		n \ge N_2 \Longrightarrow d_Y (f(x_n), \alpha) < \eps\\
		n \ge N_3 \Longrightarrow d_Y (f(x'_n), \beta) < \eps
	\end{gather*}
	을 만족하도록 잡고 \(N = \max\{N_1, N_2, N_3\}\)으로 하면
	\begin{gather*}
		d_Y (\alpha, \beta) \le d_Y (\alpha, f(x_N)) + d_Y (f(x_N), f(x'_N)) + d_Y (f'(x_N), \beta) < \eps + \eps + \eps = 3\eps
	\end{gather*}
	이다. \(\eps>0\)은 임의의 양수이므로 \(d_Y (\alpha, \beta) = 0\)이고 \(\alpha = \beta\)이다.\\
	이제 \(g: \overline{E} \subseteq X \rightarrow Y\)의 연속성을 보이는데, \(E\)의 limit point에 대해서만 보이면 된다(왜 그런가?). 위에서와 마찬가지로 \(x_0\)이 \(E\)의 limit point라고 하고, \(g(x_0) = \alpha\)라고 하자. 양수 \(\eps > 0\)에 대하여
	\begin{gather*}
		d_X(x, y) < \delta' \Longrightarrow d_Y (f(x), f(y)) < \frac{\eps}{3} \quad (x, y \in X)
	\end{gather*}
	을 만족하는 \(\delta' > 0\)을 찾는다. 우리의 주장: for \(x \in \overline{E}\), \(d_X (x, x_0) < \delta' / 2 \Longrightarrow d_Y (g(x), \alpha) < \eps\).\\
	\(x_0, x \in \overline{E}\)이므로 \(x_0, x\)으로 수렴하는 \(E\) 안의 수열 \((p_n), (q_n)\)이 존재한다. 또 정의에 의해 \(f(p_n) \rightarrow \alpha\)이다. 만약 \(x \in E'\)이면 정의에 의해 \(f(q_n) \rightarrow g(x)\)이고, \(x \in E \backslash E'\)이면 \(x\)는 isolated point이므로 \(q_n = x\) for sufficiently largle \(n\)이다. 따라서 어느 경우든 \(f(q_n) \rightarrow g(x)\)이다.
	\begin{align*}
		n \ge M_1 &\Longrightarrow d_X (p_n, x_0) < \frac{\delta'}{2}\\
		n \ge M_2 &\Longrightarrow d_Y (f(p_n), \alpha) < \frac{\eps}{3}\\
		n \ge M_3 &\Longrightarrow d_X (q_n, x) < \frac{\delta'}{2} - d_X (x, x_0)\\
		n \ge M_4 &\Longrightarrow d_Y (f(q_n), g(x)) < \frac{\eps}{3}
	\end{align*}
	인 \(M_1, M_2, M_3, M_4\)를 찾을 수 있다. 이중 최댓값을 \(M\)이라 하면,
	\begin{align*}
		d_Y (g(x), \alpha) &\le d_Y (g(x), f(q_M)) + d_Y (f(q_M), f(p_M)) + d_Y (f(p_M), \alpha)\\
		&< \frac{\eps}{3} + \frac{\eps}{3} + \frac{\eps}{3} = \eps.
	\end{align*}
\end{proof}

\subsection{Monotonic Functions} \label{sec mono}

Section \ref{sec mono}\이 끝날 때까지 \(X = Y = \RR\)인 경우만 생각합니다.

\begin{defn}
	\(f: (a, b) \rightarrow \RR\)와 \(a \le x_0 < b, \alpha \in \RR \cup \{\pm \infty\}\)에 대하여
	\begin{gather*}
		\lim_{x \rightarrow x_0+} f(x) = \lim_{x \searrow x_0} f(x) = f(x_0+) = \alpha
	\end{gather*}
	라는 것은 \(x_0\)로 수렴하는 \((x_0, b)\) 안의 임의의 수열 \((x_n)\)에 대하여 \(f(x_n) \rightarrow \alpha\)라는 것이다. 이때 \(q\)를 \textbf{right-handed limit(우극한) of \(f\) as \(x \rightarrow x_0+\)}라고 한다.\\
	\(a < x_0 \le b\)에 대하여 \textbf{left-handed limit(좌극한)} \(\lim_{x \rightarrow x_0-} f(x) = \lim_{x \nearrow x_0} f(x) = f(x_0-) = q\)도 마찬가지로 정의한다.
\end{defn}

\begin{ex} 좌극한이나 우극한이 존재하지 않는 함수의 예시입니다. \([0, 1]\)에서
	\begin{gather*}
		f(x)=
		\begin{cases}
			1, &\text{if } x \in \mathbb{Q}\\
			0, &\text{otherwise}
		\end{cases}
	\end{gather*}
	로 정의하면 \(f\)는 \(x_0 \in [0, 1]\)에서 좌극한이나 우극한을 가지지 않습니다.
\end{ex}

다음 사실은 쉽게 보일 수 있습니다(증명 생략).

\begin{prop}
	\(f: (a, b) \rightarrow \RR\)와 \(a \le x_0 < b, \alpha \in \RR\)에 대하여 다음은 동치.
	\begin{enumerate} [label=(\alph*), leftmargin=2\parindent]
		\item
		\(f(x_0+) = \alpha\).
		\item
		임의의 \(\eps > 0\)에 대하여 \(\delta > 0\)이 존재하여
		\begin{gather*}
			x_0 < x < x_0 + \delta \Longrightarrow \abs{f(x) - \alpha} < \eps \quad (x \in (a, b))
		\end{gather*}
		을 만족한다.
	\end{enumerate}
	좌극한에 대해서도 비슷한 말을 할 수 있다.
\end{prop}

\begin{cor}
	\(f: (a, b) \rightarrow \RR\)와 \(a < x_0 < b, \alpha \in \RR\)에 대하여
	\begin{gather*}
		f(x_0+) = f(x_0-) = \alpha \iff \lim_{x \rightarrow x_0} f(x) = \alpha.
	\end{gather*}
\end{cor}

\begin{defn}
	\(f: (a, b) \rightarrow \RR\)이 \textbf{monotonically increasing(단조증가) on \((a, b)\)}라는 것은 \(a < x < y < b \Longrightarrow f(x) \le f(y)\)라는 것이다. 마지막 부등호의 방향을 바꾸어 \textbf{monotonically decreasing(단조감소) on \((a, b)\)}도 정의한다. 정의역에서 montonically increasing이거나 monotonic decreasing인 함수를 \textbf{monotonic function(단조함수)}라고 한다.
\end{defn}

단조함수는 좌극한과 우극한을 항상 가집니다.

\begin{thm}
	\(f: (a, b) \rightarrow \RR\)이 \((a, b)\)에서 단조증가하면 다음이 성립한다.
	\begin{gather*}
		f(x+) = \inf\{f(t): t \in (x, b)\}, \quad a \le x < b\\
		f(x-) = \sup\{f(t): t \in (a, x)\}, \quad a < x \le b
	\end{gather*}
\end{thm}
\begin{proof}
	\(a \le x < b\)에 대하여 \(\alpha = \inf\{f(t): t \in (x, b)\}\)라고 하자.
	\begin{enumerate} [label=(\alph*), leftmargin=2\parindent]
		\item
		\(a < x < b\)인 경우: 이 경우에는 \(\alpha > -\infty\)이므로, 임의의 \(\eps > 0\)에 대해 \(f(t_0) < \alpha + \eps\)인 \(t_0 \in (x, b)\)가 존재한다. 이제 \(x < t < t_0 \Longrightarrow \alpha \le f(t) \le f(t_0) < \alpha + \eps\)이므로 \(f(x+) = \alpha\)이다. 
		\item
		\(x = a\)인 경우: 만약 \(\alpha > -\infty\)이면, (a)와 마찬가지로 생각할 수 있다. 만약 \(\alpha = -\infty\)이면 임의의 \(M < 0\)에 대해 \(f(t_0) < M\)인 \(t_0 \in (a, b)\)가 존재한다. 이제 \(a < x < t_0 \Longrightarrow f(x) \le f(t_0) < M\)이므로 \(f(a+) = -\infty\)이다.
	\end{enumerate}
	\(f(x-)\)의 경우도 마찬가지로 증명한다.
\end{proof}

\begin{cor} \label{limit}
	\(f: (a, b) \rightarrow \RR\)이 \((a, b)\)에서 단조증가하면 \(a < x < y < b\)에 대해 다음이 성립한다. 
	\begin{gather*}
		f(a+) \le f(x-) \le f(x) \le f(x+) \le f(y-) \le f(y) \le f(y+) \le f(b-)
	\end{gather*}
\end{cor}

위 정리를 이용하면 단조함수의 불연속점의 개수를 한정할 수 있습니다. \([0, 1]\)에서 정의된 함수
\begin{gather*}
	f(x)=
	\begin{cases}
		1, &\text{if } x \in \mathbb{Q}\\
		0, &\text{otherwise}
	\end{cases}
\end{gather*}
는 정의역의 모든 점에서 불연속이므로 불연속점의 집합이 uncountable입니다. 그런데 단조함수의 경우 불연속점의 집합은 countable임을 다음 따름정리를 통해 알 수 있습니다.

\begin{cor} \label{cor mono disconti}
	단조함수 \(f: (a, b) \rightarrow \RR\)의 불연속점의 집합은 at most countable.
\end{cor}
\begin{proof}
	\(f\)가 단조증가함수라고 하자. \(x \in (a, b)\)에 대하여 \(f\) is continuous at \(x\) if and only if \(f(x-) = f(x+)\)이므로, \(f\)의 불연속점의 집합을 \(D\)라고 하면 \(D = \{x \in (a, b): f(x-) < f(x+)\}\)이다. 이제 각 \(x \in D\)에 대하여 구간 \((f(x-), f(x+)) \cap \mathbb{Q}\)의 원소 \(r_x\)을 택할 수 있다. 따라서 \(x \mapsto r_x\)로 정의된 함수 \(r:D \rightarrow \mathbb{Q}\)는 injective인데(왜 그런가?) \(\mathbb{Q}\)가 countable이므로 \(D\)는 at most countable.
\end{proof}

단조증가의 정의에서, \(x < y \Longrightarrow f(x) \le f(y)\)를 \(x < y \Longrightarrow f(x) < f(y)\)로 바꾸면 \textbf{(strictly) increasing function(증가함수)}의 정의가 됩니다. \textbf{(Strictly) decreasing function(감소함수)}도 마찬가지로 정의합니다.

앞에서 정의역이 compact인 continuous bijection은 그 역함수도 연속임을 증명했는데, 정의역이 구간인 증가 연속함수 또는 감소 연속함수도 역함수가 연속입니다.

\begin{thm}
	구간 \(I\)에서 정의된 \(f: I \rightarrow \RR\)에 대하여 다음은 동치.
	\begin{enumerate} [label=(\alph*), leftmargin=2\parindent]
		\item
		\(f\) is injective.
		\item
		\(f\) is strictly increasing on \(I\) or strictly decreasing on \(I\).
	\end{enumerate}
	위 조건들이 참일 때 \(f^{-1}: f(I) \rightarrow I\)는 연속이다.
\end{thm}
\begin{proof}
	(b)$\Rightarrow$(a)는 당연하다. (b)를 부정하면,
	\begin{gather*}
		x < y, \quad f(x) > f(y), \quad z < w, \quad f(z) < f(w)
	\end{gather*}
	인 \(x, y, z, w \in I\)가 존재한다. \(f(y) < f(z), f(y) = f(z), f(y) > f(z)\) 세 가지 경우 모두를 검토해 보면 어느 경우에도 사잇값 정리에 의해 \(f\)가 injective가 될 수 없다.\\
	이제 \(f\)가 증가함수라고 하고, \(f^{-1}: f(I) \rightarrow I\)가 연속임을 보인다. \(y_0 \in f(I)\)에 대하여 \(x_0 = f^{-1}(y_0)\)라고 하자. 임의의 \(\eps > 0\)에 대해 \(\eps' = \min\{\eps, x_0-a, b-x_0\}\)로 놓고,
	\begin{gather*}
		\delta = \min \{f(x_0 + \eps') - y_0, y_0 - f(x_0 - \eps)\} > 0
	\end{gather*}
	라고 하면
	\begin{align*}
		y_0 - \delta < y < y_0 + \delta &\Longrightarrow f(x_0 - \eps') \le y_0 - \delta < y < y_0 - \delta \le f(x_0 + \eps)\\
		&\Longrightarrow x_0 - \eps' < f^{-1}(y) < x_0 + \eps'\\
		&\Longrightarrow x_0 - \eps < f^{-1}(y) < x_0 + \eps
	\end{align*}
	이므로 \(f^{-1}\)은 \(y_0\)에서 연속이다.
\end{proof}

\newpage

\section{Differentiation}

\subsection{Derivative}

실수에서 정의된 실함수의 미분가능성의 정의는 고등학교 때와 (일단은) 동일합니다.

\begin{defn}
	함수 \(f: [a, b] \rightarrow \RR\)와 \(x \in [a, b]\)에 대하여, 
	\begin{gather*}
		\phi(t) = \frac{f(t) - f(x)}{t - x} \quad (t \in [a, b] \backslash \{x\})
	\end{gather*}
	로 정의하자. 이때 \(\lim_{t \rightarrow x} \phi(t)\)가 존재하면
	\begin{gather*}
		f'(x) = \lim_{t \rightarrow x} \phi(t)
	\end{gather*}
	로 정의하고, \(f\)는 \textbf{differentiable(미분가능) at \(x\)}라고 한다. 만약 \(f\)가 정의역 \([a, b]\)의 모든 점에서 미분가능하면 \(f\)는 \textbf{differentiable on \([a, b]\)}라고 한다. 이때 \([a, b]\)에서 정의된 함수 \(f': x \mapsto f'(x)\)를 \textbf{derivative(도함수) of \(f\)}라고 한다.
\end{defn}

미분가능하면 연속이다, 덧셈, 상수배, 곱셈, 나눗셈 등에 관한 성질들 모두 고등학교 때의 증명과 똑같이 증명할 수 있으니 생략하겠습니다.

\(\RR\)에서 정의된 함수에 대해서는 위의 정의로 충분하지만, 다변수함수의 경우에 \(\phi\)를 정의할 마땅한 방법이 없습니다. 그래서 위의 정의를 포함하는 새로운 정의를 만드려고 하는데, 어떤 점 \(x_0\)에서 \(f'(x_0)\)의 의미는 함수 \(x \mapsto f(x) - f'(x_0)\)를 \(x_0\) 근처에서 가장 잘 근사하는 선형 함수\footnote{함수 \(g\)가 선형이라는 것은 \(g(x+y) = g(x) + g(y), g(cx) = cg(x)\)라는 것.}임에 주목합니다. \textbf{가장 잘 근사}한다는 것을 수학적으로 엄밀하게 쓰면 다음과 같습니다.

\begin{thm}
	\(\RR\)의 open set \(U\)에서 정의된 함수 \(f: U \rightarrow \RR\)과 \(x \in (U, L \in \RR\)에 대하여 다음은 동치.
	\begin{enumerate}[label=(\alph*), leftmargin=2\parindent]
		\item
		\(f\)는 \(x\)에서 미분가능하고 \(f'(x) = L\).
		\item
		적당한 \(\eps > 0\)에 대하여 함수 \(\eta: (-\eps, \eps) \rightarrow \RR\)가 존재하여
		\begin{gather} \label{diff}
			f(x+h) = f(x) + Lh + \eta(h) \quad (h \in (-\eps, \eps)), \quad \quad \lim_{h \rightarrow 0} \frac{\eta(h)}{\abs{h}} = 0
		\end{gather}
		을 만족한다.
	\end{enumerate}
\end{thm}
\begin{proof}
	(a)를 가정하고 \(\eta(h) = f(x + h) - f(x) - Lh\)로 정의하면 (b)를 만족하는 것은 당연하다. 이제 (b)를 가정하면,
	\begin{align*}
		f'(x) = \lim_{t \rightarrow x} \frac{f(t) - f(x)}{t - x} &= \lim_{h \rightarrow 0} \frac{f(x + h) - f(x)}{h}\\
		&= \lim_{h \rightarrow 0} \frac{Lh + \eta(h)}{h} = L + \lim_{h \rightarrow 0} \frac{\eta(h)}{h} = L.
	\end{align*}
\end{proof}

해석개론 수준에서 (b)의 일반적인 정의를 쓸 일은 거의 없는데, 다음을 증명할 때는 유용합니다(아마 마지막?).

\begin{thm} [Chain Rule]
	\(\RR\)의 open set \(U, V\)에서 정의된 두 함수 \(f: U \rightarrow \RR, g: V \subseteq \RR \rightarrow \RR\) (단, \(f(U) \subseteq V\))에 대하여 \(f\)가 \(x \in U\)에서 미분가능하고 \(g\)가 \(y = f(x) \in V\)에서 미분가능하면 함수 \(g \circ f: U \subseteq \RR \rightarrow \RR\)도 \(x\)에서 미분가능하고 \((g \circ f)'(x) = g'(y)f'(x)\).
\end{thm}
\begin{proof} [Wrong Proof]
	\begin{gather*}
		(g \circ f)'(x) = \lim_{t \rightarrow x} \frac{g(f(t)) - g(f(x))}{t - x} = \lim_{t \rightarrow x} \frac{g(f(t)) - g(f(x))}{f(t) - f(x)}\frac{f(t) - f(x)}{t - x} = g'(f(x))f'(x).
	\end{gather*}
	\(f(t) - f(x)\)가 무한히 0이 되면\ldots?
\end{proof}
\begin{proof}
	\(f'(x) = M, g'(y) = L\)이라 하고 \(f, g\)에 해당하는 (\ref{diff})의 함수를 각각 \(\eta, \zeta\)라 하자. 그러면
	\begin{align*}
		(g \circ f)(x + h) &= (g \circ f)(x) + L(f(x+h) - f(x)) + \zeta(f(x+h)-f(x))\\
		&= (g \circ f)(x) + L(Mh + \eta(h)) + \zeta(f(x+h)-f(x))\\
		&= (g \circ f)(x) + LMh + L\eta(h) + \zeta(f(x+h)-f(x))
	\end{align*}
	이제 \(LMh\) 뒤의 항을 \(\abs{h}\)로 나눈 것이 0으로 가는 것만 보이면 된다. 표기의 간단함을 위해 \(\tilde{\zeta}(h) = \zeta(h) / \abs{h}\) for \(h \neq 0\), \(\tilde{\zeta}(0) = 0\)으로 쓰면 \(\lim_{h \rightarrow 0}\tilde{\zeta}(h) = 0\)이므로
	\begin{align*}
		\lim_{h \rightarrow 0} \frac{\abs{L\eta(h) + \zeta(f(x+h)-f(x))}}{\abs{h}} &\le \lim_{h \rightarrow 0} \left (\abs{L}\frac{\abs{\eta(h)}}{\abs{h}} + \frac{\abs{f(x+h) - f(x)}}{\abs{h}}\abs{\tilde{\zeta}(f(x+h)-f(x))} \right )\\
		&\le \lim_{h \rightarrow 0} \left(0 + \abs{M}(0) \right) = 0.
	\end{align*}
\end{proof}

\subsection{Mean Value Theorem} \label{sec mvt}

고등학교에서 배운 평균값 정리를 복습하고 L'H\^opital의 법칙을 증명합니다.



\begin{prop} [Rolle's Theorem]
	\(f: [a, b] \rightarrow \RR\)이 \([a, b]\)에서 연속이고 \((a, b)\)에서 미분가능할 때, \(f(a) = f(b)\)이면 \(f'(c) = 0\)인 \(c \in (a, b)\)가 존재한다.
\end{prop}
\begin{proof}
	\(f\)가 상수함수이면 자명하므로 \(f\)가 상수함수가 아니라고 가정하자. 최대최소 정리에 의해 \(f\)는 \([a, b]\)에서 최댓값과 최솟값을 가진다. 만약 최댓값과 최솟값이 모두 \(f(a) = f(b)\)의 값과 같으면 \(f\)는 상수함수이므로, 가정에 모순이다. 따라서 최댓값과 최솟값 중 적어도 하나는 \(f(a) = f(b)\)의 값과 다르다. 일반성을 잃지 않고 최댓값이 \(f(a) = f(b)\)보다 크다고 하고, \(x = c \in (a, b)\)에서 \(f\)가 최댓값을 가진다고 하자. 그러면 \([a, b] \backslash \{c\}\)에서 정의된 함수
	\begin{gather*}
		t \mapsto \frac{f(t) - f(c)}{t - c}
	\end{gather*}
	는 \(t > c\)일 때 0보다 작거나 같고 \(t < c\)일 때 0보다 크거나 같다. \(f\)는 \(c\)에서 미분가능하므로 \(f'(c) = 0\)이다.
\end{proof}

\begin{rem}
	\([a, b]\)에서 연속이고 \((a, b)\)에서 미분가능하다는 것이 \([a, b]\)에서 미분가능함을 함축하지 않습니다(역은 당연히 참). \([0, 1]\)에서 정의된 함수
	\begin{gather*}
		f(x)=
		\begin{cases}
			0 &\text{if } x = 0\\
			x\sin\frac{1}{x} &\text{if } 0 < x \le 1
		\end{cases}
	\end{gather*}
	는 \([0, 1]\)에서 연속이고 \((0, 1)\)에서 미분가능하지만 \(f'(0)\)의 값은 존재하지 않습니다.
\end{rem}

\begin{thm} [Cauchy's Mean Value Theorem]
	\(f, g: [a, b] \rightarrow \RR\)가 \([a, b]\)에서 연속이고 \((a, b)\)에서 미분가능하면
	\begin{gather*}
		f'(c)(g(b) - g(a)) = g'(c)(f(b) - f(a))
	\end{gather*}
	를 만족하는 \(c \in (a, b)\)가 존재한다.
\end{thm}
\begin{proof}
	\([a, b]\)에서 \(h(x) = (f(b) - f(a))g(x) - (g(b) - g(a))f(x)\)로 정의하고 Rolle의 정리를 적용하면 끝.
\end{proof}

\begin{cor} [Mean Value Theorem: 평균값정리]
	\(f: [a, b] \rightarrow \RR\)가 \([a, b]\)에서 연속이고 \((a, b)\)에서 미분가능하면 \(f(b) - f(a) = f'(c)(b - a)\)를 만족하는 \(c \in (a, b)\)가 존재한다.
\end{cor}
\begin{proof}
	Cauchy의 평균값정리에서 \(g(x) = x\)로 두면 끝.
\end{proof}

드디어 우리의 오랜 숙원(!)인 L'H\^opital의 법칙을 증명할 수 있습니다.

\begin{thm} [L'H\^opital's Rule] \label{lopthm}
	적당한 양수 \(r > 0\)에 대하여 \((x_0 - r, x_0 + r) \backslash \{x_0\}\)에서 정의된 실함수 \(f, g\)가 다음 조건을 만족한다고 하자.
	\begin{enumerate}[label=(\alph*), leftmargin=2\parindent]
		\item
		\(\lim_{x \rightarrow x_0} f(x) = \lim_{x \rightarrow x_0} g(x) = 0\)
		\item
		\(f, g\)는 \((x_0 - r, x_0 + r) \backslash \{x_0\}\)에서 미분가능
		\item
		\(g'(x) \neq 0\) for \(x \in (x_0 - r, x_0 + r) \backslash \{x_0\}\)
		\item
		\(\lim_{x \rightarrow x_0} f'(x) / g'(x) = L \in \RR \cup \{\pm \infty\}\)
	\end{enumerate}
	그러면 \(\lim_{x \rightarrow x_0} f(x) / g(x) = L\)이다.
\end{thm}

\begin{proof}
	\(f(x_0) = g(x_0) = 0\)으로 정의하면 \(f, g\)는 \((x_0 - r, x_0 + r)\)에서 연속이다.\\
	먼저 \(L \in \RR\)인 경우를 보이기 위해, 임의의 양수 \(\eps > 0\)에 대하여
	\begin{gather} \label{lhopital}
		0 < \abs{x - x_0} < \delta \Longrightarrow \abs{\frac{f'(x)}{g'(x)} - L} < \eps
	\end{gather}
	를 만족하는 \(\delta > 0\)을 잡는다. 우리의 주장:
	\begin{gather*}
		0 < \abs{x - x_0} < \delta \Longrightarrow \abs{\frac{f(x)}{g(x)} - L} < \eps
	\end{gather*}
	가 성립한다. \(x \in (x_0 - \delta, x_0 + \delta)\backslash\{x_0\}\)를 고정하고 구간 \([x_0, x]\)나 \([x, x_0]\)에서 Cauchy의 평균값 정리를 적용하면
	\begin{gather*}
		\frac{f(x)}{g(x)} = \frac{f(x) - f(x_0)}{g(x) - g(x_0)} = \frac{f'(c_x)}{g'(c_x)}
	\end{gather*}
	을 만족하는 \(c_x \in (x_0, x)\) 또는 \(c_x \in (x, x_0)\)를 찾을 수 있다. 그러면 \(0 < \abs{c_x - x_0} < \delta\)이므로 (\ref{lhopital})에 의해
	\begin{gather*}
		\abs{\frac{f(x)}{g(x)} - L} = \abs{\frac{f'(c_x)}{g'(c_x)} - L} < \eps
	\end{gather*}
	이다. \(x \in (x_0 - \delta, x_0 + \delta)\backslash\{x_0\}\)은 임의의 점이었으므로 주장이 증명되었다.\\
	다음으로 \(L = \infty\)인 경우를 보이기 위해(\(-\infty\)인 경우는 마찬가지로 증명), 임의의 양수 \(M > 0\)에 대하여
	\begin{gather*}
		0 < \abs{x - x_0} < \delta \Longrightarrow \frac{f'(x)}{g'(x)} > M
	\end{gather*}
	를 만족하는 \(\delta > 0\)을 잡는다. 우리의 주장:
	\begin{gather*}
		0 < \abs{x - x_0} < \delta \Longrightarrow \frac{f(x)}{g(x)} > M
	\end{gather*}
	이 성립한다. 이 이후의 증명은 \(L \in \RR\)인 경우와 거의 같으므로 생략한다.
\end{proof}

\begin{cor} \label{cor lpt}
	Theorem \ref{lopthm}의 \(x_0\)를 \(\pm \infty\)로 바꾸거나 (a)를 \(\lim_{x \rightarrow x_0} \abs{f(x)} = \lim_{x \rightarrow x_0} \abs{g(x)} = \infty\)로 바꾸어도 된다.
\end{cor}
\begin{proof}
	\(f(1/x), g(1/x)\)이나 \(1/f(x), 1/g(x)\)에 Theorem \ref{lopthm}을 적용.
\end{proof}

\begin{rem}
	L'H\^opital의 법칙을 적용하는 사례는 고등학교 때부터 너무 많이 봤을 거니까 적용이 안 되는 사례 몇 개를 봅시다.\footnote{사실 적용이 안 되는게 아니라 애초에 정리의 조건을 만족하지 못하는 사례입니다.}
	\begin{enumerate} [label=(\alph*), leftmargin=2\parindent]
		\item
		(c)에 위배되는 경우입니다. \(f(x) = x + \sin x \cos x, g(x) = f(x) e^{\sin x}\)로 정의하면, [\(g'(x) \neq 0\) for some \(x > r\)]인 \(r\)을 찾을 수 없습니다. 그래서 애초에 \(\lim_{x \rightarrow \infty} \frac{f'(x)}{g'(x)}\)라는 식이 정의되지 않습니다. 그럼에도 불구하고 억지로 계산을 하면 
		\begin{align*}
			\lim_{x \rightarrow \infty} \frac{f'(x)}{g'(x)} &= \lim_{x \rightarrow \infty} \frac{2\cos^2 x}{((2 \cos^2 x) + (x + \sin x \cos x)\cos x) e^{\sin x}}\\
			&= \lim_{x \rightarrow \infty} \frac{2\cos x}{2 \cos x + x + \sin x \cos x}e^{-\sin x} = 0
		\end{align*}
		인데 \(\lim_{x \rightarrow \infty} f(x)/g(x) = \lim_{x \rightarrow \infty} e^{-\sin x}\)로 극한이 존재하지 않습니다.
		\item
		(d)에 위배되는 경우입니다.
		\begin{gather*}
			f(x) =
			\begin{cases}
				0 &\text{if } x = 0\\
				x^2 \sin \frac{1}{x} &\text{if } x \neq 0
			\end{cases}
			, \quad g(x) = x
		\end{gather*}
		로 정의하면 \(\lim_{x \rightarrow 0} f'(x)/g'(x)\)가 \(\RR \cup \{\pm \infty\}\)로 존재하지 않지만 \(\lim_{x \rightarrow 0} f(x)/g(x) = 0\)입니다. 즉 L'H\^opital의 법칙의 역은 성립하지 않습니다.
	\end{enumerate}
\end{rem}

다음도 고등학교 수학에서 던졌던 의문을 회수하는 정리입니다.

\begin{thm}
	\(f: [a, b] \rightarrow \RR\)이 미분가능하고 \(f'(a) < \lambda < f'(b)\)이면 \(f'(c) = \lambda\)인 \(c \in (a, b)\)가 존재한다.
\end{thm}
\begin{proof}
	\(g(x) = f(x) - \lambda x\)로 정의하면 \(g\)는 \([a, b]\)에서 최솟값을 가진다. 그런데 \(g'(a) < 0, g'(b) > 0\)이므로 \(g(a), g(b)\)는 \(g\)의 최솟값이 아니다. 따라서 \(g\)가 \([a, b]\)에서 최솟값을 가지는 점은 \(c \in (a, b)\)이다. \(g'(c) = 0\)이므로 \(f'(c) = \lambda\).
\end{proof}

\begin{cor}
	\(f: (a, b) \rightarrow \RR\)이 미분가능할 때, 다음을 만족하는 \(c \in (a, b)\)는 존재하지 않는다.
	\begin{gather*}
		f'(c-) = \alpha \in \RR, \quad f'(c+) = \beta \in \RR, \quad \alpha \neq \beta
	\end{gather*}
\end{cor}

\subsection{Taylor's Theorem}

테일러 정리를 엄밀하게 증명합니다.

\begin{defn}
	\(f: E \subseteq \RR \rightarrow \RR\)에 대하여
	\begin{enumerate} [label=(\alph*), leftmargin=2\parindent]
		\item
		\(f\)가 \(C^0\)-함수(\(f\) is of \(C^0\))라는 것은 \(f\)가 \(I\)에서 연속이라는 것이다.
		\item
		\(k = 0, 1, \ldots\)에 대하여 \(f\)가 \(C^{k+1}\)-함수(\(f\) is of \(C^{k+1}\))라는 것은 \(f\)가 \(E\)에서 미분가능하고 \(f'\)가 \(C^k\)-함수라는 것이다.
		\item
		\(f\)가 \(C^\infty\)-함수(\(f\) is of \(C^\infty\)) 또는 smooth라는 것은 모든 \(k = 0, 1, \ldots\)에 대하여 \(f\)가 \(C^k\)-함수라는 것이다.
	\end{enumerate}
\end{defn}

\begin{thm} \label{thm taylor}
	\(f, g: [a, b] \rightarrow \RR\)이 어떤 \(n = 0, 1, \ldots\)에 대하여 \(C^n\) 함수이고, \(f^{(n)}, g^{(n)}\)이 \((a, b)\)에서 미분가능하다고 하자. 그러면 각 \(x \in (a, b)\)에 대하여 다음을 만족하는 \(c_x \in (a, x)\)가 존재한다.
	\begin{align*}
		\left[ f(x) - \sum_{k=0}^n \frac{f^{(k)}(a)}{k!} (x - a)^k \right] g^{(n+1)} (c_x) = \left[ g(x) - \sum_{k=0}^n \frac{g^{(k)}(a)}{k!} (x - a)^k \right] f^{(n+1)} (c_x)
	\end{align*}
\end{thm}

\begin{proof}
	\(x \in (a, b)\)를 고정하고 함수 \(F, G\)를 다음과 같이 정의한다.
	\begin{gather*}
		F(t) = \sum_{k=0}^n \frac{f^{(k)}(t)}{k!} (x - t)^k, \quad G(t) = \sum_{k=0}^n \frac{g^{(k)}(t)}{k!} (x - t)^k
	\end{gather*}
	\(F, G\)는 \([a, x]\)에서 연속이고 \((a, x)\)에서 미분가능하다. 따라서 Cauchy의 평균값 정리에 의해
	\begin{gather*}
		[F(x) - F(a)]G'(c_x) = [G(x) - G(a)]F'(c_x)
	\end{gather*}
	를 만족하는 \(c_x \in (a, x)\)가 존재한다. 그런데
	\begin{gather*}
		F(x) = f(x), \quad F(a) =  \sum_{k=0}^n \frac{f^{(k)}(a)}{k!} (x - a)^k\\
		F'(t) = f'(t) + \sum_{k=1}^n {\frac{f^{(k+1)}(t)}{k!}(x-t)^k - \frac{f^{(k)}(t)}{(k-1)!}(x-t)^{k-1}} = \frac{f^{(n+1)}(t)}{n!} (x-t)^n\\
	\end{gather*}
	이고 \(G\)에 대해서도 비슷한 계산을 하면, 원하는 등식을 얻는다.
\end{proof}

\begin{cor}
	\(f: [a, b] \rightarrow \RR\)가 어떤 \(n = 0, 1, \ldots\)에 대하여 \(C^n\) 함수이고, \(f^{(n)}, g^{(n)}\)이 \((a, b)\)에서 미분가능하다고 하자. 그러면 각 \(x \in (a, b)\)에 대하여 다음을 만족하는 \(c_x \in (a, x)\)가 존재한다.
	\begin{equation} \label{eq taylor}
		f(x) = \sum_{k=0}^n \frac{f^{(k)}(a)}{k!} (x - a)^k + \frac{f^{(n+1)}(c_x)}{(n+1)!} (x-a)^{n+1}
	\end{equation}
	((\ref{eq taylor})에서 \((n+1)\)-차항을 \(R_{n}(x)\)라 쓰고 \textbf{나머지항(\(n\)-th remainder term)}이라고 한다.)
\end{cor}
\begin{proof}
	Theorem \ref{thm taylor}에서 \(g(x) = (x-a)^{n+1}\)로 두면 끝.
\end{proof}

\begin{cor}
	양수 \(r > 0\)에 대하여 \(f: (a-r, a+r) \rightarrow \RR\)가 어떤 \(n = 0, 1, \ldots\)에 대하여 \(C^n\) 함수이고, \(f^{(n)}, g^{(n)}\)이 \((a-r, a+r)\)에서 미분가능하다고 하자. 그러면  각 \(x \in (a-r, a+r)\)에 대하여 (\ref{eq taylor})\을 만족하는 \(c_x \in (a, x)\)가 존재한다.
\end{cor}

\begin{defn}
	\(C^\infty\)-함수 \(f: E \subseteq \RR \rightarrow \RR\)과 \(a \in \text{int}E\) 대하여 다음 급수
	\begin{gather*}
		\sum_{k=0}^\infty \frac{f^{(k)}(a)}{k!} (x-a)^k = f(a) + f'(a)(x-a) + \frac{f''(a)}{2} (x-a)^2 + \ldots + \frac{f^{(n)}(a)}{n!} (x-a)^n + \ldots
	\end{gather*}
	를 \textbf{Taylor series(테일러급수) of \(f\) at \(x=a\)}라고 한다.\\
	\(f\)가 \(x=a\)에서 \textbf{analytic(해석적)}이라는 것은 적당한 양수 \(r > 0\)에 대하여
	\begin{gather*}
		x \in (a - r, a + r) \Longrightarrow \sum_{k=0}^\infty \frac{f^{(k)}(a)}{k!} (x-a)^k = f(x)
	\end{gather*}
	라는 것이다. 정의역의 모든 점에서 analytic인 함수를 \textbf{analytic function(해석함수)}라고 한다.
\end{defn}

모든 \(C^\infty\)-함수가 analytic인 것은 아닙니다.

\begin{ex}
	\begin{equation}
		f(x) = 
		\begin{cases}
			0, &\text{if } x \le 0\\
			\exp\left(-\frac{1}{x^2}\right), &\text{if } x > 0
		\end{cases}
	\end{equation}
	로 정의하면 \(f\)는 \(\RR\)에서 \(C^\infty\)이고(이를 증명하는 것은 연습문제로 남김), 특히 \(f^{(n)}(0) = 0\) for \(n = 0, 1, \ldots\)입니다. 따라서 \(f\)의 0에서의 Taylor series는 항등적으로 0이므로 0의 근방에서 Taylor series가 \(f(x)\)로 수렴하지 않습니다. 따라서 \(f\) is not analytic at \(x=0\).
\end{ex}

급수 (\ref{eq taylor})\이 \(x \in (a-r, a+r)\)에서 \(f(x)\)로 수렴할 충분조건 하나.

\begin{obs} \label{obs factorial}
	양수 \(\alpha > 0\)에 대하여 \(\lim_n \alpha^n / n! = 0\).
\end{obs}
\begin{proof}
	\(\alpha < N\)인 자연수 \(N\)을 잡으면, \(n! > N^{n - N + 1}\) for \(n > N\)인 것을 이용.
\end{proof}

\begin{thm}
	\(C^\infty\)-함수 \(f: (a-r, a+r) \rightarrow \RR\)와 \(x \in (a - r, a + r) \backslash \{a\}\)에 대하여 \(I = [a, x]\) 또는 \(I = [x, a]\)라고 하자. 이때
	\begin{gather*}
		\abs{f^{(n)}(t)} \le M \quad (t \in I, n = 0, 1, \ldots)
	\end{gather*}
	를 만족하는 \(M > 0\)이 존재하면 (\ref{eq taylor})\은 \(f(x)\)로 수렴한다.
\end{thm}

\begin{proof}
	\(R_n(x)\)가 0으로 수렴하는 것을 보이면 되는데,
	\begin{gather*}
		\abs{\frac{f^{(n+1)}(c_x)}{(n+1)!} (x-a)^{n+1}} \le M\frac{\abs{x-a}^{n+1}}{(n+1)!} \longrightarrow 0
	\end{gather*}
	이므로 Observation \ref{obs factorial}에 의해 증명 끝.
\end{proof}

\begin{ex}
	대표적인 함수들의 \(x=0\)에서의 Taylor series와 그것이 수렴하는 \(x\)의 범위입니다.
	\begin{align*}
		\sin x &= \sum_{n=0}^\infty \frac{(-1)^n}{(2n+1)!} x^{2n+1} = x - \frac{x^3}{3!} + \frac{x^5}{5!} - \frac{x^7}{7!} + \ldots \quad &(x \in \RR)\\
		\cos x &= \sum_{n=0}^\infty \frac{(-1)^n}{(2n)!} x^{2n} = 1 - \frac{x^2}{2!} + \frac{x^4}{4!} - \frac{x^6}{6!} + \ldots \quad &(x \in \RR)\\
		e^x &= \sum_{n=0}^\infty \frac{x^n}{n!} = 1 + x + \frac{x^2}{2!} + \frac{x^3}{3!} + \ldots \quad &(x \in \RR)\\
		\arctan x &= \sum_{n=0}^\infty \frac{(-1)^n}{2n+1} x^{2n+1} = x - \frac{x^3}{3} + \frac{x^5}{5} - \frac{x^7}{7!} + \ldots \quad &(-1 \le x \le 1)\\
		\log (x+1) &= \sum_{n=1}^\infty \frac{(-1)^{n-1}}{n} x^n = x - \frac{x^2}{2} + \frac{x^3}{3} - \frac{x^4}{4} + \ldots \quad &(-1 < x \le 1)
	\end{align*}
\end{ex}

Taylor series에 대한 공부는 Section \ref{sec seq of ftn}에서 계속됩니다.

\newpage


\section{Riemann-Stieltjes Integral}
\subsection{Definition of the Integral (1)} \label{sec def int}

이번 Section의 목표는 고등학교 때의 적분을 일반화한 Riemann-Stieltjes 적분을 정의하는 것입니다. 적분의 대상이 되는 함수는 유계닫힌구간에서 정의된 유계함수입니다.

고등학교 때와 마찬가지로, 이야기의 시작은 정의역인 구간을 나누는 것입니다.

\begin{defn}
	유계닫힌구간 \([a, b]\) 안의 유한 개의 점 \(x_0, \ldots, x_n\)이
	\begin{gather*}
		x_0 = a < x_1 < \ldots < x_{n-1} < x_n = b
	\end{gather*}
	를 만족할 때, \([a, b]\)의 유한 부분집합 \(P = \{x_0, \ldots, x_n\}\)을 \textbf{partition(분할) of \([a, b]\)}라고 한다. 각 \(i = 1, \ldots, n\)에 대하여, \(\Delta x_i = x_i - x_{i-1}\)로 쓴다.\\
	\([a, b]\)의 모든 partition의 집합을 \(\mathcal{P}[a, b]\)로 쓴다.
\end{defn}

\begin{defn} \label{def stieltjes}
	유계함수 \(f: [a, b] \rightarrow \RR\)와 단조증가함수 \(\alpha: [a, b] \rightarrow \RR\)가 주어졌다고 하자.\\
	구간 \([a, b]\)의 partition \(P = \{x_0, \ldots, x_n\}\)에 대하여,\footnote{\([a, b]\)의 partition을 \(\{x_0, \ldots, x_n\}\)로 나타낼 때, 아무 말이 없으면 \(x_0 = a < x_1 < \ldots < x_{n-1} < x_n = b\)인 것으로 생각.} 
	\begin{align} \label{eq def Mi}
		\begin{split}
			\alpha_i = \alpha(x_i) \quad &(0 \le i \le n)\\
			M_i = \sup \{f(x): x_{i-1} \le x \le x_i\}, \quad m_i = \inf \{f(x): x_{i-1} \le x \le x_i\} \quad &(1 \le i \le n)\\
			\Delta\alpha_i = \alpha_i - \alpha_{i-1} \quad &(1 \le i \le n)
		\end{split}		
	\end{align}
	로 쓰자. \(f\)가 bounded이므로 \(M_i, m_i\)는 유한한 실수로 존재한다. 이때 \textbf{upper sum(상합)}과 \textbf{lower sum(하합)}을 각각 다음과 같이 정의한다.
	\begin{gather*}
		U_a^b (f, P, \alpha) = \sum_{i=1}^n M_i \Delta\alpha_i, \quad 	L_a^b (f, P, \alpha) = \sum_{i=1}^n m_i \Delta\alpha_i
	\end{gather*}
	정의역이 명확할 경우에는 \(U(f, P, \alpha), L(f, P, \alpha)\)와 같이 쓰기로 한다.\\
	이제 \textbf{upper integral(상적분)}과 \textbf{lower integral(하적분)}을 각각 다음과 같이 정의한다.\footnote{잠시 후에 이 둘의 값이 유한한 실수로 존재하며 upper integral은 항상 lower integral보다 크거나 같음을 증명할 것이다. Corollary \ref{cor int welldef} 참조.}
	\begin{gather*}
		\overline{\int_a^b} f d\alpha = \inf \{U(f, P, \alpha): P \in \mathcal{P}[a, b]\}, \quad \underline{\int_a^b} f d\alpha = \sup \{L(f, P, \alpha): P \in \mathcal{P}[a, b]\}
	\end{gather*}
	Upper integral의 값과 lower integral의 값이 같으면, 즉
	\begin{gather*}
		\overline{\int_a^b} f d\alpha = \underline{\int_a^b} f d\alpha
	\end{gather*}
	이면 \(f\)가 \textbf{Riemann-Stieltjes integrable(리만-스틸체스적분가능) with respect to \(\alpha\)}라고 하고, 위 등식의 값을
	\begin{gather*}
		\int_a^b f d\alpha
	\end{gather*}
	또는
	\begin{gather*}
		\int_a^b f(x) d\alpha(x)
	\end{gather*}
	로 쓴다.\footnote{치환적분할 때의 표기와 유사하다. 혹시\ldots?} 마지막으로, \(\alpha: [a, b] \rightarrow \RR\)에 관해 Riemann-Stieltjes integrable인 모든 함수의 집합을 \(\calR(\alpha)\)라고 쓴다. 즉 \(f: [a, b] \rightarrow \RR\)가 Riemann-Stieltjes integrable with respect to \(\alpha\)이면 \(f \in \calR(\alpha)\).
\end{defn}

\begin{defn}
	\(f, \alpha, P\)는 Definition \ref{def stieltjes}에서와 같다. 이때 특히 \(\alpha\)가 \([a, b]\)에서 정의된 항등함수인 경우에, upper sum과 lower sum을 각각
	\begin{gather*}
		U_a^b (f, P) = \sum_{i=1}^n M_i \Delta x_i, \quad 	L_a^b (f, P) = \sum_{i=1}^n m_i \Delta x_i
	\end{gather*}
	로 쓰고, 정의역이 명확하면 \(U(f, P), L(f, P)\)로 쓴다. 또 upper integral과 lower integral을 각각
	\begin{gather*}
		\overline{\int_a^b} f = \inf \{U(f, P): P \in \mathcal{P}[a, b]\}, \quad \underline{\int_a^b} f = \sup \{L(f, P): P \in \mathcal{P}[a, b]\}
	\end{gather*}
	로 쓴다. 만약 \(f\)가 Riemann-Stieltjes integrable with respecto to \(\alpha\)이면, 즉
	\begin{gather*}
		\overline{\int_a^b} f = \underline{\int_a^b} f
	\end{gather*}
	이면 \(f\)가 \textbf{Riemann integrable(리만적분가능)}이라고 하고 위 등식의 값을
	\begin{gather*}
		\int_a^b f
	\end{gather*}
	또는
	\begin{gather*}
		\int_a^b f(x) dx
	\end{gather*}
	로 쓴다. 마지막으로, \([a, b]\)에서 Riemann integrable인 모든 함수의 집합을 \(\calR([a, b])\)라고 쓴다. 즉 \(f: [a, b] \rightarrow \RR\)가 Riemann integrable이면 \(f \in \calR([a, b])\).
\end{defn}

Section \ref{sec def int}\가 끝날 때까지 아무 말이 없으면 \(f: [a, b] \rightarrow \RR\)는 유계, \(\alpha: [a, b] \rightarrow \RR\)는 단조증가함수. \(f \in \calR(\alpha)\)일 필요충분조건을 찾으려 하는데, 이를 위해서 조금의 준비가 필요합니다. 

\begin{defn}
	\([a, b]\)의 두 partition \(P, Q\)에 대하여 \(Q\)가 \(P\)의 \textbf{refinement}라는 것은 \(P \subseteq Q\)라는 것이다.
\end{defn}

\begin{prop} \label{prop refinement}
	\([a, b]\)의 두 partition \(P, Q\)에 대하여 \(Q\)가 \(P\)의 refinement이면 다음이 성립한다.
	\begin{gather*}
		L(f, P, \alpha) \le L(f, Q, \alpha) \le U(f, Q, \alpha) \le U(f, P, \alpha)
	\end{gather*}
\end{prop}
\begin{proof}
	두 번째 부등호는 정의에 의해 당연하고, 세 번째 부등호가 성립하는 것을 보이면 첫 번째 부등호가 성립하는 것은 비슷하게 보일 수 있다.\\
	\(Q\)가 \(P\)보다 원소가 1개 더 많은 경우만 보이면 된다. 
	\begin{gather*}
		P = \{x_0, \ldots, x_n\}, \quad Q = P \cup \{x'_k\}, \quad x_{k-1} < x'_k < x_k
	\end{gather*}
	로 놓고,
	\begin{gather*}
		M = \sup_{x \in [x_{k-1}, x_k]} f(x), \quad M_1 = \sup_{x \in [x_{k-1}, x'_k]} f(x), \quad M_2 = \sup_{x \in [x'_k, x_k]} f(x)
	\end{gather*}
	로 정의하면 \(M_1, M_2 \le M\)인 것은 쉽게 알 수 있다. 따라서
	\begin{gather*}
		U(f, P, \alpha) - U(f, Q, \alpha) = M(\alpha_k - \alpha_{k-1}) - [M_1(\alpha(x'_k) - \alpha_{k-1}) + M_2 (\alpha_k - \alpha(x'_k))] \ge 0.
	\end{gather*}
\end{proof}

\begin{cor} \label{cor int welldef}
	\([a, b]\)의 임의의 두 partition \(P, Q\)에 대하여 \(L(f, P, \alpha) \le U(f, Q, \alpha)\)이다. 따라서 다음이 성립한다.
	\begin{gather*}
		-\infty < \underline{\int_a^b} f d\alpha \le \overline{\int_a^b} f d\alpha < \infty.
	\end{gather*}
\end{cor}
\begin{proof}
	\(P \cup Q\)는 \(P, Q\)의 common refinement이므로
	\begin{gather*}
		L(f, P, \alpha) \le L(f, P \cup Q, \alpha) \le U(f, P \cup Q, \alpha) \le U(f, Q, \alpha).
	\end{gather*}
	두 번째 부등식이 성립하는 것은 이제 자명하다.
\end{proof}

\(f \in \calR(\alpha)\)일 첫 번째 필요충분조건.

\begin{thm} \label{thm integrable 1}
	\(f \in \calR(\alpha)\)일 필요충분조건은 임의의 양수 \(\eps > 0\)에 대하여, 적당한 partition \(P\)가 존재하여 다음을 만족하는 것이다.
	\begin{equation} \label{eq integrable 1}
		U(f, P, \alpha) - L(f, P, \alpha) < \eps
	\end{equation}
\end{thm}
\begin{proof}
	($\Rightarrow$): \(f \in \calR(\alpha)\)라고 하고 \(\eps > 0\)이 주어졌다고 하자. 정의에 의해 다음을 만족하는 두 partition \(P_1, P_2\)를 찾을 수 있다.
	\begin{gather*}
		U(f, P_1, \alpha) < \overline{\int_a^b} f d\alpha + \frac{\eps}{2}, \quad L(f, P_2, \alpha) > \underline{\int_a^b} f d\alpha - \frac{\eps}{2}
	\end{gather*}
	이제 \(P = P_1 \cup P_2\)로 두면 (\ref{eq integrable 1})\을 만족한다.\\
	($\Leftarrow$): 임의의 양수 \(\eps > 0\)에 대해 (\ref{eq integrable 1})\을 만족하는 partition \(P\)가 존재하여
	\begin{gather*}
		0 \le \overline{\int_a^b} f d\alpha - \le \underline{\int_a^b} f d\alpha \le U(f, P, \alpha) - L(f, P, \alpha) < \eps
	\end{gather*}
	인데, \(\eps > 0\)이 임의의 양수였으므로 \(\overline{\int_a^b} f d\alpha = \underline{\int_a^b} f d\alpha\)이다.
\end{proof}

\begin{ex}
	\quad
	\begin{enumerate}[label=(\alph*), leftmargin=2\parindent]
		\item
		\(f: [0, 1] \rightarrow \RR\)를
		\begin{gather*}
			f(x) =
			\begin{cases}
				1, &\text{if } x \in \QQ\\
				0, &\text{otherwise}
			\end{cases}
		\end{gather*}
		로 정의하면, 임의의 partition \(P\)에 대해 \(U(f, P) = 1, L(f, P) = 0\)이므로 Theorem \ref{thm integrable 1}의 (b)를 만족하지 않는다. 따라서 \(f\) is not Riemann integrable on \([0, 1]\).
		\item
		\(g: [-1, 1] \rightarrow \RR\)를
		\begin{gather*}
			g(x) =
			\begin{cases}
				1, &\text{if } x = 0\\
				0, &\text{otherwise}
			\end{cases}
		\end{gather*}
		로 정의하자. 임의의 양수 \(\eps > 0\)에 대하여 partition \(P\)를 다음과 같이 잡는다.
		\begin{gather*}
			P = \{-1, -\frac{\eps}{3}, \frac{\eps}{3}, 1\}
		\end{gather*}
		그러면 \(U(f, P) - L(f, P) = 2\eps/3 - 0 < \eps\)이므로, \(g\) is Riemann integrable on \([-1, 1]\).
		\item
		점 \(c \in (a, b)\)에 대하여, \(\alpha: [a, b] \rightarrow \RR\)를
		\begin{gather*}
			\alpha(x) =
			\begin{cases}
				0, &\text{if } x \le c\\
				1, &\text{otherwise}
			\end{cases}
		\end{gather*}
		로 정의하면 \(\alpha\)는 단조증가함수이다. 이때 유계함수 \(h: [a, b] \rightarrow \RR\)가 \(\lim_{x \rightarrow c+} f(x) = f(c)\)를 만족하면 \(f \in \calR(\alpha)\)이다. 임의의 양수 \(\eps > 0\)에 대하여 다음을 만족하는 양수 \(\delta > 0\)가 존재한다.
		\begin{gather*}
			c < x < c+\delta \Longrightarrow \abs{f(x) - f(c)} < \eps
		\end{gather*}
		이제 partition \(P = \{a, c, c+\delta/2, b\}\)로 잡으면,
		\begin{gather*}
			U(f, P, \alpha) = \sup\{f(x): c\le x \le c+ \delta/2 \} \le f(c) + \eps\\
			L(f, P, \alpha) = \inf\{f(x): c\le x \le c+ \delta/2 \} \le f(c) - \eps
		\end{gather*}
		이므로, \(U(f, P, \alpha) - L(f, P, \alpha) \le 2\eps\). 따라서 \(f \in \calR(\alpha)\)이다.
	\end{enumerate}
\end{ex}

\begin{obs} \label{obs refinement}
	양수 \(\eps > 0\)과 partition \(P\)가 (\ref{eq integrable 1})\을 만족하면, \(P\)의 refinement \(Q\)에 대해서도 \(U(f, Q, \alpha) - L(f, Q, \alpha) < \eps\)이 성립한다.
\end{obs}
\begin{proof}
	Proposition \ref{prop refinement}에 의해.
\end{proof}

\begin{defn}
	\([a, b]\)의 partition \(P = \{x_0, \ldots, x_n\}\)이 주어졌을 때 각 \(i = 1, \ldots, n\)에 대하여 \(t_i \in [x_{i-1}, x_i]\)를 택한다. 이때 \textbf{Riemann-Stieltjes sum(리만-스틸체스합)} \(S(f, P, \alpha)\)를 다음과 같이 정의한다.
	\begin{gather*}
		S(f, P, \alpha) = \sum_{i=1}^n f(t_i)\Delta\alpha_i
	\end{gather*}
	\(\alpha\)가 항등함수일 때의 Riemann-Stieltjes sum을 \textbf{Riemann sum(리만합)}이라고 하고 \(R(f, P)\)로 쓴다. 즉,
	\begin{gather*}
		R(f, P) = \sum_{i=1}^n f(t_i)\Delta x_i
	\end{gather*}
	이다.
\end{defn}

\begin{obs} \label{obs st sum}
	Partition \(P\)에 대해 \(L(f, P, \alpha) \le S(f, P, \alpha) \le U(f, P, \alpha)\)이다.
\end{obs}
\begin{proof}
	정의에 의해 당연하다.
\end{proof}

\(f \in \calR(\alpha)\)일 두 번째 필요충분조건. 이때는 적분의 값까지 알 수 있습니다.

\begin{thm} \label{thm st int}
	유계함수 \(f: [a, b] \rightarrow \RR\)와 단조증가함수 \(\alpha: [a, b] \rightarrow \RR\), 실수 \(A \in \RR\)에 대하여 다음은 동치.
	\begin{enumerate}[label=(\alph*), leftmargin=2\parindent]
		\item
		\(f \in \calR(\alpha)\)이고 \(\int_a^b f d\alpha = A\).
		\item
		임의의 양수 \(\eps > 0\)에 대하여 적당한 partition \(P_0\)가 존재하여 다음을 만족한다.
		\begin{gather*}
			P \in \mathcal{P}[a, b], P \supseteq P_0 \Longrightarrow \abs{S(f, P, \alpha) - A} < \eps
		\end{gather*}
	\end{enumerate}
\end{thm}
\begin{proof}
	($\Rightarrow$): (a)를 가정하면 임의의 양수 \(\eps > 0\)에 대해 partition \(P_0\)가 존재하여
	\begin{gather*}
		A - \eps < L(f, P_0, \alpha) \le U(f, P_0, \alpha) < A + \eps
	\end{gather*}
	을 만족한다. 이제 Observation \ref{obs refinement}\와 Observation \ref{obs st sum}에 의해 (b)가 성립한다.\\
	($\Leftarrow$): 양수 \(\eps > 0\)이 주어졌다고 하고, (b)에 의해 \(\abs{S(f, P, \alpha) - A} < \eps\)인 \(P\)를 찾는다. 그런데
	\begin{gather*}
		A - \eps < S(f, P, \alpha) < A + \eps\\
		\Longrightarrow \quad A - \eps \le L(f, P, \alpha) \le U(f, P, \alpha) \le A + \eps
	\end{gather*}
	이므로, Theorem \ref{thm integrable 1}에 의해 \(f \in \calR(\alpha)\)이다. 이제 \(\int_a^b f d\alpha = A\)인 것은 쉽게 알 수 있다.
\end{proof}

Riemann integral의 경우에 조건 하나를 더 추가할 수 있습니다.

\begin{defn}
	\([a, b]\)의 partition \(P = \{x_0, \ldots, x_n\}\)에 대하여 \(P\)의 \textbf{norm}을 \(\max\{\Delta x_i: i = 1, \ldots, n\}\)로 정의하고 \(\norm{P}\)로 쓴다.
\end{defn}


\begin{thm} \label{thm rm int}
	유계함수 \(f: [a, b] \rightarrow \RR\)와 실수 \(A \in \RR\)에 대하여 다음은 동치.
	\begin{enumerate}[label=(\alph*), leftmargin=2\parindent]
		\item
		\(f \in \calR([a, b])\)이고 \(\int_a^b f = A\).
		\item
		임의의 양수 \(\eps > 0\)에 대하여 적당한 partition \(P_0\)가 존재하여 다음을 만족한다.
		\begin{gather*}
			P \in \mathcal{P}[a, b], P \supseteq P_0 \Longrightarrow \abs{R(f, P) - A} < \eps
		\end{gather*}
		\item
		임의의 양수 \(\eps > 0\)에 대하여 적당한 양수 \(\delta > 0\)가 존재하여 다음을 만족한다.
		\begin{equation} \label{eq norm}
			P \in \mathcal{P}[a, b], \norm{P} < \delta \Longrightarrow \abs{R(f, P) - A} < \eps
		\end{equation}
		즉, \(\lim_{\norm{P} \rightarrow 0} R(f, P) = A\)이다.
	\end{enumerate}
\end{thm}
\begin{proof}
	(a)$\Leftrightarrow$(b)는 증명하였으므로, (c)$\Rightarrow$(b)와 (a)$\Rightarrow$(c)를 보인다.\\
	(c)$\Rightarrow$(b): 양수 \(\eps > 0\)이 주어졌다고 하고 (\ref{eq norm})\을 만족하는 \(\delta > 0\)를 찾는다. \(\norm{P_0} < \delta\)인 partition \(P_0\)는 항상 존재하는데(왜 그런가?), \(P\)가 \(P_0\)의 refinement이면 \(\norm{P} \le \norm{P_0} < \delta\)이므로 (c)에 의해 \(\abs{R(f, P) - A} < \eps\)이다.\\
	(a)$\Rightarrow$(c):양수 \(\eps > 0\)이 주어졌다고 하고 다음을 만족하는 분할 \(P_0 = \{x_0, \ldots, x_n\}\)을 잡는다.
	\begin{gather*}
		U(f, P_0) < A + \eps
	\end{gather*}
	\(f\)는 bounded이므로 \(\abs{f} < M\)인 \(M > 0\)이 존재한다. \(\delta_1 = \eps / (nM)\)으로 두자.\\
	Claim 1: \(f > 0\)일 때, \(\norm{P} < \delta_1\)이면 \(U(f, P) < A + 2\eps\)이다.
	\begin{list}{}{\leftmargin=\parindent\rightmargin=0pt}
		\item
		\begin{proof}[Proof of Claim 1]
			Partition \(P = \{y_0, \ldots, y_m\}\)이 \(\norm{P} < \delta_1\)을 만족한다고 하고,
			\begin{gather*}
				I = \{i = 1, \ldots, m : (y_{i-1}, y_i) \cap P_0 \neq P_0\}\\
				J = \{i = 1, \ldots, m : (y_{i-1}, y_i) \cap P_0 = P_0\}
			\end{gather*}
			라고 정의하자. 이때 \(\abs{I} < n\)인 것에 유의하면(왜 그런가?)
			\begin{align*}
				U(f, P) &= \sum_{i=1}^m M_i \Delta y_i\\
				&= \sum_{i \in I} M_i \Delta y_i + \sum_{i \in J} M_i \Delta y_i\\
				&< nM \norm{P} + U(f, P_0) < \eps + A + \eps = A + 2\eps.
			\end{align*}
		\end{proof}
	\end{list}
	Claim 2: 임의의 유계함수 \(f\)에 대하여 \(\norm{P} < \delta_1\)이면 \(U(f, P) < A + 2\eps\)이다.
	\begin{list}{}{\leftmargin=\parindent\rightmargin=0pt}
		\item
		\begin{proof}[Proof of Claim 2]
			적당한 상수 \(c > 0\)에 대하여 \(\tilde{f} = f + c > 0\)이게 할 수 있다.
			\begin{gather*}
				U(\tilde{f}, P) = U(f, P) + c(b-a), \quad \int_a^b \tilde{f} = A + c(b-a)
			\end{gather*}
			이므로(아직 적분의 덧셈에 관한 성질을 증명하지는 않았지만 상수 덧셈은 쉽게 확인할 수 있다) Claim 1에 의해 \(\norm{P} < \delta_1\)이면 \(U(f, P) = U(\tilde{f}, P) - c(b-a) < A + 2\eps\).
		\end{proof}
	\end{list}
	Claim 3: 임의의 유계함수 \(f\)에 대하여 다음을 만족하는 \(\delta_2 > 0\)가 존재한다.
	\begin{gather*}
		\norm{P} < \delta_2 \Longrightarrow L(f, P) > A - 2\eps
	\end{gather*}
	\begin{list}{}{\leftmargin=\parindent\rightmargin=0pt}
		\item
		\begin{proof}[Proof of Claim 3]
			\(L(f, P) = U(-f, P)\)이고 \(\int_a^b (-f) = -A\)인 것만 확인하면 Claim 2에 의해 증명된다.
		\end{proof}
	\end{list}
	이제 \(\delta = \min\{\delta_1, \delta_2\}\)로 두자. \(\norm{P} < \delta\)이면 Claim 2, Claim 3에 의해 \(\abs{R(f, P) - A} < 2\eps\)이다.
\end{proof}

\begin{rem}
	\(\alpha\)가 일반적인 단조증가함수일 경우에는 Theorem \ref{thm rm int}의 (c)와 같은 조건을 추가할 수 없다. 더 정확히 말해서,
	\begin{gather*}
		\lim_{\norm{P} \rightarrow 0} S(f, P, \alpha) = A \Longrightarrow \int_a^b f d\alpha = A
	\end{gather*}
	이지만 그 역은 성립하지 않는다. 위 명제의 증명은 Theorem \ref{thm rm int}의 (c)$\Rightarrow$(b)의 증명과 완전히 같다. 그 역에 대한 반례로, \(c \in (a, b)\)에 대하여 \(\alpha, f:[a, b] \rightarrow \RR\)를
	\begin{gather*}
		\alpha(x) =
		\begin{cases}
			0, &\text{if } x \le c\\
			1, &\text{otherwise}
		\end{cases}, \quad
		f(x) = 
		\begin{cases}
			0, &\text{if } x < c\\
			1, &\text{otherwise}
		\end{cases}
	\end{gather*}
	로 정의하자. 그러면 Theorem \ref{thm integrable 1} 아래 Example (c)에 의해 \(f \in \calR(\alpha)\)이고 특히 \(\int_a^b f d\alpha = 1\)이다. 이제 다음을 만족하는 \(\delta > 0\)가 존재한다고 가정하자.
	\begin{gather*}
		\norm{P} < \delta \Longrightarrow \abs{S(f, P, \alpha) - 1} < \frac{1}{2}
	\end{gather*}
	그러면 \(\norm{P} < \delta\)인 partition \(P\)를 다음과 같이 잡을 수 있다.
	\begin{gather*}
		P = \{x_0 = a, x_1, \ldots, x_{k-1} = c - \frac{\delta}{3}, x_{k} = c + \frac{\delta}{3}, \ldots, x_{n-1}, x_n = b\}
	\end{gather*}
	이때 각 \(t_i \in [x_{i-1}, x_i]\)를 택하면 \(S(f, P, \alpha) = f(t_k)\)인데, \(t_k \in [x_{k-1}, c)\)이면 \(S(f, P, \alpha) = 0\)이므로 모순이다.
\end{rem}

\subsection{Integrable Functions} \label{sec int ftn}

매번 Theorem \ref{thm st int}나 Theorem \ref{thm rm int} 같은 것을 써서 함수의 적분가능성을 논하기는 불편합니다. 다행히 우리가 아는 좋은 함수들은 적분가능한 것이 알려져 있습니다. Section \ref{sec int ftn}\가 끝날 때까지 아무 말이 없으면 \(f: [a, b] \rightarrow \RR\)는 유계, \(\alpha: [a, b] \rightarrow \RR\)는 단조증가함수.

\begin{thm}
	\(f\)가 \([a, b]\)에서 연속이면 \(f \in \calR(\alpha)\)이다.
\end{thm}
\begin{proof}
	\(\eps > 0\)이 주어졌다고 하고, 다음을 만족하는 \(\eta > 0\)를 잡는다.
	\begin{gather*}
		\eta[\alpha(b) - \alpha(a)] < \eps
	\end{gather*}
	\(f\)는 \([a, b]\)에서 uniformly continuous이므로 다음을 만족하는 \(\delta > 0\)가 존재한다.
	\begin{gather*}
		\abs{x - y} < \delta \Longrightarrow \abs{f(x) - f(y)} < \eta \quad (x, y \in [a, b])
	\end{gather*}
	이제 partition \(P = \{x_0, \ldots, x_n\}\)을 \(\norm{P} < \delta\)이 되도록 잡으면,
	\begin{align*}
		U(f, P, \alpha) - L(f, P, \alpha) &= \sum_{i=1}^n (M_i - m_i)\Delta\alpha_i\\
		&\le \sum_{i=1}^n \eta\Delta\alpha_i = \eta \sum_{i=1}^n \alpha_i = \eta[\alpha(b) - \alpha(a)] < \eps.
	\end{align*}
\end{proof}

고등학교 때 연속함수에 대해서만 적분할 때 적분가능성이라는 말을 쓰지 않은 이유는 모든 연속함수는 (리만)적분가능했기 때문입니다.

그런데 모든 리만적분가능한 함수가 연속인 것은 아닙니다. 리만적분의 정의를 생각해 보면, 정의역의 한 점에서 point discontinuous(극한이 존재하고 함숫값과 다른 경우)이거나 jump discontinuous(좌극한과 우극한이 각각 존재하고 그 값이 다른 경우)여도 리만적분가능성에 영향을 주지 않을 거라는 추측을 할 수 있습니다. 나아가 이런 point discontinuity나 jump discontinuity가 유한 개 있어도 리만적분가능성에 영향을 주지 않을 것입니다. 이를 더욱 확장하여, discontinuity가 \textbf{countable} 개만큼 있어도 리만적분가능하다는 것을 증명할 것입니다. 또 이를 일반적인 리만-스틸체스 적분으로 확장할 수 있는데, 이때는 \(f\)가 불연속인 점에서 \(\alpha\)가 연속이라는 조건이 필요합니다.

\begin{thm} \label{thm disconti int}
	\(f\)의 불연속점 전체의 집합을 \(D\)라고 하자. 만약 \(D\)가 at most countable이고 \(\alpha\)가 \(D\)의 각 점에서 연속이면 \(f \in \calR(\alpha)\)이다.
\end{thm}
\begin{proof}
	\(\alpha(a) = \alpha(b)\)이면 (\(\alpha\)가 상수함수이므로) 증명할 것이 없으므로 \(\alpha(a) < \alpha(b)\)라고 가정할 수 있다. \(\abs{f} < M\)인 \(M > 0\)을 고정하고, 양수 \(\eps > 0\)이 주어졌다고 하자. 먼저 다음을 만족하는 모든 점 \(x \in [a, b]\)의 집합을 \(U\)라고 정의하자.
	\begin{equation} \label{eq Ux}
		\sup_{t \in U_x}f(t) - \inf_{t \in U_x}f(t) < \frac{\eps}{\alpha(b) - \alpha(a)} \text{ for some } U_x = (x - r_x, x + r_x) \cap [a, b], r_x > 0
	\end{equation}
	\(U\)는 open in \([a, b]\)이다(왜 그런가?). 또, \(f\)가 \(x\)에서 연속이면 \(x \in U\)이므로 \([a, b] = U \cup D\)이다.\\
	한편 가정에 의해 \(D = \{d_i\}_{i \in \NN}\)으로 쓸 수 있다. 각 \(i = 1, 2,\ldots\)에 대하여 \(\alpha\)는 점 \(d_i\)에서 연속이므로 다음을 만족하는 \(\delta_i > 0\)가 존재한다.
	\begin{gather*}
		x \in (d_i - \delta_i, d_i + \delta_i) \cap [a, b] \Longrightarrow \abs{\alpha(x) - \alpha(d_i)} < \frac{\eps}{2^{i}M}
	\end{gather*}
	각 \(i\)에 대하여 \(D_i = (d_i - \delta_i/2, d_i + \delta_i/2) \cap [a, b]\)로 정의하자. 그러면 \(\{U\} \cup \{D_i\}_{i \in \NN}\)은 \([a, b]\)의 open cover이고 \([a, b]\)는 compact이므로 적당한 \(N \in \NN\)에 대해 다음이 성립한다.
	\begin{gather*}
		[a, b] = U \cup \bigcup_{i=1}^N D_i
	\end{gather*}
	\(C = [a, b] \backslash \bigcup_{i=1}^N D_i\)라고 두자. \(C\)는 finite union of disjoint closed intervals이므로(왜 그런가?) 다음과 같이 쓸 수 있다.
	\begin{gather*}
		C = \bigcup_{i=1}^{L} [a_i, b_i], \quad a_i < b_i < a_{i+1}
	\end{gather*}
	또 \(C \subseteq U\)이므로, 각 \(x \in C\)에 대해 (\ref{eq Ux})\를 만족하는 양수 \(r_x > 0\)가 존재하여 \(C \subseteq \bigcup_{x \in C} (x - r_x, x + r_x)\)이다. 그런데 \(C\)는 compact이므로, \(x_1, \ldots, x_K \in C\)가 존재하여
	\begin{gather*}
		C \subseteq \bigcup_{k=1}^K (x_k - r_{x_k}, x_k + r_{x_k})
	\end{gather*}
	이다. 이제 각 interval \([a_i, b_i]\)의 partition \(P_i = \{x_{i0}=a_i, x_{i1}, \ldots, x_{ij_i} = b_i\}\)이 존재하여 subinterval \([x_{ij}, x_{i(j+1)}] \subseteq (x_k - r_{x_k}, x_k + r_{x_k})\) for some \(k\)이도록 할 수 있다. 이때 다음이 성립함을 관찰하자.
	\begin{align*}
		\sum_{i=1}^L [U_{a_i}^{b_i}(f, P_i, \alpha) - L_{a_i}^{b_i}(f, P_i, \alpha)] &\le \sum_{i=1}^L \sum_{j=1}^{j_i} \frac{\eps}{\alpha(b) - \alpha(a)} [\alpha(x_{ij}) - \alpha(x_{i(j-1)})]\\
		&= \sum_{i=1}^L \frac{\eps}{\alpha(b) - \alpha(a)} [\alpha(b_i) - \alpha(a_i)] \le \eps
	\end{align*}
	다음으로 각 \(i = 1, \ldots, N\)에 대하여 \((d_i - \delta_i, d_i + \delta_i) \cap [a, b]\)의 closure를 \([d_{i1}, d_{i2}]\)라고 하자. 이 구간의 분할 \(P'_i = \{d_{i1}, d_{i2}\}\)에 대하여 다음이 성립한다.
	\begin{align*}
		\sum_{i=1}^N [U_{d_{i1}}^{d_{i2}} (f, P'_i, \alpha) - L_{d_{i1}}^{d_{i2}} (f, P'_i, \alpha)] &\le \sum_{i=1}^N 2M [\alpha(d_{i2}) - \alpha(d_{i1})]\\
		&\le \sum_{i=1}^N 2M \frac{\eps}{2^{i-1} M} < 4\eps
	\end{align*}
	마지막으로, \(P = P_U \cup P_D = \{y_0, \ldots, y_n\}\)로 두면 \(P\)는 \([a, b]\)의 partition이고,
	\begin{align*}
		U_a^b (f, P, \alpha) - L_a^b (f, P, \alpha) &\le \sum_{i=1}^L [U_{a_i}^{b_i}(f, P_i, \alpha) - L_{a_i}^{b_i}(f, P_i, \alpha)] + \sum_{i=1}^N [U_{d_{i1}}^{d_{i2}} (f, P'_i, \alpha) - L_{d_{i1}}^{d_{i2}} (f, P'_i, \alpha)]\\
		&< 5\eps
	\end{align*}
	이다. 따라서 \(f \in \calR(\alpha)\)이다.
\end{proof}

\begin{cor}
	\(f: [a, b] \rightarrow \RR\)가 \([a, b]\)에서 단조함수이고, \(f\)가 불연속인 점에서 \(\alpha\)가 연속이면 \(f \in \calR(\alpha)\)이다.
\end{cor}
\begin{proof}
	\(f\)의 불연속점이 at most countable인 것만 보이면 되는데, 이미 Corollary \ref{cor mono disconti}에서 증명하였다.
\end{proof}

적분가능한 함수에 연속함수를 합성해도 적분가능함수입니다.

\begin{thm}
	\(f \in \calR(\alpha)\)이고 \(m \le f \le M\)이라고 하자. 연속함수 \(\phi: [m, M] \rightarrow \RR\)에 대하여 \(\phi \circ f \in \calR(\alpha)\)이다.
\end{thm}
\begin{proof}
	\(\phi\)는 bounded이므로 \(h = \phi \circ f\)도 bounded이다. 따라서 \(h\)의 적분가능성을 논할 수 있다.\\
	양수 \(\eps > 0\)이 주어졌다고 하자. \(\phi\)는 uniformly continuous on \([m, M]\)이므로 다음을 만족하는 \(\delta \in (0, \eps)\)가 존재한다.
	\begin{gather*}
		\abs{s - t} \le \delta \Rightarrow \abs{\phi(s) - \phi(t)} < \eps \quad (s, t \in [m, M])
	\end{gather*}
	\(f \in \calR(\alpha)\)이므로 다음을 만족하는 partition \(P = \{x_0, \ldots, x_n\}\)을 잡을 수 있다.
	\begin{gather*}
		U(f, P, \alpha) - L(f, P, \alpha) < \delta^2
	\end{gather*}
	(\ref{eq def Mi})\와 같이 \(M_i, m_i\)를 정의하고, 마찬가지 방법으로 \(h = \phi \circ f\)에 대해 \(M'_i, m'_i\)를 정의한다. 다음과 같이 index set을 분할한다.
	\begin{gather*}
		I = \{i = 1, \ldots, n: M_i - m_i < \delta\}\\
		J = \{i = 1, \ldots, n: M_i - m_i \ge \delta\}
	\end{gather*}
	\(\i \in I\)이면 \(\delta\)의 정의에 의해 \(M'_i - m'_i < \eps\)이다. 한편
	\begin{gather*}
		\delta\sum_{i \in J} \Delta\alpha_i \le \sum_{i \in J} (M_i - m_i) \Delta\alpha_i \le  \sum_{i=1}^n (M_i - m_i) \Delta\alpha_i < \delta^2
	\end{gather*}
	이므로
	\begin{gather*}
		\sum_{i \in J} \Delta\alpha_i < \delta
	\end{gather*}
	이다. \(\phi\)는 bounded이므로 \(\abs{\phi} < K\)인 \(K > 0\)를 잡으면, 각 \(i\)에 대해 \(M'_i - m'_i \le 2K\)이고
	\begin{align*}
		U(h, P, \alpha) - L(h, P, \alpha) &= \sum_{i \in I} (M'_i - m'_i) \Delta\alpha_i + \sum_{i \in J} (M'_i - m'_i) \Delta\alpha_i\\
		&\le \eps \sum_{i \in I} \Delta\alpha_i + 2K \sum_{i \in J} \Delta\alpha_i \\
		&< \eps[\alpha(b) - \alpha(a)] + 2K\delta <\eps[\alpha(b) - \alpha(a) + 2K]
	\end{align*}
	이다. \(\alpha(a), \alpha(b), K\)는 고정된 값이므로, \(h \in \calR(\alpha)\)가 증명되었다.
\end{proof}

\subsection{Properties of the Integral}

\subsection{Fundamental Theorem of Calculus}

\subsection{Function of Bounded Variation}

\subsection{Definition of the Integral (2)}


\newpage

\section{Sequence of Functions} \label{sec seq of ftn}

\section{Function Space}

\section{Functions Defined by Integral}

\section{Fourier Series}


\begin{thebibliography}{99}
	
	\bibitem{Cd94}\label{ref1} 김성기, 김도한, 계승혁. \emph{해석개론}. 서울대학교출판문화원. 2011.
	\bibitem{Cd94}\label{ref2} W. Rudin. \emph{Principles of Mathematical Analysis}. McGraw-Hill. 1976.
	\bibitem{Cd94} 이인석. \emph{선형대수와 군}. 서울대학교출판문화원. 2015.
	\bibitem{Cd94} J. R. Munkres, \emph{Topology}. Prentice Hall. 2000.
	
	다음은 이 글의 한 부분에서만 인용한 경우.
	\bibitem{Cd94} \url{https://en.wikipedia.org/wiki/L%27H%C3%B4pital%27s_rule}. Corollary \ref{cor lpt} 아래의 Example에서 인용.
	\bibitem{Cd94} \url{https://math.stackexchange.com/questions/263189/proof-that-a-function-with-a-countable-set-of-discontinuities-is-riemann-integra}. Theorem \ref{thm disconti int}의 증명에서 인용.
		
\end{thebibliography}
\end{document}