\documentclass[12pt]{article}

\usepackage[a4paper,left=20mm,right=20mm,top=40mm,bottom=40mm]{geometry}
\usepackage[utf8]{inputenc}
\usepackage{amsmath, amsthm, amsfonts, kotex, enumitem, setspace, hyperref}
\setstretch{1.4}


\theoremstyle{definition}
\newtheorem{thm}{Theorem}[section]
\newtheorem{cor}[thm]{Corollary}
\newtheorem{lem}[thm]{Lemma}
\newtheorem{prop}[thm]{Proposition}
\newtheorem{defn}[thm]{Definition}
\newtheorem{notn}[thm]{Notation}
\newtheorem{obs}[thm]{Observation}
\newtheorem*{rem}{Remark}
\newtheorem*{ex}{Example}

\def\NN{\mathbb{N}}
\def\ZZ{\mathbb{Z}}
\def\QQ{\mathbb{Q}}
\def\RR{\mathbb{R}}
\def\CC{\mathbb{C}}
\def\eps{\epsilon}

\newcommand{\abs}[1]{\left\vert#1\right\vert}
\newcommand{\norm}[1]{\left\vert\left\vert#1\right\vert\right\vert}

\title{해석개론 이야기}
\author{포만한 수학하는원태인}

\begin{document}
\maketitle

\tableofcontents
\newpage

원래 다른 사이트에서 올리던 글을, 사정이 생겨 여기서 계속 이어 올리게 되었습니다.

1년 동안 공부한 해석개론의 내용을 [\ref{ref1}]\과 [\ref{ref2}]의 내용을 중심으로 복습하는 김에, {\LaTeX} 연습도 좀 하고, (정말 낮은 확률로) 여기에 관심 있는 독자가 존재한다면 재미있는 내용을 전달하기 위해서 쓰는 글입니다. 저도 일개 학부생이니 오류나 오타가 있을 수 있으며, 같은 대상을 지칭하는 영어와 한국어 용어도 혼용하고 있고, 한번에 쓰는 글이 아니므로 서술 방식과 문체도 글 내부에서 상당히 다를 수 있습니다.

\section{Completeness of \(\RR\)}

해석개론의 출발점은 실수 집합 \(\RR\)의 성질을 규명하는 것입니다. 이를 위해서 몇 가지 정의가 필요합니다.

\subsection{Field}

	\begin{defn}
		집합 \(F\)에 이항 연산 \(+\)와 \(\cdot\)이 정의되어\footnote{집합 \(X\)에 이항 연산 \(*\)이 정의되었다는 것은 함수 \(*: X \times X \rightarrow X\)가 잘 정의되었다는 뜻. 다시 말해서, 임의의 \(x, y \in F\)에 대해 \(*(x, y) = x * y\)가 \(X\) 안의 유일한 원소로 결정된다는 뜻이다.} 있을 때, 다음 (F1)-(F9)를 만족하면 \((F, +, \cdot)\) 또는 간단하게 \(F\)를 \textbf{field(체)}라고 한다. \(+\)와 \(\cdot\)을 각각 덧셈, 곱셈이라고 한다.
		\begin{enumerate} [label=(F\arabic*), leftmargin=2\parindent]
			\item
			\(x + (y+z) = (x+y)+z\) for \(x, y, z \in F\).
			\item
			\(0 \in F\)가 존재하여 [\(x+0=0+x=x\) for \(x \in F\)].
			\item
			\(x \in F\)에 대하여 \(-x \in F\)가 존재하여 \mbox{\(x+(-x)=(-x)+x=0\).}
			\item
			\(x+y=y+x\) for \(x, y \in F\).
			\item
			\(x(yz)=(xy)z\) for \(x, y, z \in F\).
			\item
			\(1 \in F \backslash \{0\}\)이 존재하여 [\(x\cdot 1=1 \cdot x=x\) for \(x \in F\)].
			\item
			\(x \in F \backslash \{0\}\)에 대하여 \(x^{-1} \in F\)가 존재하여 \(xx^{-1}=x^{-1}x=1\).
			\item
			\(xy=yx\) for \(x, y \in F\).
			\item
			\(x(y+z)=xy+xz, (x+y)z = xz+yz\) for \(x, y, z \in F\).
		\end{enumerate}
	\end{defn}
	
다음이 성립하는 것을 관찰할 수 있습니다.

	\begin{prop} \label{field addition}
		Field \(F\)에 대하여 다음이 성립한다(\(x, y, z \in F\)).
		\begin{enumerate} [label=(\alph*), leftmargin=2\parindent]
			\item
			\(x + y = x + z\)이면 \(y = z\). (\(x, y, z \in F\))
			\item
			\(y \in F\)가 [\(x + y = y + x = x\) for \(x \in F\)]를 만족하면 \(y = 0\).\footnote{사실 이것이 없으면 (F3)은 non-sense.}
			\item
			\(x + y = y + x = 0\)이면 \(y = -x\). (\(x, y \in F\))
			\item
			For \(x \in F\), \(-(-x) = x\).
		\end{enumerate}
	\end{prop}
	\begin{proof}
		\quad\\
		(a): \(y = 0 + y = (-x + x) + y = -x + (x + y) = -x + (x + z) = (-x + x) + z = 0 + z = z\).\\
		(b): \(x + y = x = x + 0\)이므로, (a)에 의해 \(y=0\).\\
		(c): \(x + y = 0 = x + (-x)\)이므로, (a)에 의해 \(y = -x\).\\
		(d): \(-x + x = x + (-x) = 0\)이므로, (c)에 의해 \(x = -(-x)\).
	\end{proof}

	\begin{prop} \label{field mult}
		Field \(F\)에 대하여 다음이 성립한다(\(x, y, z \in F\)).
		\begin{enumerate} [label=(\alph*), leftmargin=2\parindent]
			\item
			\(xy = xz\)이면 \(y = z\). (\(x, y, z \in F, x \neq 0\))
			\item
			\(y \in F\)가 [\(xy = yx = x\) for \(x \in F\)]를 만족하면 \(y = 1\).\footnote{사실 이것이 없으면 (F7)은 non-sense.}
			\item
			\(xy = yx = 1\)이면 \(y = x^{-1}\). (\(x, y \in F, x \neq 0\))
			\item
			For \(x \in F \backslash \{0\}\), \((x^{-1})^{-1} = x\).
		\end{enumerate}
	\end{prop}
	\begin{proof}
		Proposition \ref{field addition}의 증명과 거의 같다.
	\end{proof}

	\begin{prop}
		Field \(F\)에 대하여 다음이 성립한다(\(x, y \in F\)).
		\begin{enumerate} [label=(\alph*), leftmargin=2\parindent]
			\item
			\(0x = 0\).
			\item
			\(xy = 0\)이면 \(x = 0\) or \(y = 0\).
			\item
			\((-1)x = -x\).
			\item
			\((-x)y = x(-y) = -(xy)\)이고 \((-x)(-y)=xy\).
		\end{enumerate}
	\end{prop}
	\begin{proof}
		\quad\\
		(a): \(0x + 0 = 0x = (0+0)x = 0x + 0x\)이므로 Proposition \ref{field addition} (a)에 의해.\\
		(b): \(xy = 0\)이면서 \(x \neq 0, y \neq 0\)이면 \(x^{-1}, y^{-1}\)이 존재하므로
		\begin{gather*}
			0 = 0(y^{-1}x^{-1}) = (xy)(y^{-1}x^{-1}) = xx^{-1} = 1
		\end{gather*}
		이므로 모순.\\
		(c): \(x + (-x) = 0 = 0x = (1 + (-1))x = 1x + (-1)x = x + (-1)x\)이므로 Proposition \ref{field addition} (a)에 의해.\\
		(d): (c)에 의해 당연.
	\end{proof}

	\begin{ex}
		\(\QQ, \RR, \CC\) 등은 모두 field입니다. 주로 \(\RR\)을 다룰 것입니다.
	\end{ex}

\subsection{Order}

어떤 수가 다른 수보다 `크다'고 하는 것은 어떤 의미일까요? 순서의 개념을 일반화합니다.

	\begin{defn}
		집합 \(S\)에 주어진 관계(relation) \(>\)가 다음 조건을 만족하면 \(>\)를 \textbf{strict partial order}라고 한다.
		\begin{enumerate} [label=(\alph*), leftmargin=2\parindent]
			\item
			\(x > x\)인 \(x \in S\)는 존재하지 않는다.
			\item
			\(x > y\)이고 \(y > z\)이면 \(x > y > z\). (\(x, y, z \in S\))
		\end{enumerate}
		Strict partial order \(>\)가 다음 조건까지 만족하면 \(>\)를 \textbf{strict total order}라고 한다.
		\begin{enumerate}[label=(\alph*), leftmargin=2\parindent]
			\setcounter{enumi}{2}
			\item
			\(x, y \in S\)에 대해 \(x > y, x = y, y > x\) 중 오직 하나만 참.
		\end{enumerate}
	\end{defn}

	\begin{ex}
		\(\RR, \QQ, \ZZ, \NN\) 등에 주어진 일상적인 의미의 \(>\)가 strict total order임은 쉽게 확인할 수 있습니다.
	\end{ex}

	\begin{rem}
		편의를 위해 \(<, \ge, \le\)의 기호도 우리가 생각하는 의미 그대로 사용하기로 합니다.
	\end{rem}

	\begin{defn}
		Strict total order \(>\)가 주어진 field \(F\)가 다음 조건을 만족할 때 \(F\)를 \textbf{ordered field(순서체)}라고 한다.
		\begin{enumerate} [label=(\alph*), leftmargin=2\parindent]
			\item
			\(x < y\)이면 \(x + z < y + z\). (\(x, y, z \in F\))
			\item
			\(x > 0, y > 0\)이면 \(xy > 0\). (\(x, y \in F\))
		\end{enumerate}
	\end{defn}

	\begin{ex}
		\(\RR\)이나 \(\QQ\)에 일상적인 의미의 \(>\)를 주면 ordered field가 됩니다.
	\end{ex}

	\begin{defn} \label{bdd in R}
		Ordered field \(F\)의 부분집합 \(S \subseteq F\)가 \textbf{bounded above(위로 유계)}라는 것은 \(\beta \in F\)가 존재하여 [\(x \le \beta\) for all \(x \in S\)]라는 것이다. 이때 \(\beta\)를 \textbf{upper bound(상계) of \(S\)}라고 한다. 부등호 방향을 반대로 하여 \textbf{bounded below(하계)}와 \textbf{lower bound(하계)}도 정의한다. \(S\)가 bounded above and bounde below이면 \(S\)는 \textbf{bounded(유계)}라고 한다.
	\end{defn}
	
	\begin{defn}
		Ordered field \(F\)의 bounded above subset \(S\)에 대하여 \(\alpha \in F\)가 다음 조건을 만족할 때 \(\alpha\)를 \textbf{least upper bound=supremum(최소상계=상한) of \(S\)}라고 하고, \(\alpha = \sup S\)로 쓴다.
		\begin{enumerate} [label=(\alph*), leftmargin=2\parindent]
			\item
			\(\alpha\) is an upper bound of \(S\).
			\item
			If \(\gamma \in F\) is an upper bound of \(S\), then \(\alpha \le \gamma\).
		\end{enumerate}
	\end{defn}

\subsection{Complete Ordered Field}

Ordered field \(F\)의 bounded above subset \(S \subseteq F\)의 supremum은 항상 존재할까요? 이 질문에 긍정적인 답을 하게 할 수 있게 하는 \(F\)가 우리의 관심사입니다.

	\begin{defn} \label{comp axiom}
		다음 두 성질을 만족하는 ordered field \(F\)는 \textbf{least-upper-bound property} 또는 \textbf{completeness axiom(완비성공리)}을 만족시킨다고 하며 \(F\)를 \textbf{complete ordered field(완비순서체)}라고 한다.
		\begin{enumerate} [label=(\alph*), leftmargin=2\parindent]
			\item
			Nonempty bounded above subset \(S \subseteq F\) has a supremum in \(F\).
			\item
			Nonempty bounded below subset \(S \subseteq F\) has a infimum in \(F\).
		\end{enumerate}
	\end{defn}

사실 (a)와 (b)는 동치입니다.

	\begin{prop}
		Definition \ref{comp axiom}의 (a)와 (b)는 동치이다.
	\end{prop}
	\begin{proof}
		(a)$\Rightarrow$(b)만 보이면 역은 마찬가지 방법으로 보일 수 있다. (a)를 가정하고, nonempty bounded above subset \(S \subseteq F\)와 \(S\)의 모든 lower bound들의 집합 \(T\)를 생각하자. \(S\)가 bounded below이므로 \(T\)는 공집합이 아니고, \(S\)의 아무 원소를 가져오면 그 원소는 \(T\)의 upper bound이므로 \(T\)는 nonempty bounded above이다. 가정 (a)에 의해 \(T\)는 supremum \(\alpha \in F\)를 가진다. 우리의 주장: \(\alpha\)는 \(S\)의 infimum이다.\\
		임의의 \(x \in S\)에 대해 \(x\)는 \(T\)의 upper bound이므로 \(\alpha \le x\)이다. 따라서 \(x\)는 \(S\)의 lower bound이다. 한편 \(\gamma \in F\)가 \(S\)의 lower bound라고 하면 \(T\)의 정의에 의해 \(\gamma \in T\)인데, \(\alpha = \sup T\)이므로 \(\gamma \le \alpha\)이다. 따라서 \(\alpha = \inf S\)이다.
	\end{proof}

모든 ordered field가 complete인 것은 아닙니다. 대표적으로 \(\QQ\)는 complete가 아닌데, \(\{x \in \QQ: x^2 < 2\}\)는 nonempty bounded above subset of \(\QQ\)이지만 \(\QQ\)에서 supremum을 가지지 않음을 쉽게 확인할 수 있습니다.

그렇다면 실제로 complete ordered field가 존재하는지가 궁금한데, 그 답은 긍정적입니다.

	\begin{thm}
		Complete ordered field \(F\)가 존재하고, 그 \(F\)를 실수체 \(\RR\)이라고 부른다.
	\end{thm}

수학에서 어떤 것의 존재성을 보일 때는 다른 정리의 도움을 받거나, 아니면 실제로 그 대상을 구성(construct)해야 합니다.

우리가 수 체계를 배웠던 과정을 생각해보면, 자연수에서 정수, 유리수까지는 나름 자연스러운 확장이었고, 실수에서 복소수도 자연스러운 확장인데 – 유리수와 실수 사이에 어떤 gap이 있었습니다. 적어도 저한테는 그랬습니다. 제곱해서 2가 되는 유리수가 없다는 것으로 유리수체가 어떤 의미로 ‘불완전함’을 설명하곤 하지만, 이는 유리수체에서 실수체로의 확장이라기보다는 유리수체에서 field of algebraic number(algebraic number, 대수적 수: 유리계수 다항식의 근)로의 확장에 가깝습니다. 이제서야 우리는 실수가 어떻게 만들어졌는지 알았습니다. \textbf{유리수체 \(\QQ\)의 빈틈을 채워 넣어서 completeness axiom을 만족하도록 실수체 \(\RR\)을 직접 ‘구성’한 것입니다.} 그래서 실수체가 complete인 것은 증명의 대상이 아닌데, 왜냐하면 그것을 만족하도록 실수체를 만들었기 때문입니다.

\(\QQ\)로부터 \(\RR\)를 구성하여 정의하는(실수의 구성적 정의) 대표적인 방법에는 두 가지가 있는데, 하나는 Cantor의 방법이고 하나는 Dedekind의 방법입니다. 다른 글에서 이를 다룰 기회가 있을 것입니다.

여기서 드는 한 가지 의문은, 두 가지 방법으로 만든 실수체가 동일한가 하는 것입니다. 좀 더 일반적으로, complete ordered field는 유일한가라는 질문을 던질 수 있는데, 실제로 그렇습니다. 이 증명 역시 다른 글에서 다룰 기회가 있을 것입니다.

	\begin{thm}
		The complete ordered field is unique up to isomorphism. i.e., for any complete ordered fields \(F_1\) and \(F_2\), there exists a bijection \(f: f_1 \rightarrow F_2\) satisfying:
		\begin{enumerate} [label=(\alph*), leftmargin=2\parindent]
			\item
			\(x < y \Longrightarrow f(x) < f(y) \quad (x, y \in F_1)\)
			\item
			\(f(x + y) = f(x)+f(y), f(xy)=f(x)f(y) \quad (x, y \in F_1)\)
		\end{enumerate}
		따라서 \(\RR\)은 유일한 complete ordered field이다.
	\end{thm}

\newpage

\section{Sequence} \label{sec sequence}

\subsection{Metric Space}

수열의 수렴을 이야기하기 위해 거리의 개념을 도입하려고 합니다.

	\begin{defn} \label{metric}
		집합 \(X\)와 함수 \(d: X \times X \rightarrow \RR\)가 주어졌을 때 임의의 \(x, y, z \in X\)에 대해 다음이 성립하면 \((X, d)\) 또는 간단하게 \(X\)가 \textbf{metric space(거리공간)}라고 하고 \(d\)를 \textbf{metric on \(X\)}라고 한다.
		\begin{enumerate} [label=(M\arabic*), leftmargin=2\parindent]
			\item
			\(d(x, y) = 0\) if and only if \(x = y\).
			\item
			\(d(x, y) = d(y, x)\).
			\item
			\(d(x, z) \le d(x, y) + d(y, z)\)
		\end{enumerate}
	\end{defn}

	\begin{prop}
		Metric space \(X\)에서 \(d(x, y) \ge 0\) for all \(x, y \in X\).
	\end{prop}
	\begin{proof}
		\(0 = d(x, x) \le d(x, y) + d(y, x) = 2d(x, y)\).
	\end{proof}

	\begin{ex}
		\quad
		\begin{enumerate} [label=(\alph*), leftmargin=2\parindent]
			\item
			\(\RR\) 또는 \(\CC\)에서 \(d(x, y) = \abs{x - y}\)로 정의하면 \(d\)는 (M1)-(M3)을 만족하므로 \((\RR, d), (\CC, d)\)는 metric space.
			\item
			평면이나 공간에서 우리가 일상적으로 생각하는 `거리'는 (M1)-(M3)을 만족하므로\footnote{} 평면과 공간은 metric space.
			\item
			아무 집합 \(X\)에서 \(x, y \in X\)에 대해 \(d(x, y)\)를 \(x = y\)일 때 0, \(x \neq y\)일 때 1로 정의하면 (M1)-(M3)을 만족하므로 \((X, d)\)는 metric space.
		\end{enumerate}
	\end{ex}

\subsection{Convergence of a Sequence}

	\begin{defn} \label{converge}
		집합 \(X\)에 대하여 \(\NN\)(또는 \(\ZZ\))에서 \(X\)로 가는 함수를 \textbf{sequence(수열) in \(X\)}라고 하고 \((x_n)\)과 같이 쓴다.\\
		\(X\)가 metric space일 때 \(X\) 안의 sequence \((x_n)\)이 \(x \in X\)로 \textbf{converge(수렴)}한다는 것은 임의의 양수 \(\eps > 0\)에 대해 다음을 만족시키는 \(N \in \NN\)이 존재하는 것이다.
		\begin{gather*}
			n \ge N \Longrightarrow d(x_n, x) < \eps
		\end{gather*}
		이때 \(\lim_{n \rightarrow \infty}x_n = x, \lim_n x_n = x, x_n \rightarrow x\)와 같이 쓴다.
	\end{defn}

	\begin{prop}
		Metric space \(X\) 안의 sequence \((x_n)\)이 \(x \in X\)로 수렴하고 \(x' \in X\)로 수렴하면 \(x = x'\).
	\end{prop}
	\begin{proof}
		\(x \neq x'\)라고 가정하면 metric의 정의에 의해 \(d(x, x') > 0\)이다. \(r = d(x, x') / 2\)라고 하자. 극한의 정의에 의해 다음을 만족하는 \(N_1, N_2\)가 존재한다.
		\begin{gather*}
			n \ge N_1 \Longrightarrow d(x_n, x) < r\\
			n \ge N_2 \Longrightarrow d(x_n, x') < r
		\end{gather*}
		\(N = \max\{N_1, N_2\}\)라고 하면 \(d(x, x') \le d(x, x_N) + d(x, x_N') < 2r = d(x, x')\)이므로 모순.
	\end{proof}

	\begin{rem}
		이 명제는 Section \ref{cpt}의 각주에서 다시 한 번 언급될 것입니다.
	\end{rem}

	\begin{defn} \label{bdd}
		Metric space \(X\)의 부분집합 \(S \subseteq X\)가 \textbf{bounded(유계)}라는 것은 어떤 양수 \(M > 0\)이 존재하여 [\(d(x, y) < M\) for all \(x, y \in S\)]라는 것이다.\\
		Sequence \((x_n)\) in \(X\)가 \textbf{bounded}라는 것은 \(X\)의 부분집합 \(\{x_n\}_{n \in \NN}\)이 bounded라는 것이다.
	\end{defn}
	
	\begin{rem}
		\(X = \RR\)일 때 Definition \ref{bdd in R}에서의 bounded의 정의와 Definition \ref{bdd}의 정의는 동치입니다(증명 생략).
	\end{rem}

	\begin{prop} \label{conv bdd}
		Metric space \(X\) 안의 sequence \((x_n)\)이 수렴하면 \((x_n)\)은 유계이다.
	\end{prop}
	\begin{proof}
		\(x_n \rightarrow x\)라고 하면 극한의 정의에 의해 \(n \ge N \Rightarrow d(x_n, x) < 1\) for some \(N\)이다. 이때 \(\{x_n\}_{n < N}\)은 유한집합이므로 유계이고, \(\{x_n\}_{n \ge N}\)의 아무 점 2개를 잡아도 두 점 사이의 거리는 2보다 작으므로 유계이다. 유계 부분집합 2개의 합집합은 유계이므로 \(\{x_n\}_{n \in \NN}\)도 유계이다.
	\end{proof}

이제부터 Section \ref{sec sequence}\가 끝날 때까지 metric space \(X\)를 \(d(x, y) = \abs{x - y}\)가 주어진 \(\RR\)로 한정하고 이 안의 실수열만 생각합니다.

다음 성질을 증명하는 것은 연습문제로 남깁니다.

	\begin{prop}
		실수열 \((x_n\)과 \((y_n)\)이 각각 \(x, y \in \RR\)로 수렴할 때 다음이 성립한다.
		\begin{enumerate} [label=(\alph*), leftmargin=2\parindent]
			\item
			\(x_n + y_n \rightarrow x + y\).
			\item
			\(cx_n \rightarrow cx\) for \(c \in \RR\).
			\item
			\(x_ny_n \rightarrow xy\).
			\item
			\(x_n/y_n \rightarrow x/y\), if \(y_n \neq 0\) for all \(n\) and \(y \neq 0\).
		\end{enumerate}
	\end{prop}

\subsection{Monotonic Sequence}

Definition \ref{converge}\를 보면, 어떤 수열 \((x_n)\)의 수렴을 보이기 위해서 \textbf{수렴할 것으로 예상되는 점 \(x\)를 먼저 찾아야} 합니다. 그런데 그 \(x\)를 찾는 것이 쉽지 않을 때가 많습니다. 그래서 우리는 어디로 수렴하는지 모르는 상태에서 수열의 수렴을 판정할 수 있는 많은 방법들을 개발해야 하는데, 단조수열은 그 중 첫 번째입니다.

	\begin{defn}
		실수열 \((x_n)\)이 \textbf{monotonically increasing=nondecreasing sequence(단조증가수열)}이라는 것은 \(x_n \le x_{n+1}\) for all \(n\)이라는 것이다. 마찬가지 방법으로 \textbf{monotonically decreasing=nonincreasing sequence(단조감소수열)}도 정의한다. Monotonically increasing or monotonically decreasing sequence를 \textbf{monotonic sequence(단조수열)}이라고 한다.
	\end{defn}

다음은 단조수열이 \(\RR\)에서 수렴할 필요충분조건입니다.

	\begin{thm} [Monotone Convergence Theorem, 단조수렴정리]
		실수열 \((x_n)\)이 단조수열일 때, \((x_n)\) is convergent if and only if \((x_n)\) is bounded.
	\end{thm}
	\begin{proof}
		($\Rightarrow$) 방향은 Proposition \ref{conv bdd}에 의해 성립하므로 그 역만 보이면 된다. 일반성을 잃지 않고 \((x_n)\)이 유계 단조증가수열이라고 가정하자. 그러면 \(S = \{x_n\}_{n \in \NN}\)은 nonempty bounded above subset of \(\RR\)이므로, \textbf{\(\RR\)의 완비성에 의해} \(S\)의 supremum \(x \in \RR\)가 존재한다. 주장: \(x_n \rightarrow x\).\\
		임의의 \(\eps > 0\)에 대해 \(x - \eps\)은 \(S\)의 upper bound가 아니므로, \(x_N > x - \eps\)인 \(N\)이 존재한다. \((x_n)\)은 단조증가수열이므로 \(n \ge N\)에 대해 \(x - \eps < x_N \le x_n \le x\). 따라서 \((x_n)\)은 \(x\)로 수렴한다.
	\end{proof}
	\begin{rem}
		\(\RR\)의 완비성의 핵심적인 결과들 중 하나입니다. 이 정리는 complete가 아닌 field, 예를 들면 \(\QQ\)에서 성립하지 않습니다.
	\end{rem}

\subsection{Upper Limit and Lower Limit} \label{sec limsup}

	\begin{ex}
		실수열 \((x_n)\)을 다음과 같이 정의합니다.
		\begin{gather*}
			x_n = 
			\begin{cases}
				\frac{1}{n}, & \text{if } n \text{ is odd}\\
				-n, & \text{if } n \text{ is even}
			\end{cases}
		\end{gather*}
		\((x_n)\)이 수렴하지 않는 것은 명백합니다. 홀수항만 취해서 보면 이 수열의 `일부분'은 0으로 수렴하는 단조감소수열입니다. 이 0에 어떤 의미를 부여하려고 하는 것이 Section \ref{sec limsup}의 목표입니다.
	\end{ex}

먼저 표기법 하나.

	\begin{notn}
		실수열 \((x_n)\)에 대하여, 임의의 \(M > 0\)에 대해 적당한 \(N \in \NN\)이 존재하여 \(n \ge N \Rightarrow x_n > M\)을 만족시키면 \(\lim_n x_n = \infty\)로 쓴다. 마찬가지 방법으로 \(\lim_n x_n = -\infty\)로 쓴다.\\
		Nonempty \(S \subseteq \RR\)가 위로 유계가 아닐 때 \(\sup S = \infty\), 아래로 유계가 아닐 때 \(\inf S = -\infty\)로 쓴다.\\
		\(x_n \rightarrow \alpha\) for \(\alpha \in \RR \cup \{\pm \infty\}\)라는 것은 \((x_n)\)이 \(\alpha \in \RR\)로 수렴하거나, \(\lim_n x_n = \infty\) 또는 \(\lim_n x_n = -\infty\)라는 것이다. (즉, 진동하지 않는다는 의미.)
	\end{notn}

	\begin{obs} \label{obs sup}
		Nonempty subset \(A, B \subseteq \RR\)에 대하여 \(A \subseteq B\)이면 \(\sup A \le \sup B\)이고 \(\inf A \ge \inf B\). \(\sup\)이나 \(\inf\)의 값이 \(\pm\infty\)인 경우에도 성립한다.
	\end{obs}
	\begin{proof}
		당연하다.
	\end{proof}

	\begin{defn} \label{def limsup}
		실수열 \((x_n)\)의 \textbf{upper limit(상극한)}을 다음과 같이 정의하고, 그 값을 \(\limsup_n x_n \in \RR \cup \{\pm \infty\}\)으로 쓴다.
		\begin{enumerate} [label=(\alph*), leftmargin=2\parindent]
			\item
			\((x_n)\)이 위로 유계가 아니면 \(\limsup_n x_n = \infty\)로 정의.
			\item
			\((x_n)\)이 위로 유계이면 각 자연수 \(n\)에 대하여 집합 \(S_n = \{x_k\}_{k \ge n}\)을 생각한다. 이때 각 \(S_n\)은 위로 유계이므로 supremum이 유한한 실수로 존재한다. 이 값을 \(y_n\)으로 정의하자. 즉 \(y_n = \sup S_n\). Observation \ref{obs sup}에 의해 \((y_n)\)은 단조감소수열이다. 이때 \(\limsup_n x_n = \lim_n y_n \in \RR \cup \{-\infty\}\)로 정의한다. (\((y_n)\)이 아래로 유계이면 \(\lim_n y_n\)은 유한한 실수이고, 그렇지 않으면 \(-\infty\)이다.)
		\end{enumerate}
		마찬가지 방법으로 \textbf{lower limit(하극한)}을 정의하고 그 값을 \(\liminf_n x_n \in \RR \cup \{\pm \infty\}\)으로 쓴다.
	\end{defn}

도대체 이게 무슨 소리인가? 필자도 이 개념을 이해하는 데 상당히 많은 시간이 걸렸습니다. 다음 정리는 이 정의를 좀 더 구체적으로 설명해 줍니다.

	\begin{thm} \label{thm limsup}
		실수열 \((x_n)\)과 \(\alpha \in \RR\)에 대하여 \(\limsup_n x_n = \alpha\)일 필요충분조건은 다음 (a), (b)가 성립하는 것이다.
		\begin{enumerate} [label=(\alph*), leftmargin=2\parindent]
			\item
			임의의 \(\eps > 0\)에 대해 적당한 \(N \in \NN\)이 존재하여 \(n \ge N \Rightarrow x_n < \alpha + \eps\).
			\item
			임의의 \(\eps > 0\)에 대해 \(x_n > \alpha - \eps\)를 만족하는 \(n\)이 무수히 많이 존재한다.
		\end{enumerate}
	\end{thm}
	\begin{proof}
		($\Leftarrow$): 먼저 \(\limsup_n x_n = \alpha\)라고 가정하고 \(\eps > 0\)이 주어졌다고 하자. Definition \ref{def limsup}에서와 같이 단조감소수열 \((y_n)\)을 정의하면 \(\lim_n y_n = \alpha\)이므로 \(y_N < \alpha + \eps\)인 \(N\)이 존재한다. 이때 \(n \ge N\)이면 \(x_n \le y_N < \alpha + \eps\)이므로 (a)가 성립한다. 이제 (b)를 부정하여 \(x_n > \alpha + \eps\)을 만족하는 \(n\)이 유한 개만 존재한다고 가정하자. 그러면 충분히 큰 자연수 \(K\)가 존재하여 \(n \ge K \Rightarrow x_n \le \alpha - \eps\), 즉 \(y_K \le \alpha - \eps\)이다. 그런데 \((y_n)\)은 단조감소수열이므로 \(\alpha \le y_K = \alpha - \eps\)이고 따라서 모순이다.\\
		($\Rightarrow$): 이제 (a)와 (b)를 가정하고 \(\eps > 0\)이 주어졌다고 하자. (a)에 의해 \((x_n)\)은 위로 유계이므로 Definition \ref{def limsup}에서와 같이 \((y_n)\)이 잘 정의되고, \(\lim_n y_n \le \alpha + \eps\)이다. 한편 (b)에 의해, 모든 \(n\)에 대해 [\(x_n > \alpha - \eps\) for some \(m > n\)]이므로 \(y_n \ge \alpha - \eps\)이다. 따라서 \((y_n)\)은 유계 단조수열이므로 수렴하고 \(\alpha- \eps \le \lim_n y_n \le \alpha + \eps\)이다. 그런데 \(\eps > 0\)은 임의의 양수이므로 \(\limsup_n x_n = \lim_n y_n = \alpha\)이다.
	\end{proof}

하극한에 대해 마찬가지의 정리를 state하고 증명하는 것은 연습문제로 남깁니다.

상극한과 하극한이 유용한 한 가지 이유는 다음 따름정리 때문입니다.

	\begin{cor}
		실수열 \((x_n)\)과 \(\alpha \in \RR\)에 대하여 \(\lim_n x_n = \alpha\) if and only if \(\limsup_n x_n = \liminf_n x_n = \alpha\).
	\end{cor}
	\begin{proof}
		Theorem \ref{thm limsup}\과 그 하극한 version에 의해 자명.
	\end{proof}

	\begin{cor}
		\(n = 0, 1, \ldots\)에서 정의된 실수열 \((s_n)\)에 대하여
		\begin{gather*}
			\sigma_n = \frac{1}{n}\sum_{k=0}^{n-1}s_k
		\end{gather*}
		로 정의하자. \((s_n)\)이 \(s \in \RR\)로 수렴하면 \((\sigma_n)\)도 \(s\)로 수렴한다.
	\end{cor}
	\begin{proof}
		일반성을 잃지 않고 \(s = 0\)이라고 가정할 수 있다. 양수 \(\eps > 0\)을 고정하고, 다음을 만족시키는 \(N\)을 찾는다.
		\begin{gather*}
			n \ge N \Longrightarrow \abs{s_n} < \eps
		\end{gather*}
		그러면 \(n > N\)에 대해
		\begin{align*}
			\abs{\sigma_n} = \frac{1}{n}\abs{\sum_{k=0}^{n-1}s_k} &\le \frac{1}{n}\abs{\sum_{k=0}^{N}s_k} + \frac{1}{n}\sum_{k=N+1}^{n}\abs{s_k}\\
			&< \frac{1}{n}\abs{\sum_{k=0}^{N}s_k} + \frac{(n-N)\eps}{n}\\
			&<  \frac{1}{n}\abs{\sum_{k=0}^{N}s_k} + \eps
		\end{align*}
		이다. 이제 양변에 \textbf{상극한}을 취한다. (바로 극한을 못 취하는 이유는 좌변이 수렴하는지를 아직 알지 못하기 때문이다. 상극한은 항상 \(\RR \cup \{\pm \infty\}\) 안에서 존재한다.) \(\abs{\sum_{k=0}^{N}s_k}\)은 \(n\)과 무관한 상수이므로,
		\begin{gather*}
			\limsup_n \abs{\sigma_n} \le \limsup_n \left ( \frac{1}{n}\abs{\sum_{k=0}^{N}s_k} + \eps \right )= \lim_n \left ( \frac{1}{n}\abs{\sum_{k=0}^{N}s_k} + \eps \right )= \eps
		\end{gather*}
		이다. 따라서 \(0 \le \liminf_n \abs{\sigma_n} \le \limsup_n \abs{\sigma_n} \le \eps\)인데, \(\eps > 0\)은 임의의 양수였으므로 \(\liminf_n \abs{\sigma_n} = \limsup_n \abs{\sigma_n} = 0\)이고 \(\lim_n \abs{\sigma_n} = \lim_n \sigma_n = 0\)이다.
	\end{proof}



\section{Euclidean Space (작성 예정)}

\section{Open Set and Closed Set}

\section{Completeness of Euclidean Spaces}

\section{Series}

\section{Compactness} \label{cpt}

\section{Connectedness}

\section{Continuity}

\section{Differentiation}


\begin{thebibliography}{99}
	
	\bibitem{Cd94}\label{ref1} 김성기, 김도한, 계승혁. \emph{해석개론}. 서울대학교출판문화원. 2011.
	\bibitem{Cd94}\label{ref2} W. Rudin. \emph{Principles of Mathematical Analysis}. McGraw-Hill. 1976.
	\bibitem{Cd94} 이인석. \emph{선형대수와 군}. 서울대학교출판문화원. 2015.
		
\end{thebibliography}
\end{document}